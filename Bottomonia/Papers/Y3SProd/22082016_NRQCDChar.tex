\documentclass[twocolumn,amsmath,amssymb]{snp}
\pagestyle{empty}
\usepackage{graphicx}% Include figure files
\usepackage{dcolumn}% Align table columns on decimal point
\usepackage{bm}% bold math
\usepackage{amsmath}
\usepackage{ulem}
\topmargin 1.5 cm
\textwidth14.5cm
\textheight20cm
\oddsidemargin0.7cm
\columnsep0.2in
\begin{document}
\title{\Large High p$_{T}$ $\Upsilon$(3S) production at LHC energies} 
\author{\large Vineet Kumar}
\email{vineet.salar@gmail.com}
\author{\large Prashant Shukla}
\affiliation{Nuclear Physics Division, Bhabha Atomic Research Centre, Mumbai}
\maketitle
\section*{Introduction}
The quarkonia ($Q\bar Q$) have provided useful tools for probing both 
perturbative and nonperturbative aspects of Quantum Chromodynamics (QCD). 
The Quarkonia states are qualitatively different from most other hadrons since 
the velocity of the heavy constituents is small allowing a 
non-relativistic treatment of bound states. The NRQCD formalism is one 
of the most promising theoretical framework for the study of heavy quarkonium 
production~\cite{Bodwin:1994jh}. The study of the differential charmonia production 
cross sections in high energy p+p collisions is completed using 
NRQCD formalism~\cite{Kumar:2016ojy}.
%In this work we calculate the production cross section of charmonia in p$+$p collisions
%at $\sqrt{s}$ = 7 TeV using Non Relativistic QCD (NRQCD) formalism. 
\section*{Bottomonia Production in p$+$p collisions}
Under NRQCD, the cross-section for direct production of a resonance $H$ in 
a collision of particles $A$ and $B$ can be expressed in factorized form 
\begin{equation}
  \begin{split}
    &E\frac{d^{3}\sigma^{ab\rightarrow cd}}{d^{3}p}(^{(2S+1)}L_{J}) = \sum_{a,b}\int dx_a\,dx_b \\
    &G_{a/A}(x_a,\mu_{F}^{2}) G_{b/B}(x_b,\mu_{F}^{2})\frac{\hat s}{\pi}\frac{d\sigma}{d\hat t}\\
    &(ab\rightarrow^{(2S+1)}L_{J}c)\cdot \delta(\hat s + \hat t + \hat u -M^{2}) \nonumber
\end{split}
\end{equation}
where, $G_{a/A}(G_{b/B})$ is the parton distribution function (PDF) of the incoming parton $a(b)$ in the 
incident hadron $A(B)$, which depends on the momentum fraction $x_a(x_b)$ and the factorization 
scale $\mu_F$. The short distance contribution $d\sigma/d\hat t$ can be calculated 
within the framework of perturbative QCD (pQCD). $(ab\rightarrow^{(2S+1)}L_{J}c)$ 
repersents the probability of evolving a bound quarkonia state from a Q$\overline{\rm Q}$
pair. These probabilites (LDMEs) are in nonperturbative region of QCD and should 
be estimated from the experimental measurements. 
In this work we calculated the production cross section of b$\overline{\rm b}$ bound states 
in p$+$p collisions $\sqrt{s}$ = 7 TeV and use the measured data from CMS and ATLAS 
collaborations~\cite{Khachatryan:2015qpa,Aad:2012dlq} to constrain the LDMEs required for
$\Upsilon$(3S) production.
%The dominant processes in evaluating the differential 
%yields of heavy quarkonia as a function of $p_T$ are $g+q\rightarrow H+q$, $q+\bar{q}\rightarrow H+g$ and
%$g+g\rightarrow H+g$, where $H$ refers to the heavy meson. 
%The differential cross-section for the short distance contribution i.e. the heavy quark pair production 
%from the reaction of the type $a\,+\,b\,\rightarrow\,c\,+\,d$ can be written as~\cite{Owens:1986mp}
%\begin{equation}
%  \begin{split}
%    \frac{{d\sigma}^{ab\rightarrow cd}}{dp_T\,dy} = &\int dx_a\, G_{a/A}(x_a,\,\mu^{2}_{F})\\
%    &G_{b/B}(x_b,\,\mu^{2}_{F})\times 2p_T \\
%    &\frac{x_a\,x_b}{x_a-\frac{m_T}{\sqrt{s}}e^y}\frac{d\sigma}{d\hat t}(ab\rightarrow cd),
%\end{split}
%\end{equation}
%where, $\sqrt{s}$ being the total energy in the centre-of-mass and $y$ is the rapidity of 
%the $Q\bar Q$ pair. In our calculations we use CTEQ6M~\cite{Lai:2010vv} for the parton 
%distribution functions. The invariant differential cross-section is given by
%\begin{equation}
%\frac{d\sigma}{d\hat t} = \frac{|\mathcal{M}|^2}{16\pi{\hat s}^2},
%\end{equation}
%where $\hat s$ and $\hat t$ are the parton level Mandelstam variables and $\mathcal{M}$ is the 
%feynman amplitude for the process.
\begin{figure}
  \includegraphics[width=0.49\textwidth]{Fig1_Y3S_CMS.pdf}
  \caption{The NRQCD calculations of production cross section of $\Upsilon$(3S) in p+p collisions 
    as a function of transverse momentum compared with the measured CMS data at $\sqrt{s}$ = 7 
    TeV~\cite{Khachatryan:2015qpa}. The LDMEs are obtained by a combined fit of the 
    CMS and ATLAS data.}
  \label{Fig:SigmaY3SCMS}
\end{figure}
\begin{figure}
  \includegraphics[width=0.49\textwidth]{Fig2_Y3S_ATLAS.pdf}
  \caption{The NRQCD calculations of production cross section of $\Upsilon$(3S) in p+p collisions 
    as a function of transverse momentum compared with the measured ATLAS data at $\sqrt{s}$ = 7 
    TeV~\cite{Aad:2012dlq}. The LDMEs are obtained by a combined fit of the 
    CMS and ATLAS data.}
  \label{Fig:SigmaY3SATLAS}
\end{figure}
The LDMEs are predicted to scale with a definite power of the relative velocity $v$ of the heavy constituents 
inside $Q\bar Q$ bound states. In the limit $v<<1$, the production of quarkonium is based on the $^3S_1^{[1]}$ 
and $^3P_J^{[1]}$ ($J$ = 0,1,2) Color Singlet states, $^1S_0^{[8]}$, $^3S_1^{[8]}$ and $^3P_J^{[8]}$ Color 
Octet states.  The differential cross section for the direct production of $\Upsilon$(3S) can be written as the 
sum of these contributions,
\begin{eqnarray}
%\begin{split}
d\sigma(\Upsilon(3S)) &= d\sigma(Q\overline{Q}([^3S_1]_{1}))
                   \,M_{L}([^3S_1]_{1}) \nonumber \\
                &+\, d\sigma(Q\overline{Q}[^1S_0]_{8}))
                   \,M_{L}([^1S_0]_{8}) \nonumber \\ 
                &+ \, d\sigma(Q\overline{Q}([^3S_1]_{8}))
                   \,M_{L}([^3S_1]_{8}) \nonumber \\
                &+ \, d\sigma(Q\overline{Q}([^3P_J]_{8}))
                   \,M_{L}([^3P_0]_{8})\nonumber \\ \nonumber
                %&+  d\sigma(Q\overline{Q}([^3P_1]_{8}))
                 %  \,M_{L}([^3P_1]_{8})\\
                %&+  d\sigma(Q\overline{Q}([^3P_2]_{8}))
                 %  \,M_{L}([^3P_2]_{8})\\
\label{eq:dsigmaJ}
%\end{split}
\end{eqnarray}
%In our calculations, we used the expressions for the short distance CS cross-sections 
%given in Refs.~\cite{Baier:1983va,Humpert:1986cy,Gastmans:1987be} and the CO cross-sections given 
%in Refs.~\cite{Cho:1995vh,Cho:1995ce}.
In our calculations, we used the expressions for the short distance cross-sections 
from Refs.~\cite{Baier:1983va,Cho:1995vh}.
\section*{Results and discussion}
$\Upsilon$(3S) is the highest known bound state in the b$\overline{\rm b}$ spectrum 
so it must not have any feed down contribution. 
The expressions and the values for the 
color-singlet LDMEs can be obtained by solving the non-relativistic 
wavefunctions~\cite{Cho:1995vh}. 
The CO LDMEs can not be related to the non-relativistic
wavefunctions of $b \bar b$ since it involves a higher Fock state and thus
measured data~\cite{Khachatryan:2015qpa,Aad:2012dlq} is used to constrain them.
Figure~\ref{Fig:SigmaY3SCMS} shows the NRQCD calculations of production cross section of 
$\Upsilon$(3S) in p+p collisions as a function of transverse momentum compared with the 
measured data in CMS detector at LHC~\cite{Khachatryan:2015qpa}. 
Figure~\ref{Fig:SigmaY3SATLAS} shows the NRQCD calculations of production cross section of 
$\Upsilon$(3S) in p+p collisions as a function of transverse momentum compared with the 
measured data in ATLAS detector at LHC~\cite{Aad:2012dlq}.
The color-singlet contribution along with the calculated value 
and color-octet contributions fitted from data are given below for the 
$\Upsilon$(3S) production.
\begin{eqnarray}
  &M_{L}([^3S_1]_{1}) &= 4.3 \,{\rm GeV^3}\nonumber \\
  &M_{L}([^3S_1]_{8}) &= (0.0725\pm0.0013) \, {\rm GeV^3}\nonumber \\
  &M_{L}([^1S_0]_{8}) &= (0.0126\pm0.0008) \,{\rm GeV^3} \nonumber \\ \nonumber
  &&=  M_{L}([^3P_0]_{8})/5m_{b}^{2}\,\,{\rm GeV^3} \nonumber \\ \nonumber
\end{eqnarray}
We did a combined fitting of CMS and ATLAS data. The common color octet LDMEs are extracted which 
explains both the datasets simultaneously. The $\chi^2/dof$ is 3.25 for the combined fitting.
Our value of $M_{L}([^3S_1]_{8})$ is compitable with the value obtained in the recent 
calculations~\cite{Sharma:2012dy} while the value of $M_{L}([^1S_0]_{8})$ is larger than
the value of Ref.~\cite{Sharma:2012dy}. We significantly improve the large ($\approx 300\%$) error 
present on the $M_{L}([^1S_0]_{8})$ by using combined fitting and latest LHC data.
\section*{Summary}
We have calculated the differential production cross-section of $\Upsilon$(3S) meson
as a function of transverse momentum. The data from LHC experiments are used to 
obtain the values of color octer LDMEs. We plan to present a rigorous study of 
production of bottomonia states at LHC energies using NRQCD. The reevaluation of all LDMEs 
required for bottomonia production is in progress using latest data from LHC.         
\begin{thebibliography}{50}
\bibitem{Bodwin:1994jh}
G.~T.~Bodwin, E.~Braaten, and G.~P.~Lepage,
%``Rigorous QCD analysis of inclusive annihilation and production of heavy
%quarkonium,''
Phys.\ Rev.\ D {\bf 51} 1125 (1995), 
[Erratum-ibid.\ D {\bf 55} 5853 (1997)].
\bibitem{Kumar:2016ojy} 
  V.~Kumar and P.~Shukla,
  %``Charmonia production in p+p collisions under NRQCD formalism,''
  arXiv:1606.08265 [hep-ph].
  %%CITATION = ARXIV:1606.08265;%%
\bibitem{Khachatryan:2015qpa} 
  V.~Khachatryan {\it et al.} [CMS Collaboration],
%  ``Measurements of the $\Upsilon$(1S), $\Upsilon$(2S), and $\Upsilon$(3S) differential 
%  cross sections in pp collisions at $\sqrt{s} =$ 7 TeV,''
  Phys.\ Lett.\ B {\bf 749}, 14 (2015),[arXiv:1501.07750 [hep-ex]].
\bibitem{Aad:2012dlq} 
  G.~Aad {\it et al.} [ATLAS Collaboration],
 % ``Measurement of Upsilon production in 7 TeV pp collisions at ATLAS,''
  Phys.\ Rev.\ D {\bf 87}, no. 5, 052004 (2013),
  [arXiv:1211.7255 [hep-ex]].
%\bibitem{Owens:1986mp} 
%  J.~F.~Owens,
%  %``Large Momentum Transfer Production of Direct Photons, Jets, and Particles,''
%  Rev.\ Mod.\ Phys.\  {\bf 59}, 465 (1987).
%\bibitem{Lai:2010vv} 
%  H.~L.~Lai {\it et al.} %, M.~Guzzi, J.~Huston, Z.~Li, P.~M.~Nadolsky, J.~Pumplin and C.-P.~Yuan,
%  %``New parton distributions for collider physics,''
%  Phys.\ Rev.\ D {\bf 82}, 074024 (2010).
%  %[arXiv:1007.2241 [hep-ph]].
\bibitem{Baier:1983va} 
  R.~Baier and R.~Ruckl,
  %``Hadronic Collisions: A Quarkonium Factory,''
  Z.\ Phys.\ C {\bf 19}, 251 (1983).
%\bibitem{Humpert:1986cy} 
%  B.~Humpert,
%  %``Narrow Heavy Resonance Production By Gluons,''
%  Phys.\ Lett.\ B {\bf 184}, 105 (1987).
%\bibitem{Gastmans:1987be} 
%  R.~Gastmans, W.~Troost and T.~T.~Wu,
%  %``Production of Heavy Quarkonia From Gluons,''
%  Nucl.\ Phys.\ B {\bf 291}, 731 (1987).
\bibitem{Cho:1995vh} 
  P.~L.~Cho and A.~K.~Leibovich,
  %``Color octet quarkonia production,''
  Phys.\ Rev.\ D {\bf 53}, 150 (1996).
  %[hep-ph/9505329].
%\bibitem{Cho:1995ce} 
%  P.~L.~Cho and A.~K.~Leibovich,
%  %``Color octet quarkonia production. 2.,''
%  Phys.\ Rev.\ D {\bf 53}, 6203 (1996).
%  %[hep-ph/9511315].
%\bibitem{Chatrchyan:2011kc} 
%  S.~Chatrchyan {\it et al.} [CMS Collaboration],
%  %``$J/\psi$ and $\psi_{2S}$ production in $pp$ collisions at $\sqrt{s}=7$ TeV,''
%  JHEP {\bf 1202}, 011 (2012).
%%[arXiv:1111.1557 [hep-ex]].
%\bibitem{Khachatryan:2015rra} 
%  V.~Khachatryan {\it et al.} [CMS Collaboration],
%  %``Measurement of J/ψ and ψ(2S) Prompt Double-Differential Cross Sections in pp Collisions at $\sqrt{s}$=7  TeV,''
%  Phys.\ Rev.\ Lett.\  {\bf 114}, no. 19, 191802 (2015).
%  %[arXiv:1502.04155 [hep-ex]].
\bibitem{Sharma:2012dy} 
  R.~Sharma and I.~Vitev,
  %``High transverse momentum quarkonium production and dissociation in heavy ion collisions,''
  Phys.\ Rev.\ C {\bf 87}, 044905 (2013).
%  [arXiv:1203.0329 [hep-ph]].
\end{thebibliography}

\end{document}
