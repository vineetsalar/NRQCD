%\RequirePackage{lineno}
\documentclass[aps,prc,preprint,superscriptaddress,showpacs,showkeys]{revtex4-1}
\usepackage{graphicx}
\usepackage[usenames,dvipsnames,svgnames,table]{xcolor}

\begin{document}

\newcommand{\Jpsi}{J/\psi}
\newcommand{\pT}{p_{T}}
%\linenumbers
\title{{\Large High p$_{T}$ quarkonia production and suppression in Pb+Pb collisions}} 
\author{\large Vineet Kumar}
\author{\large Prashant Shukla}
\email{pshukla@barc.gov.in}
\affiliation{Nuclear Physics Division, Bhabha Atomic Research Center, Mumbai, India}
\affiliation{Homi Bhabha National Institute, Anushakti Nagar, Mumbai, India}
%\author{\large Ramona Vogt}
%\affiliation{Nuclear and Chemical Sciences Division, Lawrence Livermore National Laboratory, Livermore, CA 94551, USA}
%\affiliation{Physics Department, University of California, Davis, CA 95616, USA}

\date{\today}

\begin{abstract}
  We calculate the high p$_{T}$ quarkonia production using NRQCD method.
  Different methods of quarkonia suppression are used to explain the high
  p$_{T}$ quarkonia suppression obseved by CMS in LHC
\end{abstract}

\pacs{12.38.Mh, 24.85.+p, 25.75.-q}
\keywords{quark-gluon plasma, quarkonia, suppression, regeneration}

\maketitle

%%%%%%%%%%%%%%%%%%%%%%%%%%%%%%%%%%%%%%%%%%%%%%%%%%%%%%%%%%%%%%%%%%%%%%%%%%%%%%%%%%%%%%%%%%%%%%%%%%%%%%%%%%%%%%%%%
\section{Introduction}
Heavy-ion collisions at relativistic energies are performed to create and characterize 
quark gluon plasma (QGP), a phase of strongly-interacting matter at high energy density 
where quarks and gluons are no longer bound within hadrons.
The quarkonia states ($\Jpsi$ and $\Upsilon$) have been some of the most popular tools 
since their suppression was proposed as a signal of QGP formation \cite{Matsui:1986dk}.
The understanding of these probes has evolved substantially via measurements 
through three generations of experiments: the SPS (at CERN), RHIC (at BNL) and the LHC (at CERN) 
and by a great deal of theoretical activity. (For recent reviews see 
Refs.~\cite{Schukraft:2013wba,Kluberg:2009wc,Brambilla:2010cs}.)
Quarkonia are produced early in the heavy-ion collisions and, if they evolve
through the deconfined medium, their yields should be suppressed in comparison with those in $pp$ collisions. 
The first such measurement was the `anomalous' $\Jpsi$ suppression discovered at the SPS 
which was considered to be a hint of QGP formation. The RHIC measurements showed almost the 
same suppression at a much higher energy contrary to expectation \cite{Brambilla:2010cs,Adare:2011yf}. 
Such an observation was consistent with the scenario that, at higher collision energies, the 
expected greater suppression is compensated by  $\Jpsi$ regeneration through recombination of two 
independently-produced charm quarks~\cite{Andronic:2003zv}. 

In this paper, we calculate $\Jpsi$ and $\Upsilon$ production and
suppression 


%%%%%%%%%%%%%%%%%%%%%%%%%%%%%%%%%%%%%%%%%%%%%%%%%%%%%%%%%%%%%%%%%%%%%%%%%%%%%%%%%%%%%%%
\section{Summary}
 
 
 \section{Acknowledgement}
  The authors thank their CMS colleagues for the fruitful discussions, 
help and comments. Many of these results were presented at WHEPP and we acknowledge discussions 
with the participants of the meeting, in particular with D. Das, S. Datta, R. Gavai, S. Gupta
and R. Sharma. The work of RV was performed under the auspices of the US Department of Energy, 
Lawrence Livermore National Laboratory, Contract DE-AC52-07NA27344.


\noindent
\begin{thebibliography}{100}
\medskip

\bibitem{Matsui:1986dk} 
 T.~Matsui and H.~Satz,
 ``$J/\psi$ Suppression by Quark-Gluon Plasma Formation'',
 Phys.\ Lett.\ B {\bf 178}, 416 (1986).

\bibitem{Schukraft:2013wba} 
  J.~Schukraft,
  ``Heavy Ion Physics at the LHC: What's new ? What's next ?'',
  arXiv:1311.1429 [hep-ex].


\bibitem{Kluberg:2009wc} 
  L.~Kluberg and H.~Satz,
  ``Color Deconfinement and Charmonium Production in Nuclear Collisions,''
  arXiv:0901.3831 [hep-ph].

\bibitem{Brambilla:2010cs} 
  N.~Brambilla, S.~Eidelman, B.~K.~Heltsley, R.~Vogt, G.~T.~Bodwin, E.~Eichten, A.~D.~Frawley and A.~B.~Meyer {\it et al.},
  ``Heavy quarkonium: progress, puzzles, and opportunities,''
  Eur.\ Phys.\ J.\ C {\bf 71}, 1534 (2011).
  %[arXiv:1010.5827 [hep-ph]].

\bibitem{Adare:2011yf} 
  A.~Adare {\it et al.}  [PHENIX Collaboration],
  ``$J/\psi$ suppression at forward rapidity in Au+Au collisions at $\sqrt{s_{NN}}=200$ GeV,''
  Phys.\ Rev.\ C {\bf 84}, 054912 (2011).
  %[arXiv:1103.6269 [nucl-ex]].

\bibitem{Andronic:2003zv} 
  A.~Andronic, P.~Braun-Munzinger, K.~Redlich and J.~Stachel,
  ``Statistical hadronization of charm in heavy ion collisions at SPS, RHIC and LHC,''
  Phys.\ Lett.\ B {\bf 571}, 36 (2003).
 % [nucl-th/0303036].


  

\end{thebibliography}
\end{document}



