\section{The ATLAS detector}
\label{sec:atlas}

The ATLAS experiment~\cite{ATLASdetector} is a general-purpose detector consisting of an inner tracker, a calorimeter and a muon spectrometer. 
The inner detector (ID) directly surrounds the interaction point; 
it consists of a silicon pixel detector, a semiconductor tracker and a transition radiation tracker, and is embedded in an axial~$2$ T magnetic field.
The ID covers the pseudorapidity\footnote{ATLAS uses a right-handed coordinate system with its origin at the nominal interaction point (IP) 
in the centre of the detector and the $z$-axis along the beam pipe.
The $x$-axis points from the IP to the centre of the LHC ring, and the $y$-axis points upward. Cylindrical 
coordinates $(r,\phi)$ are used in the transverse plane,
$\phi$ being the azimuthal angle around the beam pipe. The pseudorapidity $\eta$ is defined in terms of the 
polar angle $\theta$ as $\eta=-\ln\tan(\theta/2)$ and the
transverse momentum $p_{\rm T}$ is defined as $p_{\rm T}=p\sin\theta$. The rapidity is defined as 
$y=0.5\ln\left[\left( E + p_z \right)/ \left( E - p_z \right)\right]$,
where $E$ and $p_z$ refer to energy and longitudinal momentum, respectively. The $\eta$--$\phi$ distance 
between two particles is defined as
 $\Delta R=\sqrt{(\Delta\eta)^2 + (\Delta\phi)^2}$.}
range $|\eta| = $ 2.5 and is enclosed by a calorimeter system containing
electromagnetic and hadronic sections.
The calorimeter is surrounded by a large muon spectrometer (MS) in a toroidal magnet system.
The MS consists of monitored drift tubes and cathode strip chambers, designed to provide
precise position measurements in the bending plane in the range $|\eta| <$ 2.7.
Momentum measurements in the muon spectrometer are based on track segments formed in at least two of the three precision chamber planes.

The ATLAS trigger system \cite{ATLAS:trig} is separated into three levels: the hardware-based Level-1 trigger
and the two-stage High Level Trigger (HLT), comprising the Level-2 trigger and Event
Filter, which reduce the 20~MHz proton--proton collision rate to several-hundred Hz of events of interest for data recording to mass storage. 
At Level-1, the muon trigger searches for patterns of hits satisfying different transverse momentum thresholds with a coarse 
position resolution but a fast response time using resistive-plate chambers and thin-gap chambers in the ranges $|\eta| <$ 1.05 and $1.05 <|\eta| < 2.4$, respectively.
Around these Level-1 hit patterns ``Regions-of-Interest'' (RoI) are defined that
serve as seeds for the HLT muon reconstruction. 
The HLT uses dedicated algorithms to incorporate information from both the 
MS and the ID, achieving position and momentum resolution close to that provided by the offline muon reconstruction.


