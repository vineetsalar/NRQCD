\section{Conclusion}
\label{sec:conclusion}

The prompt and non-prompt production cross-sections, the non-prompt production fraction of the $\jpsi$ and $\psiprime$
decaying into two muons, the ratio of prompt $\psiprime$ to prompt $\jpsi$ production, and the ratio of non-prompt $\psiprime$ to non-prompt $\jpsi$ production
were measured in the rapidity range
$|y|<2.0$ for transverse momenta between $8$ and $110$~\GeV. This measurement was carried out using \lumiA (\lumiB) of $pp$
collision data at a centre-of-mass energy of $7$ \TeV\ ($8$ \TeV) recorded by the ATLAS experiment at the LHC. It is the latest in a series of 
related measurements of the production of charmonium states
made by ATLAS.
In line with previous measurements, the central values were obtained 
assuming isotropic $\psi \to \mu\mu$ decays.
Correction factors for these cross-sections, computed for a number of extreme spin-alignment scenarios, are between $-35\%$ and $+100\%$ at the lowest transverse momenta studied, and between $-14\%$ and $+9\%$ at the highest transverse momenta, depending on the specific scenario.

The ATLAS measurements presented here extend the range of existing measurements to higher transverse momenta, and to a higher collision energy of $\sqrt{s} = 8$ \TeV, and are consistent with previous measurements in overlapping phase-space regions made by ATLAS 
and other LHC experiments.
For the prompt production mechanism, the predictions from the NRQCD model, which includes colour-octet contributions with various matrix elements 
tuned to earlier collider data, are found to be in good agreement with the observed data points. 
For the non-prompt production, the fixed-order next-to-leading-logarithm calculations reproduce the data reasonably well, with a slight overestimation of the differential cross-sections at the highest 
transverse momenta reached in this analysis.
