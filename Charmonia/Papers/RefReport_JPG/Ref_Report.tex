\documentclass[aps,prc,preprint,superscriptaddress,showpacs,showkeys,amsmath]{revtex4-1}
\newcommand{\barc}{{\bar{c}}}
\usepackage{graphicx}
\usepackage[usenames,dvipsnames,svgnames,table]{xcolor}
\usepackage{rotating}
\usepackage{graphicx}% Include  files
\usepackage{dcolumn}% Align table columns on decimal point
\usepackage{bm}% bold math
\usepackage{epsfig}
\usepackage{hyperref}
\usepackage{ulem}
\usepackage{appendix}

\begin{document}
%%%%%\newcommand{\1S08And3P08}{$\frac{[^1S_{0}]_{8}}{3}+\frac{[^3P_0]_{8}}{m_{c}^2}$}

{\bf Report of Referee 1 } \\


In this paper the authors present a refit of quarkonium data in pp collisions
based on the NRQCD formalism coupled with pQCD production amplitudes. 
The new aspect is essentially the inclusion of the new LHC data
which, in many cases, leads to considerably different values of the NRQCD
matrix elemnets compared to existing literature.
By and large this is a reasonable work that can be published in JPG. However,
several points have to be addressed prior to that.\\
{\color{blue}
Ans: We thank the referee for his/her comments. We address all the points and modify the 
manuscript as described below:\\}

\begin{enumerate}

\item In the intorduction, a more broad representation of the literature should
be given, e.g. for the use of quarkonia in heavy-ion collisions. I find it 
a bit overstated to claim that measuring their ratios for excited over ground 
states "has become the most important goals of Pb-Pb collisions at LHC".

{\color{blue}
    
  The discussions of heavy ion collisions are modified and references are updated as follows:\\
    
  The quarkonia yields are modified in the heavy ion collision due to QGP
  and cold nuclear matter effects which has been demonstrated for J/$\psi$ and $\Upsilon$ 
  in PbPb collisions~\cite{Strickland:2011mw,Kumar:2014kfa,Zhou:2016wbo}. The ratios of excited to ground state quarkonia 
  yields are considered as better probes of QGP since the cold matter effects, 
  which are similar for the ground and excited states, are expected to cancel in the 
  ratio. At the LHC, the production of charmonium (J/$\psi$, $\psi$(2S)) 
  and bottomonium ($\Upsilon$(1S),$\Upsilon$(2S),$\Upsilon$(3S) ) states has been studied in 
  PbPb collisions at $\sqrt{s_{NN}} = 2.76$~TeV and $\sqrt{s_{NN}} = 5.02$~TeV
  ~\cite{Chatrchyan:2012lxa,Khachatryan:2014bva,Khachatryan:2016ypw,Sirunyan:2016znt,Khachatryan:2016xxp,Abelev:2013ila}
  showing the importance of quarkonia measurements in heavy ion collisions.
}

\item I am not sure the authors deliver on the promise made in the introduction
of quantifying NLO corrections. The paragraph on top of page 13 is a bit thin 
in this regard, and seems to be contradictory (factor 3 vs. 2-3 orders of magnitude 
enhancement of the psi' cross sectoin).

{\color{blue}
  The NLO corrections increase the total color-singlet J/$\psi$ cross section by a 
  factor of two, although at high $p_T$ the corrections can enhance the production by
  2-3 orders of magnitude~\cite{Gong:2008sn}.
  The NLO corrections to J/$\psi$ production via S-wave 
  color octet (CO) states ($^1S_{0}^{[8]}\,^3S_{1}^{[8]}$) are studied in 
  Ref.~\cite{Gong:2008ft} and the corrections to $p_{T}$ distributions of both 
  J/$\psi$ yield and polarization are found to be small.
  We have given an estimate of uncertainty in the LDMEs due to enhancement of 
  color-singlet J/$\psi$ cross-section by a factor of three expected from NLO 
  corrections. 
 
  First paragraph of page 13 is modified as follows:
  
  Here the first error is due to fitting and the second error is 
  obtained by enhancing the CS cross section 3 times. 
  It is due to the fact that NLO corrections enhance the total color-singlet J/$\psi$ 
  production by a factor of 2~\cite{Gong:2008sn}.
  The NLO corrections to J/$\psi$ production via S-wave color octet (CO) states 
  ($^1S_{0}^{[8]}\,^3S_{1}^{[8]}$) are found to be small Ref.~\cite{Gong:2008ft}.
}



\item I believe that appendix A referred to in the text is missing altogether. It
should be supplied.

{\color{blue} Appendix A is included in the new version of paper. }


\item I am wondering why the authors cut off their fits at pt=5GeV. How is that
choice motivated?

{\color{blue} 
  It is known that the perturbative expansion calculations are not applicable to the regions 
with small transverse  momentum. It is shown by previous NRQCD calculations~\cite{Sharma:2012dy} that the 
formalism for production works well only for large pT. If we use a lower pT cut off 
it would give a significantly larger $\chi^2$/dof. 
}


\item The chi$^2$/dof fit values of typical 2.5-3 are not very good. The authors
should comment on that. For example, does that imply the breakdown of the 
NRQCD formalism, or the production cross sections, or else?

{\color{blue} 
  The fitting method consist of simultaneous fitting of several (seven-eight) datasets
  covering different collision energies and different kinematic ranges. Our Chi$^{2}$
  are comparable to other~\cite{Sharma:2012dy,Butenschoen:2012qr} such analysis. 
}

\item The summary is a bit too short. In addition to emphasizing the main
discrepances of this work with existing literture, the authors should 
provide a more general perspetive of the implications of their
results.

{\color{blue} 
  We have extended the summary in the new manuscript.
  
  %We have presented NRQCD calculations for the differential production 
  %cross sections of prompt J/$\psi$ and prompt $\psi$(2S) in  p+p collisions.
  %For the J/$\psi$ meson, all the relevant contributions from higher mass states 
  %are estimated. 
  %Measured transverse momentum distributions of $\psi$(2S), $\chi_{\rm c}$ and J/$\psi$ 
  %in p +{$\bar {\rm p}$} collisions at $\sqrt{s}=$ 1.8, 1.96 TeV and in p+p collisions at 
  %7, 8 and 13 TeV are used to constrain LDMEs. 
  %The calculations for  prompt J/$\psi$ and prompt $\psi$(2S) are compared with the measured 
  %data at Tevatron and LHC. 
  %The formalism provides  very good description of the data in wide energy range. 
  %The values of LDMEs are used to predict the charmonia cross sections in p+p collisions 
  %at 13 and 5 TeV in kinematic bins relevant for LHC detectors. 
  %We compare the LDMEs for charmonia obtained in this analysis with the results from earlier works.
  %At high $p_T$, the color singlet contribution is very small and thus the LHC data in large $p_T$ range 
  %helps to constrain the relative contributions of different colour octet contributions.
  %The high energy LHC data require a smaller value of the LDME $M_{L}(c\barc([^3S_1]_{8}\rightarrow \psi))$ 
  %and a larger value for the combination  $M_{L}(c\barc([^1S_0]_{8},[^3P_0]_{8})\rightarrow \psi)$ of LDMEs.   
  %In summary, we present a comprehensive lowest-order analysis of hadroproduction data, 
  %including very recent LHC data. The values of fitted LDMEs will be useful for 
  %predictions of quarkonia cross section and for the purpose of a comparison with those 
  %obtained using NLO formulations.

}



\item I could not find some of the collision systems listed in Section 3 in the
figures, for example 1.8TeV for chi$_c$ and 7 TeV LHCb result fot psi'
Once the aboe points are satisfactorily addressed in the ms. it can be
published.

{\color{blue} 
We have included these calculations in the new version of manuscript .
}



\end{enumerate}









{\bf Report of Referee 2 } \\

The manuscript describes a leading-order fit of chi$_{c0}$, psi(2S) and J/psi NRQCD color-octet long distance matrix elements (CO LDMEs) to 
Tevatron and LHC hadroproduction data. The work is certainly done thoroughly, and, besides some minor misunderstandings, the article is 
clearly written. The article does however not fulfill JPhysG's criterion of having "new results that substantially advance their 
relevant field": Leading order fits of this kind have already been published 20 years ago. The only relevant difference to these 
early works seems to be the inclusion of the chi$_{c2}$/chi$_{c1}$ ratio in the chi$_{c0}$ CO LDME fit, which have only become available with 
the LHC. Current CO LDME fits are done at next-to-leading and include a much wider range of observables, including photoproduction data, 
eta$_c$ production rates and polarization observables on top of the observables used in the manuscript assessed here.

{\color{blue}

  A LO NRQCD analysis is useful as it is straightforward and unique and once
  the parameters are obtained by fitting over large datasets it has excellent 
  predictability power for unknown cross sections.

}



{\bf Report of Referee 3 (Adjudication report)} \\

Dear Editors,

Although this work remains below the present technological state of the art, which is next-to-leading order as pointed out 
correctly by Referee B, it should still be of some interest. In fact, it present a comprehensive lowest-order analysis of 
hadroproduction data, including very recent LHC data. The resulting long-distance matrix elements may be useful as input 
for LO analysis by other authors.

However, the authors are missing some relevant references that underline the current charmonium crisis of NRQCD 
factorization which has recently emerged in various studies:

\begin{enumerate}

\item The problem of performing a joint analysis of the e$^{+}$e$^{-}$, photoproduction, hadroproduction and polarization 
  data:: arXiv:1212.2037 [hep-ph].

\item The problem of describing the eta$_c$ hadroproduction data on the basis of heavy-quark spin symmetyry:arXiv:1411.5287 [hep-ph]

\item The problem of describing double J/psi hadropoduction at large invariant masses and rapidity 
  separations of the J/psi pair: arXiv:1609.02786 [hep-ph]
  Furthermore, they should add the following paper to Ref. 31 because it present the global fit: arXiv:1105.0820 [hep-ph].
  I recommend that the discussion of the NRQCD crisis be substantiated and that these reference be properly cited. 



{\color{blue}

  We have included following discussion regarding above points in the manuscript:\\ 
  
  Authors in Ref.~\cite{Butenschoen:2012qr} extracted leading color-octet LDMEs
  through a global fit to experimental data of unpolarized J/$\psi$ production in pp, p$\overline{\rm p}$, ep, $\gamma \gamma$, 
  and e$^{+}$e$^{-}$ collisions. The extracted LDMEs give excellent discription of the 
  unpolarized J/$\psi$ yields but fail to reproduce the  polarization measured at CDF~\cite{Abulencia:2007us}.
  In another study~\cite{Chao:2012iv}, it is shown that the measured hadroproduction cross sections 
  and the CDF polarization measurement~\cite{Abulencia:2007us} can be simultaneously described by NRQCD at NLO.
  
   Recently, the LHCb measurements of $\eta_{C}$ production~\cite{Aaij:2014bga} 
   is investigated from different points of views by several groups using  
   NRQCD formalism~\cite{Butenschoen:2014dra,Han:2014jya,Zhang:2014ybe}.   
   Ref.~\cite{Butenschoen:2014dra} considered the $\eta_{ c}$ measurement as a challenge of NRQCD 
   while Ref.~\cite{Han:2014jya} shows that the LHCb measurement results in a very 
   strong constraint on the upper bound of the color-octet LDME of J/$\psi$.
   Refs.~\cite{Zhang:2014ybe} obtain the color-singlet LDME for 
   $\eta_c$ by fitting the experiment data to obtain good discription of $\eta_c$ production.
   

   The prompt double heavy quarkonium production is a sensitive testing 
   ground for NRQCD factorization. The experiments at LHC recently published 
   the measurement of double J/$\psi$ production in proton-proton collision at $ \sqrt{s} $ 
   = 7, 8 and 13 TeV~\cite{Aaij:2011yc,Khachatryan:2014iia,Aaboud:2016fzt,Aaij:2016bqq}. 
   Full NLO calculations including all color singlet and color octet contributions 
   for this process in the NRQCD framework are not fully established yet. 
   Authors in Ref~\cite{He:2015qya} showed that the LO calculations of the prompt double J/$\psi$ 
   production by NRQCD formalism describes the data only qualitatively.
   Authors in Ref~\cite{Sun:2014gca} present the NLO calculations for the color-singlet channel
   which describe the measured LHCb cross section reasonably well, but fail to reproduce the CMS 
   measurements. The complicated situation suggests that, further study and phenomenological test of NRQCD is still an 
   urgent task.



}





\end{enumerate}
 






%\\

\noindent
\begin{thebibliography}{100}
\medskip


\bibitem{Strickland:2011mw} 
  M.~Strickland,
  ``Thermal $\Upsilon$(1S) and $\chi_{b1}$ suppression in $\sqrt{s_{NN}}=2.76$ TeV Pb-Pb collisions at the LHC,''
  Phys.\ Rev.\ Lett.\  {\bf 107}, 132301 (2011), [arXiv:1106.2571 [hep-ph]].

\bibitem{Kumar:2014kfa} 
  V.~Kumar, P.~Shukla and R.~Vogt,
  ``Quarkonia suppression in PbPb collisions at $\sqrt{s_{NN}}$ = 2.76 TeV,''
  Phys.\ Rev.\ C {\bf 92}, 024908 (2015).

\bibitem{Zhou:2016wbo} 
  K.~Zhou, Z.~Chen, C.~Greiner and P.~Zhuang,
  ``Thermal Charm and Charmonium Production in Quark Gluon Plasma,''
  Phys.\ Lett.\ B {\bf 758}, 434 (2016), [arXiv:1602.01667 [hep-ph]].


\bibitem{Chatrchyan:2012lxa} 
  S.~Chatrchyan {\it et al.} [CMS Collaboration],
  ``Observation of sequential Upsilon suppression in PbPb collisions,''
  Phys.\ Rev.\ Lett.\  {\bf 109}, 222301 (2012).


%\cite{Khachatryan:2014bva}
\bibitem{Khachatryan:2014bva} 
  V.~Khachatryan {\it et al.} [CMS Collaboration],
  ``Measurement of Prompt $\psi(2S) \to J/\psi$ Yield Ratios in Pb-Pb and $p-p$ Collisions at $\sqrt {s_{NN}}=$ 2.76  TeV,''
  Phys.\ Rev.\ Lett.\  {\bf 113}, no. 26, 262301 (2014).


%\cite{Khachatryan:2016ypw}
\bibitem{Khachatryan:2016ypw} 
  V.~Khachatryan {\it et al.} [CMS Collaboration],
  ``Suppression and azimuthal anisotropy of prompt and nonprompt $J/\psi$ production in PbPb collisions at $\sqrt{s_{NN}}$ = 2.76 TeV,''
  Eur.\ Phys.\ J.\ C, [arXiv:1610.00613 [nucl-ex]].



%\cite{Sirunyan:2016znt}
\bibitem{Sirunyan:2016znt} 
  A.~M.~Sirunyan {\it et al.} [CMS Collaboration],
  ``Relative modification of prompt $ {\psi\mathrm{(2S)}} $ and $\mathrm{J}/\psi $ yields from pp to PbPb collisions at ${\sqrt{s_{\mathrm{NN}}}} = $ 5.02 TeV,''
  [Phys.\ Rev.\ Lett.\  {\bf 118}, 162301 (2017)], [arXiv:1611.01438 [nucl-ex]].


\bibitem{Khachatryan:2016xxp} 
  V.~Khachatryan {\it et al.} [CMS Collaboration],
  ``Suppression of $\Upsilon(1S)$, $\Upsilon(2S)$ and $\Upsilon(3S)$ production in PbPb collisions at $\sqrt{s_{NN}}$ = 2.76 TeV,''
  Phys.Lett.B, [arXiv:1611.01510 [nucl-ex]].

\bibitem{Abelev:2013ila} 
  B.~B.~Abelev {\it et al.} [ALICE Collaboration],
  ``Centrality, rapidity and transverse momentum dependence of $J/\psi$ suppression in Pb-Pb collisions at $\sqrt{s_{\rm NN}}$=2.76 TeV,''
  Phys.\ Lett.\ B {\bf 734}, 314 (2014), [arXiv:1311.0214 [nucl-ex]].


\bibitem{Campbell:2007ws}
  J.~M.~Campbell, F.~Maltoni, and F.~Tramontano,
  ``QCD corrections to J/$\psi$ and $\Upsilon$ production at hadron colliders,''
  Phys.\ Rev.\ Lett.\  {\bf 98} 252002 (2007).


\bibitem{Gong:2008sn} 
  B.~Gong and J.~X.~Wang,
  ``Next-to-leading-order QCD corrections to $J/\psi$ polarization at Tevatron and Large-Hadron-Collider energies,''
  Phys.\ Rev.\ Lett.\  {\bf 100}, 232001 (2008).
%  [arXiv:0802.3727 [hep-ph]].
  

\bibitem{Gong:2008ft} 
  B.~Gong, X.~Q.~Li and J.~X.~Wang,
  ``QCD corrections to J/$\psi$ production via color octet states at Tevatron and LHC,''
  Phys.\ Lett.\ B {\bf 673}, 197 (2009),
  [Phys.\ Lett.\  {\bf 693}, 612 (2010)].


\bibitem{Sharma:2012dy} 
  R.~Sharma and I.~Vitev,
  ``High transverse momentum quarkonium production and dissociation in heavy ion collisions,''
  Phys.\ Rev.\ C {\bf 87}, no. 4, 044905 (2013), [arXiv:1203.0329 [hep-ph]].



\bibitem{Butenschoen:2012qr} 
  M.~Butenschoen and B.~A.~Kniehl,
  ``Next-to-leading-order tests of NRQCD factorization with $J/\psi$ yield and polarization,''
  Mod.\ Phys.\ Lett.\ A {\bf 28}, 1350027 (2013), [arXiv:1212.2037 [hep-ph]].

\bibitem{Abulencia:2007us} 
  A.~Abulencia {\it et al.} [CDF Collaboration],
  ``Polarization of $J/\psi$ and $\psi_{2S}$ mesons produced in $p \bar{p}$ collisions at $\sqrt{s}$ = 1.96-TeV,''
  Phys.\ Rev.\ Lett.\  {\bf 99}, 132001 (2007), [arXiv:0704.0638 [hep-ex]].

\bibitem{Chao:2012iv} 
  K.~T.~Chao, Y.~Q.~Ma, H.~S.~Shao, K.~Wang and Y.~J.~Zhang,
  ``$J/\psi$ Polarization at Hadron Colliders in Nonrelativistic QCD,''
  Phys.\ Rev.\ Lett.\  {\bf 108}, 242004 (2012), [arXiv:1201.2675 [hep-ph]].
  

\bibitem{Aaij:2014bga} 
  R.~Aaij {\it et al.} [LHCb Collaboration],
  ``Measurement of the $\eta_c (1S)$ production cross-section in proton-proton collisions via the decay $\eta_c (1S) \rightarrow p \bar{p}$,''
  Eur.\ Phys.\ J.\ C {\bf 75}, no. 7, 311 (2015), [arXiv:1409.3612 [hep-ex]].
 

\bibitem{Butenschoen:2014dra} 
  M.~Butenschoen, Z.~G.~He and B.~A.~Kniehl,
  ``$\eta_c$ production at the LHC challenges nonrelativistic-QCD factorization,''
  Phys.\ Rev.\ Lett.\  {\bf 114}, no. 9, 092004 (2015), [arXiv:1411.5287 [hep-ph]].


\bibitem{Han:2014jya} 
  H.~Han, Y.~Q.~Ma, C.~Meng, H.~S.~Shao and K.~T.~Chao,
  ``$\eta_c$ production at LHC and indications on the understanding of $J/\psi$ production,''
  Phys.\ Rev.\ Lett.\  {\bf 114}, no. 9, 092005 (2015), [arXiv:1411.7350 [hep-ph]].
  

\bibitem{Zhang:2014ybe} 
  H.~F.~Zhang, Z.~Sun, W.~L.~Sang and R.~Li,
  ``Impact of $\eta_c$ hadroproduction data on charmonium production and polarization within NRQCD framework,''
  Phys.\ Rev.\ Lett.\  {\bf 114}, 092006 (2015), [arXiv:1412.0508 [hep-ph]].





\bibitem{Aaij:2011yc} 
  R.~Aaij {\it et al.} [LHCb Collaboration],
  ``Observation of $J/\psi$ pair production in $pp$ collisions at $\sqrt{s}=7 TeV$,''
  Phys.\ Lett.\ B {\bf 707}, 52 (2012), [arXiv:1109.0963 [hep-ex]].


\bibitem{Khachatryan:2014iia} 
  V.~Khachatryan {\it et al.} [CMS Collaboration],
  ``Measurement of prompt $J/\psi$ pair production in pp collisions at $ \sqrt{s} $ = 7 Tev,''
  JHEP {\bf 1409}, 094 (2014), [arXiv:1406.0484 [hep-ex]].
 

\bibitem{Aaboud:2016fzt} 
  M.~Aaboud {\it et al.} [ATLAS Collaboration],
  ``Measurement of the prompt J/ $\psi $ pair production cross-section in pp collisions at $\sqrt{s} = 8$  TeV with the ATLAS detector,''
  Eur.\ Phys.\ J.\ C {\bf 77}, no. 2, 76 (2017), [arXiv:1612.02950 [hep-ex]].


\bibitem{Aaij:2016bqq} 
  R.~Aaij {\it et al.} [LHCb Collaboration],
  ``Measurement of the $J/\psi$ pair production cross-section in $pp$ collisions at $\sqrt{s} = 13 \,{\mathrm{TeV}}$,''
  arXiv:1612.07451 [hep-ex].
  

\bibitem{He:2015qya} 
  Z.~G.~He and B.~A.~Kniehl,
  ``Complete Nonrelativistic-QCD Prediction for Prompt Double J/ψ Hadroproduction,''
  Phys.\ Rev.\ Lett.\  {\bf 115}, 022002 (2015), [arXiv:1609.02786 [hep-ph]].


\bibitem{Sun:2014gca} 
  L.~P.~Sun, H.~Han and K.~T.~Chao,
  ``Impact of $J/\psi$ pair production at the LHC and predictions in nonrelativistic QCD,''
  Phys.\ Rev.\ D {\bf 94}, 074033 (2016), [arXiv:1404.4042 [hep-ph]].





\end{thebibliography}

\end{document}










