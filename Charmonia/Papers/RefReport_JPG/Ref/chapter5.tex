%9/12/04

%\documentclass[11pt,twoside]{cernrep} 
%\usepackage{graphicx,epsfig} 
%\usepackage{here,cite}
%\usepackage{bm}

% Counter commands
%\setcounter{page}{0}
%\setcounter{chapter}{0}
%\setcounter{secnumdepth}{3}
%\setcounter{tocdepth}{3}

% Page Headings
%pagestyle{plain}

%\input{newcommand.tex}

%\begin{document}
%%%%%%%%%%%%%%  CHAPTER V : 
%\setcounter{chapter}{4}
\chapter {PRODUCTION}
\label{chap:V}
{{\it Conveners:} G.T.~Bodwin, E.~Braaten, M.~Kr\"amer, A.B.~Meyer,
V.~Papadimitriou}\par\noindent 
{{\it Authors:} G.T.~Bodwin, E.~Braaten, C.-H.~Chang, M.~Kr\"amer,
J.~Lee, A.B.~Meyer, V.~Papadimitriou, R.~Vogt}
\par\noindent

%%%%%%%%%%%%%%%%%%%%%%%%%%%%%%%%%%%%%%%%%%%%%%%%%%%%%%%%%%%%%%
%\input{secnrqcdintro}
\section{Formalism for inclusive quarkonium production}
\label{prodsec:nrqcd}

\subsection{NRQCD factorization method}
\label{prodsec:nrqcdfact}

In both heavy-quarkonium annihilation decays and hard-scattering
production, large energy-momentum scales appear. The heavy-quark mass
$m$ is much larger than $\Lambda_{\rm QCD}$, and, in the case of
production, the transverse momentum $p_T$ can be much larger than
$\Lambda_{\rm QCD}$ as well. This implies that the associated values of
the QCD running coupling constant are much less than one.
($\alpha_s(m_c)\approx 0.25$ and $\alpha_s(m_b)\approx 0.18$.)
Therefore, one might hope that it would be possible to calculate the
rates for heavy quarkonium decay and production accurately in
perturbation theory. However, there are clearly low-momentum,
nonperturbative effects associated with the dynamics of the quarkonium
bound state that invalidate the direct application of perturbation
theory.

In order to make use of perturbative methods, one must first separate
the short-distance/high-momentum, perturbative effects from the
long-distance/low-momentum, nonperturbative effects---a process which is
known as ``factorization.'' One convenient way to carry out this
separation is through the use of the effective field theory
Nonrelativistic QCD (NRQCD) \cite{Caswell:1985ui,Thacker:1990bm,Bodwin:1994jh}.
NRQCD reproduces full QCD accurately at momentum scales of order $mv$
and smaller, where $v$ is the typical heavy-quark velocity in the bound
state in the CM frame. ($v^2\approx 0.3$ for charmonium, and 
$v^2\approx 0.1$ for bottomonium.) Virtual processes involving momentum
scales of order $m$ and larger can affect the lower-momentum processes,
and their effects are taken into account through the short-distance
coefficients of the operators that appear in the NRQCD action.

Because $Q\bar Q$ production occurs at momentum scales of order $m$ or
larger, it manifests itself in NRQCD through contact interactions. As a
result, the inclusive cross section for the direct production of the
quarkonium $H$ at large transverse momentum ($p_T$ of order $m$ or
larger) in hadron or $ep$ colliders or at large momentum in the CM frame
($p^*$ of order $m$ or larger) in $e^+e^-$ colliders can be written as a
sum of products of NRQCD matrix elements and short-distance coefficients:
%
\begin{equation}
\sigma[H]=\sum_n \sigma_n(\Lambda) \langle {\cal O}_n^H(\Lambda) \rangle.
\label{prod:fact}
\end{equation}
%
Here, $\Lambda$ is the ultraviolet cutoff of the effective theory, the
$\sigma_n$ are short-distance coefficients, and the 
$\langle {\cal O}_n^H \rangle$ 
are vacuum-expectation values of four-fermion
operators in NRQCD. There is a formula analogous to Eq.~(\ref{prod:fact})
for inclusive quarkonium annihilation rates, except that the
vacuum-to-vacuum matrix elements are replaced by
quarkonium-to-quarkonium matrix elements
\cite{Bodwin:1994jh}.

The short-distance coefficients $\sigma_n(\Lambda)$ in
(\ref{prod:fact}) are essentially the process-dependent partonic cross
sections to make a $Q\bar Q$ pair, convolved with parton distributions
if there are hadrons in the initial state. The $Q\bar Q$ pair can be
produced in a color-singlet state or in a color-octet state. Its spin
state can be singlet or triplet, and it also can have orbital angular
momentum. The short-distance coefficients are determined by matching
the square of the production amplitude in NRQCD to full QCD. Because
the scale of the $Q\bar Q$ production is of order $m$ or greater, this
matching can be carried out in perturbation theory.

The four-fermion operators in Eq.~(\ref{prod:fact}) create a $Q\bar Q
$ pair in the NRQCD vacuum, project it onto 
a state that in the asymptotic future
consists of a heavy quarkonium plus anything, and then annihilate the
$Q\bar Q$ pair. The vacuum matrix element of such an operator is the
probability for a $Q\bar Q$ pair to form a quarkonium plus anything.
These matrix elements are somewhat analogous to parton fragmentation
functions. They contain all of the nonperturbative physics associated
with evolution of the $Q\bar Q$ pair into a quarkonium state. An
important property of the matrix elements, which greatly increases the
predictive power of NRQCD, is the fact that they are universal, {\it
i.e.}, process independent.

The color-singlet and color-octet four-fermion operators 
that appear in Eq.~(\ref{prod:fact}) correspond to the evolution
into a color-singlet quarkonium of a $Q\bar Q$ pair 
created at short distance in a color-singlet state
or a color-octet state, respectively. In the case of decay, 
the color-octet matrix elements have the interpretation of the
probability to find the quarkonium in a Fock state consisting of a
$Q\bar Q$ pair plus some number of gluons. It is a common
misconception that color-octet production proceeds
through the production of a higher Fock state of the quarkonium.
However, in the leading color-octet production mechanisms,
the gluons that neutralize the color are not present 
at the time of the creation of the color-octet $Q \bar Q$ pair, 
but are emitted during the subsequent hadronization process. 
The production of the quarkonium through a higher Fock state
requires the production of gluons that are nearly
collinear to the $Q\bar Q$ pair, and it is suppressed by additional
powers of $v$.

NRQCD power-counting rules allow one to organize the sum over
operators in Eq.~(\ref{prod:fact}) as an expansion in powers of
$v$. Through a given order in $v$, only a finite set of matrix
elements contributes. Furthermore, there are simplifying relations
between matrix elements, such as the heavy-quark spin symmetry and the
vacuum-saturation approximation, that reduce the number of independent
matrix elements \cite{Bodwin:1994jh}. 
Some examples of relations between color-singlet matrix elements
that follow from heavy-quark spin symmetry are 
%
\begin{eqnarray}
\langle{\cal O}^{J/\psi}_1({}^3S_1)\rangle &=&
3 \;\langle{\cal O}^{\eta_c}_1({}^1S_0)\rangle,
\label{s1-symmetry}
\\ 
\langle{\cal O}^{\chi_{cJ}}_1({}^3P_J)\rangle &=& 
\mbox{$1\over 3$} (2J+1) \langle{\cal O}^{h_c}_1({}^1P_1)\rangle. 
\label{p1-symmetry}    
\end{eqnarray}                                                  
%
These relations hold up to corrections of order $v^2$. 
The prefactors on the right side of 
Eqs.~(\ref{s1-symmetry})-(\ref{p1-symmetry})
are just ratios of the numbers of spin states.
Since the operators in Eqs.~(\ref{s1-symmetry}) and 
(\ref{p1-symmetry}) have the same angular momentum 
quantum numbers as the $Q \bar Q$ pair in the dominant 
Fock state of the quarkonium, 
the vacuum-saturation approximation can be used to express the
matrix elements in terms of the squares of wave functions 
or their derivatives at the origin, 
up to corrections of order $v^4$. 
Heavy-quark spin symmetry also gives relations between 
color-octet matrix elements, such as
%
\begin{eqnarray}
\langle{\cal O}^{J/\psi}_8({}^3S_1)\rangle &=&
3\langle{\cal O}^{\eta_c}_8({}^1S_0)\rangle,
\label{s8-symmetry}
\\
\langle{\cal O}^{J/\psi}_8({}^1S_0)\rangle &=&
3 \; \langle{\cal O}^{\eta_c}_8({}^3S_1)\rangle, 
\\
\langle{\cal O}^{J/\psi}_8({}^3P_J)\rangle &=&
3 \; \langle{\cal O}^{\eta_c}_8({}^1P_1)\rangle,
\\
\langle{\cal O}^{\chi_{cJ}}_8({}^3S_1)\rangle &=& 
\mbox{$1\over 3$} (2J+1) \langle{\cal O}^{h_c}_8({}^1S_0)\rangle.
\label{p8-symmetry}
\end{eqnarray}                                                  
%
These relations hold up to corrections of order $v^2$. The prefactors
on the right side of Eqs.~(\ref{s8-symmetry})-(\ref{p8-symmetry}) are
again just ratios of the numbers of spin states.  The
vacuum-saturation approximation is not applicable to color-octet
matrix elements.

The relative importance of the terms in the factorization formula in
Eq.~(\ref{prod:fact}) is determined not only by the sizes of the
matrix elements but also by the sizes of the coefficients $\sigma_n$
in Eq.~(\ref{prod:fact}).  The size of the coefficient depends on its
order in $\alpha_s$, color factors, and dimensionless kinematic
factors, such as $m^2/p_T^2$.

The NRQCD factorization approach is sometimes erroneously called the
``color-octet model,'' because color-octet terms are expected to
dominate in some situations, such as $J/\psi$ production at large
$p_T$ in hadron colliders.  However, there are also situations in
which color-singlet terms are expected to dominate, such as $J/\psi$
production in continuum $e^+ e^-$ annihilation at the $B$ factories.
Moreover, NRQCD factorization is not a model, but a rigorous
consequence of QCD in the limit $\Lambda_{\rm QCD}/m\rightarrow 0$.

A specific truncation of the NRQCD expansion in Eq.~(\ref{prod:fact})
could be called a model, although, unlike most models, 
it is in principle systematically improvable. 
In truncating at a given order in $v$, one
can reduce the number of independent matrix elements by making use of
approximate relations, such as 
Eqs.~(\ref{s1-symmetry})--(\ref{p1-symmetry}) and
Eqs.~(\ref{s8-symmetry})--(\ref{p8-symmetry}).
The simplest truncation of the NRQCD
expansion in Eq.~(\ref{prod:fact}) that is both phenomenologically
viable and corresponds to a consistent truncation in $v$ includes four
independent NRQCD matrix elements for each S-wave multiplet 
(one color-singlet and three color-octet) 
and two independent NRQCD matrix elements for each P-wave multiplet 
(one color-singlet and one color-octet).  
We will refer to this truncation as the standard truncation in $v$.
For the S-wave charmonium multiplet
consisting of $J/\psi$ and $\eta_c$, one can take the 
four independent matrix elements to be
$\langle {\cal O}^{J/\psi}_1({}^3S_1) \rangle$,
$\langle {\cal O}^{J/\psi}_8({}^1S_0) \rangle$,
$\langle {\cal O}^{J/\psi}_8({}^3S_1) \rangle$, and
$\langle {\cal O}^{J/\psi}_8({}^3P_0) \rangle$.  
Their relative orders in $v$ are $v^0$, $v^3$, $v^4$, and $v^4$,
respectively. It is convenient to define the linear combination
%
\begin{equation}
M_k^H = 
\langle {\cal O}^H_8({}^1S_0) \rangle +
 {k \over m_c^2} \langle {\cal O}^H_8({}^3P_0) \rangle \, ,
\label{prod:lincomb}
\end{equation}
%
because many observables are sensitive only to the linear combination
of these two color-octet matrix elements corresponding to a specific
value of $k$.  
These four independent matrix elements can be used to calculate 
the cross sections for the $\eta_c$ and each of the 3 spin states 
of the $J/\psi$. Thus, this truncation of NRQCD gives
unambiguous predictions for the polarization of the $J/\psi$. 
For the P-wave charmonium multiplet consisting of
$\chi_{c0}$, $\chi_{c1}$, $\chi_{c2}$, and $h_c$, we can take the
two independent matrix elements to be 
$\langle {\cal O}^{\chi_{c0}}_1({}^3P_0) \rangle$ and 
$\langle {\cal O}^{\chi_{c0}}_8({}^3S_1) \rangle$. 
Their orders in $v$ relative to
$\langle {\cal O}^{J/\psi}_1({}^3S_1) \rangle$ are both $v^2$.  
These two independent matrix elements can be used to calculate 
the cross sections for each of the 12 spin states in the 
P-wave multiplet.  Thus, this truncation of NRQCD gives
unambiguous predictions for the polarizations of 
the $\chi_{c1}$, $\chi_{c2}$, and $h_c$.

The NRQCD {\it decay matrix} elements can be calculated in lattice
simulations \cite{Bodwin:1993wf,Bodwin:1994js,Bodwin:1996tg,%
Bodwin:1996mf,Bodwin:2001mk} or determined from
phenomenology. However, it is not yet known how to formulate the
calculation of production matrix elements in lattice simulations, and,
so, the production matrix elements must be obtained phenomenologically.
In general, the production matrix elements are different from the decay
matrix elements. The exceptions are the color-singlet production matrix
elements in which the $Q \bar Q$ pair has the same quantum numbers as
the quarkonium state, such as those in Eqs.~(\ref{s1-symmetry}) and
(\ref{p1-symmetry}). They can be related to the corresponding decay
matrix elements through the vacuum-saturation approximation, up to
corrections of relative order $v^4$ \cite{Bodwin:1994jh}.
Phenomenological determinations of the 
production matrix elements for charmonium states
are given in Section~\ref{prodsec:tevatroncharm}.

The proof of the factorization formula in Eq.~(\ref{prod:fact}) relies both on
NRQCD and on the all-orders perturbative machinery for proving
hard-scattering factorization.  A detailed proof does not yet exist, but
work is in progress \cite{qiu-sterman}. At a large transverse momentum
($p_T$ of order $m$ or larger), corrections to hard-scattering 
factorization are thought to be of order $(mv)^2/p_T^2$ (not
$m^2/p_T^2$) in the unpolarized case and of order $mv/p_T$ (not $m/p_T$)
in the polarized case. At a small transverse momentum, $p_T$ of order
$mv$ or smaller, the presence of soft gluons in the quarkonium binding
process makes the application of the standard factorization techniques
problematic. It is not known if there is a factorization formula for
$d\sigma/dp_T^2$ at small $p_T$ or for $d\sigma/dp_T^2$ integrated over
$p_T$.

In practical calculations of
the rates of quarkonium decay and production, a number of significant
uncertainties arise. In many instances, the series in $\alpha_s$ and
$v$ in the factorization formula in Eq.~(\ref{prod:fact}) converge slowly,
and the uncertainties from their truncation are large---sometimes
$100\%$ or larger. In addition, the matrix elements are often poorly
determined, either from phenomenology or lattice measurements, and the
important linear combinations of matrix elements vary from process to
process, making tests of universality difficult. There are also large
uncertainties in the heavy-quark masses (approximately 8\% for $m_c$
and approximately 2.4\% for $m_b$) that can be very significant for
quarkonium rates that are proportional to a large power of the mass.

Many of the largest uncertainties in the theoretical predictions, as
well as some of the experimental uncertainties, cancel in the ratios of
cross sections. Examples in charmonium production are the ratio $R_\psi$
of the inclusive cross sections for $\psi(2S)$ and $J/\psi$
production and the ratio $R_{\chi_c}$ of the inclusive cross
sections for $\chi_{c1}$ and $\chi_{c2}$ production.  These ratios
are defined by
%
\begin{eqnarray}
R_\psi &=& {\sigma[\psi(2S)] \over \sigma[J/\psi]} \, ,
\label{prod:psirat}
\\
R_{\chi_c} &=& {\sigma[\chi_{c1}] \over \sigma[\chi_{c2}]} \, .
\label{prod:chirat}
\end{eqnarray}
%
Other useful ratios are the fractions $F_H$ of $J/\psi$'s that come from 
decays of higher quarkonium states $H$.
The fractions that come from decays of $\psi(2S)$ and from $\chi_c(1P)$ 
are defined by
%
\begin{eqnarray}
F_{\psi(2S)} &=& 
{\rm Br}[\psi(2S) \to J/\psi + X] \; {\sigma[\psi(2S)] \over \sigma[J/\psi]},
\label{FJpsipsi2S}
\\
F_{\chi_c} &=& \sum_{J=0}^2
{\rm Br}[\chi_{cJ}(1P) \to J/\psi + X] \;
        {\sigma[\chi_{cJ}(1P)] \over \sigma[J/\psi]}.
\label{prod:chifrac}
\end{eqnarray}
%
The $J=0$ term in (\ref{prod:chifrac}) is usually negligible, because the 
branching fraction $\rm{Br}[\chi_{c0}\to J/\psi\,X]$ is so small.
The fraction of $J/\psi$'s that are produced directly
can be denoted by $F_{J/\psi}$.


Another set of observables in which many of the uncertainties cancel out
consists of polarization variables, which can be defined as ratios
of cross sections for the production of different spin states of the same 
quarkonium. The angular distribution of
the decay products of the quarkonium depends on the spin state of the 
quarkonium. The polarization of a $1^{--}$
state, such as the $J/\psi$, can be measured from the
angular distribution of its decays into lepton pairs. Let $\theta$ be
the angle in the $J/\psi$ rest frame between the positive lepton
momentum and the chosen polarization axis.
The most convenient choice of polarization axis depends on the process.
The differential cross section has the form
%
\begin{equation}
{ d\sigma \over d(\cos \theta)} \propto 1+\alpha\cos^2\theta,
\label{prod:alphadef}
\end{equation}
%
which defines a polarization variable $\alpha$
whose range is $-1 \le \alpha \le +1$. We can define longitudinally 
and transversely polarized $J/\psi$'s to be ones whose spin components 
along the polarization axis are 0 and $\pm 1$, respectively. 
The polarization variable $\alpha$ can 
then be expressed as $(1-3\xi)/(1+\xi)$, where $\xi$ is the 
fraction of the $J/\psi$'s that are longitudinally polarized.
The value $\alpha=1$ corresponds to $J/\psi$ with 100\% transverse
polarization, while $\alpha=-1$ corresponds to $J/\psi$ with 100\%
longitudinal polarization. 

One short-coming of the NRQCD factorization approach is that, at
leading order in $v$, some of the kinematics of production are treated
inaccurately. Specifically, the mass of the light hadronic state that
forms during the evolution of the $Q\bar Q$ pair into the quarkonium
state is neglected, and no distinction is made between $2m$ and the
quarkonium mass. While the corrections to these approximations are
formally of higher order in $v$, they can be important numerically in
the cases of rapidly varying quarkonium-production distributions, such
as $p_T$ distributions at the Tevatron and $z$ distributions at the $B$
factories and HERA near the kinematic limit $z=1$. These effects can be
taken into account through the resummation of certain operator matrix
elements of higher order in $v$ \cite{Beneke:1997qw}. The resummation
results in universal nonperturbative shape functions that give the
probability distributions for a $Q\bar Q$ pair with a given set of
quantum numbers to evolve into a quarkonium with a given fraction of the 
pair's momentum. The shape functions could, in principle, be extracted 
from the data for one process and applied to another
process. Effects from resummation of logarithms of $1-z$ and model shape
functions have been calculated for the process $e^+e^-\to J/\psi+X$
\cite{Fleming:2003gt}. For shape functions that satisfy the
velocity-scaling rules, these effects are comparable in size. It may be
possible to use this resummed theoretical prediction to extract the
dominant shape function from the Belle and BaBar data for $e^+e^-\to
J/\psi+X$ and then use it to make predictions for $J/\psi$
photoproduction near $z=1$~\cite{Beneke:1999gq}. 

\subsection{Color-singlet model}
\label{prodsec:nrqcdCSM}

The color-singlet model (CSM) was first proposed shortly after the
discovery of the $J/\psi$. The initial applications were to $\eta_c$ and
$\chi_c$ production through two-gluon fusion
\cite{Einhorn:1975ua,Ellis:1976fj,Carlson:1976cd,Kuhn:1979kb}. Somewhat
later, the CSM was applied to the production of $J/\psi$ and $\eta_c$ in
$B$-meson decays \cite{DeGrand:wf,Kuhn:1979zb,Wise:1979tp} and to the
production of $J/\psi$ plus a gluon
\cite{Chang:1979nn,Baier:1981zz,Baier:1981uk,Baier:1983va,%
Berger:1980ni,Keung:1981gs} through two-gluon fusion and photon-gluon
fusion.  The CSM was taken seriously until around 1995, when
experiments at the Tevatron showed that it under-predicts the cross
section for prompt charmonium production in $p \bar p$ collisions by
more than an order of magnitude.  An extensive review of the
color-singlet model can be found in Ref.~\cite{Schuler:1994hy}.

The color-singlet model can be obtained from the NRQCD factorization
formula in Eq.~(\ref{prod:fact}) by dropping all of the color-octet
terms and all but one of the color-singlet terms. The term that is
retained is the one in which the quantum numbers of the $Q \bar Q$ pair
are the same as those of the quarkonium. The CSM production matrix
elements are related to the corresponding decay matrix elements by the
vacuum-saturation approximation, and, so, they can be determined from
annihilation decay rates. Thus, the CSM gives absolutely
normalized predictions for production cross sections. 
The heavy-quark spin symmetry relates the CSM
matrix elements of the $4(2L+1)$ states within an
orbital-angular-momentum multiplet with quantum number $L$. 
Thus, the CSM also gives nontrivial predictions for polarization.

In the case of an $S$-wave state, the CSM term in Eq.~(\ref{prod:fact})
is the one whose matrix element is of leading order in $v$. However,
owing to kinematic factors or factors of $\alpha_s$ in the short-distance
coefficients, the CSM term may not be dominant. In the case of a
$P$-wave state or a state of higher orbital angular momentum, the CSM
term is only one of the terms whose matrix element is of leading
order in $v$. For these states, the CSM leads to infrared divergences
that cancel only when one includes color-octet terms whose matrix
elements are also of leading order in $v$. Thus, the CSM is
theoretically inconsistent for quarkonium states with nonzero orbital
angular momentum.


\subsection{Color-evaporation model}
\label{prodsec:nrqcdCEM}

The color evaporation model (CEM) was first proposed in 1977 
\cite{Fritzsch:1977ay,Halzen:1977rs,Gluck:1977zm,Barger:1979js}
and has enjoyed considerable phenomenological success. 
In the CEM, the cross section for a quarkonium state $H$
is some fraction $F_H$ of the cross section for producing
$Q \bar Q$ pairs with invariant mass below the $M \bar M$ threshold, 
where $M$ is the lowest mass meson containing the heavy quark $Q$. 
(The CEM parameter $F_H$ should not be confused with the fraction
of $J/\psi$'s that come from decay of $H$.)
This cross section has an upper limit on the 
$Q \bar Q$ pair mass but no constraints on the 
color or spin of the final state.  The $Q \bar Q$ pair 
is assumed to neutralize its color by interaction with the
collision-induced color field, that is, by ``color evaporation.'' The
$Q$ and the $\bar Q$ either combine with light quarks to produce
heavy-flavored hadrons or bind with each other to form quarkonium.  If
the $Q \bar Q$ invariant mass is less than the heavy-meson threshold
$2m_M$, then the additional energy that is needed to produce
heavy-flavored hadrons can be obtained from the nonperturbative color
field. Thus, the sum of the fractions $F_H$ over all quarkonium states
$H$ can be less than unity. The fractions $F_H$ are assumed to be
universal so that, once they are determined by data, they can be used to
predict the cross sections in other processes and in other kinematic
regions.

In the CEM at leading order in $\alpha_s$, the production cross section 
for the quarkonium state $H$ in collisions of the light hadrons 
$h_A$ and $h_B$ is
%
\begin{eqnarray}
\lefteqn{\sigma_{\rm CEM}^{\rm (LO)}[h_A h_B \to H + X]  =} \nonumber\\
&& F_H \sum_{i,j} 
\int_{4m^2}^{4m_M^2} d\hat{s}
\int \! dx_1 dx_2~f_i^{h_A}(x_1,\mu)~f_j^{h_B}(x_2,\mu)~ 
\hat\sigma_{ij}(\hat{s})~\delta(\hat{s} - x_1x_2s)  \, , 
\label{prod:sigtil}
\end{eqnarray} 
%
where $ij = q \bar q$ or $g g$, $\hat s$ is the square of the partonic
center-of-mass energy, and $\hat\sigma_{ij}(\hat s)$ is the
$ij\rightarrow Q\bar Q$ subprocess cross section. The leading-order
calculation cannot describe the quarkonium $p_T$ distribution, since
the $p_T$ of the $Q \bar Q$ pair is zero at LO. At NLO in $\alpha_s$,
the subprocesses $ij\rightarrow k Q \bar Q$, where $i$, $j$, and $k$
are light quarks, antiquarks, and gluons, produce $Q \bar Q$ pairs
with nonzero $p_T$. Complete NLO calculations of quarkonium production
in hadronic collisions using the CEM have been carried out in
Refs.~\cite{Gavai:1994in,Schuler:1996ku}, using the exclusive $Q \bar
Q$ production code of Ref.~\cite{Mangano:kq} to obtain the $Q \bar Q$
pair distributions. The resulting values of the 
parameters $F_H$ are given 
in Section~\ref{prodsec:fixed-targetCEM}. There are also
calculations in the CEM beyond LO that use only a subset of the NLO
diagrams~\cite{Amundson:1996qr} and calculations that describe the
soft color interaction within the framework of a Monte Carlo event
generator~\cite{Edin:1997zb}. Calculations beyond LO in the CEM have
also been carried out for $\gamma p$, $\gamma\gamma$ and
neutrino-nucleon collisions and for $Z^0$ decays
\cite{Eboli:1998xx,Eboli:2003fr,Eboli:2003qg,Eboli:2001hc,Gregores:1996ek}.  
Apparently, the color-evaporation model has not been applied to
quarkonium production in $e^+ e^-$ annihilation.

The most basic prediction of the CEM is that the ratio of the cross
sections for any two quarkonium states should be constant, independent
of the process and the kinematic region. Some variations in these ratios
have been observed. For example, the ratio of the cross sections for
$\chi_c$ and $J/\psi$ are rather different in photoproduction and
hadroproduction. Such variations present a serious challenge to the
status of the CEM as a quantitative phenomenological model for
quarkonium production.

In some papers on the Color Evaporation Model
\cite{Amundson:1996qr}, the collision-induced color field that
neutralizes the color of the $Q \bar Q$ pair is also assumed to
randomize its spin. This leads to the prediction that the quarkonium
production rate is independent of the quarkonium spin. This prediction
is contradicted by measurements of nonzero polarization of the
$J/\psi$, the $\psi(2S)$, and the $\Upsilon(nS)$ in several
experiments. The assumption of the randomization of the $Q \bar Q$ spin
also implies simple spin-counting ratios for the cross sections 
for the direct production of
quarkonium states in the same orbital-angular-momentum multiplet.
For example, the CEM with spin randomization predicts that 
the direct-production cross sections for charmonium satisfy 
$\sigma_{\rm dir}[\eta_c]:\sigma_{\rm dir}[J/\psi] = 1:3$ and  
$\sigma_{\rm dir}[\chi_{c0}]:\sigma_{\rm dir}[\chi_{c1}]:
        \sigma_{\rm dir}[\chi_{c2}] = 1:3:5$. 
The inclusive cross sections need not satisfy these spin-counting relations
if there is significant feeddown from decay of higher quarkonium states, 
as is the case for $J/\psi$.
Deviations from the predicted spin-counting ratio for
$\chi_{c1}$ to $\chi_{c2}$ have been observed.
One might conclude that the CEM is ruled out by the
observations of nonzero polarization 
and of deviations from the spin-counting relations. 
On the other hand, the assumption of
the randomization of the $Q \bar Q$ spin is really independent of the
assumption of color evaporation. 
Some proponents of the CEM omit the assumption of spin randomization.
Alternatively, since the CEM is just a model, one can simply declare 
it to apply only to spin-averaged cross sections.
In the remainder of this chapter, when we mention the predictions of the
CEM for the relative production rates of quarkonium states that differ
only in their spin or total-angular-momentum quantum numbers, we are
referring to the version of the CEM that includes the assumption of spin
randomization.

There is a simple correspondence between
the CEM and the NRQCD factorization approach. The CEM amounts to the
assumption that an NRQCD production matrix element 
$\langle {\cal O}_n^H(\Lambda) \rangle$
is proportional to the expectation value of
the operator that is obtained by replacing the projector onto the
hadronic state $H$ with a projector onto the set of $Q\bar Q$ states
with invariant mass less
than $2m_M$. In addition to an integral over the $Q\bar Q$ phase space,
the projector contains sums over the $Q\bar Q$ spins and colors.
The only dependence on
the quarkonium $H$ is through a common factor $F_H$ in the
proportionality constant for each NRQCD matrix element. Since, in this
picture, the probability of forming a specific quarkonium state $H$ is
independent of the color and spin state of the $Q \bar Q$ pair, NRQCD
matrix elements that differ only  by color and spin quantum numbers are
equal up to simple group theory factors. This picture also implies a
hierarchy of NRQCD matrix elements according to their
orbital-angular-momentum quantum number $L$. In the integration over the
$Q\bar Q$ phase space of an NRQCD operator with orbital-angular-momentum
quantum number $L$, the leading term scales as $k^{2L+1}$, where $k$ is
the $Q$ or $\bar Q$ momentum in the $Q\bar Q$ rest frame. The difference
$s_{\rm max}- 4 m^2$ is proportional to $k^2$. Hence, there is an
orbital-angular-momentum suppression factor $[(s_{\rm max}- 4 m^2)/4
m^2]^L \sim v^{2L}$ in the matrix elements.\footnote{From the
perspective of NRQCD, the upper limit $s_{\rm max} = 4m_M^2$ on the $Q
\bar Q$ invariant mass that traditionally has been used in the CEM is
quite arbitrary.  Any choice that satisfies $s_{\rm max}- 4m_Q^2 \sim 4
m_Q^2 v^2$ leads to the same velocity-scaling rules.} That is, the CEM
implies that $S$-wave NRQCD matrix elements dominate and that those with
orbital-angular-momentum quantum number $L\ge 1$ are suppressed as
$v^{2L}$. One way to test the assumptions of the CEM is to extract the
NRQCD matrix elements from data and compare them with the predictions of
the CEM. 

The qualifier NLO in ``the CEM at NLO'' is somewhat misleading. As is
described in Section~\ref{prodsec:nrqcdmge}, the NLO cross section for
$Q \bar Q$ production that is used in computing the CEM predictions is
accurate through order $\alpha_s^3$, which is next-to-leading order at
zero $p_T$, but leading order at nonzero $p_T$. This is the same
accuracy in $\alpha_s$ as the existing predictions in the NRQCD
factorization approach. The NLO $Q\bar Q$ $p_T$ distribution is singular
at $p_T=0$, but integrable.  The existing NLO calculations in the CEM
obtain a smooth $p_T$ distribution at small $p_T$ by using a smearing
prescription to mimic the effects of multiple gluon emission. The
smearing has a significant effect on the shape of the $p_T$
distribution, except at very large $p_T$.




\subsection{Multiple gluon emission}
\label{prodsec:nrqcdmge}

Multiple gluon emission can be very important for transverse momentum
distributions, distributions near kinematic limits, and in situations in
which production near threshold is important.  For example, a
fixed-order perturbative calculation typically gives a transverse
momentum distribution $d \sigma/dp_T^2$ for quarkonium that includes
terms proportional to $\delta(p_T^2)$ and $1/p_T^2$ that are singular as
$p_T \rightarrow 0$. (However, the distribution has a well-behaved
integral over $p_T$.) This singular distribution becomes a smooth one
when the effects of multiple gluon emission are taken into account to
all orders in perturbation theory. Several methods, which we now
describe, have been developed to take into account some of these
effects.

{\it Resummation} methods sum, to all orders in $\alpha_s$,  certain
logarithmically enhanced terms that are associated with soft- and
collinear-gluon emission. The resummations can be carried out at various
levels of precision in the logarithmic enhancements, that is, in leading 
logarithmic (LL) order, in next-to-leading logarithmic (NLL) order,
etc. Resummation can, in principle, be extended to arbitrarily high
precision in the logarithmic enhancements. However, in practice, it is
seldom carried out beyond LL or NLL accuracy. Generally, logarithms of
$p_T^2/M^2$ have the largest effect on $p_T$ distributions
\cite{Collins:1984kg}, although logarithms of the available partonic
energy above threshold (threshold logarithms) and logarithms of
$s/p_T^2$ (small-x logarithms) can also be important for particular
processes and kinematic regions\footnote{For a general discussion of
resummation techniques for logarithms of $p_T^2/M^2$ and threshold
logarithms, see Ref.~\cite{Contopanagos:1996nh}.}. Because arbitrarily
soft or collinear gluon emissions are resummed, the resummed expressions
depend on nonperturbative functions. This dependence lessens as the mass
and transverse momentum scales of the process increase, and it may be
insignificant at large masses and/or transverse momenta. Some
practical disadvantages of the resummation method are that it has to be
reformulated, to some extent, for every process and that it usually does
not yield results that are fully differential in all of the kinematic
variables. Since resummation calculations retain only soft and collinear
logarithmically enhanced terms, they generally do not describe
accurately processes in which hard gluons are emitted at large
angles---so called ``Mercedes events.'' This situation can be remedied
to some extent by combining resummation with exact next-to-leading order
(NLO) calculations, which retain all contributions associated with gluon
emission at NLO, not just logarithmically enhanced contributions
\cite{Cacciari:1998it}. 

{\it Parton-shower Monte Carlos} share with resummation methods the
approach of modeling multiple gluon emission by retaining certain
logarithmically enhanced terms in the cross section. The Monte Carlos
take into account a finite, but arbitrarily large, number of gluon
emissions. The original implementations of shower-Monte-Carlo methods,
such as ISAJET \cite{Paige:fb,Paige:2003mg}, generally treat only the
leading collinear logarithmic enhancements correctly, while more recent
implementations, such as \textsc{Pythia}
\cite{Sjostrand:1985ys,Sjostrand:1986hx} and HERWIG
\cite{Marchesini:1987cf,Marchesini:1991ch} treat both the leading
collinear and soft logarithmic enhancements correctly. Generally, the
showering processes are cut off so that they do not become so soft or
collinear as to be nonperturbative in nature. The showering may then be
supplemented with nonperturbative models that describe the 
hadronization of the
partons. A practical advantage of the
shower-Monte-Carlo approach is that it is generally applied easily to
any Born-level production process. Furthermore, it produces results that
are differential in all of the kinematic variables that are associated
with the final-state particles. Hence, it lends itself to the
application of experimental cuts. As is the case with resummation
methods, the shower-Monte-Carlo approach does not yield an accurate
modeling of processes in which hard gluons are emitted at large angles.
A partial remedy for this problem is to use shower Monte Carlos in
conjunction with exact NLO calculations, rather than LO calculations.
Recently, important progress has been made in this direction
\cite{Frixione:2002ik,Frixione:2002bd,Frixione:2003ei,%
Frixione:2004wy,Kramer:2003jk,Soper:2003ya}.  In contrast with
resummation methods, some shower Monte Carlos do not take into account
virtual gluon emission. Such shower Monte Carlos do not yield reliable
estimates of the total cross section.

The {\it $k_T$-factorization} method is an attempt to take into account
initial-state radiation through parton distributions that depend the
parton's transverse momentum $k_T$, as well as on the parton's
longitudinal momentum fraction $x$. It generally gives answers that are
very different from those of collinear factorization.  The
$k_T$-dependent parton distributions are not very well known
phenomenologically, and there are possibly unresolved theoretical
issues, such as the universality of the $k_T$-dependent parton
distributions.

The {\it $k_T$-smearing} method is a phenomenological model for multiple
initial-state radiation.  As in the $k_T$-factorization method, the
$k_T$ smearing method makes use of $k_T$-dependent parton distributions.
It is assumed that the distribution factors into the $x$-dependent PDF's
that are defined by collinear factorization and a Gaussian distribution
in the transverse momentum $k_T$.  The width $\langle k_T^2 \rangle$ of
the Gaussian can be treated as a process-dependent phenomenological
parameter.  One advantage of this model is that it is easy to implement.
On the other hand, while this model may capture some of the crude
features of soft- and collinear-gluon emission, it is probably incorrect
in detail: resummation methods and shower Monte Carlos
yield transverse-momentum distributions that have longer tails than
those of a Gaussian distribution. The impact of a parton shower on the
quarkonium transverse momentum distribution is, in general, larger than
for the Gaussian $k_T$ smearing, and it extends out to larger values of 
$p_T$.


\subsection{Production in nuclear matter}
\label{prodsec:nrqcdnuc}

The existing factorization ``theorems'' for quarkonium production in
hadronic collisions are for cold hadronic matter. These theorems predict
that nuclear matter is ``transparent'' for $J/\psi$ production at large
$p_T$. That is, at large $p_T$, all of the nuclear effects are contained
in the nuclear parton distributions. The corrections to this
transparency are of order $(mv)^2/p_T^2$ for unpolarized cross
sections and of order $mv/p_T$ for polarized cross sections.

The effects of transverse-momentum kicks from multiple elastic
collisions between active partons and spectators in the nucleons are
among those effects that are suppressed by $(mv)^2/p_T^2$. Nevertheless,
these multiple-scattering effects can be important because the
production cross section falls steeply with $p_T$ and because the number
of scatterings grows linearly with the length of the path through the
nuclear matter. Such elastic interactions can be expressed in terms of
eikonal interactions \cite{Bodwin:1988fs} or higher-twist matrix
elements \cite{Qiu:2001hj}.

Inelastic scattering of the quarkonium by the nuclear matter is also an
effect of higher order in $(mv)^2/p_T^2$. However, it can become
dominant when the amount of nuclear matter that is traversed by the
quarkonium is sufficiently large. Factorization breaks down when the
length $L$ of the quarkonium path in the nucleus satisfies
\begin{equation}
L \gsim {{\rm Min}(z_Q,z_{\bar Q})P_{\rm onium}^2\over 
M_A (k_T^{\rm tot})^2},
\end{equation}
where $M_A$ is
the mass of the nucleus, $z$ is the parton longitudinal momentum
fraction, $P_{\rm onium}$ is the momentum of the quarkonium in the
parton CM frame, and $k_T^{\rm tot}$ is the accumulated 
transverse-momentum ``kick'' from passage through the nuclear matter. 
This condition for the break-down of factorization is similar to
``target-length condition'' in Drell-Yan production
\cite{Bodwin:1981fv,Bodwin:1984hc}. Such a breakdown of
factorization is observed in the Cronin effect at low $p_T$ and in
Drell-Yan production at low $Q^2$, where the cross section is
proportional to the nucleon number raised to a power less than unity.

It is possible that multiple-scattering effects may be
larger for color-octet production than for color-singlet production.
In the case of color-octet production, the pre-quarkonium $Q\bar Q$
system carries a nonzero color charge and, therefore, has a larger 
amplitude to exchange soft gluons with spectator partons.

At present, there is no complete, rigorous theory to account for all of
the effects of multiple scattering, and we must resort to
``QCD-inspired'' models. A reasonable requirement for models is that
they be constructed so that they are compatible with the factorization
result in the large-$p_T$ limit. Many models treat interactions of the
pre-quarkonium with the nucleus as on-shell scattering (Glauber
scattering). This assumption should be examined carefully, as on-shell
scattering is known, from the factorization proofs, not to be a valid
approximation in leading order in $(mv)^2/p_T^2$.



\section{Quarkonium production at the Tevatron}
\label{prodsec:tevatron}
Charmonium and bottomonium are produced copiously in high energy hadron
colliders. The present and future hadron colliders include
%
\begin{itemize}

\item the Tevatron, 
a $p \bar p$ collider operating at Fermilab with center-of-mass energy 
of 1.8 TeV in Run I and 1.96 TeV in Run II,
 
\item RHIC, 
a heavy-ion or $p p$ collider operating at Brookhaven with 
center-of-mass energy of up to 200 GeV per nucleon-nucleon collision,

\item 
the LHC at CERN, a $p p$ collider under construction at CERN 
with center-of-mass energy of 17 TeV.
\end{itemize}
%
In this section, we focus on the Tevatron, because it has produced 
the most extensive and precise data on quarkonium production.
The photoproduction of quarkonium at high-energy $p\bar{p}$, $pp$, 
and heavy ion colliders is discussed in Chapter 7 ``Quarkonium
in Media'' of this report. 


\subsection{Charmonium cross sections}
\label{prodsec:tevatroncharm}

%%%%%%%%%%%%%%%%%%%%%%%%%%%%%%%%%%%%%%%%%%%%%%%%%%%%%%%%%%%%%%%%%%%%%%%%%%%%%%%%%%%%%%%%%%%%%
\begin{figure}[htbp]
\begin{center}
\vspace{-0.5cm}
\epsfig{figure=kramer-fig2.eps,width=9cm}
\vspace{-0.5cm}
\epsfig{figure=kramer-fig3.eps,width=9cm}
\vspace{-0.5cm}
\epsfig{figure=kramer-fig4.eps,width=9cm}
\caption{Differential cross sections for the production of direct
$J/\psi$ (top), prompt $\psi(2S)$ (middle), and prompt $J/\psi$ from
decay of $\chi_c$ (bottom) at the Tevatron as a function of $p_T$.
The data points are CDF measurements from Run I
\cite{Abe:1997jz,Abe:1997yz}.  The dotted curves are the CSM
contributions.  The solid curves are the NRQCD factorization fits, and
the other curves are individual color-octet contributions to the
fits. From Ref.~\cite{Kramer:2001hh}.  }
\label{fig-tevatron_psi}
\end{center}
\end{figure}
%%%%%%%%%%%%%%%%%%%%%%%%%%%%%%%%%%%%%%%%%%%%%%%%%%%%%%%%%%%%%%%%%%%%%%%%%%%%%%%%%%%%%%%%%%%%%

In high energy collisions, charmonium is produced both through direct
production mechanisms and through decays of other hadrons. In the case of
charmonium production through $B$-hadron decays, the charmonium is
produced at a secondary vertex, and a vertex detector can be used to
identify this contribution to the measured production rate.  We refer to
the inclusive cross section for production of a charmonium state with
the contribution from $B$ decays removed as the {\it prompt} cross
section. The prompt cross section includes both 
the direct production of the charmonium and its
production through decays of higher charmonium states.

In Run I of the Tevatron, the CDF collaboration measured the prompt
cross sections for the production of several charmonium states
in $p \bar p$ collisions at
a center-of-mass energy of 1.8 TeV \cite{Abe:1997jz,Abe:1997yz}. The
CDF data for production of direct $J/\psi$, prompt $\psi(2S)$, and
prompt $J/\psi$ from decay of $\chi_c$ are shown in
Fig.~\ref{fig-tevatron_psi}. In the CDF analysis, prompt $J/\psi$'s
that do not come from decays of $\psi(2S)$ or $\chi_c$ were assumed to
be produced directly. 

At non-vanishing transverse momentum, the leading parton processes for
producing charmonium ($ij \rightarrow c \bar{c} + k$, where $i$, $j$,
and $k$ are light quarks, antiquarks, and gluons) occur at order
$\alpha_s^3$. The color-singlet-model (CSM) predictions
are shown as dotted lines in Fig.~\ref{fig-tevatron_psi}.  In the top
two panels of Fig.~\ref{fig-tevatron_psi}, the more steeply falling
dotted lines are the predictions of the CSM at leading order in
$\alpha_s$. The other dotted lines in the top two panels of
Fig.~\ref{fig-tevatron_psi} are contributions of higher order in
$\alpha_s$ involving gluon fragmentation.  As can be seen in the top
panel of Fig.~\ref{fig-tevatron_psi}, the gluon-fragmentation
contribution renders the shape of the CSM prediction for direct
$J/\psi$ production roughly compatible with the CDF data. However, the
normalization is too small by more than an order of magnitude. There
is a similar discrepancy in the normalization for prompt $\psi(2S)$
production, as can be seen in the middle panel of
Fig.~\ref{fig-tevatron_psi}. In the case of production of prompt
$J/\psi$ from decay of $\chi_c$, which is shown in the bottom panel of
Fig.~\ref{fig-tevatron_psi}, the discrepancy is less dramatic, but the
cross section is still under-predicted by the CSM.  The large
discrepancies between the measurements and the CSM predictions for the
production cross section for S-wave charmonium states rules out the
CSM as a credible model for quarkonium production.

%%%%%%%%%%%%%%%%%%%%%%%%%%%%%%%%%%%%%%%%%%%%%%%%%%%%%%%%%%%%%%%%%%%%%%%%%%%%%%%%%%%%%%%%%%%%%
\begin{table}
%\vspace*{5mm}
\begin{center}
\renewcommand{\arraystretch}{1.5}
$$
\begin{array}{|c|ccc|}
\hline\hline
 H & \langle {\cal{O}}_1^{H} \rangle  & \langle
 {\cal{O}}^{H}_8({}^3S_1) \rangle  &
 M_{3.5}^{H}\\ \hline
 J/\psi   & 1.16~{\rm GeV^3} & (1.19 \pm 0.14)\times 10^{-2}~{\rm GeV}^3 &  
 (4.54 \pm 1.11)\times 10^{-2}~{\rm GeV}^3 \\[-1mm] 
 \psi(2S) & 0.76~{\rm GeV^3} & (0.50 \pm 0.06)\times 10^{-2}~{\rm GeV}^3 & 
 (1.89 \pm 0.46)\times 10^{-2}~{\rm GeV}^3 
 \\[-1mm]
 \chi_{c0} & 0.11~{\rm GeV^5} & (0.31 \pm 0.04)\times 10^{-2}~{\rm GeV}^3 & 
 \\[1mm] \hline \hline
\end{array}
$$
\renewcommand{\arraystretch}{1.0}
\caption{NRQCD production matrix elements for charmonium states
obtained from the transverse momentum distributions at the
Tevatron~\cite{Kramer:2001hh}. The errors quoted are statistical
only.}
\label{tab:me-1}
\end{center}
\end{table}
%%%%%%%%%%%%%%%%%%%%%%%%%%%%%%%%%%%%%%%%%%%%%%%%%%%%%%%%%%%%%%%%%%%%%%%%%%%%%%%%%%%%%%%%%%%%%

According to the NRQCD factorization approach, the charmonium
production cross section contains not only the CSM terms, which are
absolutely normalized, but also color-octet terms, whose
normalizations are determined by color-octet matrix elements.  In the
case of $J/\psi$ and $\psi(2S)$ production, the most important
color-octet matrix elements are $\langle{\cal O}^H_8({}^3S_1)\rangle$,
$\langle{\cal O}^H_8({}^3P_0)\rangle$, and $\langle{\cal
O}^H_8({}^1S_0)\rangle$. At large $p_T$, the $J/\psi$ and $\psi(2S)$
cross sections are dominated by gluon fragmentation into color-octet
${}^3S_1$ charm pairs~\cite{Braaten:1995vv}, which falls as
$d\hat{\sigma}/dp_T^2\sim1/p_T^4$. The color-octet ${}^1S_0$ and
${}^3P_J$ channels are significant in the region $p_T\; \rlap{\lower
3.5 pt \hbox{$\mathchar \sim$}} \raise 1pt \hbox {$<$}\; 10$~GeV, but
fall as $d\hat{\sigma}/dp_T^2\sim1/p_T^6$ and become negligible at
large $p_t$.  Because the ${}^1S_0^{(8)}$ and ${}^3P_J^{(8)}$
short-distance cross sections have a similar $p_t$ dependence, the
transverse momentum distribution is sensitive only to the linear
combination $M^H_k$ defined in (\ref{prod:lincomb}), with $k\approx
3$.  As can be seen in the top panel of Fig.~\ref{fig-tevatron_psi}, a
good fit to the normalization and shape of the direct $J/\psi$ cross
section can be obtained by adjusting $\langle{\cal
O}^{J/\psi}_8({}^3S_1)\rangle$ and $M^{J/\psi}_{3.5}$. As is shown in
the middle panel of Fig.~\ref{fig-tevatron_psi}, a similarly good fit
to the prompt $\psi(2S)$ cross section can be obtained by adjusting
the corresponding parameters for $\psi(2S)$.  In the case of
production of the $\chi_{cJ}$ states, the most important color-octet
matrix element is $\langle{\cal O}^H_8({}^3S_1)\rangle$.  As can be
seen in the bottom panel of Fig.~\ref{fig-tevatron_psi}, the fit to
the cross section for production of prompt $J/\psi$ from decay of
$\chi_c$ can be improved by adjusting $\langle{\cal
O}^{\chi_{c0}}_8({}^3S_1)\rangle$.  Table~\ref{tab:me-1} shows the
values of the quarkonium matrix elements that are obtained in the fit
of Ref.~\cite{Kramer:2001hh,Beneke:1996yw}.  The color-singlet matrix
elements are taken from the potential-model calculation of
Refs.~\cite{Buchmuller:1980su,Eichten:1995ch}.  The color-octet matrix
elements have been extracted from the CDF data
\cite{Abe:1997jz,Abe:1997yz}.  The CTEQ5L parton distribution
functions \cite{Lai:1999wy} were used, with renormalization and
factorization scales $\mu=(p_T^2+4 m_c^2)^{1/2}$ and
$m_c=1.5\,$GeV. The Altarelli-Parisi evolution has been included for
the $\langle{\cal O}^{\chi_{c0}}_8({}^3S_1)\rangle$ fragmentation
contribution. See Ref.~\cite{Beneke:1996yw} for further details. The
extraction of the various color-octet matrix elements relies on the
differences in their $p_T$ dependences. Smaller experimental error
bars could help to resolve the different $p_T$ dependences with
greater precision.

%%%%%%%%%%%%%%%%%%%%%%%%%%%%%%%%%%%%%%%%%%%%%%%%%%%%%%%%%%%%%%%%%%%%%%%%%%%%%%%%%%%%%%%%%%%%%
\begin{figure}[htb]
\setlength{\epsfxsize=0.95\textwidth}
\setlength{\epsfysize=0.5\textheight}
\centerline{\epsffile{psicdfpt3.ps}}
\caption{Differential cross sections for production of direct $J/\psi$
(top left), prompt $J/\psi$ from decays of $\psi(2S)$ (top right), and
prompt $J/\psi$ from decays of $\chi_c$ (bottom) at the Tevatron as a
function of $p_T$.  The data points are the CDF measurements
\cite{Abe:1997jz,Abe:1997yz}.  The dotted and solid curves are the CEM 
predictions at NLO with $\langle k_T^2\rangle = 2.5$ GeV$^2$, 
using the first and fourth charmonium parameter sets in 
Table~\ref{prodsec:qqbparams}.  }
\label{psicdfptdep}
\end{figure}
%%%%%%%%%%%%%%%%%%%%%%%%%%%%%%%%%%%%%%%%%%%%%%%%%%%%%%%%%%%%%%%%%%%%%%%%%%%%%%%%%%%%%%%%%%%%%

The normalization and the shape of the prompt charmonium cross section
at the Tevatron can also be described reasonably well by the
color-evaporation model (CEM).  
The CEM parameters can be fixed
by fitting to the data from $pN$ collisions and by
using the measured branching fractions for charmonium decays. The
predictions of the CEM at next-to-leading order in $\alpha_s$ (NLO)
can be calculated using the NLO parameter sets that are described in
Section~\ref{prodsec:fixed-targetCEM}. The normalization of the predicted
cross section for prompt $J/\psi$ production is in reasonable
agreement with the CDF data from Run~I. The shape can be brought into
good agreement by adding $k_T$ smearing, with $\langle k_T^2\rangle =
2.5$~GeV$^2$. In Fig.~\ref{psicdfptdep}, the resulting CEM predictions
are compared with the CDF charmonium data for production of direct
$J/\psi$, prompt $J/\psi$ from decay of $\psi(2S)$, and prompt $J/\psi$
from decay of $\chi_c$. The predictions are all in good agreement with
the CDF data.

In the case of the $S$-wave production matrix elements, the NRQCD
velocity-scaling rules predict that 
\begin{equation} 
{ \langle{\cal O}_8\rangle \over \langle{\cal O}_1\rangle}
\sim {v^4 \over 2N_c},
\end{equation} 
where this estimate includes color factors that are associated with
the expectation values of the NRQCD operators, as advocated by Petrelli
{\it et al.} \cite{Petrelli:1997ge}. As can be seen from
Table~\ref{tab:me-1}, the extracted color-octet matrix elements are
roughly compatible with this estimate [$v^4/(2N_c)\approx 0.015$].
However, a much more stringent test of the theory is to check the
universality of the extracted matrix elements in other processes. In
the case of the $P$-wave production matrix elements, the velocity
scaling rules yield the estimate
\begin{equation}
{\langle{\cal O}_8\rangle \over \langle{\cal O}_1\rangle/m_c^2}
\sim {v^0 \over 2N_c}.
\end{equation}
The $P$-wave color-octet matrix element in Table~\ref{tab:me-1} is
somewhat smaller than this estimate would suggest. That is also the
case for the matrix elements that appear in $P$-wave quarkonium
decays, which have been determined phenomenologically
\cite{Maltoni:2000km} and in lattice calculations
\cite{Bodwin:1993wf,Bodwin:1994js,Bodwin:1996tg,Bodwin:1996mf,Bodwin:2001mk}.

% ----------------------------------------------------------------------
\begin{table}
\begin{center}
\renewcommand{\arraystretch}{1.3}
\begin{tabular}{|c|cc|ccc|}
\hline\hline
 \mbox{Reference} & \multicolumn{2}{c|}{\mbox{PDF}} & $\langle
 {\cal{O}}^{J/\psi}_8({}^3S_1) \rangle$ &
 $M_{k}^{J/\psi} $ & $ k $ \\ \hline\hline
 \multicolumn{6}{|c|}{\mbox{LO collinear factorization}} \\ \hline
 {\rm CL} \cite{Cho:1995ce} &
 \multicolumn{2}{c|}{\mbox{MRS(D0)~\cite{Martin:1992zi}}} & $ 0.66 \pm
 0.21 $ & $6.6 \pm 1.5$ & 3 \\ \hline &
 \multicolumn{2}{c|}{\mbox{CTEQ4L~\cite{Lai:1996mg}}} & $1.06 \pm
 0.14^{+1.05}_{-0.59}$ & $4.38 \pm 1.15^{+1.52}_{-0.74}$ & \\
 {\rm BK~\cite{Beneke:1996yw}} &
 \multicolumn{2}{c|}{\mbox{GRV-LO(94)~\cite{Gluck:1994uf}}} & $ 1.12 \pm
 0.14^{+0.99}_{-0.56} $ & $3.90 \pm 1.14^{+1.46}_{-1.07}$ & 3.5 \\ &
 \multicolumn{2}{c|}{\mbox{MRS(R2)~\cite{Martin:1996as}}} & $ 1.40 \pm
 0.22^{+1.35}_{-0.79} $ & $10.9 \pm 2.07^{+2.79}_{-1.26}$ & \\ \hline &
 \multicolumn{2}{c|}{\mbox{MRST-LO(98)~\cite{Martin:1998sq}}} & $ 0.44
 \pm 0.07 $ & $ 8.7 \pm 0.9$  & \\
 \raisebox{2ex}[-2ex]{BKL~\cite{Braaten:1999qk}} &
 \multicolumn{2}{c|}{\mbox{CTEQ5L~\cite{Lai:1999wy}}} & $0.39 \pm 0.07$
 & $6.6 \pm 0.7 $  & \raisebox{2ex}[-2ex]{3.4} \\[0.5mm] \hline\hline
 \multicolumn{6}{|c|}{\mbox{Parton shower radiation}} \\ \hline &
 \multicolumn{2}{c|}{\mbox{CTEQ2L~\cite{Tung:ua}}} & $0.96 \pm 0.15$
 & $1.32 \pm 0.21 $ & \\ {\rm S~\cite{Sanchis-Lozano:1999um}} &
 \multicolumn{2}{c|}{\mbox{MRS(D0)~\cite{Martin:1992zi}}} & $0.68 \pm
 0.16$ & $1.32 \pm 0.21$ & 3  \\ &
 \multicolumn{2}{c|}{\mbox{GRV-HO(94)~\cite{Gluck:1994uf}}} & $0.92 \pm
 0.11$ & $0.45 \pm 0.09$ & \\ \hline {\rm KK~\cite{Kniehl:1998qy}} &
 \multicolumn{2}{c|}{\mbox{CTEQ4M~\cite{Lai:1996mg}}} & $0.27 \pm 0.05$
 & $0.57 \pm 0.18 $ & 3.5 \\[0.5mm] \hline \hline
 \multicolumn{6}{|c|}{\mbox{$k_T$-smearing}} \\ \hline & & $ \langle k_T
 \rangle \mbox{[GeV]} $ & & & \\ & & 1 & $1.5\pm 0.22$ & $8.6\pm 2.1$ & \\
 \raisebox{2ex}[-2ex]{P~\cite{Petrelli:1999rh}} &
 \raisebox{2ex}[-2ex]{CTEQ4M~\cite{Lai:1996mg}} & 1.5 & $1.7 \pm 0.19$ & $
 4.5 \pm 1.5 $ & \raisebox{2ex}[-2ex]{3.5}\\ \hline & & 0.7 & $ 1.35 \pm
 0.30 $ & $ 8.46 \pm 1.41 $ & \\ \raisebox{2ex}[-2ex]{SMS~\cite{Sridhar:1998rt}}
 & \raisebox{2ex}[-2ex]{MRS(D$'_-$)~\cite{Martin:1992zi}} & 1 & $ 1.5
 \pm 0.29 $  & $ 7.05 \pm 1.17 $ & \raisebox{2ex}[-2ex]{3} \\[0.5mm]
 \hline\hline \multicolumn{6}{|c|}{\mbox{$k_T$-factorization}} \\
 \hline {\rm HKSST1~\cite{Hagler:2000eu}} &
 \multicolumn{2}{c|}{\mbox{KMS~\cite{Kwiecinski:1997ee}}} & $ \approx
 0.04 \pm 0.01 $  & $ \approx 6.5 \pm 0.5 $ & 5\\ \hline \hline
\end{tabular}
\renewcommand{\arraystretch}{1.0}
\end{center}
\caption{$J/\psi$ production matrix elements in units of
$10^{-2}$~GeV${}^3$~\cite{Kramer:2001hh}.
The first error bar is statistical; the second
error bar (where present) is obtained by varying the factorization and
renormalization scales. 
}
\label{tab:me-2}
\end{table}
% -----------------------------------------------------------------------

In Table~\ref{tab:me-2}, we show matrix elements for $J/\psi$ production
that have been obtained from various other fits to the transverse
momentum distribution. We see that there is a large uncertainty that
arises from the dependence of the matrix elements on the factorization
and renormalization scales, as well as a large dependence on the choice
of parton distributions. The  extracted values of the color-octet matrix
elements (especially $M_k$) are very sensitive to the small-$p_T$
behavior of the cross section and this, in turn, leads to a sensitivity
to the behavior of the gluon distribution at small $x$. Furthermore, the
effects of multiple gluon emission are important, and their
omission in the fixed-order perturbative calculations leads to
overestimates of the sizes of the matrix elements. In
Table~\ref{tab:me-2}, one can see the results of various attempts to
estimate the effects of multiple gluon emission. Sanchis-Lozano (S)
and Kniehl and Kramer (KK) made use of parton-shower Monte Carlos,
while Petrelli (P) and Sridhar, Martin, and Stirling (SMS) employed
models containing Gaussian $k_T$ smearing. In addition, Sanchis-Lozano
included a resummation of logarithms of $p_T^2/m^2$. H\"agler,
Kirschner, Sch\"afer, Szymanowski, and Teryaev (HKSST) used the
$k_T$-factorization formalism to resum large logarithms in the limit $s
\gg 4m_c^2$. (See also the calculations by Yuan and Chao \cite{Yuan:2000cp,Yuan:2000qe}.) 
Similar large dependences on the choices of factorization and
renormalization scales, parton distributions, and multiple gluon
emission can be seen in the matrix elements that have been extracted
from the $\psi(2S)$ and $\chi_c$ transverse momentum
distributions. See Ref.~\cite{Kramer:2001hh} for details.

Effects of corrections of higher order in $\alpha_s$ are a further
uncertainty in the fits to the data in Table~\ref{tab:me-2}. Such
corrections are known to be large in the case of charmonium decays. In
the case of charmonium production, a new channel for color-singlet
production, involving $t$-channel gluon exchange, first appears in order
$\alpha_s$ and could yield a large correction. Maltoni and Petrelli
\cite{Petrelli:1999rh} have found that real-gluon corrections to
color-singlet ${}^3S_1$ production give a large contribution.
Next-to-leading order (NLO) corrections in $\alpha_s$ for 
$\chi_{c0}$ and $\chi_{c2}$
production have been calculated \cite{Petrelli:1997ge}, as have
NLO corrections for the fragmentation process
\cite{Beneke:1995yb,Ma:1995ci,Braaten:2000pc}. Large corrections from
the resummation of logarithms of $p_T^2/m^2$ in the fragmentation of
partons into quarkonium have also been calculated
\cite{Cacciari:1994dr,Braaten:1994xb,Roy:1994ie,Sanchis-Lozano:1999um}.

Similar theoretical uncertainties arise in the extraction of the NRQCD
production matrix elements for the $\psi(2S)$ and $\chi_c$ states. The
statistical uncertainties are larger for $\psi(2S)$ and $\chi_c$
production than for $J/\psi$ production. We refer the reader to
Ref.~\cite{Kramer:2001hh} for some examples of the NRQCD matrix elements
that have been extracted for these states.

%%%%%%%%%%%%%%%%%%%%%%%%%%%%%%%%%%%%%%%%%%%%%%%%%%%%%%%%%%%%%%%%%%%%%%%%%%%%%%%%%%
\begin{table}[ht]
\begin{center}
\begin{tabular}{|c|c|} 
\hline \hline
$H$ & $F_H$ (in \%) \\ 
\hline
$J/\psi$        & $64 \pm 6$ \\ 
$\psi(2S)$      & $7 \pm 2$ to $15\pm 5$ \\ 
$\chi_c(1P)$    & $29.7 \pm 1.7(\hbox{stat.}) \pm 5.7(\hbox{sys.})$ \\
\hline \hline
\end{tabular}
\caption{The fractions $F_H$ of prompt $J/\psi$ mesons
that are produced by the decay of higher charmonium states $H$
and the fraction $F_{J/\psi}$ that are produced directly.}
\label{prodsec:Jpsifractions}
\end{center}
\end{table}
%%%%%%%%%%%%%%%%%%%%%%%%%%%%%%%%%%%%%%%%%%%%%%%%%%%%%%%%%%%%%%%%%%%%%%%%%%%%%%%%%%

The CDF collaboration has measured the fraction of prompt 
$J/\psi$'s that come from decays of $\psi(2S)$ and $\chi_c(1P)$ 
states and the fractions that are produced directly \cite{Abe:1997yz}.  
The CDF measurements were made for $J/\psi$'s 
with transverse momentum $p_T > 4$ GeV
and pseudo-rapidity $|\eta| < 0.6$.
The fractions, which are defined in Eqs.~(\ref{FJpsipsi2S}) 
and (\ref{prod:chifrac}), are given in Table~\ref{prodsec:Jpsifractions}.
The fraction of $J/\psi$'s that are directly produced is 
approximately constant over the range 5 GeV $< p_T <$ 15 GeV.
The fraction from decays of $\psi(2S)$ increases from
$(7 \pm 2)$\% at $p_T = 5$ GeV to $(15 \pm 5)$\% at $p_T = 15$ GeV.
The fraction from decays of $\chi_c(1P)$ seems to decrease
slowly over this range of $p_T$.
Such variations with $p_T$ are counter to the predictions of the
color-evaporation model.  

The CDF collaboration has also measured the ratio of the prompt $\chi_{c1}$
and $\chi_{c2}$ cross sections at the Tevatron \cite{Affolder:2001ij}.
The measured value of the ratio $R_{\chi_c}$ defined in 
Eq.~(\ref{prod:chirat}) is
%
\begin{equation}
R_{\chi_c} = 
1.04 \pm 0.29(\hbox{stat.}) \pm 0.12(\hbox{sys.}).
\label{Rchi12Tev}
\end{equation}
%
The $\chi_{c2}$ and $\chi_{c1}$ were observed through their radiative decays
into a $J/\psi$ and a photon, which were required to have transverse
momenta exceeding 4 GeV and 1 GeV, respectively. The color-evaporation
model predicts that this ratio should be 
close to the spin-counting ratio $3/5$, 
since the feeddown from the $\psi(2S)$ is small.
The NRQCD factorization fit to the prompt $\chi_c$ cross section in the
region $p_T> 5$ GeV implies a ratio of $0.9\pm 0.2$
\cite{maltoni-chi-ratio}. The CDF result slightly favors the
NRQCD factorization prediction.


%%%%%%%%%%%%%%%%%%%%%%%%%%%%%%%%%%%%%%%%%%%%%%%%%%%%%%%%%%%%%%%%%%%%%%%%%%%%%%%%%%%%%%%%%%%%%
\begin{figure}[htb]
  \centerline{\hbox{ \hspace{0.2cm}
    \includegraphics[width=8cm]{cdf_xsec_psi_new.eps}
    \hspace{0.3cm}
    \includegraphics[width=8cm]{cdf_bxsec_psi_new.eps}}}
 \caption{Differential inclusive cross section for $p\bar{ p}
\rightarrow J/\psi X$ (left).
Differential  cross section distribution of $J/\psi$ events from $b$-hadron
decay (right).
Both cross sections are plotted as a function of the transverse momentum $p_T$
of the $J/\psi$ and are integrated over the rapidity range $|y(J/\psi)|<0.6$. 
\label{CDF_xsec} }
\end{figure}
%%%%%%%%%%%%%%%%%%%%%%%%%%%%%%%%%%%%%%%%%%%%%%%%%%%%%%%%%%%%%%%%%%%%%%%%%%%%%%%%%%%%%%%%%%%%%

Charmonium production data from Tevatron Run~II have recently become
available. Using a 39.7~pb$^{-1}$ data sample from Run~II, the CDF
Collaboration has measured the inclusive cross section for $J/\psi$
production and subsequent decay into $\mu^+\mu^-$ \cite{CDF_LP03}. 
The inclusive cross section includes both prompt $J/\psi$'s 
and $J/\psi$'s from decays of $b$-hadrons.
The inclusive differential cross section as a function of $p_T$ for rapidity 
$|y|<0.6$ has been obtained down to zero transverse momentum and is 
shown in the left panel of Fig.~\ref{CDF_xsec}. The total integrated cross
section for inclusive $J/\psi$ production in $p\bar{p}$ interactions at
$\sqrt{s} = 1.96$~TeV is measured to be
\begin{equation}
\sigma[p\bar{p} \rightarrow J/\psi X, |y(J/\psi)|<0.6 ] \; 
= 4.08 \pm 0.02({\rm stat}) \pm 0.36(\rm {syst})~{\mu\rm b}.
\label{exeq1}
\end{equation}
These new measurements await comparison
with updated theoretical calculations in the low $p_T$ region.

Using a sample of 4.7~pb$^{-1}$ of Run II data, the D0 collaboration
has verified that the $J/\psi$ cross section is independent of the
rapidity of the $J/\psi$ for a rapidity range 0$~<|y|<~$2.  
This analysis has been performed for $p_T(J/\psi)
>5$~GeV and $p_T(J/\psi) > 8$~GeV~\cite{D0_LP03}.
The CDF and D0 collaborations have performed studies of forward 
 differential $J/\psi$ production cross sections 
in the pseudo-rapidity regions
2.1$~<|\eta (J/\psi)| <~$2.6 and 2.5$~\leq |\eta (J/\psi)| \leq~$3.7, 
respectively, using their Run I data  \cite{CDF_Iforw, D0_Iforw}. 

  Using 39.7~pb$^{-1}$ of the Run II data, the CDF Collaboration has also
measured the differential cross section as a function of $p_T$ and the
cross section integrated over $p_T$ for the production of $b$-hadrons
that decay in the channel $H_b \rightarrow J/\psi X$ \cite{CDF_LP03}.
The differential cross section multiplied by the branching fraction for
$J/\psi \to \mu^+ \mu^-$ is shown in the right panel of
Fig.~\ref{CDF_xsec}. A recent QCD prediction that is based on a fixed
order (FO) calculation plus a resummation of next-to-leading order logs
(NLL) \cite{Cacciari:2004} is overlaid. The cross section integrated
over $p_T$ was found to be
\begin{equation}
\sigma[p\bar{p} \rightarrow H_b X, p_T(J/\psi)> 1.25~ {\rm GeV}, 
|y(J/\psi)|<0.6] \;
= 28.4 \pm 0.4({\rm stat}) ^{+4.0}_{-3.8} (\rm {syst})~{\mu\rm b}.
\label{exeq2}
\end{equation}
This measurement can be used to extract the total inclusive $b$-hadron
cross section.


\subsection{Bottomonium cross sections}
\label{prodsec:tevatronbottom}

%%%%%%%%%%%%%%%%%%%%%%%%%%%%%%%%%%%%%%%%%%%%%%%%%%%%%%%%%%%%%%%%%%%%%%%%%%%%%%%%%%%%%%%%%%%%%
\begin{figure}[htb]
\begin{center}
\epsfig{figure=kramer-fig7.eps,width=10cm}
\caption{Inclusive $\Upsilon(1S)$ cross section at the Tevatron as a
function of $p_T$. The data points are the CDF measurements
\cite{Abe:1997jz}. The solid curve is the NRQCD factorization fit, and
the other curves are individual contributions to the
NRQCD factorization fit.  From
Ref.~\cite{Kramer:2001hh,LHC-workshop}.}
\label{fig-tevatron_ups1S}
\end{center}
\end{figure}
%%%%%%%%%%%%%%%%%%%%%%%%%%%%%%%%%%%%%%%%%%%%%%%%%%%%%%%%%%%%%%%%%%%%%%%%%%%%%%%%%%%%%%%%%%%%%

Using Run I data, the CDF Collaboration has reported inclusive
production cross sections for the $\Upsilon(1S)$, $\Upsilon(2S)$ and
$\Upsilon(3S)$ states in the region 0 $< p_T < 20$~GeV
\cite{Acosta:2001gv}.  The rates of inclusive
production of the $\Upsilon(1S)$, $\Upsilon(2S)$ and $\Upsilon(3S)$
states for $p_T > 4$~GeV were found to be higher than the rates
predicted by CSM calculations by a factor of about five. 
Inclusion of color-octet production mechanisms within the 
NRQCD framework can account for the observed cross sections for 
$p_T > 8$~GeV~\cite{Cho:1995vh,Cho:1995ce,LHC-workshop,Braaten:2000cm}, 
as is shown for $\Upsilon(1S)$ production in Fig.~\ref{fig-tevatron_ups1S}. 
An accurate description of the $\Upsilon$ cross section in the low-$p_T$
region requires NLO corrections and a resummation of multiple gluon 
radiation. A fit to the CDF data using a parton shower Monte Carlo 
to model the effects of multiple gluon emission has given much 
smaller values of the color-octet matrix elements that are compatible 
with zero~\cite{Domenech:2000ri}.

%%%%%%%%%%%%%%%%%%%%%%%%%%%%%%%%%%%%%%%%%%%%%%%%%%%%%%%%%%%%%%%%%%%%%%%%%%%%%%%%%%%%%%%%%%%%%
\begin{figure}[htb]
\setlength{\epsfxsize=0.95\textwidth}
\setlength{\epsfysize=0.5\textheight}
\centerline{\epsffile{upscdfpt3.ps}}
\caption{Differential cross sections for $\Upsilon(1S)$ (top left),
$\Upsilon(2S)$ (top right), and $\Upsilon(3S)$ (bottom) at the Tevatron
as a function of $p_T$. The data points are the CDF measurements
\cite{Acosta:2001gv}. The solid curves are the CEM predictions at NLO
with $\langle k_T^2\rangle = 3.0$ GeV$^2$, using the first bottomonium
parameter set in Tables~\ref{prodsec:qqbparams}. The dashed curves are
multiplied by a $K$-factor of 1.4. }
\label{upscdfptdep}
\end{figure}
%%%%%%%%%%%%%%%%%%%%%%%%%%%%%%%%%%%%%%%%%%%%%%%%%%%%%%%%%%%%%%%%%%%%%%%%%%%%%%%%%%%%%%%%%%%%%

The normalization and the shape of the bottomonium cross sections at the
Tevatron can also be described reasonably well by the color-evaporation
model (CEM).  The CEM predictions are compared
with the CDF data for $\Upsilon(1S)$, $\Upsilon(2S)$, and
$\Upsilon(3S)$ in Fig.~\ref{upscdfptdep}. Most of the relevant
parameters can be fixed completely by fitting data from $pN$
collisions and by using measured branching fractions for bottomonium
decays.  The predictions of the CEM at NLO that are shown in
Fig.~\ref{upscdfptdep} have been calculated using the NLO parameter
sets that are described in Section~\ref{prodsec:fixed-targetCEM}.  The
predicted cross sections for $\Upsilon(1S)$ and $\Upsilon(3S)$
production are a little below the data; the normalizations can be
improved by multiplying the cross sections by a K-factor of 1.4.  The
shapes have been brought into good agreement with the data by
including $k_T$ smearing, with $\langle k_T^2\rangle = 3.0$
GeV$^2$.  This value of $\langle k_T^2\rangle$ is a little larger
than the value $\langle k_T^2\rangle = 2.5$ GeV$^2$ that gives the
best fit to the charmonium cross sections.  

%%%%%%%%%%%%%%%%%%%%%%%%%%%%%%%%%%%%%%%%%%%%%%%%%%%%%%%%%%%%%%%%%%%%%%%%%%%%%%%%%%%%%%%%%%%%%
\begin{figure}
\centerline{\includegraphics[width=5.4cm]{bqw-up1.eps}
\hfil       \includegraphics[width=5.4cm]{bqw-up2.eps}
\hfil       \includegraphics[width=5.4cm]{bqw-up3.eps}}
\caption{Calculated differential cross sections times leptonic branching
fractions $B$, evaluated at $y = 0$, as functions of transverse momentum
for hadronic production of (a) $\Upsilon(1S)$, (b) $\Upsilon(2S)$, and
(c) $\Upsilon(3S)$ \cite{Berger:2004cc}, along with CDF
data~\cite{Abe:1995an,Acosta:2001gv} at $\sqrt S = 1.8$~TeV.  The solid lines
show the result of the full calculation. The 1995 CDF cross sections are
multiplied a factor $0.88$.}
\label{fig:berger-qiu-wang}
\end{figure}
%%%%%%%%%%%%%%%%%%%%%%%%%%%%%%%%%%%%%%%%%%%%%%%%%%%%%%%%%%%%%%%%%%%%%%%%%%%%%%%%%%%%%%%%%%%%%

A recent calculation of the production cross sections for the
$\Upsilon(1S)$, $\Upsilon(2S)$, and $\Upsilon(3S)$ at the Tevatron
combines a resummation of logarithms of $M_{\Upsilon}^2/p_T^2$ with a
calculation at leading order in $\alpha_s$ in what is, in essence, the
color-evaporation model \cite{Berger:2004cc}. 
The resummation of the effects of multiple gluon 
emission in the CEM has some simplifications that do not occur in 
the NRQCD factorization approach.
The results of the calculation of Ref.~\cite{Berger:2004cc} are shown, 
along with CDF data, in Fig.~\ref{fig:berger-qiu-wang}. The resummation 
of logarithms of $M_{\Upsilon}^2/p_T^2$ allows the calculation to 
reproduce the shape of the data at small $p_T$. The normalizations have 
been adjusted to obtain the best fit to the data.
the best fit to the data.

%%%%%%%%%%%%%%%%%%%%%%%%%%%%%%%%%%%%%%%%%%%%%%%%%%%%%%%%%%%%%%%%%%%%%%%%%%%%%%%%%%
\begin{table}[ht]
\begin{center}
\begin{tabular}{|c|c|} 
\hline \hline
$H$ & $F_H$ (in \%) \\ 
\hline
$\Upsilon(1S)$  & $50.9 \pm 8.2(\hbox{stat.}) \pm 9.0(\hbox{sys.})$ \\ 
$\Upsilon(2S)$  & $10.7^{+7.7}_{-4.8}$ \\ 
$\Upsilon(3S)$  & $ 0.8^{+0.6}_{-0.4}$ \\ 
$\chi_b(1P)$    & $27.1 \pm 6.9(\hbox{stat.}) \pm 4.4(\hbox{sys.})$ \\
$\chi_b(2P)$    & $10.5 \pm 4.4(\hbox{stat.}) \pm 1.4(\hbox{sys.})$ \\
$\chi_b(3P)$    & $< 6$ \\
\hline \hline
\end{tabular}
\caption{The fractions $F_H$ of $\Upsilon(1S)$ mesons that are produced 
by the decay of a higher bottomonium state $H$ and the fraction 
$F_{\Upsilon(1S)}$ that are produced directly.}
\label{prodsec:Upsfractions}
\end{center}
\end{table}
%%%%%%%%%%%%%%%%%%%%%%%%%%%%%%%%%%%%%%%%%%%%%%%%%%%%%%%%%%%%%%%%%%%%%%%%%%%%%%%%%%

The CDF Collaboration has also reported the fractions of $\Upsilon(1S)$
mesons, for $p_T> 8$~GeV, that come from decays of $\chi_b(1P)$, 
$\chi_b(2P)$, $\chi_b(3P)$, $\Upsilon(2S)$, and $\Upsilon(3S)$ and the 
fraction that originate from direct production \cite{Affolder:2000nn}. 
The fractions from decays of $\Upsilon(nS)$ and for $\chi_b(nP)$ 
are defined by
%
\begin{eqnarray}
F_{\Upsilon(nS)} &=& 
{\rm Br}[\Upsilon(nS) \to \Upsilon(1S) + X] \; 
{\sigma[\Upsilon(nS)] \over \sigma[\Upsilon(1S)]},
\label{FUpsUps}
\\
F_{\chi_b(nP)} &=& \sum_{J=0}^3
{\rm Br}[\chi_{bJ}(nP) \to \Upsilon(1S) + X] \;
{\sigma[\chi_{bJ}(nP)] \over \sigma[\Upsilon(1S)]}.
\label{FUpschi}
\end{eqnarray}
%
The fraction of $\Upsilon(1S)$'s that are produced directly
can be denoted by $F_{\Upsilon(1S)}$.
The fractions are given in Table~\ref{prodsec:Upsfractions}.


\subsection{Polarization} 
\label{prodsec:tevatronpol}

%%%%%%%%%%%%%%%%%%%%%%%%%%%%%%%%%%%%%%%%%%%%%%%%%%%%%%%%%%%%%%%%%%%%%%%%%%%%%%%%%%%%%%%%%%%%%
\begin{figure}[htb]
\begin{center}
$ \begin{array}{cc}
\includegraphics[width=3in]{psi-pol.eps}
&
\includegraphics[width=3in]{psip-pol.eps} \\
\end{array} $
\caption{
Polarization variable $\alpha$ for prompt $J/\psi$ (left) and for
prompt $\psi(2S)$ (right) at the Tevatron as a function of $p_T$.  The
data points are the CDF measurements from Run I
\cite{Affolder:2000nn}. In the left panel (prompt $J/\psi$), the band
is the NRQCD factorization prediction of Ref.~\cite{Braaten:1999qk},
and the other curves are the values of $\alpha$ for individual
components of the prompt $J/\psi$ signal. In the right panel (prompt
$\psi(2S)$), the bands are various NRQCD factorization
predictions~\cite{Beneke:1996yw,Leibovich:1996pa,Braaten:1999qk}.}
\label{fig-pol}
\end{center}
\end{figure}
%%%%%%%%%%%%%%%%%%%%%%%%%%%%%%%%%%%%%%%%%%%%%%%%%%%%%%%%%%%%%%%%%%%%%%%%%%%%%%%%%%%%%%%%%%%%%

The polarization of the quarkonium contains important information 
about the production mechanism. The polarization variable $\alpha$ 
for a $1^{--}$ state, such as $J/\psi$, $\psi(2S)$, or $\Upsilon(1S)$, 
is defined by Eq.~(\ref{prod:alphadef}), where the angle $\theta$
is measured with respect to some polarization axis. 
At a hadron collider, a convenient choice of the polarization axis
is the direction of the boost vector from the quarkonium rest frame 
to the center-of-momentum frame of the colliding hadrons.  

The NRQCD factorization approach gives a simple prediction for the 
polarization variable $\alpha$ at very large transverse momentum.
The production of a quarkonium with $p_T$ that
is much larger than the quarkonium mass is dominated by gluon
fragmentation---a process in which the quarkonium is formed in the
hadronization of a gluon that is created with even larger transverse
momentum. The NRQCD factorization approach predicts that the dominant
gluon-fragmentation process is gluon fragmentation into a $Q\bar Q$ pair
in a color-octet ${}^3S_1$ state.  The fragmentation probability for
this process is of order $\alpha_s$, while the fragmentation
probabilities for all other processes are of order $\alpha_s^2$ or
higher. The NRQCD matrix element for this fragmentation process is
$\langle{\cal O}^H_8({}^3S_1)\rangle$. At large $p_T$, the fragmenting
gluon is nearly on its mass shell, and, so, is transversely polarized.
Furthermore, the velocity-scaling rules predict that the color-octet
$Q\bar Q$ state retains the transverse polarization as it evolves into
an $S$-wave quarkonium state \cite{Cho:1994ih}, up to corrections of
relative order $v^2$. Radiative corrections and color-singlet production
dilute the quarkonium polarization somewhat
\cite{Beneke:1995yb,Beneke:1996yw}. In the case of $J/\psi$ production,
feeddown from higher quarkonium states is also important
\cite{Braaten:1999qk}. Feeddown from $\chi_c$ states is about 30\% of
the $J/\psi$ sample and dilutes the polarization. Feeddown from the
$\psi(2S)$ is about 10\% of the $J/\psi$ sample and is largely
transversely polarized. Despite these various diluting effects, a
substantial polarization is expected at large $p_T$, and its detection
would be a ``smoking gun'' for the presence of the color-octet production
mechanism. In contrast, the color-evaporation model predicts zero
quarkonium polarization.

The CDF measurement of the $J/\psi$ polarization as a function of
$p_T$ \cite{Affolder:2000nn} is shown in the left panel of
Fig.~\ref{fig-pol}, along with the NRQCD factorization prediction
\cite{Braaten:1999qk}.  The observed $J/\psi$ polarization is in
agreement with the prediction, except for the highest $p_T$
bin. However, the prediction of increasing polarization with
increasing $p_T$ is not in evidence. The CDF data
\cite{Affolder:2000nn} and the NRQCD factorization prediction 
\cite{Beneke:1996yw,Leibovich:1996pa,Braaten:1999qk} for $\psi(2S)$
polarization are shown in the right panel of Fig.~\ref{fig-pol}.  The
theoretical analysis of $\psi(2S)$ polarization is simpler than for
the $J/\psi$, since feeddown does not play a r\^ole. However, the
experimental statistics are not as good for the $\psi(2S)$ as for
$J/\psi$. Again, the expectation of increasing polarization with
increasing $p_T$ is not confirmed.

Because the polarization depends on ratios of matrix elements, some of
the theoretical uncertainties are reduced compared with those in the
production cross section. The polarization is probably not strongly
affected by multiple gluon emission or $K$-factors. Uncertainties
from contributions of higher-order in $\alpha_s$ could conceivably
change the rates for the various spin states by a factor of two. 
Therefore, it is important to carry out the NLO calculation, but that
calculation is very difficult technically and is computing intensive. 
Order-$v^2$ corrections to parton fragmentation to quarkonium can be
quite large. Bodwin and Lee \cite{Bodwin:2003wh} have found that the $v^2$
corrections to gluon fragmentation to $J/\psi$ are about $+70\%$ for
the color-singlet channel and $-50\%$ for the color-octet channel.
The color-singlet correction shifts $\alpha$ down by about 10\% at the
largest $p_T$. Since the color-octet matrix element is fit to Tevatron
data, the $v^2$ correction merely changes the size of the matrix element
and has no immediate effect on the theoretical prediction. An additional
theoretical uncertainty comes from the presence of order-$v^2$ spin-flip
processes in the evolution of the $Q\bar Q$ pair into the quarkonium.
It could turn out that spin-flip contributions are large, either
because their velocity-scaling power laws happen to have large
coefficients or because, as has been suggested in
Refs.~\cite{Beneke:1997av,Brambilla:1999xf,Fleming:2000ib,
Sanchis-Lozano:2001rr,Brambilla:2002nu}, the
velocity scaling rules themselves need to be modified. Then spin-flip
contributions could significantly dilute the $J/\psi$ polarization.
Nevertheless, it is is difficult to see how there could not be
substantial polarization in $J/\psi$ or $\psi(2S)$ production for
$p_T>4m_c$.\footnote{It has been argued that re-scattering interactions
between the intermediate charm-quark pair and a co-moving color field
could yield unpolarized quarkonium~\cite{Marchal:2000wd,Maul:2001fw}.
The theoretical analysis of these effects, however, relies on several
simplifying assumptions, and further work is needed to establish the
existence of re-scattering corrections in charmonium hadroproduction at
large $p_T$.}

%%%%%%%%%%%%%%%%%%%%%%%%%%%%%%%%%%%%%%%%%%%%%%%%%%%%%%%%%%%%%%%%%%%%%%%%%%%%%%%%%%%%%%%%%%%%%
\begin{figure}[htb]
\begin{center}
\epsfig{figure=cdf_ups1s_pol_nrqcd.eps,width=10cm}
\caption{ Polarization variable $\alpha$ for inclusive $\Upsilon(1S)$
production at the Tevatron as a function of $p_T$. The data points
are the CDF measurements from Run I \cite{Acosta:2001gv}.
The theoretical band represents 
the NRQCD factorization prediction \cite{Braaten:2000gw}.}
\label{fig-upsilon-pol}
\end{center}
\end{figure}
%%%%%%%%%%%%%%%%%%%%%%%%%%%%%%%%%%%%%%%%%%%%%%%%%%%%%%%%%%%%%%%%%%%%%%%%%%%%%%%%%%%%%%%%%%%%%

The CDF data for $\Upsilon$ polarization is shown in 
Fig.~\ref{fig-upsilon-pol}, along with the NRQCD factorization prediction.
Averaging over a range of $p_T$, the CDF
Collaboration finds $\alpha=-0.06\pm 0.20$ for $1\hbox{
GeV}<p_T<20\hbox{ GeV}$ \cite{Cropp:1999ub,Papadimitriou:2001bb}, which
is consistent with the NRQCD factorization prediction
\cite{Braaten:2000gw}.
In comparison with the prediction for $J/\psi$ polarization, the
prediction for $\Upsilon$ polarization has smaller $v$-expansion
uncertainties. However, in the case of $\Upsilon$ production, the
fragmentation mechanism does not dominate until relatively large values
of $p_T$ are reached, and, hence, the transverse
polarization is predicted to be small for $p_T$ below about 10~GeV. 
Unfortunately, the current Tevatron data sets run out of statistics 
in the high-$p_T$ region.

\subsection{Prospects for the Tevatron Run II}

Run~II at the Tevatron will provide a substantial increase in luminosity
and will allow the collider experiments to determine the $J/\psi$,
$\psi(2S)$ and $\chi_c$ cross sections more precisely and at larger
values of $p_t$. An accurate measurement of the $J/\psi$ and $\psi(2S)$
polarization at large transverse momentum will be the most crucial test
of NRQCD factorization. In addition, improved data on the $J/\psi$ and
$\psi(2S)$ cross sections will help to reduce some of the ambiguities in
extracting the color-octet matrix elements.
 
With increased statistics it might be possible to access other
charmonium states such as the $\eta_c(nS)$ or the $h_c(nP)$.  Heavy-quark
spin symmetry provides approximate relations between the
nonperturbative matrix elements that describe spin-singlet and
spin-triplet states. The matrix elements for $\eta_c(nS)$ are related to
those for $\psi(nS)$, while the leading matrix elements for $h_c(nP)$
can be obtained from those for $\chi_c(nP)$. [See
Eqs.~(\ref{s1-symmetry}-\ref{p8-symmetry}).] Within NRQCD, the rates for
$\eta(nS)$ and $h(nP)$ production can thus be predicted unambiguously in
terms of the nonperturbative matrix elements that describe the $J/\psi$,
$\psi(2S)$ and $\chi_c$ production cross sections. A comparison of the
various charmonium production rates would therefore provide a stringent
test of NRQCD factorization and the heavy-quark spin symmetry.  The
cross sections for producing the $\eta_c$ and the $h_c$ at Run~II of the
Tevatron are large~\cite{Mathews:1998nk,Sridhar:1996vd}, but the
acceptances and efficiencies for observing the decay modes on which one
can trigger are, in general, small, and detailed experimental studies
are needed to quantify the prospects.  Other charmonium processes that
have been studied in the literature include the production of $D$-wave
states~\cite{Qiao:1997wb}, $J/\psi$ production in association with
photons~\cite{Kim:1997bb,Mathews:1999ye}, and double gluon fragmentation
to $J/\psi$ pairs~\cite{Barger:1996vx}.
 
The larger statistics expected at Run~II of the Tevatron will also allow
the collider experiments to improve the measurements of the bottomonium
cross sections. As yet undiscovered states, such as the $\eta_b(1S)$,
could be detected, for example, in the decay $\eta_b \to J/\psi +
J/\psi$~\cite{Braaten:2000cm} or in the decay $\eta_b \to D^* +
D^{(*)}$~\cite{Maltoni:2004hv}, and the associated production of
$\Upsilon$ and electroweak bosons might be
accessible~\cite{Braaten:1999th}. If sufficient statistics can be
accumulated, the onset of transverse $\Upsilon(nS)$ polarization may be
visible at $p_{T,\Upsilon}\gsim 15$~GeV.
 

\section{Quarkonium production in fixed-target experiments}
\label{prodsec:fixed-target}

\subsection{Cross sections}
\label{prodsec:fixed-targetxsec}

Several collaborations have made predictions for fixed-target quarkonium
production within the NRQCD factorization formalism
\cite{Beneke:1996tk,Tang:1996rm,Gupta:1996ut}. The predictions of
Ref.~\cite{Beneke:1996tk} for $J/\psi$ and $\psi(2S)$ production in $pN$
collisions are shown, along with the experimental data, 
in the left panels of 
Figs.~\ref{fig:fixed-target-psi} and \ref{fig:fixed-target-pri}. The
calculation is at leading-order in $\alpha_s$ and uses the standard
truncation in $v$ that is described in Section~\ref{prodsec:nrqcdfact}.
The data are from the compilation in Ref.~\cite{Schuler:1994hy}, with
additional results from
Refs.~\cite{Tzamarias:1990ij,Schub:1995pu,Alexopoulos:1995dt}. In the
case of $pN$ production of $J/\psi$, the data clearly require a
color-octet contribution, in addition to a color-singlet contribution.
In the case of $\psi(2S)$ production, it is less clear that a
color-octet contribution is essential. One should keep in mind that the
color-singlet contribution is quite uncertain, owing to uncertainties in
the values of $m_c$ and the renormalization scale \cite{Beneke:1997av}.
One can reduce these uncertainties by considering the ratio of the cross
sections for direct and inclusive $J/\psi$ production, which is
predicted to be approximately 0.6 in the NRQCD factorization approach and
approximately 0.2 in the color-singlet model \cite{Beneke:1997av}.
Clearly, experiment favors the NRQCD factorization prediction. However,
the prediction for the ratio depends on our knowledge of feed-down from
$\chi_c$ states, and, as we shall see, NRQCD factorization predictions
for $\chi_c$ production in fixed-target experiments
are not in good agreement with the data.

%%%%%%%%%%%%%%%%%%%%%%%%%%%%%%%%%%%%%%%%%%%%%%%%%%%%%%%%%%%%%%%%%%%%%%%%%%%%%%%%%%%%%%%%%%%
\begin{figure}[htb]
\begin{center}
$ \begin{array}{cc}
\includegraphics[width=7.5cm]{beneke-rothstein-psi.eps}
&
\includegraphics[width=7.5cm]{beneke-rothstein-psipi.eps}
\end{array} $
\caption{
Forward cross section ($x_F>0$) for $J/\psi$ production 
in $pN$ collisions (left) and $\pi N$ collisions (right).  
The curves are the CSM predictions for direct $J/\psi$ (dashed lines), 
the NRQCD factorization predictions for direct $J/\psi$ with 
$M_7^{J/\psi}=3.0\times 10^{-2}$~GeV$^3$ (dotted lines), 
and the inclusive cross sections for $J/\psi$ including 
radiative feed-down from $\chi_{cJ}$ and $\psi(2S)$ 
(solid lines).  From Ref.~\cite{Beneke:1996tk}.
}
\label{fig:fixed-target-psi}
\end{center}
\end{figure}
%%%%%%%%%%%%%%%%%%%%%%%%%%%%%%%%%%%%%%%%%%%%%%%%%%%%%%%%%%%%%%%%%%%%%%%%%%%%%%%%%%%%%%%%%%%

%%%%%%%%%%%%%%%%%%%%%%%%%%%%%%%%%%%%%%%%%%%%%%%%%%%%%%%%%%%%%%%%%%%%%%%%%%%%%%%%%%%%%%%%%%%
\begin{figure}[htb]
\begin{center}
$ \begin{array}{cc}
\includegraphics[width=7.5cm]{beneke-rothstein-pri.eps} 
&
\includegraphics[width=7.5cm]{beneke-rothstein-pripi.eps}
\end{array} $
\caption{
Forward cross section ($x_F>0$) for $\psi(2S)$ production 
in $pN$ collisions (left) and $\pi N$ collisions (right).  
The curves are the CSM predictions (dotted lines) 
and the NRQCD factorization predictions with 
$M_7^{\psi(2S)}=5.2\times 10^{-3}$~GeV$^3$ (solid lines).  
From Ref.~\cite{Beneke:1996tk}.
}
\label{fig:fixed-target-pri}
\end{center}
\end{figure}
%%%%%%%%%%%%%%%%%%%%%%%%%%%%%%%%%%%%%%%%%%%%%%%%%%%%%%%%%%%%%%%%%%%%%%%%%%%%%%%%%%%%%%%%%%%

In fixed-target production of $J/\psi$ and $\psi(2S)$ at
leading order in $\alpha_s$ (LO), the
relevant production matrix elements are 
$\langle{\cal O}^H_8({}^3S_1)\rangle$, $\langle{\cal O}^H_8({}^1S_0)\rangle$, 
and $\langle{\cal O}^H_8({}^3P_0)\rangle$, 
but the cross section is sensitive only to the linear combination 
$M^H_k$ defined in (\ref{prod:lincomb}) with $k\approx 7$. 
The fits of the LO predictions for $J/\psi$ and $\psi(2S)$
production in $pN$ collisions \cite{Beneke:1996tk} yield
$M_7^{J/\psi}=3.0\times 10^{-2}\hbox{~GeV}^3$ and
$M_7^{\psi(2S)}=5.2\times 10^{-3}\hbox{~GeV}^3$.
The corrections at next-to-leading order in $\alpha_s$ (NLO)
give a large $K$-factor in the
color-octet contributions \cite{Petrelli:1997ge}. A fit to the data
using the NLO result for the color-octet
contributions gives $M_{6.4}^{J/\psi}=1.8\times 10^{-2}\hbox{~GeV}^3$
and $M_{6.4}^{\psi(2S)}=2.6\times 10^{-3}\hbox{~GeV}^3$ 
\cite{Maltoni:2000km}.  The NLO value of $M_{6.4}^{J/\psi}$
is about a factor $2$ smaller than the LO value of $M_7^{J/\psi}$. Note
that the NLO fit uses CTEQ4M \cite{Lai:1996mg} parton distributions,
while the  LO fit uses the CTEQ3L \cite{Lai:1994bb} parton
distributions. The LO result for $M^{J/\psi}$ is somewhat smaller than
the LO result from the Tevatron, and the NLO result for $M^{J/\psi}$ is
somewhat larger than the parton-shower result from the Tevatron.
However, given the large uncertainties in these quantities, the
agreement is reasonable. It should also be remembered that the Tevatron
cross sections are sensitive to $M_k^H$ with $k \approx 3$ rather than
$k \approx 7$, and, so, comparisons are somewhat uncertain. Attempts to
constrain this uncertainty are hampered by the fact that the
$\overline{\rm MS}$ matrix elements need not be positive. One can also
question whether hard-scattering factorization holds for the total cross
section, which is dominated by small $p_T$-contributions. Furthermore,
kinematic corrections from the difference between $2m$ and the
quarkonium mass may be large.

The predictions of Ref.~\cite{Beneke:1996tk} for $J/\psi$ and
$\psi(2S)$ production in $\pi N$ collisions are shown, along with the
experimental data, in the right panels of 
Figs.~\ref{fig:fixed-target-psi} and \ref{fig:fixed-target-pri}. 
% in Figs.~\ref{fig:fixed-target-psipi} and \ref{fig:fixed-target-pripi}. 
The calculation is at leading-order in $\alpha_s$ and uses the
standard truncation in $v$ that is described in
Section~\ref{prodsec:nrqcdfact}. Again, the data are from the compilation
in Ref.~\cite{Schuler:1994hy}, with additional results from
Refs.~\cite{Tzamarias:1990ij,Schub:1995pu,Alexopoulos:1995dt}. In the
NRQCD predictions in Figs.~\ref{fig:fixed-target-psi} and
\ref{fig:fixed-target-pri}, the values of $M_7$ that are used are the
ones that were obtained from the fits to the $pN$ production data. The
$\pi N$ production data clearly show an excess over these predictions
that cannot be accounted for by the color-octet contributions. This
discrepancy has been discussed extensively in
Ref.~\cite{Schuler:1994hy}, and it may reflect our lack of knowledge
of the gluon distribution in the pion or the presence of different
higher-twist effects in the proton and the pion. Such higher-twist
effects are not accounted for in the standard NRQCD factorization
formulas, which are based on leading-twist hard-scattering
factorization.

Some of the largest uncertainties in the predictions cancel out
if we consider ratios of cross sections.  
The uncertainties in the NRQCD factorization predictions 
can still be very large. They arise
from uncertainties in the color-octet matrix elements, uncalculated
corrections of higher order in $v$ and $\alpha_s$, and uncertainties
from the choices of renormalization and factorization scales. In
addition, one can question whether hard-scattering factorization holds
for the cross section integrated over $p_T$.

%%%%%%%%%%%%%%%%%%%%%%%%%%%%%%%%%%%%%%%%%%%%%%%%%%%%%%%%%%%%%%%%%%%%%%%%%%%%%%%%%%%%%%%%%%%
\begin{table}[ht]
\addtolength{\arraycolsep}{0.2cm}
\renewcommand{\arraystretch}{1.25} 
\begin{center}
\begin{tabular}{|c|ccc|} 
\hline
\hline
\mbox{Experiment} & \mbox{beam/target} & $\sqrt{s}/\mbox{GeV}$ & 
$R_\psi$ \\
\hline 
\mbox{E537}~\cite{Tzamarias:1990ij}& $\bar p\mbox{W}$ & $15.3$ 
& $0.185\pm 0.0925$\\ 
\mbox{E705}~\cite{Arenton:pw}& $p\mbox{Li}$ & $23.7$ & $0.14\pm 0.02 \pm 
0.004\pm 0.02$\\ 
\mbox{E705}~\cite{Arenton:pw}& $\bar p\mbox{Li}$ & $23.7$ & $0.25\pm 
0.22 \pm 0.007\pm 0.04$\\ 
\mbox{E771}~\cite{Alexopoulos:1995dt}& $p\mbox{Si}$ & $38.8$
& $0.14\pm 0.02$ \\ 
\mbox{HERA-B}~\cite{Spengler:2004gr} & p\mbox{(C, W)} & $41.5$ 
&$0.13\pm 0.02 $\\ 
\hline
\mbox{E537}~\cite{Tzamarias:1990ij}& $\pi^-\mbox{W}$ & $15.3$ 
& $0.2405\pm 0.0650$\\ 
\mbox{E673}~\cite{Hahn:tz}    & $\pi\mbox{Be}$ &$20.6$&$0.20\pm 0.09$\\ 
\mbox{E705}~\cite{Arenton:pw} & $\pi^+\mbox{Li}$ &$23.7$ &
$0.14\pm 0.02\pm 0.004\pm 0.02$\\ 
\mbox{E705}~\cite{Arenton:pw} & $\pi^-\mbox{Li}$ &$23.7$ &
$0.12\pm 0.03\pm 0.03\pm 0.02$\\ 
\mbox{E672/706}~\cite{Gribushin:1995rt} & $\pi^-\mbox{Be}$ &$31.1$
&$0.15\pm 0.03\pm 0.02$\\ 
\hline
\hline 
\end{tabular}
\end{center}
\caption{\label{tab:psi} 
Experimental results for the ratio $R_\psi$
of the inclusive cross sections for $\psi(2S)$ and $J/\psi$ 
production.}
\end{table} 
%%%%%%%%%%%%%%%%%%%%%%%%%%%%%%%%%%%%%%%%%%%%%%%%%%%%%%%%%%%%%%%%%%%%%%%%%%%%%%%%%%%%%%%%%%%

The $\psi(2S)$-to-$J/\psi$ ratio $R_\psi$ is defined in Eq.~(\ref{FJpsipsi2S}).
The experimental results for $R_\psi$ from fixed-target experiments
are compiled in Table~\ref{tab:psi}.
The result from experiment E673 is obtained
by dividing the observed fraction of $J/\psi$'s from decays of $\psi(2S)$ 
by the branching fraction for $\psi(2S) \to J/\psi X$
given by the Particle Data Group \cite{Eidelman:2004wy}.
The result from experiment E771
is obtained by dividing the observed ratio of the 
products of the cross sections and the branching fractions into 
$\mu^+ \mu^-$ by the ratio of the branching fractions into $\mu^+ \mu^-$
given by the Particle Data Group \cite{Eidelman:2004wy}.
The NRQCD factorization approach gives the values 
$R_\psi=0.16$ for both $pN$ collisions and $\pi^-N$ collisions
\cite{Beneke:1996tk}.  The color-singlet model gives 
$R_\psi=0.14$ for $pN$ collisions and 
$R_\psi=0.16$ for $\pi^-N$ collisions 
\cite{Beneke:1996tk}. In the color-evaporation model, 
this ratio is simply an input. Thus the ratio $R_\psi$
is not able to discriminate between any of these approaches.

%%%%%%%%%%%%%%%%%%%%%%%%%%%%%%%%%%%%%%%%%%%%%%%%%%%%%%%%%%%%%%%%%%%%%%%%%%%%%%%%%%%%%%%%%%%
\begin{table}[ht]
\addtolength{\arraycolsep}{0.2cm}
\renewcommand{\arraystretch}{1.25} 
\begin{center}
\begin{tabular}{|c|cccc|} 
\hline
\hline
\mbox{Experiment} & \mbox{beam/target} & $\sqrt{s}/\mbox{GeV}$ & 
$F_{\chi_c}$ &$R_{\chi_c}$ \\
\hline 
\mbox{E673}~\cite{Bauer:yf} & $p\mbox{Be}$ & $19.4/21.7$ &$0.47\pm 0.23$ 
&$0.24\pm 0.28$ \\ 
\mbox{E705}~\cite{Antoniazzi:1993yf}& $p\mbox{Li}$ & $23.7$ & ---
&$0.08_{-0.15}^{+0.25}$ \\ 
\mbox{E705}~\cite{Arenton:pw}& $p\mbox{Li}$ & $23.7$ & $0.30\pm 0.04$
&--- \\ 
\mbox{E771}~\cite{Alexopoulos:1999wp}& $p\mbox{Si}$ & $38.8$ &---
& $0.53\pm 0.20\pm 0.07$ \\ 
\mbox{HERA-B}~\cite{Spengler:2004gr} & p\mbox{(C, W)} & $41.5$ 
&$0.32\pm 0.06\pm 0.04 $&--- \\ 
\hline
\mbox{WA11}~\cite{Lemoigne:1982jc}& $\pi\mbox{Be}$ & $18.6$ 
&$0.305\pm 0.050$ & $0.68\pm 0.28$ \\ 
\mbox{E673}~\cite{Bauer:yf} & $\pi\mbox{Be}$ &$18.9$ &$0.31\pm 0.10$
& $0.96\pm 0.64 $\\ 
\mbox{E673}~\cite{Hahn:tz}    & $\pi\mbox{Be}$ &$20.6$&$0.37\pm 0.09$
& $0.9\pm 0.4$\\ 
\mbox{E705}~\cite{Antoniazzi:1993yf} & $\pi\mbox{Li}$ &$23.7$ &--- &
$0.52_{-0.27}^{+0.57}$ \\ 
\mbox{E705}~\cite{Arenton:pw} & $\pi^+\mbox{Li}$ &$23.7$ &
$0.40\pm 0.04$ &--- \\ 
\mbox{E705}~\cite{Arenton:pw} & $\pi^-\mbox{Li}$ &$23.7$ &
$0.37\pm 0.03$ &--- \\ 
\mbox{E672/706}~\cite{Koreshev:1996wd} & $\pi^-\mbox{Be}$ &$31.1$
&$0.443\pm 0.041\pm 0.035$ & $0.57\pm 0.18\pm 0.06$ \\ 
\hline
\hline 
\end{tabular}
\end{center}
\caption{\label{tab:chi} Experimental results for the 
fraction of $J/\psi$'s from $\chi_c$ decay, $F_{\chi_c}$,
and the $\chi_{c1}$-to-$\chi_{c2}$ ratio, $R_{\chi_c}$. In view of the
experimental uncertainties, no attempt has been made to rescale older
measurements to account for the latest $\chi_c$ branching fractions.
Modified version of a table from Ref.~\cite{Beneke:1997av}.}
\end{table} 
%%%%%%%%%%%%%%%%%%%%%%%%%%%%%%%%%%%%%%%%%%%%%%%%%%%%%%%%%%%%%%%%%%%%%%%%%%%%%%%%%%%%%%%%%%%

The fraction $F_{\chi_c}$ of $J/\psi$'s that come from $\chi_c$ decays
is defined in Eq.~(\ref{prod:chifrac}).
The experimental results for $F_{\chi_c}$ from fixed-target experiments
are compiled in Table~\ref{tab:chi}.
The NRQCD factorization approach gives the values 
$F_{\chi_c}=0.27$ for $pN$ collisions 
and $F_{\chi_c}=0.28$ for $\pi^-N$ collisions
\cite{Beneke:1996tk}. The color-singlet model gives 
$F_{\chi_c}=0.68$ for $pN$ collisions and 
$F_{\chi_c}=0.66$ for $\pi^-N$ collisions 
\cite{Beneke:1996tk}. In the color-evaporation model, 
this ratio is simply an input. 
Clearly, the experimental results favor the NRQCD factorization
approach over the color-singlet model.  
The most precise results from $pN$ fixed target experiments are 
compatible with the Tevatron result in Table~\ref{prodsec:Jpsifractions}.
The most precise results from $\pi N$ fixed target experiments
are somewhat larger.


The $\chi_{c1}$-to-$\chi_{c2}$ ratio $R_{\chi_c}$ 
is defined in Eq.~(\ref{prod:chirat}).
There are substantial variations among the NRQCD factorization
predictions for $R_{\chi_c}$ in fixed-target
experiments. Beneke and Rothstein \cite{Beneke:1996tk} give the values
$R_{\chi_c}=0.07$ for $pN$ collisions and $R_{\chi_c}=0.05$ for $\pi^-N$
collisions. Their calculation is carried out at leading order in
$\alpha_s$ and uses the standard truncation in $v$ that is described in
Section~\ref{prodsec:nrqcdfact}. Beneke and Rothstein \cite{Beneke:1996tk}
suggest that corrections to hard-scattering factorization may be large.
Beneke \cite{Beneke:1997av} gives the estimate $R_{\chi_c}\approx 0.3$
for both $pN$ and $\pi N$ collisions. This estimate is based on the
assumption that the ${}^3P_2$ and ${}^3P_0$ color-octet matrix elements
dominate the $\chi_{c1}$ production. It is consistent with the
estimate in Ref.~\cite{Gupta:1997me}, once that estimate is modified to
take into account the dominant color-singlet channel in $\chi_{c2}$
production \cite{Beneke:1997av}. Maltoni \cite{Maltoni:2000km} gives
central values of $R_{\chi_c}$ for $pN$ collisions that range from
$R_{\chi_c}=0.04$ to $R_{\chi_c}=0.1$ as the beam energy ranges from
200~GeV to 800~GeV. Maltoni's calculation takes into account matrix
elements at leading order in $v$, but contains corrections of
next-to-leading order in $\alpha_s$. His calculation displays a very
large dependence on the renormalization scale. In summary, the existing
predictions for $R_\chi$ based on NRQCD factorization are in the range
0.04--0.3 for both $pN$ and $\pi N$ collisions. The color-singlet model
predicts that $R_{\chi_c}\approx 0.05\hbox{--}0.07$ for both $pN$
and $\pi N$ collisions \cite{Beneke:1996tk,Beneke:1997av}. The
color-evaporation model predicts that $R_{\chi_c} \simeq 3/5$
\cite{Amundson:1995em,Amundson:1996qr}.

The experimental results for $R_{\chi_c}$ are compiled in
Table~\ref{tab:chi}. As can be seen, the data are somewhat inconsistent
with each other. 
The results from the most precise experiments are significantly smaller 
than the Tevatron result in Eq.~(\ref{Rchi12Tev}).
There seems to be a trend toward larger values of
$R_{\chi_c}$ in $\pi N$ experiments than in $pN$ experiments. Such a
dependence on the beam type is contrary to the predictions of the
color-evaporation model. It also would not be expected in the NRQCD
factorization approach, unless there is an unusual enhancement in
the $q\bar q$ production channel \cite{Beneke:1997av}. Both the $pN$ and
$\pi N$ data yield results that are significantly larger than the
predictions of the color-singlet model. The $pN$ experiments seem to
favor the NRQCD factorization predictions, while the $\pi N$ experiments
seem to favor the color-evaporation prediction. However, in light of the
large theoretical and experimental uncertainties, no firm conclusions
can be drawn.


\subsection{Polarization}
\label{prodsec:fixed-targetpol}

%%%%%%%%%%%%%%%%%%%%%%%%%%%%%%%%%%%%%%%%%%%%%%%%%%%%%%%%%%%%%%%%%%%%%%%%%%%%%%%%%%%%%%%%%%%
\begin{table}[ht]
\addtolength{\arraycolsep}{0.2cm}
\renewcommand{\arraystretch}{1.25} 
\begin{center}
\begin{tabular}{|c|ccc|} 
\hline
\hline
\mbox{Experiment} & \mbox{beam/target} & \mbox{Beam Energy}/\mbox{GeV}
&$\alpha$  \\ 
\hline \mbox{E537}~\cite{Tzamarias:1990ij} & $(\pi, p)$\mbox{(Be, Cu, W)} 
& $125$ &$0.024\mbox{--}0.032$\\
\mbox{E672/706}~\cite{Gribushin:1999ha} & p\mbox{Be} & $530$ 
& $0.01\pm 0.15$ \\
\mbox{E672/706}~\cite{Gribushin:1999ha}& p\mbox{Be} & $800$ 
& $-0.11\pm 0.15$ \\
\mbox{E771}~\cite{Introzzi:yi}& p\mbox{Si} & $800$ & $-0.09\pm 0.12$ \\
\mbox{E866}~\cite{Chang:2003rz}& p\mbox{Cu} & $800$ & $0.069\pm 0.08$\\
\mbox{HERA-B}~\cite{Spengler:2004gr} & p\mbox{(C, W)} & $920$ 
& $(-0.5,\, +0.1)\pm 0.1$ \\ 
\hline
\hline 
\end{tabular} 
\end{center}
\caption{\label{tab:fixed-target-pol}
Experimental results for the polarization variable $\alpha$ in $J/\psi$
production. Modified version of a table from Ref.~\cite{Spengler:2004gr}.}
\end{table}
%%%%%%%%%%%%%%%%%%%%%%%%%%%%%%%%%%%%%%%%%%%%%%%%%%%%%%%%%%%%%%%%%%%%%%%%%%%%%%%%%%%%%%%%%%%

The polarization variable $\alpha$ for $J/\psi$ production is
defined  by the angular distribution in Eq.~(\ref{prod:alphadef}).  
In fixed-target experiments,
the most convenient choice of the polarization axis is the direction 
of the boost vector from the $J/\psi$ rest frame to the lab frame. 
Experimental results for $\alpha$  
are shown in Table~\ref{tab:fixed-target-pol}.
The prediction of the NRQCD factorization approach is
$0.31<\alpha<0.63$~\cite{Beneke:1996tk}. Both the theoretical prediction
and the data include feeddown from $\chi_c$ states. The prediction is
largely independent of the target and beam types. It was made
specifically for the beam energy 117~GeV. However, the energy dependence
of the prediction is quite mild, and the prediction would be expected to
hold with little error even at a beam energy of 800~GeV. The
color-singlet model predicts a substantial transverse polarization
\cite{Vanttinen:1994sd}. The color-evaporation model predicts that
$\alpha = 0$ for all processes. There are also specific predictions for
the HERA-B experiment in which the region of small $p_T$ is excluded.
The predictions for the range $p_T=1.5$--$4$~GeV are $\alpha=0$--$0.1$
in the NRQCD factorization approach and $\alpha=0.2$--$0.4$ in the
color-singlet model~\cite{lee-2000}. Experimental results for the
polarization variable $\alpha$ in $J/\psi$ production are shown in
Table~\ref{tab:fixed-target-pol}. The data from the conventional
fixed-target experiments are consistent with $\alpha=0$ and favor the
prediction of the color-evaporation model over the predictions of NRQCD
factorization or the color-singlet model \cite{Beneke:1996tk}.  At the
smaller values of $p_T$, one can question whether resummation of the
perturbation series is needed and whether hard-scattering factorization
would be expected to hold. The HERA-B data are also consistent with
$\alpha=0$  and favor the predictions of the NRQCD factorization
approach and the color-evaporation model over the prediction of the
color-singlet model.

There is also a measurement of the polarization of $\psi(2S)$ in a
fixed-target experiment. The E615 experiment measured $\alpha$ for
$\psi(2S)$ mesons produced in $\pi N$ collisions at 253~GeV
\cite{Heinrich:zm}. The data yield $-0.12<\alpha<0.16$, while the
prediction of the NRQCD factorization approach is
$0.15<\alpha<0.44$~\cite{Beneke:1996tk}.

%%%%%%%%%%%%%%%%%%%%%%%%%%%%%%%%%%%%%%%%%%%%%%%%%%%%%%%%%%%%%%%%%%%%%%%%%%%%%%%%%%
\begin{figure}[htb]
\begin{center}
\epsfig{figure=e866-upsilon-pol.eps,width=12cm}
%\includegraphics[width=15cm]{e866-upsilon-pol.eps}
\caption{Polarization of $\Upsilon$ mesons and Drell-Yan pairs as a function
of $p_T$ and $x_F$ in $p$-Cu collisions in the E866 experiment. From
Ref.~\cite{Brown:2000bz}.}
\label{fig:e866-upsilon-pol}
\end{center}
\end{figure}
%%%%%%%%%%%%%%%%%%%%%%%%%%%%%%%%%%%%%%%%%%%%%%%%%%%%%%%%%%%%%%%%%%%%%%%%%%%%%%%%%%

The E866/NuSea experiment has studied the production of dimuons in the
collision of 800~GeV protons with copper \cite{Brown:2000bz}. The
experiment used the angular distributions of dimuons in
the mass range 8.1--15.0~GeV to measure
the polarization variable $\alpha$ for Drell-Yan
pairs, for $\Upsilon(1S)$ mesons, and for a mixture of $\Upsilon(2S)$
and $\Upsilon(3S)$ mesons. The data cover the kinematic ranges 0.0 $<
x_F <$ 0.6 and $p_T <$ 4.0~GeV. The results for the polarization
variable $\alpha$ as a function of $p_T$ and $x_F$ are shown in
Fig.~\ref{fig:e866-upsilon-pol}. The $\Upsilon(1S)$ data show almost no
polarization at small $x_F$ and $p_T$, but show nonzero transverse
polarization at either large $p_T$ or large $x_F$. A fit at the
$\Upsilon$(1S) mass for a polarization that is independent of $x_F$ and
$p_T$ gives $\alpha =$0.07 $\pm$ 0.04. This
observation is substantially smaller than a prediction that is based
on the NRQCD factorization approach, which gives $\alpha$ in the range
0.28--0.31~\cite{Kharchilava:1998wa,Tkabladze:1999mb}. However, it
also disagrees with the prediction of the color-evaporation model
that the polarization should be zero \cite{Amundson:1996qr}. The most
remarkable result from this experiment is that the
$\Upsilon(2S)$ and $\Upsilon(3S)$ were found to be strongly
transversely polarized, with the polarization variable $\alpha$ close to
its maximal value $\alpha = +1$ for all $x_F$ and $p_T$, as in the
case of Drell-Yan pairs. This result provides strong motivation
for measuring the polarizations of the $\Upsilon(2S)$ and $\Upsilon(3S)$
at the Tevatron to see if these states are also produced with
substantial polarizations in $p \bar p$ collisions.


\subsection{Color-evaporation model parameters}
\label{prodsec:fixed-targetCEM}

%%%%%%%%%%%%%%%%%%%%%%%%%%%%%%%%%%%%%%%%%%%%%%%%%%%%%%%%%%%%%%%%%%%%%%%%%%%%%%%%%%
\begin{table}[ht]
\begin{center}
\begin{tabular}{|llcc||llcc|}
\hline \hline
PDF & $m_c$ & $\mu/m_{cT}$ & $F_{J/\psi}$ & 
PDF & $m_b$ & $\mu/m_{bT}$ & $F_{\Upsilon(1S)}$ \\ \hline
MRST HO & 1.2 & 2   & 0.0144  &
MRST HO & 4.75 & 1   & 0.0276 \\
MRST HO & 1.4 & 1   & 0.0248  &
MRST HO & 4.5  & 2   & 0.0201 \\
CTEQ 5M & 1.2 & 2   & 0.0155  &
MRST HO & 5.0  & 0.5 & 0.0508 \\
GRV 98 HO & 1.3 & 1 & 0.0229  &
GRV 98 HO & 4.75 & 1 & 0.0225 \\ 
\hline \hline
\end{tabular}
\caption{
Inclusive CEM parameters $F_{J/\psi}$ and $F_{\Upsilon(1S)}$
from Ref.~\cite{Bedjidian:2003gd}
for various choices of PDF's, quark masses (in GeV), and scales.
}
\label{prodsec:qqbparams}
\end{center}
\end{table}
%%%%%%%%%%%%%%%%%%%%%%%%%%%%%%%%%%%%%%%%%%%%%%%%%%%%%%%%%%%%%%%%%%%%%%%%%%%%%%%%%%

%%%%%%%%%%%%%%%%%%%%%%%%%%%%%%%%%%%%%%%%%%%%%%%%%%%%%%%%%%%%%%%%%%%%%%%%%%%%%%%%%%
\begin{table}[ht]
\begin{center}
\begin{tabular}{|c|ccccc|} 
\hline \hline
$H$ & $J/\psi$ & $\psi(2S)$ & $\chi_{c1}$ & $\chi_{c2}$ & \\ \hline
$F_H^{\rm dir}/F_{J/\psi}$       & 0.62 & 0.14 & 0.60 & 0.99 & \\ 
\hline \hline
$H$ & $\Upsilon(1S)$ & $\Upsilon(2S)$ & $\Upsilon(3S)$ 
        & $\chi_b(1P)$ & $\chi_b(2P)$ \\ \hline
$F_H^{\rm dir}/F_{\Upsilon(1S)}$ & 0.52 & 0.33 & 0.20 & 1.08 & 0.84 \\ 
\hline \hline
\end{tabular}
\caption{Ratios of the direct CEM parameters $F_H^{\rm dir}$ to the
inclusive CEM parameter $F_{J/\psi}$ in the case of charmonium states
and to the inclusive CEM parameter $F_{\Upsilon(1S)}$ in the case of
bottomonium states. From Ref.~\protect \cite{Digal:2001ue}.}
\label{prodsec:ratios}
\end{center}
\end{table}
%%%%%%%%%%%%%%%%%%%%%%%%%%%%%%%%%%%%%%%%%%%%%%%%%%%%%%%%%%%%%%%%%%%%%%%%%%%%%%%%%%

Data from $pp$ and $pA$ collisions have been used
to extract the parameters $F_H$ of the color-evaporation model.
(The CEM parameter $F_H$ should not be confused with the fraction 
of $J/\psi$'s that come from decay of $H$.)
The results of these extractions are 
given in Tables~\ref{prodsec:qqbparams} and~\ref{prodsec:ratios}.
The numerical values of the CEM parameters $F_H$ that are obtained by
fitting data depend on the choices of the parton densities (PDF's), the
heavy quark mass $m_Q$, the renormalization/factorization scale $\mu$,
and the order in $\alpha_s$ of the calculation. The CEM parameters have
been calculated using several sets of parton densities
\cite{Martin:1998sq,Lai:1999wy,Gluck:1998xa}, quark masses, and scales
\cite{Vogt:2002vx,Vogt:2002ve} that reproduce the measured $Q \bar Q$
cross section. In these calculations, the scale $\mu$ was set to a
constant times $m_{QT} = (m_Q^2 + p_T^2)^{1/2}$, where $p_T$ is the sum
of the transverse momenta of the $Q$ and the $\bar Q$.

We first describe the extraction of the CEM parameters $F_H$ for
charmonium states. The inclusive cross section for $J/\psi$ production
has been measured in $pp$ and $pA$ collisions up to $\sqrt{s} = 63$ GeV.
The data are of two types: the forward cross section, $\sigma(x_F > 0)$,
and the cross section at zero rapidity, $d\sigma/dy|_{y=0}$.  These
cross sections include the feeddown from decays of $\chi_{cJ}$ and
$\psi(2S)$. The parameters $F_{J/\psi}$ that were obtained by fitting
the inclusive $J/\psi$ cross sections measured in $pp$ and $pA$
collisions are given in Table~\ref{prodsec:qqbparams} for four sets of
PDF's and parameters. The ratio of the parameter $F_H^{\rm dir}$ for the
direct production of a charmonium state $H$ to the parameter $F_{J/\psi}$
for the inclusive production of $J/\psi$ can be determined from the
measured ratios of the inclusive cross sections for $H$ and $J/\psi$
using the known branching fractions for the feeddown decays. These
ratios are given in Table~\ref{prodsec:ratios} for various charmonium
states.

A similar procedure can be used to determine the CEM parameters $F_H$
for bottomonium states. In most data on $pp$ and $pA$ collisions below
$\sqrt{s} = 100$ GeV, only the sum of the $\Upsilon(1S)$,
$\Upsilon(2S)$, and $\Upsilon(3S)$ cross sections weighted by their
branching fractions to decay into lepton pairs is reported. A fit to the
lepton-pair cross section in the $\Upsilon$ region at zero rapidity
therefore gives a linear combination of the inclusive parameters 
$F_{\Upsilon(nS)}$ weighted by the branching fractions 
$B[\Upsilon(nS) \to \ell^+ \ell^-]$. 
The inclusive parameters $F_{\Upsilon(1S)}$ given in
Table~\ref{prodsec:qqbparams} were extracted by using the known
branching fractions and the measured ratios of the inclusive cross
sections for $\Upsilon(nS)$ in $p \bar p$ collisions at the Tevatron
\cite{Affolder:1999wm}. The ratios of the parameters $F_H^{\rm dir}$ for
the direct production of a bottomonium state $H$ to the parameter
$F_{\Upsilon(1S)}$ for the inclusive production of $\Upsilon(1S)$ that
were obtained in Ref.~\cite{Gunion:1996qc} have been updated in
Ref.~\cite{Digal:2001ue} by using recent CDF data on $\chi_b$ production
and are given in Table~\ref{prodsec:ratios}.

%%%%%%%%%%%%%%%%%%%%%%%%%%%%%%%%%%%%%%%%%%%%%%%%%%%%%%%%%%%%%%%%%%%%%%%%%%%%%%%%%%
\begin{figure}[htb]
\setlength{\epsfxsize=0.95\textwidth}
\setlength{\epsfysize=0.4\textheight}
\begin{center}
\epsfig{figure=psiups_fixt.eps,width=14cm}
%\centerline{\epsffile{psiups_fixt.eps}}
\end{center}
\caption{Forward $J/\psi$ production cross section (left) and weighted
average of the $\Upsilon(nS)$ production cross sections at zero rapidity
(right) as a function of the center-of-mass energy $\sqrt{s}$. The
$J/\psi$ data are from $pp$ experiments and from $pA$ experiments with
light targets $A \leq 12$. It has been assumed that the cross sections
scale as $A^{0.9}$.  The low-energy $\Upsilon$ data are from $pp$ and
$pA$ experiments. It has been assumed that the cross sections are linear
in $A$. The high-energy $\Upsilon$ data are from $p \bar p$ experiments.
The curves are the cross sections calculated to NLO in the CEM using the
four charmonium parameter sets and the four bottomonium parameter sets
in Table~\ref{prodsec:qqbparams}.}
\label{psiupsfixt}
\end{figure}
%%%%%%%%%%%%%%%%%%%%%%%%%%%%%%%%%%%%%%%%%%%%%%%%%%%%%%%%%%%%%%%%%%%%%%%%%%%%%%%%%%

The forward cross section for $J/\psi$ and the weighted cross section at
zero rapidity for $\Upsilon(nS)$ are shown  as a function of the
center-of-mass energy in Fig.~\ref{psiupsfixt}. The energy dependence of
both cross sections is well reproduced by the CEM at NLO. All of the CEM
parameter sets give good fits to the data for $\sqrt{s} \leq 63$ GeV,
but their predictions for $\Upsilon(nS)$ differ by up to a factor of two
when extrapolated to 2~TeV. The extrapolation of the forward $J/\psi$
cross section to 2~TeV cannot be compared with data from Run I of the
Tevatron because the lepton-$p_T$ cut excludes a measurement of the
cross section for $J/\psi$ in the region $p_T < 5$ GeV that dominates 
the integrated cross section.



%%%%%%%%%%%%%%%%%%%%%%%%%%%%%%%%%%%%%%%%%%%%%%%%%%%%%%%%%%%%%%%%%%%%%%%%%%%%%%%%%%

%\input{sechera}
\section{Quarkonium production at HERA} 
\label{prodsec:hera}

\subsection{Inelastic photoproduction of charmonium}
\label{prodsec:hera:gp}

\begin{figure}
\epsfig{figure=herafeyn.eps,width=15cm}
\caption{Generic Feynman diagrams for inelastic \jpsi\ production.  a,b)
direct-photon processes; c) resolved-photon process. In diagrams
 a) and c), the \ccbar\ pair leading to the formation of the \jpsi\ can 
be in a color-singlet or a color-octet state while in b) it can
only be in a  color-octet state. Additional soft gluons emitted
during the  hadronization process are not shown.}
\label{herafeyn}
\end{figure}

At the $ep$ collider HERA, the inelastic charmonium production process
is dominantly virtual-photon-gluon fusion: a photon emitted from the
incoming electron or positron interacts with a gluon from the proton to
produce a \ccbar\ pair that subsequently forms a charmonium state.  In
photoproduction, the photon virtuality $Q^2$ is small and the photon is
quasi-real.  In this case, the photon can either couple to the $c$ quark
directly (``direct'' processes, Fig.~\ref{herafeyn}a or b) or it can
interact via its hadronic component (``resolved'' processes,
Fig.~\ref{herafeyn}c).
Many models have been suggested to describe inelastic charmonium
production in the framework of perturbative QCD, such as the 
color-singlet model
(CSM) \cite{Berger:1980ni,Baier:1981uk,Baier:1981zz,Baier:1983va}
described in Section~\ref{prodsec:nrqcdCSM}, the
color-evaporation model \cite{Halzen:1977rs,Eboli:1998xx} 
described in Section~\ref{prodsec:nrqcdCEM}, and soft
color interactions \cite{Edin:1997zb}.

%%%%%%%%%%%% Color singlet and NRQCD

For $J/\psi$ and $\psi(2S)$ photoproduction, the CSM calculations are
available to next-to-leading order\cite{Kramer:1994zi,Kramer:1995nb}. 
These are performed using standard 
hard-scattering factorization in which the gluon density 
depends only on the momentum fraction $x$.
Alternatively, using
the CSM, inelastic \jpsi\ production can be modeled in the
$k_T$-factorization approach (see Section~\ref{prodsec:nrqcdmge}) using
an unintegrated ($k_T$-dependent) gluon density in the proton.

Theoretical calculations based on the NRQCD factorization approach 
\cite{Caswell:1985ui,Thacker:1990bm,Bodwin:1994jh} are 
available in leading order. For \jpsi\ and $\psi(2S)$ photoproduction
at HERA, these have been performed by Cacciari and Kr\"amer
\cite{Cacciari:1996dg}, Beneke, Kr\"amer, and V\"anttinen
\cite{Beneke:1998re}, Amundson, Fleming, and Maksymyk
\cite{Amundson:1996ik}, Ko, Lee, and Song \cite{Ko:1996xw},
Godbole, Roy, and Sridhar \cite{Godbole:1995ie}, and Kniehl and
G.~Kramer \cite{Kniehl:1997fv,Kniehl:1997gh}. The theoretical
calculations use the standard truncation in $v$, in which 
the independent NRQCD matrix elements are 
$\langle {\cal O}_1^{J/\psi}({}^3S_1) \rangle$, $\langle {\cal
O}_8^{J/\psi}({}^1S_0)
\rangle$, $\langle {\cal O}_8^{J/\psi}({}^3S_1) \rangle$, and $\langle
{\cal O}_8^{J/\psi}({}^3P_0) \rangle$.  The relative strength of the
color-octet contributions depends crucially on the size of the
corresponding NRQCD matrix elements. Unfortunately the values of the
matrix elements $\langle {\cal O}_8^{J/\psi}({}^1S_0) \rangle$ and
$\langle {\cal O}_8^{J/\psi}({}^3P_0) \rangle$, which are most
important in \jpsi\ and $\psi(2S)$ photoproduction at HERA, still show
large uncertainties. (See Section~\ref{prodsec:tevatroncharm} and
Ref.~\cite{Kramer:2001hh}.)

The theoretical predictions are sensitive to a number of input
parameters, {\it e.g.}, the parton distributions, the values of
$\alpha_s$, and the charm-quark mass $m_c$, as well as the choice of the
renormalization and factorization scales. In the NRQCD factorization
approach, the values of the color-octet NRQCD matrix elements are
additional parameters.  The comparison with the data in the NRQCD
approach also suffers from the uncertainties associated with LO
calculations. Next-to-leading-order corrections might change the results
substantially. Although the NLO terms have not been calculated in the
NRQCD approach, effects that are similar to those in the CSM may be
expected, in which the NLO terms lead to an increase in the cross
section of typically a factor two, with a strong \ensuremath{p_{T,\psi}}
dependence.

\begin{figure}
\begin{center}
\epsfig{figure=gammap-jpsiz.eps,width=9cm}
\caption{The rate for inelastic $J/\psi$ photoproduction at HERA
 as a function of $z$. The open band represents the LO
 NRQCD factorization prediction~\cite{Kramer:2001hh}. The solid band
 represents the NLO color-singlet
 contribution~\cite{Kramer:1995nb,Kramer:2001hh}. The data points are
 from the H1
\cite{Adloff:2002ex} and ZEUS
\cite{Chekanov:2002at} measurements.}
\label{fig-photo-prod}
\end{center}
\end{figure}

\begin{figure}
\begin{center}
\epsfig{figure=gammap-jpsipt2.eps,width=9cm}
\caption{The rate for inelastic $J/\psi$ photoproduction at HERA
as a function of $p_{T,\psi}$. The solid band represents the NLO
color-singlet contribution~\cite{Kramer:1995nb,Kramer:2001hh}. The 
dotted line is the LO color-singlet contribution. The
data points are from the H1~\cite{Adloff:2002ex} and ZEUS
\cite{Chekanov:2002at} measurements.}
\label{fig-photo-prod-csm}
\end{center}
\end{figure}

Figure~\ref{fig-photo-prod} shows the measurements of the prompt
\jpsi\ cross section by the H1 collaboration \cite{Adloff:2002ex} 
and the ZEUS collaboration \cite{Chekanov:2002at}, compared with the
theoretical predictions given in Ref.~\cite{Kramer:2001hh}.
%
The variable $z$ denotes the fraction of the photon energy that is
transferred to the $J/\psi$ and is defined as
\begin{equation}
z={(E-p_z)_{J/\psi} \over (E-p_z)_{\rm hadrons}}, 
\label{eq:zdef}
\end{equation}
where $E$ and $p_z$ in the numerator are the
energy and $z$-component of the momentum of the $J/\psi$ and  
$E$ and $p_z$ in the denominator are the sums of the energies 
and $z$-components of the momenta of all the hadrons in the final state.

The $J/\psi$ data points shown in Fig.~\ref{fig-photo-prod} are not corrected
for feeddown processes, such as diffractive and inelastic production
of $\psi(2S)$ mesons ($\approx 15\% $), the production of $b$ hadrons
with subsequent decays to \jpsi\ mesons, or feeddown from the
production of $\chi_c$ states. The latter two contributions are
estimated to contribute between 5\% at medium $z$ and 30\% at the
lowest values of $z$.
%
The open band in Fig.~\ref{fig-photo-prod} represents the sum of the
color-singlet and color-octet contributions, calculated in leading
order in QCD perturbation theory.  The uncertainty is due to the
uncertainty in the color-octet NRQCD matrix elements. The NRQCD
prediction deviates from the data near $z=1$, owing to the large
color-octet contribution in that region.  The shaded band shows the
calculation of the color-singlet contribution to next-to-leading order
in $\alpha_s$
\cite{Kramer:1994zi,Kramer:1995nb}. The NLO corrections increase 
the color-singlet contribution by about a factor of two, so that it
accounts for the data quite well without the inclusion of a
color-octet contribution.
%

Uncertainties in $m_c$ could lower the color-singlet contribution by
about a factor of two, leaving more room for color-octet
contributions.  In the experimental data, the cut $p_{T,\psi}>1$~GeV
is employed. One can question whether hard-scattering factorization is
valid at such small values of $p_{T,\psi}$. However, the data
differential in $p_{T,\psi}$ are compatible with color-singlet
production alone at large $p_{T,\psi}$
(Fig.~\ref{fig-photo-prod-csm}).

The next-to-leading-order QCD corrections are crucial in describing the
shape of the transverse-momentum distribution of the $J/\psi$. The NLO
color-singlet cross section includes processes such as $\gamma + g \to
(c\bar{c}) + g  g $, which are dominated by $t$-channel gluon exchange
and scale as $\alpha_s^3 m_c^2 / p_{T,\psi}^6$. At
$p_{T,\psi}\;\rlap{\lower 3.5 pt \hbox{$\mathchar \sim$}} \raise 1pt
\hbox {$>$}\;m_c$ their contribution is enhanced with respect to the
leading-order cross section, which scales as $\sim \alpha_s^2
m_c^4/p_{T,\psi}^8$. The comparison with the experimental data in
Fig.~\ref{fig-photo-prod-csm} confirms the importance of the
higher-order corrections.

\begin{figure}
\begin{center}
\epsfig{figure=gammap-beneke.eps,width=16cm}
\caption{Differential cross sections ${\rm d}\sigma/{\rm d}z$
($60<W_{\gamma p}<240\,{\rm GeV}$) for $p_{T,\psi}>2\,{\rm GeV}$ (left
panel) and $p_{T,\psi}>3\,{\rm GeV}$ (right panel) in comparison with
NRQCD  calculations that include color-octet and color-singlet
contributions and resummations of soft contributions at high $z$
\cite{Beneke:1999gq}. The curves correspond to three values of the
parameter $\Lambda$: $\Lambda=0$, {\it i.e.}, no resummation (dashed
line), $\Lambda=300$~MeV (solid line), and $\Lambda= 500$~MeV
(dash-dotted line). The theoretical curves have  been scaled with a
common factor 2 in the left panel and 3 in the right panel.}
\label{gammap-beneke}
\end{center}
\end{figure}


At large $z$, 
the emission of soft gluons in the conversion of the
\ccbar\ pairs to \jpsi\ mesons is suppressed, owing to phase-space
limitations. Furthermore, the velocity expansion of the NRQCD
factorization approach is expected to break down \cite{Beneke:1997qw}. 
These effects are not taken into account in the theoretical calculation
that is shown in Fig~\ref{fig-photo-prod}.
In Ref.~\cite{Beneke:1999gq}, a resummation of the nonrelativistic
expansion was carried out, leading to a decrease of the predicted cross
section at high $z$.  
The resummation involves a parameter $\Lambda$ that describes 
the energy in the \ccbar\ rest frame that is lost by the \ccbar\ system
in its conversion into the \jpsi\ meson.
In Fig.~\ref{gammap-beneke}, the measured cross
sections $d\sigma/dz$ for ${p_{T,\psi}}>2$ GeV and for ${p_{T,\psi}}>3$
GeV are compared with the results of these resummed calculations.  The
calculated curves have been roughly normalized to the data points at low
$z$. The resummed calculation for $\Lambda=500$ MeV gives an acceptable
description of the data at \ensuremath{p_{T,\psi}}$ > 3$ GeV, while the
agreement between data and calculation is still poor for
\ensuremath{p_{T,\psi}}$ > 2$ GeV or for lower $\Lambda$ values.

Effects from resummation of logarithms of $1-z$ and model shape
functions have also been calculated for the process $e^+e^-\to
J/\psi+X$ \cite{Fleming:2003gt}. It may be possible to use this
resummed theoretical prediction to extract the dominant shape function
from the Belle and BaBar data for $e^+e^-\to J/\psi+X$ and then use it
to make predictions for $J/\psi$ photoproduction near $z=1$.

Measurements of the $J/\psi$ production cross section at large $z$ are 
available from H1~\cite{Aktas:2003zi} and from ZEUS~\cite{Chekanov:2002at}. 
In this region, the contribution from diffractively produced $J/\psi$ 
mesons is expected to be large, as is discussed below 
in Section~\ref{subsechera:diffjpsi}.

\begin{figure}
\begin{center}
\epsfig{figure=gammap-cascade.eps,width=14cm}
\caption{Inelastic \jpsi\ production in the region  $60<W_{\gamma
p}<240$ GeV, $0.3<z<0.9$, and $p_{T,\psi}^2>1$ GeV$^2$, in comparison
with a $k_T$-factorization model implemented in the  Monte Carlo
generator CASCADE\cite{Jung:2001hk,Jung:2001hx}. Left panel:
differential cross section ${\rm d}\sigma/{\rm d}z$; right panel:
$d\sigma/d\ensuremath{p_{T,\psi}^2}$ in the range $0.3<z<0.9$.}
\label{gammap-cascade}
\end{center}
\end{figure}

The ZEUS Collaboration has also measured the $\psi'$ to $J/\psi$ cross
section ratio \cite{Chekanov:2002at} in the range $0.55 < z < 0.9$ and
$50 < W < 180$ GeV. It is found to be consistent with being
independent of the kinematic variables $z$, $p_{t,\psi}$ and $W$, as
is expected if the underlying production mechanisms for the $J/\psi$
and the $\psi'$ are the same.
An average value $\sigma(\psi')/\sigma(J/\psi)=
0.33\pm0.10 ({\rm stat.}) ^{+0.01}_{-0.02} ({\rm syst.})$ is found
which compares well with the prediction from the leading-order
color-singlet model \cite{Kramer:1994zi}.


The $k_T$-factorization approach %~\cite{Catani:1990eg,Collins:1991ty}
(see Section~\ref{prodsec:nrqcdmge}) has recently been applied
successfully to the description of a variety of
processes\cite{Jung:2001hk,Jung:2001hx,Saleev:1994fg}. In this approach,
the \jpsi\ production process is factorized into a $k_T$-dependent gluon
density and a matrix element for off-shell partons. A leading-order
calculation within this approach is implemented in the Monte Carlo
generator CASCADE~\cite{Jung:2001hk,Jung:2001hx}.
Figure~\ref{gammap-cascade} shows a comparison of the data with the
predictions from the $k_T$-factorization approach. Good agreement is
observed between data and predictions for $z\lsim 0.8$. At high $z$
values, the CASCADE calculation underestimates the cross section. This
may be due to missing higher-order effects, or missing relativistic
corrections, which are not available for the off-shell matrix element.
It could also indicate a possible missing color-octet contribution.  The
CASCADE predictions for the the \ensuremath{p_{T,\psi}^2} dependence of
the cross section (Fig.~\ref{gammap-cascade}c) fit the data considerably
better than the collinear LO calculations. This improved fit is due to
the transverse momentum of the gluons from the proton, which contributes
to the transverse momentum of the \jpsi\ meson. Note, however, that the
NLO color-singlet calculation in collinear factorization
\cite{Kramer:1995nb} also describes the \ensuremath{p_{T,\psi}^2}
distribution.

The polarization of the \jpsi\ meson is expected to differ in the
various theoretical approaches discussed here and could in principle
be used to distinguish between them, independently of normalization
uncertainties. The general decay angular distribution can be
parametrized as
\begin{equation}
  \frac{d\Gamma(J/\psi\to l^+l^-)}{d\Omega}
  \propto
  1 + \lambda \cos^2\theta + \mu \sin 2\theta \cos\phi
  + \frac{\nu}{2} \sin^2\theta \cos 2\phi,
  \label{paramdef}
\end{equation}
where $\theta$ and $\phi$ refer to the polar and azimuthal angle of the
$l^+$ three-momentum with respect to a coordinate system that is defined
in the $J/\psi$ rest frame. (See, for example, Ref.~\cite{Beneke:1998re}
for details.) The parameters $\lambda, \mu, \nu$ can be calculated
within NRQCD or the CSM as a function of the kinematic variables, 
such as $z$ and $p_{T,\psi}$.

\begin{figure}
\unitlength1.0cm
\begin{picture}(6.,10.)
\put(2.0,6.){\epsfig{figure=lavszpt1cv.eps,width=6cm}}
\put(8.5,6.){\epsfig{figure=nuvszpt1cv.eps,width=6cm}}
\put(2.0,0.){\epsfig{figure=lavsptz1cv.eps,width=6cm}}
\put(8.5,0.){\epsfig{figure=nuvsptz1cv.eps,width=6cm}}
\end{picture}
\caption{Polarization parameters $\lambda$ (left panels) and $\nu$
(right panels) in the target rest frame as functions of $z$ (top
panels) and \ensuremath{p_{T,\psi}} (bottom panels). The error
bars on the data points correspond to the total experimental error. The
theoretical calculations shown are from the NRQCD approach
\cite{Beneke:1998re} (shaded bands) with color-octet and color-singlet
contributions, while the curves show the result from the
color-singlet contribution separately. 
}
\label{gammap-pola}
\end{figure}


\begin{figure}
\begin{center}
\epsfig{figure=ep-kniehl.eps,width=10cm}
\end{center}
\caption{Generic diagrams for charmonium production 
 mechanisms: photon-gluon fusion via a ``$2\rightarrow 1$''
 process (top left diagram) and ``$2\rightarrow 2$'' processes 
 (remaining diagrams). 
 All the diagrams contribute via color-octet mechanisms, 
 while the top right diagram can also contribute
 via the color-singlet mechanism.  Additional soft gluons emitted during
 the hadronization process are not shown.}
\label{fig-ep-kniehl}
\end{figure}


In Fig.~\ref{gammap-pola}, the data are shown, together with the results
from two LO calculations: the NRQCD prediction, including color-octet
and color-singlet contributions \cite{Beneke:1998re}, and the color-singlet
contribution alone.
A calculation that uses a $k_T$-factorization 
approach and off-shell gluons is also available~\cite{Baranov:1998af}.
In contrast to the predictions shown in the Fig.~\ref{gammap-pola},
in which $\lambda$ is zero or positive, the prediction of 
the $k_T$-factorization approach is that $\lambda$ should become
increasingly negative toward larger values of $p_{t,J/\psi}$,
reaching $\lambda \sim -0.5$ at $p_{T,\psi}=6$ GeV.
However, at present, the errors in the
data preclude any firm conclusions. In this range of $p_{T,\psi}$
none of the calculations
predicts a decrease in $\lambda$ with increasing $z$. 
In order to distinguish between
full NRQCD and the color-singlet contribution alone, measurements at
larger \ensuremath{p_{T,\psi}} are required. The measured values of
$\nu$, for which no prediction is available from the $k_T$-factorization
approach, favor the full NRQCD prediction.

\subsection{Inelastic electroproduction of charmonium}
\label{prodsec:hera:dis}


As in photoproduction, inelastic leptoproduction of $J/\psi$ mesons at
HERA ($e+p\rightarrow e+J/\psi+X$) is dominated by virtual-photon-gluon
fusion ($\gamma^*g\rightarrow \ccbar$). In leptoproduction, or deep
inelastic $ep$-scattering (DIS), the exchanged photon has a nonzero
virtuality $Q^2=-q^2$, where $q$ is the four-momentum of the virtual
photon. For events with a photon virtuality of $Q^2\gtrsim 1$ GeV$^2$,
the electron scattering angle is large enough for the electron to be
detected in the central detectors.

The analysis of leptoproduction at finite $Q^2$ has experimental and
theoretical advantages in comparison with the analysis of
photoproduction. At high $Q^2$, theoretical uncertainties in the models
decrease and resolved-photon processes are expected to be negligible.
Furthermore, the background from diffractive production of charmonia is
expected to decrease faster with $Q^2$ than the inelastic process, and
the distinct signature of the scattered lepton makes the inelastic
process easier to detect.

A first comparison between data and NRQCD calculations was presented in
Ref.~\cite{Adloff:1999zs}.  The NRQCD calculations in
Ref.~\cite{Adloff:1999zs} were performed by taking into account only
``$2\rightarrow 1$'' diagrams (see the top left diagram of
Fig.\ref{fig-ep-kniehl}) \cite{Fleming:1997fq}, and disagreement between
data and theory was observed both in the absolute values of the cross
sections and in their shapes as functions of the variables that were
studied.

More recently, the cross section for $J/\psi$ production in
deep-inelastic $ep$ scattering at HERA was calculated in the NRQCD
factorization approach at leading order in $\alpha_s$ by Kniehl and
Zwirner \cite{Kniehl:2001tk}, taking into account diagrams of the type
``$2\rightarrow 2$'', as are shown in the top right and bottom diagrams
of Fig.~\ref{fig-ep-kniehl}. The calculation made use of the matrix
elements of Ref.~\cite{Braaten:1999qk} and MRST98LO \cite{Martin:1998sq}
and CTEQ5L \cite{Lai:1999wy} parton distributions.

\begin{figure}
\epsfig{figure=ep-q2.eps,width=14cm}
\caption{Differential cross sections $d\sigma / dQ^2$ (top left panel)
and $d\sigma/dp_{T,\psi}^{*2}$ (bottom left panel) and the corresponding
ratios of data to theory (right panels). 
The data from H1~\cite{Adloff:2002ey} are compared
with the NRQCD  calculation~\cite{Kniehl:2001tk} (CO+CS, dark band) and 
the color-singlet contribution~\cite{Kniehl:2001tk} (CS, light band).}
%
\label{fig-ep-q2}
\end{figure}

\begin{figure}
\unitlength1cm
\begin{picture}(10.,6.)
\put(1.,-0.5){\epsfig{figure=03-012-02.ps,width=6.5cm}}
\put(8.5,-0.5){\epsfig{figure=03-012-08.ps,width=6.5cm}}
\end{picture}
\caption{Differential cross sections $d\sigma / dQ^2$ (left)
and $d\sigma/dz$ (right) and theory predictions. 
The data from ZEUS~\cite{zeus-eps03-565} are compared
with the NRQCD calculation~\cite{Kniehl:2001tk} (CO+CS, dark band), 
the color-singlet contribution (CS, light band), and with
the prediction LZ(kt,CS) from the $k_T$-factorization approach
within the CSM~\cite{Lipatov:2002tc}. The solid lines delimit 
the uncertainties, and the
dashed line show the central values. The CSM prediction
LZ (CS) in the collinear-factorization approach, as given by the
authors of Ref.~\cite{Lipatov:2002tc},
is also shown (dotted line). }
%
\label{fig-zeus-q2z}
\end{figure}

\begin{figure}
\epsfig{figure=ep-z.eps,width=16cm}
\caption{Normalized differential cross sections. $1/\sigma\,d\sigma /
dz$ (top left panel), $1/\sigma\,d\sigma / dW$ (top right panel),
$1/\sigma\,d\sigma / dp_{T,\psi}^2$ (middle left panel),
$1/\sigma\,d\sigma /dY^\ast$ (middle right panel), $1/\sigma\,d\sigma /
dp_{T,\psi}^{*2}$ (bottom left panel), and $1/\sigma\,d\sigma /dY_{lab}$
(bottom right panel).  The histograms show calculations for inelastic
$J/\psi$ production within the NRQCD factorization
approach\protect\cite{Kniehl:2001tk}, which have been normalized to the
integrated cross section. The dark band represents the sum of CS and CO
contributions, and the light band represents the CSM contribution alone.
These contributions are normalized separately. The error bands reflect
the theoretical uncertainties.}
%
\label{fig-ep-z}
\end{figure}

In Fig.~\ref{fig-ep-q2}, the results of this calculation are plotted as
a function of $Q^2$ and $p^2_{T,\psi}$, along with the H1 data
\cite{Adloff:2002ey}. The NRQCD results that are shown in
Fig.~\ref{fig-ep-q2} include the contributions from the color-octet
channels $^{3}\!S_1$, $^{3}\!P_{J=0,1,2}$, $^{1}\!S_0$, as well as from
the color-singlet channel $^{3}\!S_1$. The contribution of the
color-singlet channel is also shown separately. The values of the NRQCD
matrix elements were determined from the distribution of transverse
momenta of \jpsi\ mesons produced in \ppbar\
collisions~\cite{Braaten:1999qk}.\footnote{\label{fn} The extracted
values for the NRQCD matrix elements depend on the parton distributions.
For the set MRST98LO \cite{Martin:1998sq}, the values are $\langle {\cal
O}_1^{J/\psi}(^3\!S_1)\rangle=1.3\pm 0.1 \,$GeV$^3$, $\langle {\cal
O}_8^{J/\psi}(^3\!S_1)\rangle=(4.4\pm0.7) \times 10^{-3}
\,$GeV$^3$ and $M_{3.4}^{J/\psi}=(8.7\pm 0.9) \times 10^{-2}\,$GeV$^3$,
where $M_{3.4}^{J/\psi}$ is the linear combination of two NRQCD matrix
elements that is defined in Eq.~(\ref{prod:lincomb}).}
%
The bands include theoretical uncertainties, which originate from the
uncertainty in the charm-quark mass $m_c=1.5\pm0.1$ GeV, the variation
of renormalization and factorization scales by factors 1/2 and 2, and
the uncertainties in the NRQCD matrix elements, all of which result
mainly in normalization uncertainties that do not affect the shapes of
the distributions.

Fig.~\ref{fig-zeus-q2z} shows the differential electroproduction
cross sections for $J/\psi$ mesons as functions of $Q^2$ and $z$,
as measured by the ZEUS collaboration~\cite{zeus-eps03-565}. 
The ZEUS data, which are consistent with the H1 results 
shown in Fig~\ref{fig-ep-q2}, are compared with 
predictions in the framework of 
NRQCD (CS+CO)~\cite{Kniehl:2001tk} and also with predictions 
in the $k_T$-factorization approach in which only 
the color-singlet contribution (CS) is included~\cite{Lipatov:2002tc}.
As in Fig.~\ref{fig-ep-q2}, the uncertainties in the 
NRQCD calculations are indicated in 
Fig.~\ref{fig-zeus-q2z} as shaded bands. 
For the prediction within the $k_T$-factorization approach 
(LZ(kt,CS)),
only one of the sources of uncertainty is presented, namely the 
uncertainty in the pomeron intercept $\Delta$, 
which controls the normalization
of the unintegrated gluon density.

In Fig.~\ref{fig-ep-z}, the normalization uncertainties of the
theory, which are dominant, are removed 
by normalizing the differential
cross sections measured by H1~\cite{Adloff:2002ey} and the
theory predictions to the integrated 
cross sections in the measured range for each distribution.
The comparisons in Figs.~\ref{fig-ep-q2}--\ref{fig-ep-z} 
indicate that the color-singlet contribution
follows the shape of the data from H1 and ZEUS quite well.
In general, the CSM predictions are below the H1
and ZEUS data, but are consistent with the data, given the 
uncertainties, both in shape and normalization.
However, the differential cross sections as a function
of the transverse momentum squared of the $J/\psi$ are too 
steep compared to the data (lower left plot in Fig.~\ref{fig-ep-q2}). 
A similar observation
was made for photoproduction (Section~\ref{prodsec:hera:gp},
Fig~\ref{fig-photo-prod-csm}), in which the LO CSM calculation is too
steep and the NLO CSM calculation is found to describe the data
well.
The $z$ distribution (Figs.~\ref{fig-zeus-q2z} and \ref{fig-ep-z})
is very poorly described by the full calculation that includes
color-octet contributions, while the color-singlet contribution
alone reproduces the shape of the data rather well.
The failure of the 
color-octet calculations could be due to the fact that resummations
of soft-gluons are not included here.
It is worth noting that the calculation of Kniehl and Zwirner
disagrees with a number of previous results
\cite{Korner:1982fm,Guillet:1987xr,Merabet:sm,Krucker:1995uz,Yuan:2000cn}, 
which themselves are not fully consistent.


\subsection{Diffractive vector meson production}
\label{subsechera:diffjpsi}

\begin{figure}
\begin{center}
\epsfig{figure=feyn_hera_jpsi_diffractive.eps,width=5cm}
\end{center}
\caption{Diagram for diffractive charmonium production via exchange of two gluons
 in a color-singlet state.}
\label{fig-ep-feyn-diff}
\end{figure}

\begin{figure}
\epsfig{figure=gp-vmdiffractive.eps,width=16cm}
\caption{Total cross section and cross sections for production of 
various vector mesons in $\gamma p$ collisions as a function of 
$W_{\gamma p}$, as measured at HERA and in fixed-target experiments.}
%
\label{fig-gp-vmdiffractive}
\end{figure}

At HERA, the dominant production channel for quarkonia with quantum
numbers of real photons (i.e.\ $J^{PC}=1^{--}$) is through diffractive
processes.
%
In perturbative QCD, the diffractive production of vector mesons can be
modeled in the proton rest frame by a process in which the photon
fluctuates into a $q\bar{q}$ pair at a long distance from the proton
target.  The $q\bar{q}$ subsequently interacts with the proton via a
color-singlet exchange, i.e.\,in lowest order QCD via the exchange of a
pair of gluons with opposite color
(see Fig.~\ref{fig-ep-feyn-diff})~\cite{Ryskin:1992ui,
Brodsky:1994kf, %
Collins:1996fb, %
Collins:1997sr, %
Teubner:1999pm, %
Bartels:2000ze, % confinement
Hayashigaki:2004iw}.
At small $|t|$, where $t$ is the momentum-transfer-squared at the proton
vertex, the elastic process in which the proton stays intact dominates.
Toward larger values of $|t|$, the dissociation of the proton into a
small-invariant-mass state becomes dominant. Measurements of diffractive
vector-meson production cross sections and helicity structure from the
H1~\cite{Aid:1996bs,Adloff:1999kg, Adloff:1999zs,Adloff:2000vm,
Adloff:2000nx,Adloff:2002tb, Adloff:2002re,Aktas:2003zi,h1-ichep04-6-0180} and
ZEUS~\cite{Breitweg:1998nh,Breitweg:1998ki,
Breitweg:1999jy,Breitweg:1999fm,Breitweg:2000mu,Chekanov:2002rm,
Chekanov:2002xi,Chekanov:2004mw} collaborations are available for
$\rho^0$, $\omega$, $\phi$, $J/\psi$, $\psi'$, and $\Upsilon$
production, spanning the ranges of $0 \simeq Q^2 < 100 $ GeV$^2$, $ 0
\simeq |t| < 20 $ GeV$^2$, and $ 20 < W_{\gamma p} < 290 $ GeV. 
($W_{\gamma p}$ is the $\gamma p$ center-of-mass energy.) In
Fig.~\ref{fig-gp-vmdiffractive}, the elastic photoproduction cross
sections are shown.
Perturbative calculations in QCD are available for the kinematic
regions in which at least one of the energy scales $\mu^2$ (i.e.~$Q^2$, 
$M_V^2$ or $|t|$) is large and the strong-coupling constant
$\alpha_s(\mu^2)$ is small~\cite{Frankfurt:1997fj,%
McDermott:1999fa,% 
Frankfurt:2000ez,% gamma p (high W)
Ryskin:1995hz,% gluon density 
Martin:1997wy,% skewed parton densities
Martin:1999wb,% sigma_l/sigma_t
Ivanov:2004vd% dipole calculations
}.

In the presence of such a `hard' scale, QCD predicts a steep rise of the
cross section with $W_{\gamma p}$.  At small $Q^2$, $|t|$ and meson
masses $M_V$, vector-meson production is known to show a
non-perturbative ``soft'' behavior that is described, for example, by
Regge-type models~\cite{Regge:mz, Regge:1960zc, Sakurai:1969ss,
% VDM 
Donnachie:1992ny,Donnachie:1998gm}.
%
Toward larger values of $|t|$, in the leading logarithmic approximation,
diffractive $J/\psi$ production can be described by the effective
exchange of a gluonic ladder. At sufficiently low values of Bjorken $x$
(i.e. large values of $W_{\gamma p}$), the gluon ladder is expected to
include contributions from BFKL
evolution~\cite{ FKL:1976, Kuraev:fs, BL:1978, Mueller:1994jq, % bfkl
Nikolaev:1993th % qcd pomeron 
}, as well as from DGLAP evolution~\cite{Gribov:ri}.

Experimentally, diffractive events are generally distinguishable from
inelastic events, since, aside from meson-decay products, only a few
final-state particles are produced in the central rapidity range in
proton dissociation and no particles are produced in the central
rapidity range in elastic diffraction.
%
The elasticity variable $z$ defined in Eq.~(\ref{eq:zdef}) 
is often used to demark
the boundary between the elastic and inelastic regions, with a typical
demarcation for $J/\psi$ production being $z>0.95$ for the diffractive
region and $z<0.95$ for the inelastic region. However, at large $z$,
there is actually no clear distinction between inelastic $J/\psi$
production and diffractive $J/\psi$ production 
in which the proton dissociates
into a final state with large invariant mass, owing to the fact that the
two processes can produce the same final-state particles.
%
In the region of large $z$, both inelastic and
diffractive processes are expected to contribute to the cross section.
In calculations that are based on the NRQCD factorization approach, the
cross section increases toward large $z$, owing to large contributions
from color-octet $c \bar c$ pairs, as is explained in
Section~\ref{prodsec:hera:gp}. These contributions are, however,
substantially reduced when one takes into account multiple soft gluon
emission, {\it e.g.}, in resummation calculations~\cite{Beneke:1999gq}. 
At the same time, calculations in perturbative QCD that assume a
diffractive color-singlet exchange are capable of describing the
production cross sections at large
$z$~\cite{Aktas:2003zi,Chekanov:2002rm,zeus-eps03-549}. 
A unified description in QCD
of the large $z$ region, taking into account both inelastic and
diffractive contributions, has yet to be developed.

\subsection{Prospects for the upgraded HERA collider}

With the HERA luminosity upgrade, a wealth of new quarkonium data will
become available. The existing \jpsi\ and $\psi(2S)$ measurements can
be improved and extended into new kinematic regions, and other
quarkonium final states may become accessible.  The future analyses of
quarkonium production at HERA offer unique possibilities to test the
theoretical framework of NRQCD factorization. 
It should be noted here that calculations to next-to-leading order, which
are not yet available in the framework of NRQCD factorization, could 
be an essential ingredient in a 
full quantitative understanding of charmonium production at HERA, and
also at other experiments, such as those at the Tevatron.
Some of the most interesting reactions will be discussed briefly below. See
Refs.~\cite{Cacciari:1996dy,Kramer:2001hh} for more details.

The measurement of inelastic {\em $\chi_c$ photoproduction} is a
particularly powerful way to discriminate between NRQCD and the
color-evaporation model. The assumption of a single, universal
long-distance factor in the color-evaporation model implies a
universal $\sigma[\chi_c]/\sigma[J/\psi]$ ratio.  A large $\chi_c$ cross
section is predicted for photon-proton collisions.  The ratio of
$\chi_c$ production to $J/\psi$ production is expected to be similar to
that at hadron colliders, for which $\sigma[\chi_c] / \sigma[J/\psi]
\approx 0.5$~\cite{Abe:1997yz}. In NRQCD, on the other hand, the
$\sigma[\chi_c]/\sigma[J/\psi]$ ratio is process-dependent and
strongly suppressed in photoproduction. Up to corrections of 
${\cal O}(\alpha_s,v^2)$ one finds that~\cite{Kramer:2001hh}
\begin{equation}
\frac{\sigma[\gamma p \to \chi_{cJ}\,X]}
     {\sigma[\gamma p \to J/\psi \,X]}
\approx \frac{15}{8}\, (2J+1)\, \frac{\langle {\cal O}^{\chi_{c0}}_8
   (^3\!S_1)\rangle}
   {\langle {\cal O}^{J/\psi}_1(^3\!S_1)\rangle}
 \approx (2J+1)\,0.005 ,
\end{equation}
where the last approximation makes use of the NRQCD matrix elements
that are listed in Table~\ref{tab:me-1}. A search for $\chi_c$
production at HERA that results in a cross section measurement or an
upper limit on the cross section would probe directly the color-octet
matrix element $\langle {\cal O}^{\chi_J}_8(^3\!S_1)\rangle$ and would
test the assumption of a single, universal long-distance factor that is
implicit in the color-evaporation model.

The inclusion of color-octet processes is crucial in describing the
photoproduction of the {\em spin-singlet states} $\eta_c(1S)$,
$\eta_c(2S)$, and $h_c(1P)$. With regard to the $P$-wave state $h_c$,
the color-octet contribution is required to cancel the infrared
divergence that is present in the color-singlet cross
section~\cite{Fleming:1998md}. The production of the $\eta_c$, on the
other hand, is dominated by color-octet processes, since the
color-singlet cross section vanishes at leading-order, owing to
charge-conjugation invariance~\cite{Hao:1999kq,Hao:2000ci}, as is the
case for $\chi_c$ photoproduction. The cross sections for $\eta_c(1S)$,
$\eta_c(2S)$, and $h_c(1P)$ photoproduction are
sizable~\cite{Fleming:1998md,Hao:1999kq}, but it is not obvious that
these particles can be detected experimentally, even with
high-statistics data.

The energy spectrum of {\em $J/\psi$'s produced in association with a
photon} via the process $\gamma p \to J/\psi + \gamma\,X$ is a
distinctive probe of color-octet
processes~\cite{Kim:1993at,Cacciari:1996dy,Mehen:1997vx,Cacciari:1997zu}.
In the color-singlet channel and at leading-order in $\alpha_s$, $J/\psi
+ \gamma$ can be produced only through resolved-photon interactions. The
corresponding energy distribution is therefore peaked at low values of
$z$. The intermediate-$z$ and large-$z$ regions of the energy spectrum
are  expected to be dominated by the color-octet process $\gamma g \to
c\bar c_8({}^3\!S_1)\,\gamma$. Observation of a substantial fraction of
$J/\psi + \gamma$ events at $z\;\rlap{\lower 3.5 pt \hbox{$\mathchar
\sim$}} \raise 1pt \hbox {$>$}\; 0.5$ would provide clear evidence for
the presence of color-octet processes in quarkonium photoproduction.
Experimentally, this measurement is very difficult due to the
large background from photons from $\pi^0$ decays which are produced
in the final state.

With the significant increase in statistics at the upgraded HERA
collider, it might be possible to study {\em inelastic photoproduction
of bottomonium states} for the first time. The large value of the
$b$-quark mass makes the perturbative QCD predictions more reliable than
for charm production, and the application of NRQCD should be on safer
ground for the bottomonium system, in which $v^2 \approx 0.1$. However,
the production rates for $\Upsilon$ states are suppressed compared with
those for $J/\psi$ by more than two orders of magnitude at HERA---a
consequence of the smaller $b$-quark electric charge and the phase-space
reduction that follows from the larger $b$-quark mass.


%\input{seclep}

\section{Quarkonium production at LEP}
\label{prodsec:lep}

\subsection{$J/\psi$ production}

The LEP collider was used to study $e^+ e^-$ collisions at the $Z^0$
resonance. Charmonium was produced at LEP through direct production
in $Z^0$ decay, through the decays of $b$ hadrons from $Z^0$ decay,
and through $\gamma \gamma$ collisions.  
The contributions from the decays of $b$ hadrons can be separated from 
those from direct production by using a vertex detector.
Charmonium that is produced directly will be referred to as ``prompt.''

%%%%%%%%%%%%%%%%%%%%%%%%%%%%%%%%%%%%%%%%%%%%%%%%%%%%%%%%%%%%%%%%%%%%%%%%%%%%%%%%%%%%%%%%
\begin{figure}[ht]
\begin{center}
\epsfig{figure=boyd-lep.eps,width=10cm}
\caption{
Differential rate $d \Gamma/ dz$ for inclusive decay of $Z^0$ 
into $J/\psi$. The data is from the ALEPH collaboration 
\cite{ALEPH:1997zj}.  The dashed line is the sum of the tree-level 
color-singlet and color-octet terms.  The solid line  is an 
interpolation between resummed calculations in the small-$z$ 
and large-$z$ regions. From Ref.~\cite{Boyd:1998km}. 
}
\label{fig-boyd-lep}
\end{center}
\end{figure}
%%%%%%%%%%%%%%%%%%%%%%%%%%%%%%%%%%%%%%%%%%%%%%%%%%%%%%%%%%%%%%%%%%%%%%%%%%%%%%%%%%%%%%%%

%%%%%%%%%%%%%%%%%%%%%%%%%%%%%%%%%%%%%%%%%%%%%%%%%%%%%%%%%%%%%%%%%%%%%%%%%%%%%%%%%%%%%%%%
\begin{figure}[ht]
\begin{center}
\epsfig{figure=delphi1.eps,width=10cm}
\caption{
Differential cross section for the process ${\gamma
\gamma\rightarrow J/\psi +X}$ as a function of $p_T^2$. 
The data points are from the DELPHI Collaboration
\cite{Todorova-Nova:2001pt,Abdallah:2003du}. The upper set of curves is the
NRQCD factorization predictions, and the lower set of curves is the
color-singlet model prediction. The solid and dashed curves correspond
to the MRST98LO \cite{Martin:1998sq} and CTEQ5L \cite{Lai:1999wy} parton
distributions, respectively. The arrows indicate the relative
contributions at $p_T=0$ from parton processes $ij \to c \bar c$,
which were ignored in the analysis. 
Here $ij = \gamma \gamma$, $gg$, or $q \bar q$. 
From Ref.~\cite{Klasen:2001cu}.}
\label{fig-gamma-gamma-psi}
\end{center}
\end{figure}
%%%%%%%%%%%%%%%%%%%%%%%%%%%%%%%%%%%%%%%%%%%%%%%%%%%%%%%%%%%%%%%%%%%%%%%%%%%%%%%%%%%%%%%%


In $Z^0$ decay, the dominant mechanism for charmonium production is the 
decay of the $Z^0$ into $b \bar b$, followed by the fragmentation 
of the $b$ or $\bar b$ into a heavy hadron and the subsequent
decay of the heavy hadron into charmonium.
The inclusive branching fraction of the $Z^0$ into a charmonium state
$H$ is to a good approximation the product of the branching fraction 
for $Z^0 \to b \bar b$, a weighted average of the inclusive branching 
fractions of $b$ hadrons into $H$, and a factor of two to account for the 
$b$ and the $\bar b$:
%
\begin{equation}
{\rm Br}[ Z^0 \to H X] \approx 
2 \; {\rm Br}[ Z^0 \to b \bar b ] \sum_B D_{b \to B} \; {\rm Br}[B \to H X].
\end{equation}
%
The branching fraction for the $b$ hadron $B$ to decay into 
a state that includes $H$ is weighted by the probability 
$D_{b \to B}$ for a 45~GeV $b$ quark to fragment into $B$.
The inclusive branching fractions for $Z^0$ decay into several 
charmonium states have been measured.
Since these measurements have more to do with $b$-hadron decay
than $Z^0$ decay, they are presented in Section~\ref{prodsec:bdecays}.

The ALEPH, DELPHI, L3, and OPAL collaborations at LEP have measured
the inclusive branching fraction of $Z^0$ into prompt $J/\psi$
\cite{ALEPH:1997zj,Abreu:1994rk,Wadhwa:1998mt,Alexander:1996jp}.
In the NRQCD factorization approach, there are two
mechanisms that dominate direct $J/\psi$ production.  The first is
$Z^0$ decay into $c \bar c$, followed by the fragmentation of the $c$
or $\bar c$ into $J/\psi$ via the color-singlet channel $c \bar
c_1({}^3S_1)$. The second is $Z^0$ decay into $q \bar g g$, followed by
the fragmentation of the gluon into $J/\psi$ via the color-octet channel
$c \bar c_8({}^3S_1)$.  Boyd, Leibovich, and Rothstein
\cite{Boyd:1998km} have used the results from the four LEP
collaborations to extract the color-octet matrix element: $\langle{\cal
O}^{J/\psi}_8({}^3S_1) \rangle =(1.9\pm 0.5_{stat}\pm1.0_{theory})
\times 10^{-2}\hbox{ GeV}^3$. This is about a factor of two larger than
the Tevatron value and has smaller theory errors, but feeddown from
$\chi_c$ and $\psi(2S)$ states was not taken into account in the
theoretical analysis. Boyd, Leibovich, and Rothstein \cite{Boyd:1998km}
also carried out a resummation of the logarithms of $M_Z^2/M_\psi^2$ and
$z^2$, where $z = 2E_{c \bar c}/m_Z$. Their result for the resummed $z$
distribution for prompt $J/\psi$ production is compared with
data from the ALEPH collaboration \cite{ALEPH:1997zj} in
Fig.~\ref{fig-boyd-lep}. Their analysis predicts an enhancement in the
production rate near $z=0.15$. The uncertainties in the data are too
large to make a definitive statement about the presence or absence of
this feature.

The inclusive cross section for ${\gamma \gamma\rightarrow J/\psi +X}$
at LEP has been measured by the DELPHI Collaboration
\cite{Todorova-Nova:2001pt,Abdallah:2003du}. The cross section at nonzero
$p_T$ has been computed at leading order in $\alpha_s$. 
The computation includes the direct-photon process 
$\gamma \gamma \to (c \bar c ) + g$, which is of order $\alpha^2 \alpha_s$, 
the single-resolved-photon process $i \gamma \to (c \bar c ) + i$, 
which is of order $\alpha \alpha_s^2$, 
and the double-resolved-photon process $i j \to (c \bar c ) + k$, 
which is of order $\alpha_s^3$
\cite{Ma:1997bi,Japaridze:1998ss,Godbole:2001pj,Klasen:2001mi,Klasen:2001cu}.
(Here, $ij = gg$, $g q$, $g \bar q$, or $q \bar q$.) Note that all
processes contribute formally at the same order in perturbation theory since
the leading behavior of the parton distributions in the photon is
$\propto \alpha/\alpha_s$. The contribution to the 
$\gamma\gamma\rightarrow J/\psi +X$ cross section at LEP
that is by far dominant numerically is that
from single-resolved processes, {\it i.e.}, photon-gluon fusion.

The results of the LO computation~\cite{Klasen:2001cu} are
shown in Fig.~\ref{fig-gamma-gamma-psi}. The computation uses the
NRQCD matrix elements of Ref.~\cite{Braaten:1999qk}. Theoretical
uncertainties were estimated by varying the renormalization and
factorization scales by a factor two and by incorporating the effects
of uncertainties in the values of the color-octet matrix elements. As
can be seen from Fig.~\ref{fig-gamma-gamma-psi}, the comparison with
the DELPHI data
\cite{Todorova-Nova:2001pt,Abdallah:2003du} clearly favors the
NRQCD factorization approach over the color-singlet model. However,
the comparison of Fig.~\ref{fig-gamma-gamma-psi} is based on a
leading-order calculation. It is known from the related process of
$J/\psi$ photoproduction at HERA, which is also dominated by
photon-gluon fusion, that the LO color-singlet cross section fails to
describe the $J/\psi$ data at nonzero $p_T$. Inclusion of the NLO
correction, however, brings the color-singlet prediction in line with
experiment. Similarly large NLO corrections can be expected for
${\gamma \gamma\rightarrow J/\psi +X}$ production at LEP, and a
complete NLO analysis is needed before firm conclusions on the
importance of color-octet contributions can be drawn.  A first step in
this direction has been taken recently in Ref.~\cite{Klasen:2004tz},
where the NLO corrections to the direct process $\gamma \gamma \to (c
\bar c ) + g$ have been calculated. 

\subsection{$\Upsilon(1S)$ production}

The OPAL collaboration has measured the inclusive branching fraction for
the decay of $Z^0$ into $\Upsilon(1S)$ \cite{Alexander:1995vh}. The
NRQCD factorization prediction for ${\rm Br}[Z^0\rightarrow \Upsilon(1S)
+ X]$ is $5.9\times 10^{-5}$ \cite{Cho:1995vv}. The color-singlet-model
prediction is $1.7\times 10^{-5}$
\cite{Keung:1980ev,Kuhn:1981jy,Abraham:1989ri,Hagiwara:1991mt,%
Braaten:1993mp,Cho:1995vv}. The experimental result from OPAL
\cite{Alexander:1995vh} is $[1.0\pm 0.4(\hbox{stat.})\pm
0.1(\hbox{sys.}) \pm 0.2(\hbox{prod.\ mech.})]\times 10^{-4}$.
This is compatible with the NRQCD factorization prediction, but not with
the color-singlet-model prediction.


\section{Charmonium production in $e^+e^-$ annihilations at 10.6 GeV}
\label{prodsec:eecontinuum}

The $B$ factories have proved to be a rich source of data on charmonium
production in $e^+ e^-$ annihilation. The $B$ factories operate near the
peak of the $\Upsilon(4S)$ in order to maximize the production rate for
$B$ mesons, but about 75\% of the events are continuum $e^+e^-$
annihilation events. The enormous data samples that have been accumulated
compensate for the relatively small cross sections for 
$e^+ e^-$ annihilation into states that include charmonium.

\subsection{$J/\psi$ production}

%%%%%%%%%%%%%%%%%%%%%%%%%%%%%%%%%%%%%%%%%%%%%%%%%%%%%%%%%%%%%%%%%%%%%%%%%%%%%%%%%%%%%%
\begin{figure}
\begin{center}
$ \begin{array}{cc}
\includegraphics[width=3in]{fleming-belle-mom.eps}
&
\includegraphics[width=3in]{fleming-babar-mom.eps} \\
{\mathrm (a)} \hspace{.5in} & {\mathrm (b)}
\end{array} $
\caption{$J/\psi$ production rate in $e^+e^-$ annihilation at
$10.6$~GeV as a function of $p^* = p_{\psi}$, the $J/\psi$ momentum in
the CM frame. The vertical axis is the number of $J/\psi$ events per
0.5~GeV$/c$. The data points are from (a) the Belle Collaboration
\cite{Abe:2001za} and (b) the BaBar Collaboration
\cite{Aubert:2001pd}. The upper lines are the sum of the leading-order
color-singlet contribution and the color-octet contribution, which
includes a resummation of logarithms of $1-z$ and a phenomenological
shape function. The lower lines are the leading-order color-singlet
contribution alone. From Ref.~\cite{Fleming:2003gt}.}
\label{belle-babar}
\end{center}
\end{figure}
%%%%%%%%%%%%%%%%%%%%%%%%%%%%%%%%%%%%%%%%%%%%%%%%%%%%%%%%%%%%%%%%%%%%%%%%%%%%%%%%%%%%%%

The Belle and BaBar Collaborations have measured the inclusive cross
section $\sigma[e^+e^-\rightarrow J/\psi\, X]$. The Belle Collaboration
obtains $2.52\pm 0.21\pm 0.21\hbox{~pb}$ \cite{Abe:2001za}, while
the BaBar Collaboration obtains $1.47\pm 0.10\pm 0.13\hbox{~pb}$
\cite{Aubert:2001pd}. The leading-order parton process in the
color-singlet model is $e^+ e^- \to (c \bar c) + gg$, which is of order
$\alpha^2 \alpha_s^2$. The leading color-octet contributions in the
NRQCD factorization approach come from $e^+ e^-$ annihilation into $(c
\bar c) + g$, which is order $\alpha^2 \alpha_s$, and into $(c \bar c) +
q \bar q$ and $(c \bar c) + g g$, which are order $\alpha^2 \alpha_s^2$.
The prediction for the cross section $\sigma[e^+e^-\rightarrow
J/\psi\,X]$ in the color-singlet model is $0.45-0.81\hbox{~pb}$
\cite{Cho:1996cg,Yuan:1996ep,Yuan:1997sn,Schuler:1998az}, while the
NRQCD factorization prediction is $1.1-1.6\hbox{~pb}$
\cite{Yuan:1996ep,Yuan:1997sn,Schuler:1998az}. There is a $3\sigma$
discrepancy between the experiments, but the NRQCD factorization
prediction seems to be favored.
The discrepancies between the two experiments in this and other 
measurements may be due partly to differences in cuts that were used 
to suppress contributions from processes in which the charmonium 
is not produced by annihilation of $e^+$ and $e^-$ with a center-of-mass 
energy of 10.6 GeV.  These include radiative-return processes, 
in which the $e^+$ or $e^-$ loses a substantial fraction of its 
momentum by radiating a collinear photon before the collision, 
virtual photon radiation, in which the $e^+$ or $e^-$ radiates 
a virtual photon that becomes a $J/\psi$ or $\psi(2S)$, 
and two-photon collisions, which produce $\eta_c$, $\chi_{c0}$, 
and $\chi_{c2}$. 


The momentum distribution of the $J/\psi$ provides information about the
production mechanism. The momentum of the $J/\psi$ in the CM frame
can be characterized in terms of its magnitude $p^*$ and its angle
$\theta^*$ with respect to the beam direction. The Belle
\cite{Abe:2001za} and BaBar \cite{Aubert:2001pd} measurements for the
differential cross section for $J/\psi$ production as a function of
$p^*$ are shown in Fig.~\ref{belle-babar}. The color-singlet prediction,
which is shown in the lower curves in Fig.~\ref{belle-babar}, is far too
small to describe the data. The measurements from Belle and BaBar do not
show any enhancement at the maximum value of $p^*$, as might be expected
from the color-octet process $e^+ e^- \to (c \bar c) + g$ that is of
leading order in $\alpha_s$. However, there are two effects that are
expected to modify the leading-order result. The first effect is that
the $v$ expansion of NRQCD breaks down near the kinematic maximum value
of $p^*$. Resummation of the expansion is required
\cite{Beneke:1997qw,Beneke:1999gq}, and it leads to a nonperturbative
shape function \cite{Beneke:1997qw}, which smears out the peak in the
leading-order result. A second effect near the maximum value of $p^*$ is
that there are large logarithms of $1-z$, where $z=E_{c\bar c}/E_{c\bar
c}^{\rm max}$, that must also be resummed. The effect of that
resummation is again to smear out the peak in the leading-order result.
A resummation of logarithms of $1-z$ has been combined with a
phenomenological shape function in Ref.~\cite{Fleming:2003gt}. The
results of this calculation are shown in the upper curves in
Fig.~\ref{belle-babar}. The shape function has been chosen to fit the
Belle and BaBar data. The normalization of the shape function is fixed
by the color-octet NRQCD matrix elements, which were taken to be
$\langle{\cal O}^{J/\psi}_8({}^1S_0)\rangle= \langle{\cal
O}^{J/\psi}_8({}^3P_0)\rangle=6.6\times10^{-2}$~GeV.  These values of
the color-octet matrix elements are consistent with data from
photoproduction and hadroproduction
\cite{Beneke:1996tk,Amundson:1996ik}. As can be seen, the resummations of
the $v$ expansion and the logarithms of $1-z$ produce reasonable fits to
the data. The resummation prediction is not expected to be valid at
small values of $p^*$. It should also be kept in
mind that hard-scattering factorization may not hold unless
$p^*\gg\Lambda_{\rm QCD}$. While the comparison of the resummed theory
with experiment indicates that it is plausible that the NRQCD
factorization approach can describe the experimental data, the
theoretical results rely heavily on the phenomenological shape function,
whose shape is tuned to fit the data. The resummed theory will receive a
much more stringent test when a phenomenological shape function that has
been extracted from the $e^+e^-$ data is used to predict the $J/\psi$
production cross section in some other process, for example,
photoproduction at HERA.

%%%%%%%%%%%%%%%%%%%%%%%%%%%%%%%%%%%%%%%%%%%%%%%%%%%%%%%%%%%%%%%%%%%%%%%%%%%%%%%%%%%%%%
\renewcommand{\arraystretch}{1.1}
\begin{table}[ht]
\begin{center}
\begin{tabular}{|ccc|ccc|}
\hline   
\hline   
\multicolumn{3}{|c|}{Belle} &
\multicolumn{3}{c|}{BaBar} \\
\hline
Range of $p^*$ (GeV) & $A$ & $\alpha$ & 
Range of $p^*$ (GeV) & $A$ & $\alpha$ \\
\hline
$2.0 < p^* < 2.6$ & $0.82^{+0.95}_{-0.63}$ & $-0.62^{+0.30}_{-0.24}$ &
        $p^* < 3.5$ & $0.05\pm 0.22$ & $-0.46\pm 0.21$ \\
$2.6 < p^* < 3.4$ & $1.44^{+0.42}_{-0.38}$ & $-0.34^{+0.18}_{-0.16}$ & 
                  &                       & \\
$3.4 < p^* < 5.0$ & $1.08^{+0.44}_{-0.33}$ & $-0.32^{+0.20}_{-0.18}$ &
        $3.5 < p^*$ & $1.5 \pm 0.6$ & $-0.80\pm0.09$ \\
\hline
\hline   
\end{tabular}
\renewcommand{\arraystretch}{1}
\caption{Angular asymmetry variable $A$ and polarization variable $\alpha$
for various ranges of the CM momentum $p^*$ of the $J/\psi$
in $e^+ e^- \to J/\psi X$ at $\sqrt{s} = 10.6$ GeV, as measured by the 
Belle \cite{Abe:2001za} and BaBar \cite{Aubert:2001pd} Collaborations.}
\label{prod:eetab}
\end{center}
\end{table}
%%%%%%%%%%%%%%%%%%%%%%%%%%%%%%%%%%%%%%%%%%%%%%%%%%%%%%%%%%%%%%%%%%%%%%%%%%%%%%%%%%%%%%

The other variable that characterizes the momentum of the $J/\psi$ is
its angle $\theta^*$ with respect to the beam direction in the CM frame.
The angular distribution $d\sigma/d(\cos \theta^*)$ is proportional to
$1+A\cos^2 \theta^*$, which defines an angular asymmetry variable $A$.
The Belle \cite{Abe:2001za} and BaBar \cite{Aubert:2001pd}
Collaborations have measured $A$ in several bins of $p^*$. Their results
are shown in Table~\ref{prod:eetab}. The NRQCD factorization approach
predicts that $A\approx 0$ at small $p^*$ and $0.6<A<1.0$ at large $p^*$
\cite{Braaten:1995ez} . The color-singlet model predicts that $A\approx
0$ at small $p_T$ and $A\approx -0.8$ at large $p^*$
\cite{Braaten:1995ez}. The Belle and BaBar results favor the
NRQCD factorization prediction, but the uncertainties are large.

The polarization of the $J/\psi$ provides further information about 
the production mechanism.  
The polarization variable $\alpha$ for $J/\psi$ production 
is defined by the angular distribution in Eq.~(\ref{prod:alphadef}).
In $e^+ e^-$ annihilation, the most convenient choice for the
polarization axis is the boost vector from the quarkonium rest frame 
to the $e^+ e^-$ center-of-momentum frame. 
The Belle and BaBar Collaborations have measured the polarization 
variable $\alpha$ in several bins of $p^*$. Their results are shown in
Table~\ref{prod:eetab}. The polarization of 
$J/\psi$'s from $e^+ e^-$ annihilation
at the $B$ factories has not yet been calculated within the NRQCD 
factorization approach. In contrast to the production of $J/\psi$'s 
with large $p_T$ at the Tevatron, 
where the dominance of gluon fragmentation into
color-octet ${}^3S_1$ $c \bar c$ states implies a large transverse
polarization, production of $J/\psi$'s at the $B$ factories occurs at
values of $p^*$ for which there are no simple qualitative expectations
for the polarization.  A comparison between  theory and experiment must
await an actual calculation of the $J/\psi$ polarization,
including the effects of feeddown from higher charmonium states. It may
be necessary to include in such a calculation resummations of the $v$
expansion and logarithms of $1-z$ in order to make precise quantitative
statements. However, the effects of these resummations is mainly to
re-distribute the $J/\psi$'s that are produced via the color-octet
mechanism over a range in $p^*$ without affecting the total number of
such $J/\psi$'s.

A surprising result from the Belle Collaboration is that most of the
$J/\psi$'s that are produced in $e^+e^-$ annihilation at $10.6$~GeV are
accompanied by charmed hadrons. The presence of a charmed hadron
indicates the creation of a second $c \bar c$ pair in addition to the
pair that forms the $J/\psi$. A convenient measure of the probability
for creating the second $c \bar c$ pair is the ratio
%
\begin{eqnarray}
R_{\rm double} =
{\sigma[e^+e^-\to J/\psi+X_{c\bar c}] 
        \over \sigma[e^+e^-\to J/\psi+X]} \;. 
\label{prodsec:Rdouble}
\end{eqnarray}
%
The Belle Collaboration finds that $R_{\rm double} =0.82\pm 0.15\pm
0.14$ with $R_{\rm double} > 0.48$ at the 90\% confidence level
\cite{belle-eps2003}. The NRQCD factorization approach leads to the
prediction $R_{\rm double}\approx 0.1$
\cite{Cho:1996cg,Baek:1996kq,Yuan:1996ep}, which disagrees with the
Belle result by almost an order of magnitude. The discrepancy seems to
arise primarily from the cross section in the numerator of
(\ref{prodsec:Rdouble}). The Belle result for this cross section is
about 0.6--1.1~pb \cite{Abe:2002rb}, while the prediction is about
0.10--0.15 pb \cite{Cho:1996cg,Baek:1996kq,Yuan:1996ep,Liu:2003jj}. At
leading order in $\alpha_s$, which is $\alpha_s^2$, the process of
$e^+e^-$ annihilation into $J/\psi+X_{c\bar c}$ proceeds through $(c
\bar c)+c\bar c$. The contributions to this cross section in which the
$J/\psi$ is formed from a color-octet $c\bar c$ pair are suppressed by
a factor $v^4\approx 0.1$, and they have been found to yield only
about 7\% of the total cross section \cite{Liu:2003jj}. Corrections of
order $\alpha_s^3$ and higher are also not expected to be particularly
large. Thus, the source of the discrepancy between the Belle result
for $R_{\rm double}$ and theory remains a mystery.

%%%%%%%%%%%%%%%%%%%%%%%%%%%%%%%%%%%%%%%%%%%%%%%%%%%%%%%%%%%%%%%%%%%%%%%%%%%%%%%%%%%%%%%%
\begin{figure}[ht]
\begin{center}
\epsfig{figure=belledoublecharm.eps,width=10cm}
\caption{Distribution of masses recoiling against the reconstructed
$J/\psi$ in inclusive $e^+e^- \rightarrow J/\psi X$ events at Belle
\cite{Abe:2004ww}. The solid line is the best fit, including
contributions from the $\eta_c$, $\chi_{c0}(1P)$, and $\eta_c(2S)$.
The dotted line is a fit in which additional contributions from the
$J/\psi$, $\chi_{c1}(1P)$, $\chi_{c2}(1P)$, and $\psi(2S)$ have been set
at their largest possible values within the 90\%-confidence-level
limits.
}
\label{fig-belledoublecharm}
\end{center}
\end{figure}
%%%%%%%%%%%%%%%%%%%%%%%%%%%%%%%%%%%%%%%%%%%%%%%%%%%%%%%%%%%%%%%%%%%%%%%%%%%%%%%%%%%%%%%%

There is also a large discrepancy between theory and experiment in an
exclusive double-$c \bar c$ cross section. For the double-charmonium
process $e^+e^-\to J/\psi+\eta_c$, the Belle Collaboration measures the
product of the cross section and the branching fraction for the $\eta_c$ to
decay into at least two charged tracks to be $25.6\pm
2.8\pm 3.4~\hbox{fb}$ \cite{Abe:2004ww}. In contrast, leading-order
calculations predict a cross section of $2.31\pm 1.09~\hbox{fb}$
\cite{Braaten:2002fi,Liu:2002wq, brodsky-ji-lee}. There are some
uncertainties from uncalculated corrections of higher-order in
$\alpha_s$ and $v$ and from NRQCD matrix elements. However, because this
is an exclusive process, only color-singlet matrix elements enter, and
these are fairly well determined from the decays $J/\psi\to e^+e^-$ and
$\eta_c\to \gamma\gamma$.

Since the Belle mass resolution is 110 MeV but the $J/\psi$-$\eta_c$
mass difference is only 120~MeV, it has been suggested that some of the
$J/\psi+\eta_c$ data sample may consist of $J/\psi+J/\psi$ events
\cite{Bodwin:2002fk,Bodwin:2002kk}. The state $J/\psi+J/\psi$ has
charge-parity $C=+1$, and consequently, is produced in a two-photon
process, whose rate is suppressed by a factor $(\alpha/\alpha_s)^2$
relative to the rate for $J/\psi+\eta_c$. However, as was pointed out in
Refs.~\cite{Bodwin:2002fk,Bodwin:2002kk}, the two-photon process
contains photon-fragmentation contributions that are enhanced by factors
$(E_{\rm beam}/2m_c)^4$ from photon propagators and $\log[8(E_{\rm
beam}/2m_c)^4]$ from a would-be collinear divergence. As a result, the
predicted cross section  $\sigma[e^+e^-\to J/\psi+J/\psi]=8.70\pm
2.94$~fb is larger than the predicted cross section $\sigma[e^+e^-\to
J/\psi+\eta_c]=2.31\pm 1.09$~fb \cite{Bodwin:2002fk,Bodwin:2002kk}.
Corrections of higher order in $\alpha_s$ and $v$ are likely to
reduce the prediction for the $J/\psi+J/\psi$ cross section by about a
factor of three \cite{Bodwin:2002kk,Luchinsky:2003yh}. These
predictions spurred a re-analysis of the Belle data \cite{Abe:2003ja}.
The invariant mass distribution of $X$ in $e^+e^-\to J/\psi+X$ is shown
in Fig.~\ref{fig-belledoublecharm}. No significant $J/\psi+J/\psi$
signal was observed. The upper limit on the cross section times the
branching fraction into at least two charged tracks \cite{Abe:2004ww} is
$\sigma[e^+e^-\to J/\psi+J/\psi]<9.1~\hbox{fb}$, which is consistent
with the prediction of Refs.~\cite{Bodwin:2002fk,Bodwin:2002kk}.

\subsection{Prospects at BaBar and Belle}

The BaBar and Belle detectors are accumulating ever larger data 
samples of charmonium that is produced directly in $e^+ e^-$ annihilation.
The simplicity of the initial state makes the theoretical analysis 
of this process particularly clean.  These two factors
make charmonium production in continuum $e^+ e^-$ annihilation a
particularly attractive process in which to compare theoretical
predictions with experiment.  The experimental results that have already 
emerged from these detectors provide further motivation for 
understanding this process.  There are 
significant discrepancies between previous measurements 
by BaBar and Belle.  There are also surprising results from Belle 
on double $c \bar c$ production that differ dramatically from theoretical
expectations.  The resolution of these problems will inevitably 
lead to progress in our understanding of charmonium production.


The surprising double-$c \bar c$ results from Belle include 
an inclusive measurement, the ratio $R_{\rm double}$ defined in 
Eq.~(\ref{prodsec:Rdouble}), and exclusive double-charmonium 
cross sections, such as $\sigma[e^+e^-\to J/\psi+\eta_c]$.
The discrepancies between theory and experiment in these measurements
are among the largest in the standard model. Theory and
experiment differ by about an order of magnitude---a discrepancy which
is larger than any known QCD $K$-factor.  It is important to
recognize that these discrepancies are problems not just for NRQCD
factorization, but for perturbative QCD in general. 
It is difficult to imagine how any perturbative calculation 
of $R_{\rm double}$ could give a value as large as 80\%.
With regard to the cross section for $e^+e^-\to J/\psi+\eta_c$,
the perturbative QCD formalism for exclusive processes 
\cite{brodsky-ji-lee} gives a result that reduces to that of
NRQCD factorization \cite{Braaten:2002fi,Liu:2002wq} in the
nonrelativistic limit and is well-approximated by it if one uses
realistic light-cone wave functions for $J/\psi$ and $\eta_c$.\footnote{
The Belle result can be accommodated by using asymptotic light-cone
wave functions that are appropriate for light hadrons
\cite{Ma:2004qf}, but there is no justification for using such
wave functions for charmonium.} 
Clearly, it is very important to have independent checks of the Belle
inclusive and exclusive double-$c \bar c$ results. If the Belle
results are confirmed, then we would be forced to entertain some
unorthodox possibilities: the inapplicability of perturbative QCD to
double-$c \bar c$ production, new charmonium production mechanisms
within the standard model, or perhaps even physics beyond the standard
model. 

The larger data samples that are now available should allow much more 
accurate measurements of the inclusive process $e^+ e^- \to J/\psi X$,
including the momentum distribution of the $J/\psi$ and its polarization.
The measurements of the $J/\psi$ momentum distribution may allow the
determination of all the relevant NRQCD matrix elements.  A comparison
with the NRQCD matrix elements measured at the Tevatron would then provide 
a test of their universality.  Once the NRQCD matrix elements 
are determined, they can be used to predict the polarization of the 
$J/\psi$ as a function of its momentum, which would provide a stringent 
test of the NRQCD predictions for spin.
Instead of imposing cuts to suppress contributions from radiative return,
virtual photon radiation, and two-photon collisions, it might be better 
to choose cuts in order to maximize the precision of the measurements,
without any regard to the production mechanism.  The contributions 
from other mechanisms could instead be taken into account in the 
theoretical analyses.  

The large data samples of BaBar and Belle should also allow 
measurements of the inclusive production of other charmonium states, 
such as the $\psi(2S)$ and the $\chi_c(1P)$.  Such measurements could be used 
to determine the NRQCD matrix elements for these charmonium states.  
They are also important because they provide constraints on the 
contributions to inclusive $J/\psi$ production from decays of higher 
charmonium states.

There are some straightforward improvements that could be made
in the theoretical predictions for inclusive charmonium production 
in $e^+ e^-$ annihilation.  For example, there are only a few 
components missing from a complete calculation of all 
contributions through second order in $\alpha_s$.
In the contribution from the 
color-octet ${}^3S_1$ channel, the virtual corrections at order 
$\alpha_s^2$ have not been calculated.  There are also color-octet 
contributions to $e^+ e^- \to c \bar c c \bar c$ at order $\alpha_s^2$ 
that have not been calculated.  The theoretical predictions 
for inclusive charmonium production could 
also be improved by treating more systematically the contributions 
from the feeddown from decays of higher charmonium states and from 
other mechanisms, such as radiative return, virtual photon radiation, 
and two-photon collisions.


\section{Charmonium production in $B$-meson decays}
\label{prodsec:bdecays}

$B$-meson decays are an excellent laboratory for studying 
charmonium production because $B$ mesons decay into 
charmonium with branching fractions greater than a percent.  
At a $B$ factory operating near the peak of the $\Upsilon(4S)$ 
resonance, about 25\% of the events consist of
a $B^+ B^-$ pair or a $B^0 \bar B^0$ pair.  The large sample 
of $B$ mesons accumulated by the CLEO experiment allowed 
the measurements of many exclusive and inclusive branching 
fractions into final states that include charmonium.
The Belle and BaBar experiments are accumulating even larger
samples of $B$ mesons, providing a new source of 
precise data on charmonium production in $B$ decays.

The inclusive branching fractions of $B$ mesons into charmonium states
can be measured most accurately for the mixture of $B^+$, $B^0$, and
their antiparticles that are produced in the decay of the $\Upsilon(4S)$
resonance \cite{Balest:1994jf,Chen:2000ri}. 
Those that have been measured are listed in
Table~\ref{prodsec:Brbcharm}. The fraction of $J/\psi$'s that come from
decay of $\chi_c$'s, which is defined in Eq.~(\ref{prod:chifrac}), is
$F_{\chi_c} = (11 \pm 2)$\%. This is significantly smaller than
the value that is measured at the Tevatron, 
which is given in Table~\ref{prodsec:Jpsifractions}.
The $\chi_{c1}$-to-$\chi_{c2}$ ratio,
which is defined in Eq.~(\ref{prod:chirat}), is $R_{\chi_c} = 5.1\pm 3.0$. 
Although the error bar is large, this ratio seems to be
substantially larger than the value that is measured
at the Tevatron, which is given in Eq.~(\ref{Rchi12Tev}),
and the values measured in fixed-target experiments, 
which are given in Table~\ref{tab:chi}. 
Such differences in $R_{\chi_c}$ and $F_{\chi_c}$
are contrary to the predictions of the color-evaporation model.


Inclusive branching fractions into charmonium states have also been
measured at LEP for the mixtures of $B^+$, $B^0$, $B^0_s$, $b$ baryons,
and their antiparticles that are produced in 
$Z^0$ decay \cite{Buskulic:1992wp,Adriani:1993ta,Abreu:1994rk}. 
This mixture of $b$ hadrons can be interpreted as the one that
arises from the fragmentation of a $b$ quark that is produced with 
a momentum of 45 GeV.
The branching fractions that
have been measured are listed in Table~\ref{prodsec:Brbcharm}. The
branching fraction into $\chi_{c1}(1P)$ seems to be significantly larger
than for the mixture from $\Upsilon(4S)$ decay. The difference could be
due to the contribution from $b$ baryons.
It is often assumed that the mixture of $b$ hadrons that is produced 
at high-energy hadron colliders, such as the Tevatron, 
is similar to that produced in $Z^0$ decay.  This assumption could be 
tested by measuring ratios of inclusive cross sections 
for charmonium states that come from the decays of $b$ 
hadrons at the Tevatron.


%%%%%%%%%%%%%%%%%%%%%%%%%%%%%%%%%%%%%%%%%%%%%%%%%%%%%%%%%%%%%%%%%%%%%%%%%%%%%%%%%%
\begin{table}[ht]
\begin{center}
\begin{tabular}{|c|cccc|} 
\hline
mixture & $J/\psi$ & $\psi(2S)$ & $\chi_{c1}(1P)$ & $\chi_{c2}(1P)$ \\ 
\hline
from $\Upsilon(4S)$ decay      
& $11.5 \pm 0.6$ &  $3.5 \pm 0.5$ &  $3.6 \pm 0.5$ &  $0.7 \pm 0.4$ \\ 
from $Z^0$ decay   
& $11.6 \pm 1.0$ &  $4.8 \pm 2.4$ & $11.5 \pm 4.0$ &              \\ 
\hline
\end{tabular}
\caption{Inclusive branching fractions (in units of $10^{-3}$)
for mixtures of $b$ hadrons to decay into charmonium states.
}
\label{prodsec:Brbcharm}
\end{center}
\end{table}
%%%%%%%%%%%%%%%%%%%%%%%%%%%%%%%%%%%%%%%%%%%%%%%%%%%%%%%%%%%%%%%%%%%%%%%%%%%%%%%%%%


The observed inclusive branching fractions of $B$ mesons into $J/\psi$ and
$\psi(2S)$ are larger than the predictions of the color-singlet model by
about a factor of three. Ko, Lee, and Song applied the NRQCD
factorization approach to the production of $J/\psi$ and $\psi(2S)$ 
in $B$ decays~\cite{Ko:1995iv}. The color-octet ${}^3S_1$ term in the
production rate is suppressed by a factor of $v^4$ that comes from
the NRQCD matrix element. However, the production rate also involves
Wilson coefficients that arise from evolving the effective weak
Hamiltonian from the scale $M_W$ to the scale $m_b$. The Wilson
coefficient for the color-octet ${}^3S_1$ term is significantly larger
than that for the color-singlet term, although the smallness of the
color-singlet term may be the result of an accidental cancellation that
occurs in the leading-order treatment of the evolution of the
coefficients.  Moreover, the color-singlet contribution is
decreased by the relativistic correction of order $v^2$. 
The inclusion of the color-octet
${}^3S_1$ term allows one to explain the factor of three discrepancy
between the data and the color-singlet-model prediction.

The observed branching fraction for decays of $B$ directly into
$J/\psi$, which excludes the feeddown from decays of $\psi(2S)$ and
$\chi_c$, is much larger than the prediction of the color-evaporation
model.  The CEM prediction for the direct branching fraction is
$0.24-0.66$~\cite{Ko:1999zx}, where the range comes from the uncertainty
in the CEM parameters. The CLEO collaboration has made a precise
measurement of the direct branching fraction of $B$ into
$J/\psi$~\cite{Anderson:2002md}: 
${\rm Br}_{\rm dir}[B \to J/\psi+X] = (0.813\pm 0.041)\%$.  
The CEM prediction is significantly 
smaller than the data. As we have already mentioned, the data
can be accommodated within the 
NRQCD factorization approach by including color-octet terms. 

Beneke, Maltoni, and Rothstein \cite{Beneke:1998ks} have calculated the
inclusive decay rates of $B$ mesons into $J/\psi$ and $\psi(2S)$ to
next-to-leading order in $\alpha_s$. They used their results to
extract NRQCD matrix elements from the data. Their results for the
linear combinations of NRQCD matrix elements defined in
Eq.~(\ref{prod:lincomb}) are $M_{3.1}^{J/\psi}=(1.5_{-1.1}^{+0.8})\times
10^{-2}\hbox{ GeV}^3$ and $M_{3.1}^{\psi(2S)}=(0.5\pm 0.5)\times
10^{-2}\hbox{ GeV}^3$. The uncertainties arise from experiment and from
imprecise of knowledge of the matrix elements $\langle {\cal
O}^H_8({}^3S_1)\rangle$ and $\langle {\cal O}^H_1({}^3S_1)\rangle$. Ma,
taking into account initial-state hadronic corrections, has extracted
slightly different linear combinations of matrix elements
\cite{Ma:2000bz}: $M_{3.4}^{J/\psi}=2.4\times 10^{-2}\hbox{ GeV}^3$ and
$M_{3.4}^{\psi(2S)}=1.0\times 10^{-2}\hbox{ GeV}^3$. In both
extractions, the color-octet matrix elements are considerably smaller
than those from the Tevatron fits, but the uncertainties are large.

The effects of color-octet terms on the polarization of $J/\psi$ in $B$
decays were considered by Fleming, Hernandez, Maksymyk, and
Nadeau~\cite{Fleming:1996pt} and by Ko, Lee, and Song~\cite{Ko:1999zx}.
The polarization variable $\alpha$ for $J/\psi$ production is defined 
by the angular distribution in Eq.~(\ref{prod:alphadef}).
In $B$ meson decays, the most convenient choice of the polarization axis 
is the direction of the boost vector from the $J/\psi$ rest frame 
to the rest frame of the $B$ meson.
The color-evaporation model predicts no polarization.  The predictions of
NRQCD factorization and of the color-singlet model depend on the mass
of the $b$ quark.  For $m_b = 4.7 \pm 0.3$~GeV, the prediction of
NRQCD factorization is $\alpha = -0.33 \pm 0.08$~\cite{Fleming:1996pt}
and the prediction of the color-singlet model is $\alpha = -0.40 \pm
0.07$~\cite{Fleming:1996pt}. The uncertainties that arise from $m_b$
have been added in quadrature with other uncertainties. We note that the 
uncertainty in $m_b$ that was used in this calculation is 
considerably larger than the uncertainty of 2.4\% that is given in 
Chapter~6. Measurements of
the polarization by the CLEO Collaboration have given the results
$\alpha=-0.30\pm 0.08$ for $J/\psi$ and $\alpha=-0.45\pm 0.30$ for
$\psi(2S)$~\cite{Anderson:2002md}. The result for $J/\psi$ strongly
disfavors the color-evaporation model and is consistent with the
predictions of the NRQCD factorization approach and the color-singlet
model.

Bodwin, Braaten, Yuan, and Lepage have applied the NRQCD factorization
approach to the production of the $P$-wave charmonium states $\chi_{cJ}$
in $B$ decays~\cite{Bodwin:1992qr}. For $P$-wave quarkonium production,
there is a color-octet NRQCD matrix element that is of the same order in
$v$ as the leading color-singlet matrix element. Therefore, the
factorization formula must include both the color-singlet $P$-wave and
the color-octet $S$-wave contributions. The short-distance coefficient
in the color-singlet ${}^3P_J$ term for $\chi_{cJ}$ production
vanishes at leading order in $\alpha_s$ for
$J=0,2$~\cite{Kuhn:1979zb,Kuhn:1983ar}. The color-octet ${}^3S_1$ term
for $\chi_{cJ}$ production is proportional to the number of spin
states $2J+1$. Thus, the relative importance of the color-singlet and
color-octet terms can vary dramatically among the three $\chi_{cJ}$
states.  The prediction of the color-singlet model at leading order 
in $\alpha_s$ that the direct production rate of $\chi_{c2}$ should 
vanish can be tested. 
The feeddown from $\psi(2S)$ decay contributes 
$(0.24 \pm 0.04) \times 10^{-3}$ to the
inclusive branching fraction into $\chi_{c2}$ given in
Table~\ref{prodsec:Brbcharm}.  The remainder 
$(0.5 \pm 0.4) \times 10^{-3}$ is consistent with zero, 
and hence it is compatible with the prediction of the
color-singlet model, but it is also compatible with a small
color-octet contribution.


\section{$B_c$ production}
\label{prodsec:Bc}

The $B_c$ and $B_c^*$ are the ground state and the first excited state
of the $\bar b c$ quarkonium system. Their total angular momentum and
parity quantum numbers are $J^P$ = $0^-$ and $1^+$, and their dominant
Fock states have the angular momentum quantum numbers ${}^1S_0$ and
${}^3S_1$, respectively. In the following discussion, we will refer to
general $\bar b c$ quarkonium states as $B_c$ mesons and use the terms
$B_c$ and $B_c^*$ specifically for the ground state and the first 
excited state.

In contrast to charmonium
and bottomonium states, which have ``hidden flavor,'' $B_c$ mesons
contain two explicit flavors. As a consequence, the $B_c$ decays only
through the weak interactions, and the $B_c^*$ decays through an
electromagnetic transition into the $B_c$ with almost $100\%$
probability. The higher-mass $B_c$ mesons below the $B D$ threshold
decay through hadronic and electromagnetic transitions into lower-mass
$B_c$ mesons with almost $100\%$ probability. They cascade down through
the $\bar b c$ spectrum, eventually producing a $B_c$ or a $B_c^*$.
Another consequence of the explicit flavors is that the most important
production mechanisms for $B_c$ mesons are completely different from
those for hidden-flavor quarkonia. In the production of $B_c$ mesons by
strong or electromagnetic interactions, two additional heavy quarks
$\bar{c}$ and $b$ are always produced. The production cross sections for
$B_c$ mesons are suppressed compared with the production cross sections
for hidden-flavor quarkonia because the leading-order diagrams are of
higher order in the coupling constants and also because the phase-space
is suppressed, owing to the presence of the additional heavy quarks.

The small cross sections for producing $B_c$ mesons make the
prospects for observing the $B_c$ at $e^+e^-$ and $ep$ colliders rather
bleak. A possible exception to this assessment exists for the case of
production at an $e^+e^-$ collider with energy at the $Z^0$ peak,
for which the production rate of the $B_c$ is predicted to be
marginal for observation \cite{Chang:bb}. The $B_c$ was not
discovered at LEP, despite careful searches
\cite{Abreu:1996nz,Barate:1997kk,Ackerstaff:1998zf}. It was finally
discovered at the Tevatron by the CDF collaboration in 1998
\cite{Abe:1998wi,Abe:1998fb}. We restrict our attention in the remainder
of this subsection to the production of $B_c$ mesons at hadron colliders.

The production of $B_c$ mesons was studied before the discovery of the
$B_c$
\cite{Chang:bb,Braaten:1993jn,Chang:1992jb,Chang:1994aw,Chang:1996jt,
Kolodziej:1995nv,Berezhnoy:an,Baranov:wy,Berezhnoi:1997fp}. If one
assumes that all nonperturbative effects in the production of the $B_c$
in hadron-hadron collisions can be absorbed into the hadrons' parton
distribution functions (PDF's), then the inclusive production cross
section can be written in the factorized form
%
\begin{eqnarray}
d\sigma[h_A h_B \rightarrow B_c + X] &=& \sum_{ij} 
\int dx_{1} dx_{2} \; f^{h_A}_i(x_{1},\mu) f^{h_B}_j(x_{2},\mu)\; 
d\hat{\sigma}[ij \rightarrow B_c + X] \;.
\label{prodsec:Bccross}
\end{eqnarray}
%
The NRQCD factorization formula for the parton-parton cross section is
%
\begin{eqnarray}
d\hat{\sigma}[ij \rightarrow B_{c} + X] =
\sum_n d\hat{\sigma}[ij \rightarrow (\bar b c)_n + X] \; 
\langle {\cal O}_n^{B_c} \rangle \;,
\label{prodsec:Bcsubcross}
\end{eqnarray}
%
where the sum is over 4-fermion operators that create and annihilate
$\bar b c$. At the leading order in $\alpha_s$, which is $\alpha_s^4$,
the parton-parton process is $i j \to (\bar b c) + b \bar{c}$, where $ij
= gg$ (gluon fusion) or $q \bar q$ (quark-antiquark annihilation). Since
$m_{B_c}> m_b > m_c\gg \Lambda_{QCD}$, the leading-order parton-parton
process involves only hard propagators, even at small $p_T$.
Nevertheless, because of soft-gluon interactions, for example between
the $B_c$ and the recoiling $b$ and $\bar c$ quarks, it is not clear
that hard-scattering factorization holds at small $p_T$.

According to the velocity-scaling rules of NRQCD, the matrix element for
$B_c$ production that is of leading order in $v$ is 
$\langle {\cal O}^{B_c}_1({}^1S_0) \rangle$. 
The vacuum-saturation approximation
can be used to show that it is proportional to $F_{B_c}^2$, where
$F_{B_c}$ is the decay constant of the $B_c$, up to corrections of order
$v^4$. The leading matrix element for $B_c^*$ production is 
$\langle {\cal O}^{B_c^*}_1({}^3S_1) \rangle$. The vacuum-saturation
approximation and heavy-quark spin symmetry can be used to show that
this matrix element is also proportional to $F_{B_c}^2$, up to
corrections of order $v^3$.  The leading color-octet matrix elements
are suppressed as $v^3$ or $v^4$. The color-octet terms in
(\ref{prodsec:Bcsubcross}) are probably not 
as important for $B_c$ mesons as
they are for hidden-flavor quarkonia. In the case of $J/\psi$
production, the short-distance coefficients of the color-octet matrix
elements are enhanced relative to those for the color-singlet matrix
element by an inverse power of the QCD-coupling $\alpha_s$, by factors
of $p_T/m_c$ at large $p_T$, by factors of $m_c/p_T$ at small $p_T$, and
by color factors. The only one of these enhancement factors that may
apply to the $B_c$ is the color factor. Because there are many Feynman
diagrams that contribute to the parton process $i j \to (\bar b c) + b
\bar c$ at order $\alpha_s^4$, the color correlations implied by
individual Feynman diagrams tend to average out. We therefore expect a
$\bar b c$ pair to be created in a color-octet state roughly eight times
as often as in a color-singlet state. We will assume that, in spite of
the enhancement from this color factor,  the color-octet contributions
to the production cross sections for $B_c$ and $B_c^*$ are small
compared with the leading color-singlet contributions. This assumption
is equivalent to using the color-singlet model to calculate the
production rate.
 
Two approaches have been used to compute the cross sections for $B_c$
mesons: the complete order-$\alpha_s^4$ approach
\cite{Chang:1992jb,Chang:1994aw,Chang:1996jt,%
Kolodziej:1995nv,Berezhnoy:an,Baranov:wy}
and the fragmentation approach \cite{Braaten:1993jn,Cheung:1999ir}.  In
the complete order-$\alpha_s^4$ approach, the parton cross section
in Eq.~(\ref{prodsec:Bcsubcross}) is computed at leading order in
$\alpha_s$, where the only subprocesses are 
$i j \to (\bar b c) + b \bar c $:
%
\begin{eqnarray}
d\hat{\sigma}[ij \rightarrow B_{c} + X] =
d\hat{\sigma}[ij \rightarrow \bar b c_1({}^1S_0) + b \bar c] \;
\langle  {\cal O}^{B_c}_1(^1S_0) \rangle \;.
\label{prodsec:BcsubcrossLO}
\end{eqnarray}
%
The fragmentation approach is based on the fact that,
for asymptotically large $p_T \gg M_{B_c}$, 
the cross section (\ref{prodsec:Bcsubcross})
can be further factored into a cross section for producing
$\bar b$ and a fragmentation function $D_{\bar b \to B_c}(z,\mu)$
that gives the probability for the $\bar b$ to fragment 
into a $B_c$ carrying a fraction $z$ of the $\bar b$ momentum:
%
\begin{eqnarray}
d\hat{\sigma}[ij \rightarrow B_{c} + X] \approx
\int dz \; d\hat{\sigma}[ij \rightarrow \bar b  + b ] \;
D_{\bar b \to B_c}(z,\mu) \;.
\label{prodsec:BcsubcrossFRAG}
\end{eqnarray}
%
If both factors are calculated only at leading order in $\alpha_s$, this
is just an approximation to the complete order-$\alpha_s^4$ cross
section in Eq.~(\ref{prodsec:Bcsubcross}). One advantage of the
fragmentation approach is that the expressions for the $\bar b$
production cross section $ d\hat{\sigma}$ and the fragmentation
function $D_{\bar b \to B_c}$ in Eq.~(\ref{prodsec:BcsubcrossFRAG})
can be written down in a few lines. For $p_T \gg m_{B_c}$, the
fragmentation approach has another advantage in that the
Altarelli-Parisi evolution equations can be used to sum the leading
logarithms of $p_T/m_c$ to all orders. Unfortunately, as was pointed
out in Ref.~\cite{Chang:1994aw,Chang:1996jt,Kolodziej:1995nv}, the
fragmentation cross section does not become an accurate approximation to
the complete order-$\alpha_s^4$ cross section until surprisingly
large values of $p_T$. For example, if the parton center-of-mass energy
is $\sqrt{\hat{s}}=200$ GeV, the fragmentation cross section is a
good approximation only for $p_T \geq 40$ GeV.  We will therefore not
consider the fragmentation approach further.

The authors of Ref.~\cite{Chang:2003cq} recently developed a Monte
Carlo event generator for hadronic $B_c$ and $B^*_c$ production,
using the complete order-$\alpha_s^4$ approach and taking advantage
of helicity amplitude techniques \cite{Mangano:1990by}. The generator
is a Fortran package, and  it is implemented in PYTHIA
\cite{Sjostrand:1993yb}, which allows one to generate complete events.
The complete order-$\alpha_s^4$ cross section includes contributions
from gluon-gluon fusion and  quark-antiquark annihilation. At the
Tevatron, the gluon-gluon fusion mechanism is dominant over 
quark-antiquark annihilation, except in certain kinematics regions
\cite{Chang:1992jb,Chang:2003cr}. At the LHC, the gluon-gluon fusion
mechanism is always dominant. All the results below are obtained
from the gluon-gluon fusion mechanism only.

The inputs that are required to calculate the complete
order-$\alpha_s^4$ cross sections are the masses $m_b, m_c$, and
$m_{B_c}$, the decay constant $F_{B_c}$, the PDF's, the QCD coupling
constant $\alpha_s$, and the factorization scale $Q$. The masses $m_b$
and $m_c$ are known with uncertainties of about 2.4\% and 8\%,
respectively.  In the NRQCD factorization approach, one sets $m_{B_c} =
m_c+m_b$ and $m_{B_c^*} = m_c+m_b$ in the short-distance coefficients.
Contributions from operators of higher order in $v$ then account for the
binding energy in $m_{B_c}$ and $m_{B_c^*}$. This procedure is also
required in order to preserve gauge invariance if one makes use of
on-shell spin-projection operators for the $B_c$ and $B_c^*$ states
\cite{Chang:1979nn,Guberina:1980fn}. Since an experimental measurement
of the decay constant $F_{B_c}$ is not available, one has to use a value
that is obtained from potential models
\cite{Chen:fq,Eichten:1994gt,Kiselev:1994rc,AbdEl-Hady:1999xh} or from
lattice gauge theory \cite{Davies:1996gi}. The uncertainty in the factor
$F_{B_c}^2$ is about 6\%. Since the order-$\alpha_s^4$ cross section is
at leading order in perturbation theory, the running of $\alpha_s$ can
be taken at leading order, and LO versions of the PDF's can be used.

The running coupling constant and the PDF's depend on the
renormalization/factorization scale $\mu$, and, so, a
prescription for the scale $\mu$ is also required. There is no general
rule for choosing the scale in an LO calculation, especially in the
case of a $2\to 3$ subprocess, such as $ij \to B_c + b \bar c$. The
factorization formula (\ref{prodsec:BcsubcrossFRAG}) for asymptotically
large $p_T$ suggests that an appropriate choice for the scales in the
fragmentation contribution to the cross section might be to set $\mu=
m_{bT}\equiv (m_b^2 + p_T^2)^{1/2}$ in the PDF's and $\alpha_s^4
=\alpha_s^2(m_{bT}) \alpha_s^2(m_c)$ in the parton cross section.
However, the fragmentation term does not dominate until very large
$p_T$, and there are important contributions to the cross section
that have nothing to do with fragmentation
\cite{Chang:1994aw,Chang:1996jt,Kolodziej:1995nv}. For example, there
are contributions that involve the splitting of one of the colliding
gluons into a $c \bar c$ pair, followed by the creation of a $b \bar b$
pair in the hard scattering of the $c$ from the other gluon and
then by the recombination of the $\bar b$ and $c$ into a $B_c$. 
The sensitivity to the choice of $\mu$ could be decreased by 
carrying out a complete
calculation of the production cross section for the $B_c$ at
next-to-leading order in $\alpha_s$, but this is, at present,
prohibitively difficult. In the absence of such a calculation, we
can use the variation in the complete order-$\alpha_s^4$ cross section 
for several reasonable choices for the scale as an estimate of the
uncertainty that arises from the choice of scale.


%%%%%%%%%%%%%%%%%%%%%%%%%%%%%%%%%%%%%%%%%%%%%%%%%%%%%%%%%%%%%%%%%%%%%%%%%%%%%%%%%%%%%%
\begin{table}
\begin{center}
\vskip 0.6cm
\begin{tabular}{|c|ccccc|ccccc|}
\hline\hline & \multicolumn{5}{|c}{~~~Tevatron~($\sqrt s=2$
TeV)~~~}& \multicolumn{5}{|c}{~~~LHC~($\sqrt s=14$ TeV)~~~}\\
\hline $m_c$ (GeV) & ~~$1.4$~~ & ~~$1.5$~~
&~~$1.6$~~&~~$1.7$~~ & ~~$1.8$~~ & ~~$1.4$~~
 & ~~$1.5$~~ &~~$1.6$~~ &~~$1.7$~~ &~~$1.8$~~\\
$\sigma[B_{c}]$ (nb) & 3.87 & 3.12 &
2.56 & 2.12 & 1.76 & 61.0 & 49.8 & 41.4 & 34.7 & 28.9 \\
$\sigma[B^{*}_c]$ (nb) & 9.53 & 7.39 & 5.92 &
4.77 & 3.87 & 153. & 121.& 97.5 & 80.0 & 66.2 \\
\hline\hline
\end{tabular}
\caption{The cross sections (in nb) for direct production of $B_c$ and
$B_c^*$ at the Tevatron and at the LHC for various values of the
charm-quark mass $m_c$. The gluon distribution function is CTEQ5L,
the running of $\alpha_s$ is leading order, the scale is
$\mu^{2}=\hat{s}/4$, and the other parameters are $F_{B_c}=480$ MeV, and
$m_b=4.9$ GeV.}
\label{prod:Bctab}
\end{center}
\end{table}
%%%%%%%%%%%%%%%%%%%%%%%%%%%%%%%%%%%%%%%%%%%%%%%%%%%%%%%%%%%%%%%%%%%%%%%%%%%%%%%%%%%%%%

The hadronic production cross-section for $B_c$ mesons depends strongly
on the collision energy. In Table \ref{prod:Bctab}, we give the direct
cross sections for $B_c$ and $B_c^*$  production at the Tevatron and the
LHC for several values of the charm quark mass $m_c$ and for typical
values for the other parameters. The cross section for $B_c$ production
at the LHC is larger than at the Tevatron by a factor of about 16. The
cross sections for $B_c^*$ production are larger than those for $B_c$
production by a factor of about 2.4. The cross sections are fairly
sensitive to the charm-quark mass, varying by more than a factor of two
as $m_c$ is varied from 1.4 to 1.8 GeV. In Fig.~\ref{prod:Bcfig}, we
show the differential cross-sections for $B_c$ production as a function
of the $B_c$ transverse momentum $p_T$ and $B_c$ rapidity $y$ at the
Tevatron and the LHC, using four different prescriptions for the scale
$\mu$. At central rapidity, the variations among the four choices of
scale is about a factor of three at the Tevatron and a factor of two at
the LHC. The differential cross-sections decrease more slowly with $p_T$
and $|y|$ at the LHC than at the Tevatron. The total uncertainty from
combining all of the uncertainties in the direct cross section for $B_c$
production is less than an order of magnitude. The uncertainty in the
ratio of the direct-production cross sections for the $B_c^*$  and the
$B_c$ is much smaller because many of the uncertainties cancel in the
ratio.

%%%%%%%%%%%%%%%%%%%%%%%%%%%%%%%%%%%%%%%%%%%%%%%%%%%%%%%%%%%%%%%%%%%%%%%%%%%%%%%%%%%%%%
\begin{figure}
\centering
\hfill\includegraphics[width=0.43\textwidth]{cteqQpt.eps}%
\includegraphics[width=0.43\textwidth]{cteqQy.eps}\hspace*{\fill}
\caption{
The differential cross sections for the direct production
of the $B_c$ as a function of its
transverse momentum $p_T$ and its rapidity $y$ 
at the Tevatron ($\sqrt{s}=2$ TeV) and at the LHC
($\sqrt{s}=14$ TeV) for four choices of the scale:
$\mu^2=\hat{s}/4$ (solid line), 
$\mu^2=p_{T}^2+m_{B_c}^2$ (dotted line),
$\mu^2=\hat{s}$ (dashed line),
and $\mu^2=p_{Tb}^2+m_b^2$ (dash-dot line).
The gluon distribution is CTEQ5L, 
the running of $\alpha_s$ is leading order, 
and the other parameters are $F_{B_c}=480$ MeV, $m_c=1.5$ GeV, 
and $m_b=4.9$ GeV.} 
\label{prod:Bcfig}
\end{figure}
%%%%%%%%%%%%%%%%%%%%%%%%%%%%%%%%%%%%%%%%%%%%%%%%%%%%%%%%%%%%%%%%%%%%%%%%%%%%%%%%%%%%%%

The results presented above are for the direct production of the $B_c$
and the $B_c^*$. Experiments at the Tevatron and the LHC will measure
the inclusive cross sections, including the feeddown from all of the
higher states of the $\bar b c$ system. The $\bar b c$ system has a rich
spectrum of excited states below the $B D$ threshold.  They include an
additional $S$-wave  multiplet, one or two $P$-wave multiplets, and a
$D$-wave multiplet. After being produced, these excited $B_c$ mesons all
cascade eventually down to the ground state $B_c$. Since the $B_c^*$
decays into the $B_c$ with a probability of almost 100\%, the feeddown
from directly-produced $B_c^*$'s increases the cross section for the
$B_c$ by about a factor of 3.4. The complete order-$\alpha_s^4$ cross
sections for $B_c$ and $B_c^*$ production can be applied equally well to
the $2S$ multiplet. The direct-production cross sections for these
states are smaller than those for the $1S$ states by the ratio of the
squares of the wave functions at the origin, which is about 0.6. Thus,
the inclusive cross section for $B_c$ production, including the effect
of feeddown from the direct production of all of the $S$-wave  $B_c$
states, is larger than the cross section for direct $B_c$ production,
which is given in Table~\ref{prod:Bctab} and shown in
Fig.~\ref{prod:Bcfig}, by a factor of about 5.4. 

%%%%%%%%%%%%%%%%%%%%%%%%%%%%%%%%%%%%%%%%%%%%%%%%%%%%%%%%%%%%%%%%%%%%%%%%%%%%%%%%%%%%%%
\begin{figure}[htb]
\begin{center}
\epsfig{figure=bc_xsec.eps,width=10cm}
\caption{The ratio $R[J/\psi l\nu]$, which is defined in 
Eq.~(\ref{prodsec:Bcratio}), versus the $B_c$ lifetime.
The point, surrounded by a one-standard-deviation contour, shows the
values of $R[J/\psi l\nu]$ and the $B_c$ lifetime 
that were measured by CDF \cite{Abe:1998wi,Abe:1998fb}.
The shaded region represents the theoretical predictions and their
uncertainty bands from Refs.~\cite{Lusignoli1,Scora} for two different
values of the semileptonic width $\Gamma_{s.l.} = \Gamma[B_c \rightarrow
J/\psi l \nu]$. }
\label{fig-bcxsec_cdf}
\end{center}
\end{figure}
%%%%%%%%%%%%%%%%%%%%%%%%%%%%%%%%%%%%%%%%%%%%%%%%%%%%%%%%%%%%%%%%%%%%%%%%%%%%%%%%%%%%%%

The production of $B_c$ in $p \bar p$ collisons at $\sqrt{s} = 1.8$ TeV 
has been measured at the Tevatron by the CDF collaboration 
\cite{Abe:1998wi,Abe:1998fb}. 
CDF has measured the ratio
%
\begin{eqnarray}
R[J/\psi l\nu] = 
{\sigma[B_c] {\rm Br}[B_c^+ \rightarrow J/\psi l^+\nu]
\over \sigma[B^+] {\rm Br}[B^+ \rightarrow J/\psi K^+]}
\label{prodsec:Bcratio}
\end{eqnarray}
%
for $B_c^+$ and $B^+$ with transverse momenta $p_T >$ 6.0 GeV 
and with rapidities $|y| <$ 1.0. 
Their result is $R[J/\psi l\nu] = 0.132^{+0.041}_{-0.037}(\rm stat.)
\pm 0.031 (\rm syst.) ^{+0.032}_{-0.020}(\rm lifetime)$. 
This result is consistent with results from previous 
searches \cite{Abreu:1996nz,Barate:1997kk,Ackerstaff:1998zf}. 
Fig.~\ref{fig-bcxsec_cdf} compares the CDF
measurements of $R[J/\psi l\nu]$ and the $B_c$ lifetime
with theoretical predictions from Refs.~\cite{Lusignoli1,Scora}
for two different values of the semileptonic width 
$\Gamma_{s.l.} = \Gamma[B_c \rightarrow J/\psi l \nu]$. 
The theoretical predictions use the values 
$|V_{cb}| = 0.041 \pm 0.005$~\cite{PDG96}, 
$\sigma[B^+_c]/\sigma[\bar{b}] = 1.3 \times 10^{-3}$~\cite{Lusignoli2}, 
$\sigma[B^+]/\sigma[\bar{b}] = 0.378 \pm 0.022$~\cite{PDG96}, 
and ${\rm Br}[B^+ \rightarrow J/\psi K^+] 
        = (1.01 \pm 0.14) \times 10^{-3}$~\cite{PDG96}.
The predictions and the measurement are consistent within 
experimental and theoretical uncertainties.  



Quantitative predictions for the contribution to the inclusive $B_c$
production cross section from the feeddown from $P$-wave states would
require complete knowledge of the order-$\alpha_s^4$ cross sections for
the production of $P$-wave states. It is theoretically inconsistent to
use the color-singlet model to calculate these cross sections for the
$P$-wave states. There are color-octet terms in the $P$-wave production
cross sections that are of the same order in both $v$ and $\alpha_s$ as
the color-singlet terms, and they must be included. The color-singlet
production matrix elements for the $P$-wave states can be estimated from
potential models or determined from lattice gauge theory.  The
color-octet production matrix elements for the $P$-wave states can
perhaps be estimated by interpolating between the corresponding matrix
elements for charmonium and bottomonium states.

In summary, the order-$\alpha_s^4$ color-singlet production cross
section for $S$-wave $\bar b c$ mesons can be used to predict the $B_c$
production cross section, including feeddown from excited $S$-wave
states.  The uncertainty in the normalization of that prediction is less
than an order of magnitude. If the inclusive cross section for $B_c$
production that is measured at the Tevatron or the LHC is much larger
than that prediction, it could indicate that there is a large
contribution from the feeddown from $P$-wave or
higher-orbital-angular-momentum states. It could also indicate that the
color-octet contributions to the direct production of the $B_c$ and the
$B_c^*$ are important. 



\section{Summary and outlook}
\label{prodsec:summary}

NRQCD factorization, together with hard-scattering factorization,  
provides a systematic formalism for computing
inclusive quarkonium production rates in QCD. 
Nonperturbative effects associated with the binding of
a $Q \bar Q$ pair into a quarkonium are factored into
NRQCD matrix elements that scale in a definite manner with 
the typical relative velocity $v$ of the heavy quark in the quarkonium.
The NRQCD matrix elements are predicted to be universal, {\it i.e.},
independent of the process that creates the $Q \bar Q$ pair. The NRQCD
factorization formula for inclusive cross sections is believed to hold
when $p_T \gg \Lambda_{\rm QCD}$, where $p_T$ is the transverse momentum
of the quarkonium with respect to the colliding particles.  It is
well-motivated by the effective field theory NRQCD and by factorization
theorems that have been proven for simpler hard-scattering processes. 
Explicit proofs of factorization for quarkonium production would be
welcome, because they would help quantify the sizes of corrections to
the factorization formula. It is important to bear in mind that
conventional proofs of hard-scattering factorization fail at small
$p_T$. Consequently, NRQCD factorization formulas, even those that
include soft-gluon resummation, may be unreliable in this region. It
also follows that the NRQCD factorization approach may not be applicable
to total cross sections if they are dominated by contributions at
small $p_T$.

The NRQCD factorization approach incorporates elements of both the 
color-singlet model and the color-evaporation model.  It includes the 
color-singlet model terms, for which the NRQCD matrix elements can be 
determined from annihilation decays.  It also includes color-octet 
production mechanisms, as in the color-evaporation model.  The NRQCD 
factorization approach extends these models into a 
systematically improvable framework.  The color-singlet model is
emphatically ruled out by the observation of prompt $J/\psi$ and 
$\psi(2S)$ production at the Tevatron at rates that are more 
than an order 
of magnitude larger than the color-singlet-model predictions.  
The color-evaporation model is ruled out by the observations 
of nonzero polarization of $J/\psi$'s 
in $B$ meson decays and in $e^+e^-$ annihilation at 10.6 GeV and by 
the observation of nonzero polarization of $\Upsilon(2S)$'s and 
$\Upsilon(3S)$'s in fixed-target experiments.   
It is also ruled out by the fact that different values of
the fraction of $J/\psi$'s that come from $\chi_c$ decays
are measured at the Tevatron and in $B$-meson decays.
Despite having been ruled out, 
the color-singlet model and the color-evaporation model can 
still play  useful roles as ``straw men" with which to compare the
predictions of NRQCD factorization. The color-evaporation model 
has not yet been ruled out, for example,  
as a description of differential cross
sections at the Tevatron and in fixed-target experiments.

The NRQCD factorization approach provides a general phenomenological 
framework that cannot be ruled out easily.  The factorization 
formula involves infinitely many NRQCD matrix elements, most of which are 
adjustable parameters.  It is only the truncation in $v$ that reduces 
those parameters to a finite set. The standard truncation has 
four independent NRQCD matrix 
elements for each S-wave multiplet and two independent NRQCD matrix 
elements 
for each P-wave multiplet.  NRQCD factorization with the standard 
truncation in $v$ remains a phenomenologically viable description of 
inclusive quarkonium production.  As one tests NRQCD factorization at
higher levels of precision, the standard truncation must ultimately
fail. The NRQCD factorization approach itself may remain viable if one
truncates at a higher order in $v$, but only at the cost of
introducing many new adjustable parameters.

In the effort to make the predictions of the NRQCD factorization 
approach more quantitative, the most 
urgent need is to extend all calculations to next-to-leading order (NLO) in 
$\alpha_s$.  For hadron collisions at small $p_T$ ($p_T \ll m$), the 
leading-order parton process is $ij \to Q \bar Q$. NLO calculations of 
that process are 
already available, but a resummation of multiple gluon emissions is required 
in order to tame large logarithms of $m^2/p_T^2$ and 
to turn the singular $p_T$ distribution into a smooth distribution.  For 
very large $p_T$ ($p_T \gg m$), the production of quarkonium is dominated 
by gluon fragmentation.  The leading-order fragmentation process is 
$g \to Q \bar Q_8 ( ^3S_1 )$, and the NLO calculation of the gluon 
fragmentation function into $Q \bar Q$ is available.  What is still lacking 
is the NLO calculation at intermediate $p_T$, for which the 
leading-order parton process is $ij \to  Q \bar Q + k$.  By taking 
into account the NLO 
corrections in $\alpha_s$, one should significantly decrease some of the 
uncertainties in the NRQCD factorization predictions.

One problematic source of uncertainties in the NRQCD factorization 
predictions is relativistic corrections.  The first relativistic 
corrections of order $v^2$ in the channel that corresponds to the 
color-singlet model have been calculated for many processes.  In 
many cases, they have large coefficients that cast doubt on 
the validity of the 
expansion in powers of $v$ for charmonium, and even for bottomonium.  
The success of lattice NRQCD in describing bottomonium spectroscopy 
ensures the 
applicability of the velocity expansion for this system in some form.  It 
is possible that some reorganization or resummation of the velocity 
expansion might be necessary in order to make precise quantitative 
calculations of charmonium production.

The best individual experiments for determining the NRQCD production 
matrix elements for both charmonium and bottomonium are probably 
those at the Tevatron, because of the large range of $p_T$
that is accessible. It will be important to take advantage of the
measurements down to small $p_T$ that were achieved at the CDF
detector for bottomonium in Run I and for charmonium in Run II.  This
will require taking
into account the effects of multiple gluon emission in the theoretical
analysis.  
Measurements of charmonium production in other experiments
are also important because they provide tests of the universality of the
production matrix elements. These experiments include those that measure
charmonium production in $ep$ collisions at HERA,
in $e^+ e^-$ annihilation at the $B$ factories, and in
$B$ meson decays at the $B$ factories.  
One can use these experiments to extract values of
the NRQCD matrix elements or, as has typically been the practice to
date, one can use the matrix elements that have been extracted from the
Tevatron data to make predictions for charmonium production in other  
experiments.

The ratios of the production cross sections for different quarkonium
states may also provide important tests of NRQCD factorization.
(Here, particularly, one must keep in mind the {\it caveats} about the
applicability of the NRQCD factorization approach to total cross
sections.) The uncertainties in the predictions for ratios of cross
sections are much smaller than those in the individual cross
sections because many of the uncertainties cancel in the
ratio. The variations of the ratios from process to process and as
functions of kinematic variables provide important information
about the production mechanisms.  The ratios of production rates of
spin-triplet $S$-wave states, such as the 
$\psi(2S)$--to--$J/\psi$ ratio, do not seem to vary much. 
However, a significant variation has been observed in a ratio 
of the production rates of $P$-wave and $S$-wave states, 
namely the fraction of $J/\psi$'s that come from decays of $\chi_c$'s. 
A substantial variation has also been observed
in a ratio of production rates of $P$-wave states, 
namely the $\chi_{c1}$-to-$\chi_{c2}$ ratio. 
More precise measurements of these and other
ratios would be valuable.  Of particular importance would be
measurements of ratios of production rates of spin-singlet and
spin-triplet quarkonium states, such as the 
$\eta_c$-to-$J/\psi$ ratio.
 The absence of clean signatures for spin-singlet quarkonium
states makes such measurements difficult.

The polarization of quarkonium is another important test of NRQCD
factorization.  The standard truncation in $v$ leads to unambiguous
predictions for the ratios of production rates of different
spin states,  without introducing any new parameters.  The
predictions are most easily tested for the quarkonium
states with $J^{PC} = 1^{--}$, but they can also be tested for other 
states.  It is extremely important to test the simple qualitative 
predictions that in hadron collisions the $1^{--}$ states should become 
transversely polarized at sufficiently large $p_T$.  More careful 
quantitative estimates of the polarization of the $J/\psi$, the 
$\psi(2S)$, and the
$\Upsilon (nS)$ as functions of $p_T$ at the Tevatron and the LHC
would be useful.  More precise measurements of the polarization of
the $J/\psi$ and the $\psi(2S)$ in other production processes,
such as $ep$ collisions, $e^+e^-$ annihilations, and $B$ decay, would
also be valuable.

The most puzzling experimental results in quarkonium production in recent 
years have been the double-$c \bar c$ results from $e^+e^-$ annihilation at 
the $B$ factories.  The measurements by the Belle collaboration of the 
fraction of $J/\psi$'s that are accompanied by charmed hadrons and 
of the 
exclusive cross section for $J/\psi + \eta_c$ production are both much 
larger than 
expected.  No satisfactory theoretical explanation of these results has 
emerged.  It would be worthwhile to measure the fraction of $J/\psi$'s 
accompanied by charm hadrons in other processes, such as $p \bar p$ 
annihilation at the Tevatron and $ep$ collisions at HERA, to see if there 
are any surprises.

The outlook for progress in understanding quarkonium production is very 
bright.  The NRQCD factorization approach provides a general framework for 
describing inclusive quarkonium production.  Current experiments will 
provide severe tests of NRQCD factorization with the standard truncation of 
the velocity expansion.  These tests will either provide a firm 
foundation for predictions of quarkonium production in future experiments
or lead us to new insights into the physics of quarkonium production.



\begin{thebibliography}{999}

%%% section1: Formalism for inclusive quarkonium production

%%% 1.1 NRQCD factorization method 

%\cite{Caswell:1985ui}
\bibitem{Caswell:1985ui}
W.~E.~Caswell and G.~P.~Lepage,
%``Effective Lagrangians For Bound State Problems In QED, QCD, And Other
%Field Theories,''
Phys.\ Lett.\ B {\bf 167} (1986) 437.
%%CITATION = PHLTA,B167,437;%%

%\cite{Thacker:1990bm}
\bibitem{Thacker:1990bm}
B.~A.~Thacker and G.~P.~Lepage,
%``Heavy Quark Bound States In Lattice QCD,''
Phys.\ Rev.\ D {\bf 43} (1991) 196.
%%CITATION = PHRVA,D43,196;%%

%\cite{Bodwin:1994jh}
\bibitem{Bodwin:1994jh}
G.~T.~Bodwin, E.~Braaten and G.~P.~Lepage,
%``Rigorous QCD analysis of inclusive annihilation and production of heavy
%quarkonium,''
Phys.\ Rev.\ D {\bf 51} (1995) 1125
[Erratum-ibid.\ D {\bf 55} (1997) 5853]
[hep-ph/9407339].
%%CITATION = HEP-PH 9407339;%%

%% \cite{bks-charm-bottom} --> \cite{Bodwin:1993wf,Bodwin:1994js,Bodwin:1996tg,Bodwin:1996mf}
%\cite{Bodwin:1993wf}
\bibitem{Bodwin:1993wf}
G.~T.~Bodwin, S.~Kim, and D.~K.~Sinclair,
%``Matrix elements for the decays of S and P wave quarkonium:
%An exploratory study,''
Nucl.\ Phys.\ B (Proc.\ Suppl.)  {\bf 34} (1994) 434.
%%CITATION = NUPHZ,34,434;%%

%\cite{Bodwin:1994js}
\bibitem{Bodwin:1994js}
G.~T.~Bodwin, S.~Kim, and D.~K.~Sinclair,
%``Decays rates for S wave and P wave bottomonium,''
Nucl.\ Phys.\ Proc.\ Suppl.\
{\bf 42} (1995) 306
[hep-lat/9412011].
%%CITATION = HEP-LAT 9412011;%%

%\cite{Bodwin:1996tg}
\bibitem{Bodwin:1996tg}
G.~T.~Bodwin, D.~K.~Sinclair, and S.~Kim,
%``Quarkonium decay matrix elements from quenched lattice QCD,''
Phys.\ Rev.\ Lett.\  {\bf 77} (1996) 2376
[hep-lat/9605023].
%%CITATION = HEP-LAT 9605023;%%

%\cite{Bodwin:1996mf}
\bibitem{Bodwin:1996mf}
G.~T.~Bodwin, D.~K.~Sinclair, and S.~Kim,
%``Lattice calculation of quarkonium decay matrix elements,''
Int.\ J.\ Mod.\ Phys.\ A {\bf 12} (1997) 4019
[hep-ph/9609371].
%%CITATION = HEP-PH 9609371;

%\cite{Bodwin:2001mk}
\bibitem{Bodwin:2001mk}
G.~T.~Bodwin, D.~K.~Sinclair, and S.~Kim,
%``Bottomonium decay matrix elements from lattice QCD with two light
%quarks,''
Phys.\ Rev.\ D {\bf 65} (2002) 054504
[hep-lat/0107011].
%%CITATION = HEP-LAT 0107011;%%

\bibitem{qiu-sterman}
J.-w.~Qiu and G.~Sterman (private communication).

\bibitem{Beneke:1997qw}
M.~Beneke, I.~Z.~Rothstein, and M.~B.~Wise,
%``Kinematic enhancement of non-perturbative corrections to quarkonium  
%production,''
Phys.\ Lett.\ B {\bf 408} (1997) 373
[hep-ph/9705286].
%%CITATION = HEP-PH 9705286;%%

%\cite{Fleming:2003gt}
\bibitem{Fleming:2003gt}
S.~Fleming, A.~K.~Leibovich and T.~Mehen,
%``Resumming the color-octet contribution to e+ e- $\to$ J/psi + X,''
Phys.\ Rev.\ D {\bf 68} (2003) 094011 
[hep-ph/0306139].
%%CITATION = HEP-PH 0306139;%%

\bibitem{Beneke:1999gq}
M.~Beneke, G.~A.~Schuler, and S.~Wolf,
%``Quarkonium momentum distributions in photoproduction and B decay,''
Phys.\ Rev.\ D {\bf 62} (2000) 034004
[hep-ph/0001062].
%%CITATION = HEP-PH 0001062;%%

%%% 1.2 Color-singlet model

%\cite{Einhorn:1975ua}
\bibitem{Einhorn:1975ua}
M.~B.~Einhorn and S.~D.~Ellis,
%``Hadronic Production Of The New Resonances: Probing Gluon
%Distributions,''
Phys.\ Rev.\ D {\bf 12} (1975) 2007.
%%CITATION = PHRVA,D12,2007;%%

%\cite{Ellis:1976fj}
\bibitem{Ellis:1976fj}
S.~D.~Ellis, M.~B.~Einhorn, and C.~Quigg,
%``Comment On Hadronic Production Of Psions,''
Phys.\ Rev.\ Lett.\  {\bf 36} (1976) 1263.
%%CITATION = PRLTA,36,1263;%%

%\cite{Carlson:1976cd}
\bibitem{Carlson:1976cd}
C.~E.~Carlson and R.~Suaya,
%``Hadronic Production Of Psi / J Mesons,''
Phys.\ Rev.\ D {\bf 14} (1976) 3115.
%%CITATION = PHRVA,D14,3115;%%

%\cite{Kuhn:1979kb}
\bibitem{Kuhn:1979kb}
J.~H.~K\"uhn,
%``Hadronic Production Of P Wave Charmonium States,''
Phys.\ Lett.\ B {\bf 89} (1980) 385.
%%CITATION = PHLTA,B89,385;%%

%\cite{DeGrand:wf}
\bibitem{DeGrand:wf}
T.~A.~DeGrand and D.~Toussaint,
%``The Decay Of B Quarks Into Psi's,''
Phys.\ Lett.\ B {\bf 89} (1980) 256.
%%CITATION = PHLTA,B89,256;%%

%\cite{Kuhn:1979zb}
\bibitem{Kuhn:1979zb}
J.~H.~K\"uhn, S.~Nussinov, and R.~R\"uckl,
%``Charmonium Production In B Decays,''
Z.\ Phys.\ C {\bf 5} (1980) 117.
%%CITATION = ZEPYA,C5,117;%%

%\cite{Wise:1979tp}
\bibitem{Wise:1979tp}
M.~B.~Wise,
%``An Estimate Of J / Psi Production In B Decays,''
Phys.\ Lett.\ B {\bf 89} (1980) 229.
%%CITATION = PHLTA,B89,229;%%

%\cite{Chang:1979nn}
\bibitem{Chang:1979nn}
C.~H.~Chang,
%``Hadronic Production Of J / Psi Associated With A Gluon,''
Nucl.\ Phys.\ B {\bf 172} (1980) 425.
%%CITATION = NUPHA,B172,425;%%

%\cite{Baier:1981zz}
\bibitem{Baier:1981zz}
R.~Baier and R.~R\"uckl,
%``On Inelastic Leptoproduction Of Heavy Quarkonium States,''
Nucl.\ Phys.\ B {\bf 201} (1982) 1.
%%CITATION = NUPHA,B201,1;%%

%\cite{Baier:1981uk}
\bibitem{Baier:1981uk}
R.~Baier and R.~R\"uckl,
%``Hadronic Production Of J / Psi And Upsilon: Transverse Momentum
%Distributions,''
Phys.\ Lett.\ B {\bf 102} (1981) 364.
%%CITATION = PHLTA,B102,364;%%

%\cite{Berger:1980ni}
\bibitem{Berger:1980ni}
E.~L.~Berger and D.~L.~Jones,
%``Inelastic Photoproduction Of J / Psi And Upsilon By Gluons,''
Phys.\ Rev.\ D {\bf 23} (1981) 1521.
%%CITATION = PHRVA,D23,1521;%%

%\cite{Baier:1983va}
\bibitem{Baier:1983va}
R.~Baier and R.~R\"uckl,
%``Hadronic Collisions: A Quarkonium Factory,''
Z.\ Phys.\ C {\bf 19} (1983) 251.
%%CITATION = ZEPYA,C19,251;%%

%\cite{Keung:1981gs}
\bibitem{Keung:1981gs}
W.~Y.~Keung,
%``Inclusive Quarkonium Production,''
Print-81-0161 (BNL)
%\href{http://www.slac.stanford.edu/spires/find/hep/www?r=print-81-0161\%2F(bnl)}{SPIRES entry}
{\it Presented at Z0 Physics Workshop, Ithaca, N.Y., Feb 6-8, 1981}.

\bibitem{Schuler:1994hy}
G.~A.~Schuler,
%``Quarkonium production and decays,''
hep-ph/9403387.

%%% 1.3 Color-evaporation model

\bibitem{Fritzsch:1977ay}
H.~Fritzsch,
%``Producing Heavy Quark Flavors In Hadronic Collisions: A Test Of Quantum
%Chromodynamics,''
Phys.\ Lett.\ B {\bf 67} (1977) 217.
%%CITATION = PHLTA,B67,217;%%

%\cite{Halzen:1977rs}
\bibitem{Halzen:1977rs}
F.~Halzen,
%``Cvc For Gluons And Hadroproduction Of Quark Flavors,''
Phys.\ Lett.\ B {\bf 69} (1977) 105.
%%CITATION = PHLTA,B69,105;%%

%\cite{Gluck:1977zm}
\bibitem{Gluck:1977zm}
M.~Gl\"uck, J.~F.~Owens and E.~Reya,
%``Gluon Contribution To Hadronic J / Psi Production,''
Phys.\ Rev.\ D {\bf 17} (1978) 2324.
%\cite{Barger:1979js}
\bibitem{Barger:1979js}
V.~D.~Barger, W.~Y.~Keung and R.~J.~Phillips,
%``On Psi And Upsilon Production Via Gluons,''
Phys.\ Lett.\ B {\bf 91} (1980) 253.
%%CITATION = PHLTA,B91,253;%%

%\cite{Gavai:1994in}
\bibitem{Gavai:1994in}
R.~Gavai, D.~Kharzeev, H.~Satz, G.~A.~Schuler, K.~Sridhar and R.~Vogt,
%``Quarkonium production in hadronic collisions,''
Int.\ J.\ Mod.\ Phys.\ A {\bf 10} (1995) 3043
[hep-ph/9502270].
%%CITATION = HEP-PH 9502270;%%

%\cite{Schuler:1996ku}
\bibitem{Schuler:1996ku}
G.~A.~Schuler and R.~Vogt,
%``Systematics of quarkonium production,''
Phys.\ Lett.\ B {\bf 387} (1996) 181
[hep-ph/9606410].
%%CITATION = HEP-PH 9606410;%%

%\cite{Mangano:kq}
\bibitem{Mangano:kq}
M.~L.~Mangano, P.~Nason and G.~Ridolfi,
%``Fixed Target Hadroproduction Of Heavy Quarks,''
Nucl.\ Phys.\ B {\bf 405} (1993) 507.
%%CITATION = NUPHA,B405,507;%%

%\cite{Amundson:1996qr}
\bibitem{Amundson:1996qr}
J.~F.~Amundson, O.~J.~P.~Eboli, E.~M.~Gregores and F.~Halzen,
%``Quantitative tests of color evaporation: Charmonium production,''
Phys.\ Lett.\ B {\bf 390} (1997) 323
[hep-ph/9605295].
%%CITATION = HEP-PH 9605295;%%

\bibitem{Edin:1997zb}
A.~Edin, G.~Ingelman and J.~Rathsman,
%``Quarkonium production at the Tevatron through soft colour interactions,''
Phys.\ Rev.\ D {\bf 56} (1997) 7317
[hep-ph/9705311].
%%CITATION = HEP-PH 9705311;%%

\bibitem{Eboli:1998xx}
O.~J.~P.~Eboli, E.~M.~Gregores and F.~Halzen,
%``Inelastic photoproduction at HERA: A second charmonium crisis?,''
Phys.\ Lett.\ B {\bf 451} (1999) 241
[hep-ph/9802421].
%%CITATION = HEP-PH 9802421;%%

\bibitem{Eboli:2003fr}
O.~J.~P.~Eboli, E.~M.~Gregores and F.~Halzen,
%``Color evaporation description of inelastic photoproduction of J/ psi at DESY
%HERA,''
Phys.\ Rev.\ D {\bf 67} (2003) 054002.
%%CITATION = PHRVA,D67,054002;%%

\bibitem{Eboli:2003qg}
O.~J.~P.~Eboli, E.~M.~Gregores and J.~K.~Mizukoshi,
%``Testing color evaporation in photon photon production of J/psi at CERN LEP
%II,''
Phys.\ Rev.\ D {\bf 68} (2003) 094009
[hep-ph/0308121].
%%CITATION = HEP-PH 0308121;%%

\bibitem{Eboli:2001hc}
O.~J.~P.~Eboli, E.~M.~Gregores and F.~Halzen,
%``Soft color enhancement of the production of J/psi's by neutrinos,''
Phys.\ Rev.\ D {\bf 64} (2001) 093015
[hep-ph/0107026].
%%CITATION = HEP-PH 0107026;%%

\bibitem{Gregores:1996ek}
E.~M.~Gregores, F.~Halzen and O.~J.~P.~Eboli,
%``Prompt charmonium production in Z decays,''
Phys.\ Lett.\ B {\bf 395} (1997) 113
[hep-ph/9607324].
%%CITATION = HEP-PH 9607324;%%

%\cite{Berger:2004cc}
\bibitem{Berger:2004cc}
E.~L.~Berger, J.~w.~Qiu, and Y.~l.~Wang,
%``Transverse momentum distribution of upsilon production in hadronic
%collisions,''
hep-ph/0404158.
%%CITATION = HEP-PH 0404158;%%

%%% 1.4 Multiple gluon emission

%\cite{Collins:1984kg}
\bibitem{Collins:1984kg}
J.~C.~Collins, D.~E.~Soper and G.~Sterman,
%``Transverse Momentum Distribution In Drell-Yan Pair And W And Z Boson
%Production,''
Nucl.\ Phys.\ B {\bf 250} (1985) 199.
%%CITATION = NUPHA,B250,199;%%

%\cite{Contopanagos:1996nh}
\bibitem{Contopanagos:1996nh}
H.~Contopanagos, E.~Laenen and G.~Sterman,
%``Sudakov factorization and resummation,''
Nucl.\ Phys.\ B {\bf 484} (1997) 303
[hep-ph/9604313].
%%CITATION = HEP-PH 9604313;%%

%\cite{Cacciari:1998it}
\bibitem{Cacciari:1998it}
see e.g.\ M.~Cacciari, M.~Greco, and P.~Nason,
%``The p(T) spectrum in heavy-flavour hadroproduction,''
JHEP {\bf 9805} (1998) 007
[hep-ph/9803400].
%%CITATION = HEP-PH 9803400;%%

%\cite{Paige:fb}
\bibitem{Paige:fb}
F.~E.~Paige and S.~D.~Protopopescu,
%``Isajet: A Monte Carlo Event Generator For Isabelle, Version 2,''
BNL-29777.
%\href{http://www.slac.stanford.edu/spires/find/hep/www?r=bnl-29777}

%\cite{Paige:2003mg}
\bibitem{Paige:2003mg}
F.~E.~Paige, S.~D.~Protopescu, H.~Baer, and X.~Tata,
%``ISAJET 7.69: A Monte Carlo event generator for p p, anti-p p, and e+ 
%e- reactions,''
hep-ph/0312045.
%%CITATION = HEP-PH 0312045;%%

%\cite{Sjostrand:1985ys}
\bibitem{Sjostrand:1985ys}
T.~Sj\"ostrand,
%``The Lund Monte Carlo For Jet Fragmentation And E+ E- Physics: Jetset
%Version 6.2,''
Comput.\ Phys.\ Commun.\  {\bf 39} (1986) 347.
%%CITATION = CPHCB,39,347;%%

\bibitem{Sjostrand:1986hx}
T.~Sj\"ostrand and M.~Bengtsson,
%``The Lund Monte Carlo For Jet Fragmentation And E+ E- Physics: Jetset
%Version 6.3: An Update,''
Comput.\ Phys.\ Commun.\  {\bf 43} (1987) 367.
%%CITATION = CPHCB,43,367;%%

%\cite{Marchesini:1987cf}
\bibitem{Marchesini:1987cf}
G.~Marchesini and B.~R.~Webber,
%``Monte Carlo Simulation Of General Hard Processes With Coherent QCD
%Radiation,''
Nucl.\ Phys.\ B {\bf 310} (1988) 461.
%%CITATION = NUPHA,B310,461;%

%\cite{Marchesini:1991ch}
\bibitem{Marchesini:1991ch}
G.~Marchesini, B.~R.~Webber, G.~Abbiendi, I.~G.~Knowles, M.~H.~Seymour,
and L.~Stanco,
%``HERWIG: A Monte Carlo event generator for simulating hadron emission
%reactions with interfering gluons. Version 5.1 - April 1991,''
Comput.\ Phys.\ Commun.\  {\bf 67} (1992) 465.
%%CITATION = CPHCB,67,465;%%

%\cite{Frixione:2002ik}
\bibitem{Frixione:2002ik}
S.~Frixione and B.~R.~Webber,
%``Matching NLO QCD computations and parton shower simulations,''
JHEP {\bf 0206} (2002) 029
[hep-ph/0204244].
%%CITATION = HEP-PH 0204244;%%

%\cite{Frixione:2002bd}
\bibitem{Frixione:2002bd}
S.~Frixione and B.~R.~Webber,
%``The MC@NLO event generator,''
hep-ph/0207182.
%%CITATION = HEP-PH 0207182;%%

%\cite{Frixione:2003ei}
\bibitem{Frixione:2003ei}
S.~Frixione, P.~Nason and B.~R.~Webber,
%``Matching NLO QCD and parton showers in heavy flavour production,''
JHEP {\bf 0308} (2003) 007
[hep-ph/0305252].
%%CITATION = HEP-PH 0305252;%%

%\cite{Frixione:2004wy}
\bibitem{Frixione:2004wy}
S.~Frixione and B.~R.~Webber,
%``The MC@NLO 2.3 event generator,''
hep-ph/0402116.
%%CITATION = HEP-PH 0402116;%%

\bibitem{Soper:2003ya}
D.~E.~Soper,
%``Next-to-leading order QCD calculations with parton showers. II: Soft
%singularities,''
Phys.\ Rev.\ D {\bf 69} (2004) 054020
[hep-ph/0306268].
%%CITATION = HEP-PH 0306268;%%

%\cite{Kramer:2003jk}
\bibitem{Kramer:2003jk}
M.~Kr\"amer and D.~E.~Soper,
%``Next-to-leading order QCD calculations with parton showers. I: Collinear
%singularities,''
Phys.\ Rev.\ D {\bf 69} (2004) 054019
[hep-ph/0306222].
%%CITATION = HEP-PH 0306222;%%

%%% 1.5 Production in nuclear matter

\bibitem{Bodwin:1988fs}
G.~T.~Bodwin, S.~J.~Brodsky, and G.~P.~Lepage,
%``Effects Of Initial State QCD Interactions In The Drell-Yan Process,''
Phys.\ Rev.\ D {\bf 39} (1989) 3287.
%%CITATION = PHRVA,D39,3287;%%

\bibitem{Qiu:2001hj}
J.~w.~Qiu and G.~Sterman,
%``QCD and rescattering in nuclear targets,''
Int.\ J.\ Mod.\ Phys.\ E {\bf 12} (2003) 149
[hep-ph/0111002].
%%CITATION = HEP-PH 0111002;%%

\bibitem{Bodwin:1981fv}
G.~T.~Bodwin, S.~J.~Brodsky, and G.~P.~Lepage,
%``Initial State Interactions And The Drell-Yan Process,''
Phys.\ Rev.\ Lett.\  {\bf 47} (1981) 1799.
%%CITATION = PRLTA,47,1799;%%

%\cite{Bodwin:1984hc}
\bibitem{Bodwin:1984hc}
G.~T.~Bodwin,
%``Factorization Of The Drell-Yan Cross-Section In Perturbation Theory,''
Phys.\ Rev.\ D {\bf 31} (1985) 2616
[Erratum-ibid.\ D {\bf 34} (1986) 3932].
%%CITATION = PHRVA,D31,2616;%%

%%% Section 2 Quarkonium production at the Tevatron

%%% 2.1. Charmonium cross sections 

%\cite{Abe:1997jz}
\bibitem{Abe:1997jz}
F.~Abe {\it et al.}  [CDF Collaboration],
%``J/psi and psi(2S) production in p anti-p collisions at s**(1/2) =  
%1.8-TeV,''
Phys.\ Rev.\ Lett.\  {\bf 79} (1997) 572.
%%CITATION = PRLTA,79,572;%%

%\cite{Abe:1997yz}
\bibitem{Abe:1997yz}
F.~Abe {\it et al.}  [CDF Collaboration],
%``Production of J/psi mesons from chi/c meson decays in p anti-p  collisions at
%s**(1/2) = 1.8-TeV,''
Phys.\ Rev.\ Lett.\  {\bf 79} (1997) 578.
%%CITATION = PRLTA,79,578;%%

\bibitem{Klein:2003vd}
S.~R.~Klein and J.~Nystrand,
%``Photoproduction of quarkonium in proton proton and nucleus nucleus
%collisions,''
Phys.\ Rev.\ Lett.\  {\bf 92} (2004) 142003
[hep-ph/0311164].
%%CITATION = HEP-PH 0311164;%%

%% \bibitem{kramer}
%\cite{Kramer:2001hh}
\bibitem{Kramer:2001hh}
M.~Kr\"amer,
%``Quarkonium production at high-energy colliders,''
Prog.\ Part.\ Nucl.\ Phys.\  {\bf 47} (2001) 141
[hep-ph/0106120].
%%CITATION = HEP-PH 0106120;%%

\bibitem{Braaten:1995vv}
E.~Braaten and S.~Fleming,
%``Color octet fragmentation and the psi-prime surplus at the Tevatron,''
Phys.\ Rev.\ Lett.\  {\bf 74} (1995) 3327
[hep-ph/9411365].
%%CITATION = HEP-PH 9411365;%%

%\cite{Beneke:1996yw}
\bibitem{Beneke:1996yw}
M.~Beneke and M.~Kr\"amer,
%``Direct J/psi and psi' polarization and cross-sections at the  
%Tevatron,''
Phys.\ Rev.\ D {\bf 55} (1997) 5269
[hep-ph/9611218].

\bibitem{Buchmuller:1980su}
W.~Buchm\"uller and S.~H.~Tye,
%``Quarkonia And Quantum Chromodynamics,''
Phys.\ Rev.\ D {\bf 24} (1981) 132.
%%CITATION = PHRVA,D24,132;%%

\bibitem{Eichten:1995ch}
E.~J.~Eichten and C.~Quigg,
%``Quarkonium wave functions at the origin,''
Phys.\ Rev.\ D {\bf 52} (1995) 1726
[hep-ph/9503356].
%%CITATION = HEP-PH 9503356;%%

%\cite{Lai:1999wy}
\bibitem{Lai:1999wy}
H.~L.~Lai {\it et al.}  [CTEQ Collaboration],
%``Global {QCD} analysis of parton structure of the nucleon: CTEQ5 
%parton  distributions,''
Eur.\ Phys.\ J.\ C {\bf 12} (2000) 375
[hep-ph/9903282].
%%CITATION = HEP-PH 9903282;%%

\bibitem{Petrelli:1997ge}
A.~Petrelli, M.~Cacciari, M.~Greco, F.~Maltoni, and M.~L.~Mangano,
%``NLO production and decay of quarkonium,''
Nucl.\ Phys.\ {\bf B514} (1998) 245
[hep-ph/9707223].
%%CITATION = HEP-PH 9707223;%%

%\cite{Maltoni:2000km}
\bibitem{Maltoni:2000km}
F.~Maltoni,
%``Quarkonium decays and production in NRQCD,''
hep-ph/0007003.
%%CITATION = HEP-PH 0007003;%%

%\cite{Cho:1995vh}
\bibitem{Cho:1995vh}
P.~L.~Cho and A.~K.~Leibovich,
%``Color octet quarkonia production,''
Phys.\ Rev.\ D {\bf 53} (1996) 150
[hep-ph/9505329].
%%CITATION = HEP-PH 9505329;%%

%\cite{Cho:1995ce}
\bibitem{Cho:1995ce}
P.~L.~Cho and A.~K.~Leibovich,
%``Color-octet quarkonia production II,''
Phys.\ Rev.\ D {\bf 53} (1996) 6203
[hep-ph/9511315].
%%CITATION = HEP-PH 9511315;%%

\bibitem{Martin:1992zi}
A.~D.~Martin, W.~J.~Stirling, and R.~G.~Roberts,
%``Parton distributions updated,''
Phys.\ Lett.\ B {\bf 306} (1993) 145;
{\bf 309} (1993) 492 (erratum).
%%CITATION = PHLTA,B306,145;%%

%\cite{Lai:1996mg}
\bibitem{Lai:1996mg}
H.~L.~Lai {\it et al.},
%``Improved parton distributions from global analysis of recent deep  inelastic
%scattering and inclusive jet data,''
Phys.\ Rev.\ D {\bf 55} (1997) 1280
[hep-ph/9606399].
 %%CITATION = HEP-PH 9606399;%%

\bibitem{Gluck:1994uf}
M.~Gl\"uck, E.~Reya and A.~Vogt,
%``Dynamical parton distributions of the proton and small x physics,''
Z.\ Phys.\ C {\bf 67} (1995) 433.
%%CITATION = ZEPYA,C67,433;%%

\bibitem{Martin:1996as}
A.~D.~Martin, R.~G.~Roberts and W.~J.~Stirling,
%``Parton distributions: A study of the new HERA data, alpha(s),  the 
%gluon and p anti-p jet production,''
Phys.\ Lett.\ B {\bf 387} (1996) 419
[hep-ph/9606345].
%%CITATION = HEP-PH 9606345;%%

% \bibitem{BKL}
%\cite{Braaten:1999qk}
\bibitem{Braaten:1999qk}
E.~Braaten, B.~A.~Kniehl and J.~Lee,
%``Polarization of prompt J/psi at the Tevatron,''
Phys.\ Rev.\ D {\bf 62} (2000) 094005
[hep-ph/9911436].
%%CITATION = HEP-PH 9911436;%%

\bibitem{Martin:1998sq}
A.~D.~Martin, R.~G.~Roberts, W.~J.~Stirling, and R.~S.~Thorne,
%``Parton distributions: A new global analysis,''
Eur.\ Phys.\ J.\ C {\bf 4} (1998) 463
[hep-ph/9803445].
%%CITATION = HEP-PH 9803445;%%

%\cite{Sanchis-Lozano:1999um}
\bibitem{Sanchis-Lozano:1999um}
M.~A.~Sanchis-Lozano,
%``New extraction of color-octet NRQCD matrix elements from charmonium
%hadroproduction,''
Nucl.\ Phys.\ Proc.\ Suppl.\  {\bf 86} (2000) 543
[hep-ph/9907497].
%%CITATION = HEP-PH 9907497;%%

%\cite{Tung:ua}
\bibitem{Tung:ua}
W.~K.~Tung,
%``Perspectives On Global QCD Analysis,''
{\it Prepared for International Workshop on Deep Inelastic Scattering 
and Related Subjects, Eilat, Israel, 6-11 Feb 1994}.

%\cite{Kniehl:1998qy}
\bibitem{Kniehl:1998qy}
B.~A.~Kniehl and G.~Kramer,
%``TEVATRON-HERA colour-octet charmonium anomaly versus higher-order {QCD}
%effects,''
Eur.\ Phys.\ J.\ C {\bf 6} (1999) 493
[hep-ph/9803256].
%%CITATION = HEP-PH 9803256;%%

%\cite{Petrelli:1999rh}
\bibitem{Petrelli:1999rh}
A.~Petrelli,
%``J/psi production: Tevatron and fixed-target collisions,''
Nucl.\ Phys.\ Proc.\ Suppl.\  {\bf 86} (2000) 533
[hep-ph/9910274].
%%CITATION = HEP-PH 9910274;%%

%% \bibitem{SMS} 
%\cite{Sridhar:1998rt}
\bibitem{Sridhar:1998rt}
K.~Sridhar, A.~D.~Martin and W.~J.~Stirling,
%``J/psi production at the Tevatron and HERA: The effect of k(T) smearing,''
Phys.\ Lett.\ B {\bf 438} (1998) 211
[hep-ph/9806253].
%%CITATION = HEP-PH 9806253;%%

%\cite{Hagler:2000eu}
\bibitem{Hagler:2000eu}
P.~H\"agler, R.~Kirschner, A.~Sch\"afer, L.~Szymanowski and O.~V.~Teryaev,
%``Direct J/psi hadroproduction in k(T)-factorization and the color octet
%mechanism,''
Phys.\ Rev.\ D {\bf 63} (2001) 077501
[hep-ph/0008316].
%%CITATION = HEP-PH 0008316;%%

%\cite{Kwiecinski:1997ee}
\bibitem{Kwiecinski:1997ee}
J.~Kwiecinski, A.~D.~Martin and A.~M.~Stasto,
%``A unified BFKL and GLAP description of F2 data,''
Phys.\ Rev.\ D {\bf 56} (1997) 3991
[hep-ph/9703445].
%%CITATION = HEP-PH 9703445;%%

%\cite{Yuan:2000cp}
\bibitem{Yuan:2000cp}
F.~Yuan and K.~T.~Chao,
%``Color-singlet direct J/psi and psi' production at Tevatron in the k(t)
%factorization approach,''
Phys.\ Rev.\ D {\bf 63} (2001) 034006
[hep-ph/0008302].
%%CITATION = HEP-PH 0008302;%%

%\cite{Yuan:2000qe}
\bibitem{Yuan:2000qe}
F.~Yuan and K.~T.~Chao,
%``Polarizations of J/psi and psi' in hadroproduction at Tevatron in the k(t)
%factorization approach,''
Phys.\ Rev.\ Lett.\  {\bf 87} (2001) 022002
[hep-ph/0009224].
%%CITATION = HEP-PH 0009224;%%

%\cite{Beneke:1995yb}
\bibitem{Beneke:1995yb}
M.~Beneke and I.~Z.~Rothstein,
%``Psi' polarization as a test of colour octet quarkonium production,''
Phys.\ Lett.\ B {\bf 372} (1996) 157
[Erratum-ibid.\ B {\bf 389} (1996) 769]
[hep-ph/9509375].
%%CITATION = HEP-PH 9509375;%%

%\cite{Ma:1995ci}
\bibitem{Ma:1995ci}
J.~P.~Ma,
%``Gluon fragmentation into P wave triplet quarkonium,''
Nucl.\ Phys.\ {\bf B447} (1995) 405
[hep-ph/9503346].
%%CITATION = HEP-PH 9503346;%%

%\cite{Braaten:2000pc}
\bibitem{Braaten:2000pc}
E.~Braaten and J.~Lee,
%``Next-to-leading order calculation of the color octet (3)S(1) gluon  
%fragmentation function for heavy quarkonium,''
Nucl.\ Phys.\ B {\bf 586} (2000) 427 
[hep-ph/0004228].
%%CITATION = HEP-PH 0004228;%%

\bibitem{Cacciari:1994dr}
M.~Cacciari and M.~Greco,
%``J / psi production via fragmentation at the Tevatron,''
Phys.\ Rev.\ Lett.\  {\bf 73} (1994) 1586
[hep-ph/9405241].
%%CITATION = HEP-PH 9405241;%%

%\cite{Braaten:1994xb}
\bibitem{Braaten:1994xb}
E.~Braaten, M.~A.~Doncheski, S.~Fleming, and M.~L.~Mangano,
%``Fragmentation production of J / psi and psi-prime at the Tevatron,''
Phys.\ Lett.\ B {\bf 333} (1994) 548
[hep-ph/9405407].
%%CITATION = HEP-PH 9405407;%%

\bibitem{Roy:1994ie}
D.~P.~Roy and K.~Sridhar,
%``Fragmentation contribution to quarkonium production in hadron collision,''
Phys.\ Lett.\  {\bf B339} (1994) 141
[hep-ph/9406386].
%%CITATION = HEP-PH 9406386;%%

%\cite{Affolder:2001ij}
\bibitem{Affolder:2001ij}
T.~Affolder {\it et al.}  [CDF Collaboration],
%``Production Of Chi/C1 And Chi/C2 In P Anti-P Collisions At S**(1/2) = 
%1.8-Tev,''
Phys.\ Rev.\ Lett.\  {\bf 86} (2001) 3963.
%%CITATION = PRLTA,86,3963;%%

\bibitem{maltoni-chi-ratio} 
F.~Maltoni (unpublished).

\bibitem{CDF_LP03}
http://www-cdf.fnal.gov/physics/new/bottom/030904.blessed-bxsec-jpsi/.

\bibitem{D0_LP03}
http://www-d0.fnal.gov/Run2Physics/ckm/approved\_results/approved\_results.html;
http://www-d0.fnal.gov/Run2Physics/ckm/Moriond\_2003/index2.html.

\bibitem{CDF_Iforw}
D.~Acosta  {\it et al.}  [CDF Collaboration],
Phys.\ Rev.\ D {\bf 66} (2002) 092001.

\bibitem{D0_Iforw}
B.~Abbott  {\it et al.}  [D0 Collaboration],
Phys.\ Rev.\ Lett.\  {\bf 82} (1999) 35.

\bibitem{Cacciari:2004}
M.~Cacciari, S.~Frixione, M.~L.~Mangano, P.~Nason and G.~Ridolfi,
%``QCD analysis of first b cross section data at 1.96-TeV,''
JHEP {\bf 0407} (2004) 033
[hep-ph/0312132].
%%CITATION = HEP-PH 0312132;%%

%%% 2.2 Bottomonium cross sections

%\cite{Acosta:2001gv}
\bibitem{Acosta:2001gv}
D.~Acosta {\it et al.}  [CDF Collaboration],
%``Upsilon production and polarization in p anti-p collisions at  s**(1/2) =
%1.8-TeV,''
Phys.\ Rev.\ Lett.\  {\bf 88} (2002) 161802.
%%CITATION = PRLTA,88,161802;%%

%\cite{Affolder:2000nn}
\bibitem{Affolder:2000nn}
T.~Affolder {\it et al.}  [CDF Collaboration],
%``Measurement of J/psi and psi(2S) polarization in p anti-p collisions  at
%s**(1/2) = 1.8-TeV,''
Phys.\ Rev.\ Lett.\  {\bf 85} (2000) 2886
[hep-ex/0004027].
%%CITATION = HEP-EX 0004027;%%

\bibitem{LHC-workshop}
M.~Kr\"amer and F.~Maltoni,
in 'Bottom Production',
P.~Nason, G.~Ridolfi, O.~Schneider, G.F.~Tartarelli, P.~Vikas et al.,
[hep-ph/0003142], published in CERN-YR-2000/01, G.~Altarelli and
M.L.~Mangano editors.
%%CITATION = HEP-PH 0003142;%%

%\cite{Braaten:2000cm}
\bibitem{Braaten:2000cm}
E.~Braaten, S.~Fleming, and A.~K.~Leibovich,
%``NRQCD analysis of bottomonium production at the Tevatron,''
Phys.\ Rev.\ D {\bf 63} (2001) 094006
[hep-ph/0008091].
%%CITATION = HEP-PH 0008091;%%

%\cite{Domenech:2000ri}
\bibitem{Domenech:2000ri}
J.~L.~Domenech and M.~A.~Sanchis-Lozano,
%``Results from bottomonia production at the Tevatron and prospects for  the
%LHC,''
Nucl.\ Phys.\ B {\bf 601} (2001) 395
[hep-ph/0012296].

%\cite{Abe:1995an}
\bibitem{Abe:1995an}
F.~Abe {\it et al.}  [CDF Collaboration],
%``Upsilon production in p anti-p collisions at s**(1/2) = 1.8-TeV,''
Phys.\ Rev.\ Lett.\  {\bf 75} (1995) 4358.
%%CITATION = PRLTA,75,4358;%%

%%% 2.3 Polarization

%\cite{Cho:1994ih}
\bibitem{Cho:1994ih}
P.~L.~Cho and M.~B.~Wise,
%``Spin symmetry predictions for heavy quarkonia alignment,''
Phys.\ Lett.\ B {\bf 346} (1995) 129
[hep-ph/9411303].
%%CITATION = HEP-PH 9411303;%%

%\cite{Leibovich:1996pa}
\bibitem{Leibovich:1996pa}
A.~K.~Leibovich,
%``Psi' polarization due to color-octet quarkonia production,''
Phys.\ Rev.\ D {\bf 56} (1997) 4412
[hep-ph/9610381].
%%CITATION = HEP-PH 9610381;%%

%\cite{Bodwin:2003wh}
\bibitem{Bodwin:2003wh}
G.~T.~Bodwin and J.~Lee,
%``Relativistic corrections to gluon fragmentation into spin-triplet
% S-wave quarkonium,''
Phys.\ Rev.\ D {\bf 69} (2004) 054003 
[hep-ph/0308016].
%%CITATION = HEP-PH 0308016;%%

\bibitem{Beneke:1997av}
M.~Beneke,
%``Nonrelativistic effective theory for quarkonium production in hadron
%collisions,''
hep-ph/9703429.
%%CITATION = HEP-PH 9703429;%%

\bibitem{Brambilla:1999xf}
N.~Brambilla, A.~Pineda, J.~Soto and A.~Vairo,
%``Potential NRQCD: An effective theory for heavy quarkonium,''
Nucl.\ Phys.\ B {\bf 566} (2000) 275
[hep-ph/9907240].

\bibitem{Fleming:2000ib}
S.~Fleming, I.~Z.~Rothstein and A.~K.~Leibovich,
%``Power counting and effective field theory for charmonium,''
Phys.\ Rev.\ D {\bf 64} (2001) 036002 
[hep-ph/0012062].
%%CITATION = HEP-PH 0012062;%%
 
\bibitem{Sanchis-Lozano:2001rr}
M.~A.~Sanchis-Lozano,
%``Power scaling rules for charmonia production and HQEFT,''
Int.\ J.\ Mod.\ Phys.\ A {\bf 16} (2001) 4189 
[hep-ph/0103140].
%%CITATION = HEP-PH 0103140;%%

\bibitem{Brambilla:2002nu}
N.~Brambilla, D.~Eiras, A.~Pineda, J.~Soto and A.~Vairo,
%``Inclusive decays of heavy quarkonium to light particles,''
Phys.\ Rev.\ D {\bf 67} (2003) 034018
[hep-ph/0208019].
%%CITATION = HEP-PH 0208019;%%

\bibitem{Marchal:2000wd}
N.~Marchal, S.~Peigne and P.~Hoyer,
%``Quarkonium production through hard comover scattering. II,''
Phys.\ Rev.\ D {\bf 62} (2000) 114001
[hep-ph/0004234].
%%CITATION = HEP-PH 0004234;%%
 
\bibitem{Maul:2001fw}
M.~Maul,
%``Quarkonium production through hard comover rescattering in polarized  and unpolarized p p scattering,''
Nucl.\ Phys.\ B {\bf 594} (2001) 89
[hep-ph/0009279].
%%CITATION = HEP-PH 0009279;%%

%\cite{Braaten:2000gw}
\bibitem{Braaten:2000gw}
E.~Braaten and J.~Lee,
%``Polarization of Upsilon(nS) at the Tevatron,''
Phys.\ Rev.\ D {\bf 63} (2001) 071501
[hep-ph/0012244].
%%CITATION = HEP-PH 0012244;%%

%\cite{Cropp:1999ub}
\bibitem{Cropp:1999ub}
R.~Cropp  [CDF collaboration],
%``Recent results on J/psi, psi(2S) and Upsilon production at CDF,''
hep-ex/9910003.
%%CITATION = HEP-EX 9910003;%%

%\cite{Papadimitriou:2001bb}
\bibitem{Papadimitriou:2001bb}
V.~Papadimitriou  [CDF Collaboration],
%``Quarkonia production and polarization studies with CDF,''
Int.\ J.\ Mod.\ Phys.\ A {\bf 16S1A} (2001) 160.
%%CITATION = IMPAE,A16S1A,160;%%

\bibitem{Mathews:1998nk}
P.~Mathews, P.~Poulose and K.~Sridhar,
%``eta/c production at the Tevatron: A test of NR{QCD},''
Phys.\ Lett.\  {\bf B438} (1998) 336
[hep-ph/9803424].
%%CITATION = HEP-PH 9803424;%%
 
\bibitem{Sridhar:1996vd}
K.~Sridhar,
%``(1)P(1) charmonium production at the Tevatron,''
Phys.\ Rev.\ Lett.\  {\bf 77} (1996) 4880
[hep-ph/9609285].
%%CITATION = HEP-PH 9609285;%%

\bibitem{Qiao:1997wb}
C.~Qiao, F.~Yuan and K.~Chao,
%``Gluon fragmentation to (3)D(J) quarkonia,''
Phys.\ Rev.\ D {\bf 55} (1997) 5437
[hep-ph/9701249].
%%CITATION = HEP-PH 9701249;%%
 
\bibitem{Kim:1997bb}
C.~S.~Kim, J.~Lee and H.~S.~Song,
%``Color-octet contributions in the associate J/psi + gamma  hadroproduction,''
Phys.\ Rev.\  {\bf D55} (1997) 5429
[hep-ph/9610294].
%%CITATION = HEP-PH 9610294;%%
 
\bibitem{Mathews:1999ye}
P.~Mathews, K.~Sridhar and R.~Basu,
%``J/psi + gamma production at the LHC,''
Phys.\ Rev.\  {\bf D60} (1999) 014009
[hep-ph/9901276].
%%CITATION = HEP-PH 9901276;%%
 
\bibitem{Barger:1996vx}
V.~Barger, S.~Fleming and R.~J.~Phillips,
%``Double gluon fragmentation to $J/\psi$ pairs at the Tevatron,''
Phys.\ Lett.\ B {\bf 371} (1996) 111
[hep-ph/9510457].
%%CITATION = HEP-PH 9510457;%%

\bibitem{Maltoni:2004hv}
F.~Maltoni and A.~D.~Polosa,
%``Observation potential for eta/b at the Tevatron,''
hep-ph/0405082.
%%CITATION = HEP-PH 0405082;%%

\bibitem{Braaten:1999th}
E.~Braaten, J.~Lee and S.~Fleming,
%``Associated production of Upsilon and weak gauge bosons at the Tevatron,''
Phys.\ Rev.\  {\bf D60} (1999) 091501
[hep-ph/9812505].
%%CITATION = HEP-PH 9812505;%%

%%% Section 3: Quarkonium production in fixed target experiments

%%% 3.1 Cross sections 

\bibitem{Beneke:1996tk}
M.~Beneke and I.~Z.~Rothstein,
%``Hadro-production of Quarkonia in Fixed Target Experiments,''
Phys.\ Rev.\ D {\bf 54} (1996) 2005
[Erratum-ibid.\ D {\bf 54} (1996) 7082]
[hep-ph/9603400].
%%CITATION = HEP-PH 9603400;%%

\bibitem{Tang:1996rm}
W.~K.~Tang and M.~Vanttinen,
%``Color-Octet $J/\psi$ Production at Low $p_\perp$,''
Phys.\ Rev.\ D {\bf 54} (1996) 4349
[hep-ph/9603266].
%%CITATION = HEP-PH 9603266;%%

\bibitem{Gupta:1996ut}
S.~Gupta and K.~Sridhar,
 %``Colour-octet Contributions to J/psi Hadroproduction at Fixed Target
%Energies,''
Phys.\ Rev.\ D {\bf 54} (1996) 5545
[hep-ph/9601349].
%%CITATION = HEP-PH 9601349;%%

\bibitem{Tzamarias:1990ij}
C.~Akerlof {\it et al.},
%``Psi Production And Anti-P N And Pi- N Interactions At 125-Gev/C And 
%A Determination Of The Gluon Structure Functions Of The Anti-P And The 
%Pi-,''
Phys.\ Rev.\ D {\bf 48} (1993) 5067.
%%CITATION = PHRVA,D48,5067;%%

\bibitem{Schub:1995pu}
M.~H.~Schub {\it et al.}  [E789 Collaboration],
%``Measurement of J / psi and psi-prime production in 800-GeV/c proton 
%- gold
%collisions,''
Phys.\ Rev.\ D {\bf 52} (1995) 1307
[Erratum-ibid.\ D {\bf 53} (1996) 570].
%%CITATION = PHRVA,D52,1307;%%

\bibitem{Alexopoulos:1995dt}
T.~Alexopoulos {\it et al.}  [E771 Collaboration],
%``Measurement of J / psi, psi-prime and upsilon total cross-sections 
%in 800-GeV/c p - Si interactions,''
Phys.\ Lett.\ B {\bf 374} (1996) 271.
%%CITATION = PHLTA,B374,271;%%

%\cite{Eidelman:2004wy}
\bibitem{Eidelman:2004wy}
S.~Eidelman {\it et al.}  [Particle Data Group Collaboration],
%``Review of particle physics,''
Phys.\ Lett.\ B {\bf 592} (2004) 1.
%%CITATION = PHLTA,B592,1;%%

\bibitem{Gupta:1997me}
S.~Gupta and P.~Mathews,
%``Higher orders in the colour-octet model of J/psi production,''
Phys.\ Rev.\ D {\bf 56} (1997) 3019
[hep-ph/9703370].
%%CITATION = HEP-PH 9703370;%%

\bibitem{Bauer:yf}
D.~A.~Bauer {\it et al.},
%``Differences Between Proton And Pi- Induced Production Of The 
%Charmonium Chi
%States,''
Phys.\ Rev.\ Lett.\  {\bf 54} (1985) 753.
%%CITATION = PRLTA,54,753;%%

\bibitem{Antoniazzi:1993yf}
L.~Antoniazzi {\it et al.}  [E705 Collaboration],
%``Production of chi charmonium via 300-GeV/c pion and proton 
%interactions on a lithium target,''
Phys.\ Rev.\ D {\bf 49} (1994) 543.
%%CITATION = PHRVA,D49,543;%%

\bibitem{Alexopoulos:1999wp}
T.~Alexopoulos {\it et al.}  [E771 Collaboration],
%``Hadroproduction of the chi1 and chi2 states of charmonium in 
%800-GeV/c
%proton silicon interactions,''
Phys.\ Rev.\ D {\bf 62} (2000) 032006
[hep-ex/9908010].
%%CITATION = HEP-EX 9908010;%%

%\cite{Lai:1994bb}
\bibitem{Lai:1994bb}
H.~L.~Lai {\it et al.},
%``Global QCD analysis and the CTEQ parton distributions,''
Phys.\ Rev.\ D {\bf 51} (1995) 4763
[hep-ph/9410404].
%%CITATION = HEP-PH 9410404;%%

%\cite{Arenton:pw}
\bibitem{Arenton:pw}
M.~Arenton  [E705 Collaboration],
%``Charmonium Production With 300-Gev/C Pi+, Pi-, P And Anti-P Beams,''
Int.\ J.\ Mod.\ Phys.\ A {\bf 12} (1997) 3837.
%%CITATION = IMPAE,A12,3837;%%

%\cite{Spengler:2004gr}
\bibitem{Spengler:2004gr}
J.~Spengler  [HERA-B Collaboration],
%``Quarkonia production with the HERA-B experiment,''
J.\ Phys.\ G {\bf 30}, S871 (2004)
[arXiv:hep-ex/0403043].
%%CITATION = HEP-EX 0403043;%%

\bibitem{Lemoigne:1982jc}
Y.~Lemoigne {\it et al.},
%``Measurement Of Hadronic Production Of The Chi 1++ (3507) And The Chi 
%2++
%(3553) Through Their Radiative Decay To J / Psi,''
Phys.\ Lett.\ B {\bf 113} (1982) 509
[Erratum-ibid.\ B {\bf 116} (1982) 470].
%%CITATION = PHLTA,B113,509;%%

\bibitem{Hahn:tz}
S.~R.~Hahn {\it et al.},
%``Hadronic Production Of Charmonium In 225-Gev/C Pi- Be Interactions,''
Phys.\ Rev.\ D {\bf 30} (1984) 671.
%%CITATION = PHRVA,D30,671;%%

\bibitem{Koreshev:1996wd}
V.~Koreshev {\it et al.}  [E672-E706 Collaborations],
%``Production of charmonium states in pi- Be collisions at 515-GeV/c,''
Phys.\ Rev.\ Lett.\  {\bf 77} (1996) 4294.
%%CITATION = PRLTA,77,4294;%%

%\cite{Gribushin:1995rt}
\bibitem{Gribushin:1995rt}
A.~Gribushin {\it et al.}  [E672 and E706 Collaborations],
%``Production of J / psi and psi (2S) mesons in pi- Be collisions at
%515-GeV/c,''
Phys.\ Rev.\ D {\bf 53} (1996) 4723.
%%CITATION = PHRVA,D53,4723;%%

\bibitem{Amundson:1995em}
J.~F.~Amundson, O.~J.~Eboli, E.~M.~Gregores, and F.~Halzen,
%``Colorless States in Perturbative QCD: Charmonium and Rapidity Gaps,''
Phys.\ Lett.\ B {\bf 372} (1996) 127 
[hep-ph/9512248].
%%CITATION = HEP-PH 9512248;%%

%%% 3.1 Polarization

\bibitem{Gribushin:1999ha}
A.~Gribushin {\it et al.}  [E672 Collaborations],
%``Production of J/psi mesons in p Be collisions at 530-GeV/c and  
%800-GeV/c,''
Phys.\ Rev.\ D {\bf 62} (2000) 012001
[hep-ex/9910005].
%%CITATION = HEP-EX 9910005;%%

\bibitem{Introzzi:yi}
G.~Introzzi  [E771 Collaboration],
%``E771 Results On Charm And Beauty,''
Nucl.\ Phys.\ Proc.\ Suppl.\  {\bf 55A} (1997) 188.
%%CITATION = NUPHZ,55A,188;%%

\bibitem{Chang:2003rz}
T.~H.~Chang {\it et al.}  [FNAL E866/NuSea collaboration],
%``J/psi polarization in 800-GeV p Cu interactions,''
Phys.\ Rev.\ Lett.\  {\bf 91} (2003) 211801
[hep-ex/0308001].
%%CITATION = HEP-EX 0308001;%%

\bibitem{Vanttinen:1994sd}
M.~Vanttinen, P.~Hoyer, S.~J.~Brodsky and W.~K.~Tang,
%``Hadroproduction and polarization of charmonium,''
Phys.\ Rev.\ D {\bf 51} (1995) 3332
[hep-ph/9410237].
%%CITATION = HEP-PH 9410237;%%

\bibitem{lee-2000}
Jungil Lee, presentation at the HERA-B Collaboration Meeting, December 7, 
2000.

\bibitem{Heinrich:zm}
J.~G.~Heinrich {\it et al.},
%``Higher Twist Effects In The Reaction Pi- N $\to$ Mu+ Mu- X At 
%253-Gev/C,''
Phys.\ Rev.\ D {\bf 44} (1991) 1909.
%%CITATION = PHRVA,D44,1909;%%

\bibitem{Brown:2000bz}
C.~N.~Brown {\it et al.}  [FNAL E866 Collaboration],
%``Observation of polarization in bottomonium production at s**(1/2) =  
%38.8-GeV,''
Phys.\ Rev.\ Lett.\  {\bf 86} (2001) 2529
[hep-ex/0011030].
%%CITATION = HEP-EX 0011030;%%

\bibitem{Kharchilava:1998wa}
A.~Kharchilava, T.~Lohse, A.~Somov, and A.~Tkabladze,
%``Upsilon polarization at HERA-B,''
Phys.\ Rev.\ D {\bf 59} (1999) 094023
[hep-ph/9811361].
%%CITATION = HEP-PH 9811361;%%

\bibitem{Tkabladze:1999mb}
A.~Tkabladze,
%``Bottomonium polarization in hadroproduction,''
Phys.\ Lett.\ B {\bf 462} (1999) 319.
%%CITATION = PHLTA,B462,319;%%

%%% 3.3 Color-evaporation parameters

%\cite{Bedjidian:2003gd}
\bibitem{Bedjidian:2003gd}
M.~Bedjidian {\it et al.},
%``Hard probes in heavy ion collisions at the LHC: Heavy flavour physics,''
hep-ph/0311048.
%%CITATION = HEP-PH 0311048;%%

%\cite{Digal:2001ue}
\bibitem{Digal:2001ue}
S.~Digal, P.~Petreczky and H.~Satz,
%``Quarkonium feed-down and sequential suppression,''
Phys.\ Rev.\ D {\bf 64} (2001) 094015
[hep-ph/0106017].
%%CITATION = HEP-PH 0106017;%%

%\cite{Gluck:1998xa}
\bibitem{Gluck:1998xa}
M.~Gl\"uck, E.~Reya and A.~Vogt,
%``Dynamical parton distributions revisited,''
Eur.\ Phys.\ J.\ C {\bf 5} (1998) 461
[hep-ph/9806404].
%%CITATION = HEP-PH 9806404;%%

%\cite{Vogt:2002vx}
\bibitem{Vogt:2002vx}
R.~Vogt,
%``Systematics of heavy quark production at RHIC,''
in proceedings of the 18$^{\rm th}$ Winter Workshop on Nuclear
Dynamics, edited by R. Bellweid {\it et al.}, Debrecen, Hungary (2002) p. 253
[hep-ph/0203151].
%%CITATION = HEP-PH 0203151;%%

%\cite{Vogt:2002ve}
\bibitem{Vogt:2002ve}
R.~Vogt,
%``Heavy quark production in heavy ion colliders,''
Heavy Ion Phys.\  {\bf 18} (2003) 11
[hep-ph/0205330].
%%CITATION = HEP-PH 0205330;%%

%\cite{Affolder:1999wm}
\bibitem{Affolder:1999wm}
T.~Affolder {\it et al.}  [CDF Collaboration],
%``Production of Upsilon(1S) mesons from chi/b decays in p anti-p  collisions at s**(1/2) =
%1.8-TeV,''
Phys.\ Rev.\ Lett.\  {\bf 84} (2000) 2094
[hep-ex/9910025].
%%CITATION = HEP-EX 9910025;%%

%\cite{Gunion:1996qc}
\bibitem{Gunion:1996qc}
J.~F.~Gunion and R.~Vogt,
%``Determining the existence and nature of the quark-gluon plasma by  Upsilon suppression at
%the LHC,''
Nucl.\ Phys.\ B {\bf 492} (1997) 301
[hep-ph/9610420].
%%CITATION = HEP-PH 9610420;%%

%%% Section 4: Quarkonium production at HERA

%%% 4.1 Inelastic charmonium photoproduction

\bibitem{Kramer:1994zi}
M.~Kr\"amer, J.~Zunft, J.~Steegborn, and P.~M.~Zerwas,
%``Inelastic J / psi photoproduction,''
Phys.\ Lett.\ B {\bf 348} (1995) 657
[hep-ph/9411372].
%%CITATION = HEP-PH 9411372;%%

\bibitem{Kramer:1995nb}
M.~Kr\"amer,
%``QCD Corrections to Inelastic $J/\psi$ Photoproduction,''
Nucl.\ Phys.\ B {\bf 459} (1996) 3
[hep-ph/9508409].
%%CITATION = HEP-PH 9508409;%%

%\cite{Cacciari:1996dg}
\bibitem{Cacciari:1996dg}
M.~Cacciari and M.~Kr\"amer,
%``Color-Octet Contributions to $J/\psi$ Photoproduction,''
Phys.\ Rev.\ Lett.\  {\bf 76} (1996) 4128
[hep-ph/9601276].
%%CITATION = HEP-PH 9601276;%%

\bibitem{Beneke:1998re}
M.~Beneke, M.~Kr\"amer, and M.~Vanttinen,
%``Inelastic photoproduction of polarised J/psi,''
Phys.\ Rev.\ D {\bf 57} (1998) 4258
[hep-ph/9709376].
%%CITATION = HEP-PH 9709376;%%

%\cite{Amundson:1996ik}
\bibitem{Amundson:1996ik}
J.~Amundson, S.~Fleming and I.~Maksymyk,
%``Photoproduction of J/$\psi$ in the forward region,''
Phys.\ Rev.\ D {\bf 56} (1997) 5844
[hep-ph/9601298].
%%CITATION = HEP-PH 9601298;%%

%\cite{Ko:1996xw}
\bibitem{Ko:1996xw}
P.~Ko, J.~Lee and H.~S.~Song,
%``Color-octet mechanism in $\gamma + p \to J/\psi + X$,''
Phys.\ Rev.\ D {\bf 54} (1996) 4312
[Erratum-ibid.\ D {\bf 60} (1999) 119902]
[hep-ph/9602223].
%%CITATION = HEP-PH 9602223;%%

\bibitem{Godbole:1995ie}
R.~M.~Godbole, D.~P.~Roy, and K.~Sridhar,
%``J/psi production via fragmentation at HERA,''
Phys.\ Lett.\ B {\bf 373} (1996) 328
[hep-ph/9511433].
%%CITATION = HEP-PH 9511433;%%

\bibitem{Kniehl:1997fv}
B.~A.~Kniehl and G.~Kramer,
%``Colour-octet contributions to J/psi photoproduction via fragmentation 
% at HERA,''
Phys.\ Lett.\ B {\bf 413} (1997) 416
[hep-ph/9703280].
%%CITATION = HEP-PH 9703280;%%

\bibitem{Kniehl:1997gh}
B.~A.~Kniehl and G.~Kramer,
%``Charmonium production via fragmentation at DESY HERA,''
Phys.\ Rev.\ D {\bf 56} (1997) 5820
[hep-ph/9706369].
%%CITATION = HEP-PH 9706369;%%

%\cite{Adloff:2002ex}
\bibitem{Adloff:2002ex}
C.~Adloff {\it et al.}  [H1 Collaboration],
%``Inelastic photoproduction of J/psi mesons at HERA,''
Eur.\ Phys.\ J.\ C {\bf 25} (2002) 25
[hep-ex/0205064].
%%CITATION = HEP-EX 0205064;%%

%\bibitem{H1-photoproduction}
%H1 Collaboration, Contributed paper 157aj, International Europhysics 
%Conference on High Energy Physics (EPS99), Tampere, Finland, 1999.

%\cite{Chekanov:2002at}
\bibitem{Chekanov:2002at}
S.~Chekanov {\it et al.}  [ZEUS Collaboration],
%``Measurements of inelastic J/psi and psi' photoproduction at HERA,''
Eur.\ Phys.\ J.\ C {\bf 27} (2003) 173
[hep-ex/0211011].
%%CITATION = HEP-EX 0211011;%%

%\bibitem{Zeus-photoproduction}
%Zeus Collaboration, Contributed Paper 851, International Conference on 
%High Energy Physics (ICHEP2000), Osaka, Japan, 2000.

\bibitem{Jung:2001hk}
H.~Jung and G.~P.~Salam,
%``Hadronic final state predictions from CCFM: The hadron-level Monte  Carlo generator CASCADE,''
Eur.\ Phys.\ J.\ C {\bf 19} (2001) 351 
[hep-ph/0012143].
%%CITATION = HEP-PH 0012143;%%

\bibitem{Jung:2001hx}
H.~Jung,
%``The CCFM Monte Carlo generator CASCADE,''
Comput.\ Phys.\ Commun.\  {\bf 143} (2002) 100
[hep-ph/0109102].
%%CITATION = HEP-PH 0109102;%%

%\cite{Saleev:1994fg}
\bibitem{Saleev:1994fg}
V.~A.~Saleev and N.~P.~Zotov,
%``Heavy Quarkonium Photoproduction At High-Energies In Semihard Approach,''
Mod.\ Phys.\ Lett.\ A {\bf 9} (1994) 151
[Erratum-ibid.\ A {\bf 9} (1994) 1517].
%%CITATION = MPLAE,A9,151;%%

%\cite{Baranov:1998af}
\bibitem{Baranov:1998af}
S.~P.~Baranov,
%``Probing The Bfkl Gluons With J/Psi Leptoproduction,''
Phys.\ Lett.\ B {\bf 428} (1998) 377.
%%CITATION = PHLTA,B428,377;%%

%%% 4.2 Inelastic Charmonium Production in DIS

%\cite{Adloff:1999zs}
\bibitem{Adloff:1999zs}
C.~Adloff {\it et al.}  [H1 Collaboration],
%``Charmonium production in deep inelastic scattering at HERA,''
Eur.\ Phys.\ J.\ C {\bf 10} (1999) 373 
[hep-ex/9903008].
%%CITATION = HEP-EX 9903008;%%

%\cite{Fleming:1997fq}
\bibitem{Fleming:1997fq}
S.~Fleming and T.~Mehen,
%``Leptoproduction of J/psi,''
Phys.\ Rev.\ D {\bf 57} (1998) 1846
[hep-ph/9707365].
%%CITATION = HEP-PH 9707365;%%

\bibitem{Kniehl:2001tk}
B.~A.~Kniehl and L.~Zwirner,
%``J/psi inclusive production in e p deep-inelastic scattering at DESY  
%HERA,''
Nucl.\ Phys.\ {\bf B621} (2002) 337
[hep-ph/0112199].
%%CITATION = HEP-PH 0112199;%%

%\cite{Adloff:2002ey}
\bibitem{Adloff:2002ey}
C.~Adloff {\it et al.}  [H1 Collaboration],
%``Inelastic leptoproduction of J/psi mesons at HERA,''
Eur.\ Phys.\ J.\ C {\bf 25} (2002) 41
[hep-ex/0205065].
%%CITATION = HEP-EX 0205065;%%

\bibitem{zeus-eps03-565}
S.~Chekanov {\it et al.}  [ZEUS Collaboration],
 %``Measurement of proton dissociative diffractive photoproduction of 
 % J/psi mesons at HERA''
contributed paper 565, International Europhysics Conference 
on High Energy Physics (EPS 2003), Aachen, Germany, 2003. 

%\cite{Lipatov:2002tc}
\bibitem{Lipatov:2002tc}
A.~V.~Lipatov and N.~P.~Zotov,
%``Inelastic J/psi production at HERA in the colour singlet model with
%k(T)-factorization,''
Eur.\ Phys.\ J.\ C {\bf 27} (2003) 87 
[hep-ph/0210310].
%%CITATION = HEP-PH 0210310;%%

\bibitem{Korner:1982fm}
J.~G.~K\"orner, J.~Cleymans, M.~Kuroda, and G.~J.~Gounaris,
%``Azimuthal Dependence Of Deep Inelastic Heavy Resonance Production,''
Phys.\ Lett.\ B {\bf 114} (1982) 195.
%%CITATION = PHLTA,B114,195;%%

\bibitem{Guillet:1987xr}
J.~P.~Guillet,
%``A Way To Measure The Spin Dependent Distribution Of The Gluon,''
Z.\ Phys.\ C {\bf 39} (1988) 75.
%%CITATION = ZEPYA,C39,75;%%

\bibitem{Merabet:sm}
H.~Merabet, J.~F.~Mathiot, and R.~Mendez-Galain,
%``Inelastic Leptoproduction Of J / Psi And The Gluon Distribution In 
%The Nucleon,''
Z.\ Phys.\ C {\bf 62} (1994) 639.
%%CITATION = ZEPYA,C62,639;%%

\bibitem{Krucker:1995uz}
D.~Kr\"ucker, Ph.D. Thesis, RWTH Aachen, 1995.
%``Model for elastic $J/\psi$--production at {HERA} (In German),''.

\bibitem{Yuan:2000cn}
F.~Yuan and K.~T.~Chao,
%``Polarized J/psi production in deep inelastic scattering at HERA,''
Phys.\ Rev.\ D {\bf 63} (2001) 034017
[hep-ph/0008301].
%%CITATION = HEP-PH 0008301;%%

%%% 4.3 Diffractive vector meson production 

%\cite{Ryskin:1992ui}
\bibitem{Ryskin:1992ui}
M.~G.~Ryskin,
%``Diffractive J / psi electroproduction in LLA QCD,''
Z.\ Phys.\ C {\bf 57} (1993) 89.
%%CITATION = ZEPYA,C57,89;%%

%\cite{Brodsky:1994kf}
\bibitem{Brodsky:1994kf}
S.~J.~Brodsky, L.~Frankfurt, J.~F.~Gunion, A.~H.~Mueller and M.~Strikman,
%``Diffractive leptoproduction of vector mesons in QCD,''
Phys.\ Rev.\ D {\bf 50} (1994) 3134 
[hep-ph/9402283].
%%CITATION = HEP-PH 9402283;%%

%\cite{Collins:1996fb}
\bibitem{Collins:1996fb}
J.~C.~Collins, L.~Frankfurt and M.~Strikman,
%``Factorization for hard exclusive electroproduction of mesons in QCD,''
Phys.\ Rev.\ D {\bf 56} (1997) 2982 
[hep-ph/9611433].
%%CITATION = HEP-PH 9611433;%%

%\cite{Collins:1997sr}
\bibitem{Collins:1997sr}
J.~C.~Collins,
%``Proof of factorization for diffractive hard scattering,''
Phys.\ Rev.\ D {\bf 57} (1998) 3051 
[Erratum-ibid.\ D {\bf 61} (2000) 019902]
[hep-ph/9709499].
%%CITATION = HEP-PH 9709499;%%

%\cite{Teubner:1999pm}
\bibitem{Teubner:1999pm}
T.~Teubner,
%``Theory of elastic vector meson production,''
hep-ph/9910329.
%%CITATION = HEP-PH 9910329;%%

%\cite{Bartels:2000ze}
\bibitem{Bartels:2000ze}
J.~Bartels and H.~Kowalski,
%``Diffraction at HERA and the confinement problem,''
Eur.\ Phys.\ J.\ C {\bf 19} (2001) 693 
[hep-ph/0010345].
%%CITATION = HEP-PH 0010345;%%

%\cite{Hayashigaki:2004iw}
\bibitem{Hayashigaki:2004iw}
A.~Hayashigaki and K.~Tanaka,
 %``Transverse quark motion inside charmonia in diffractive photo- and
%electroproductions,''
hep-ph/0401053.
%%CITATION = HEP-PH 0401053;%%

%%%%%%%%%%%%%%%%%%%%%%%%%%%%%%%%%%%%%%%%%%%%%%%%%%%%%%%%%%%%%%%%%%
% H1 Measurements of Diffractive VM Production

\bibitem{Aid:1996bs}
S.~Aid {\it et al.}  [H1 Collaboration],
%``Elastic Photoproduction of $\rho~{0}$ Mesons at HERA,''
Nucl.\ Phys.\ B {\bf 463} (1996) 3 
[hep-ex/9601004].
%%CITATION = HEP-EX 9601004;%%

%\cite{Adloff:1999kg}
\bibitem{Adloff:1999kg}
C.~Adloff {\it et al.}  [H1 Collaboration],
%``Elastic electroproduction of rho mesons at HERA,''
Eur.\ Phys.\ J.\ C {\bf 13} (2000) 371 
[hep-ex/9902019].
%%CITATION = HEP-EX 9902019;%%

%\cite{Adloff:2000vm}
\bibitem{Adloff:2000vm}
C.~Adloff {\it et al.}  [H1 Collaboration],
%``Elastic photoproduction of J/psi and Upsilon mesons at HERA,''
Phys.\ Lett.\ B {\bf 483} (2000) 23 
[hep-ex/0003020].
%%CITATION = HEP-EX 0003020;%%

%\cite{Adloff:2000nx}
\bibitem{Adloff:2000nx}
C.~Adloff {\it et al.}  [H1 Collaboration],
%``Measurement of elastic electroproduction of Phi mesons at HERA,''
Phys.\ Lett.\ B {\bf 483} (2000) 360 
[hep-ex/0005010].
%%CITATION = HEP-EX 0005010;%%

%\cite{Adloff:2002tb}
\bibitem{Adloff:2002tb}
C.~Adloff {\it et al.}  [H1 Collaboration],
%``A measurement of the t dependence of the helicity structure of  diffractive
%rho meson electroproduction at HERA,''
Phys.\ Lett.\ B {\bf 539} (2002) 25 
[hep-ex/0203022].
%%CITATION = HEP-EX 0203022;%%

%\cite{Adloff:2002re}
\bibitem{Adloff:2002re}
C.~Adloff {\it et al.}  [H1 Collaboration],
%``Diffractive photoproduction of psi(2S) mesons at HERA,''
Phys.\ Lett.\ B {\bf 541} (2002) 251 
[hep-ex/0205107].
%%CITATION = HEP-EX 0205107;%%

%\cite{Aktas:2003zi}
\bibitem{Aktas:2003zi}
A.~Aktas {\it et al.}  [H1 Collaboration],
%``Diffractive photoproduction of J/psi mesons with large momentum  transfer at
%HERA,''
Phys.\ Lett.\ B {\bf 568} (2003) 205 
[hep-ex/0306013].
%%CITATION = HEP-EX 0306013;%%

\bibitem{h1-ichep04-6-0180}
A.~Aktas {\it et al.}  [H1 Collaboration],
%``Elastic photoproduction of J/psi mesons at HERA,''
contributed paper 6-0180, International Conference 
on High Energy Physics (ICHEP04), Beijing, China, 2004. 


%%%%%%%%%%%%%%%%%%%%%%%%%%%%%%%%%%%%%%%%%%%%%%%%%%%%%%%%%%%%%%%%%%
% ZEUS Measurements of Diffractive VM Production
%\cite{Breitweg:1998nh}
\bibitem{Breitweg:1998nh}
J.~Breitweg {\it et al.}  [ZEUS Collaboration],
%``Exclusive electroproduction of rho0 and J/psi mesons at HERA,''
Eur.\ Phys.\ J.\ C {\bf 6} (1999) 603 
[hep-ex/9808020].
%%CITATION = HEP-EX 9808020;%%

%\cite{Breitweg:1998ki}
\bibitem{Breitweg:1998ki}
J.~Breitweg {\it et al.}  [ZEUS Collaboration],
%``Measurement of elastic Upsilon photoproduction at HERA,''
Phys.\ Lett.\ B {\bf 437} (1998) 432 
[hep-ex/9807020].
%%CITATION = HEP-EX 9807020;%%

%\cite{Breitweg:1999jy}
\bibitem{Breitweg:1999jy}
J.~Breitweg {\it et al.}  [ZEUS Collaboration],
 %``Measurement of diffractive photoproduction of vector mesons at large
%momentum transfer at HERA,''
Eur.\ Phys.\ J.\ C {\bf 14} (2000) 213 
[hep-ex/9910038].
%%CITATION = HEP-EX 9910038;%%

%\cite{Breitweg:1999fm}
\bibitem{Breitweg:1999fm}
J.~Breitweg {\it et al.}  [ZEUS Collaborations],
 %``Measurement of the spin-density matrix elements in exclusive
%electroproduction of rho0 mesons at HERA,''
Eur.\ Phys.\ J.\ C {\bf 12} (2000) 393 
[hep-ex/9908026].
%%CITATION = HEP-EX 9908026;%%

%\cite{Breitweg:2000mu}
\bibitem{Breitweg:2000mu}
J.~Breitweg {\it et al.}  [ZEUS Collaboration],
%``Measurement of exclusive omega electroproduction at HERA,''
Phys.\ Lett.\ B {\bf 487} (2000) 273 
[hep-ex/0006013].
%%CITATION = HEP-EX 0006013;%%

%\cite{Chekanov:2002rm}
\bibitem{Chekanov:2002rm}
S.~Chekanov {\it et al.}  [ZEUS Collaboration],
 %``Measurement of proton dissociative diffractive photoproduction of  vector
%mesons at large momentum transfer at HERA,''
Eur.\ Phys.\ J.\ C {\bf 26} (2003) 389 
[hep-ex/0205081].
%%CITATION = HEP-EX 0205081;%%

\bibitem{zeus-eps03-549}
S.~Chekanov {\it et al.}  [ZEUS Collaboration],
 %``Measurement of proton dissociative diffractive photoproduction of 
 % J/psi mesons at HERA''
contributed paper 549, International Europhysics Conference 
on High Energy Physics (EPS 2003), Aachen, Germany, 2003. 

%\cite{Chekanov:2002xi}
\bibitem{Chekanov:2002xi}
S.~Chekanov {\it et al.}  [ZEUS Collaboration],
%``Exclusive photoproduction of J/psi mesons at HERA,''
Eur.\ Phys.\ J.\ C {\bf 24} (2002) 345 
[hep-ex/0201043].
%%CITATION = HEP-EX 0201043;%%

%\cite{Chekanov:2004mw}
\bibitem{Chekanov:2004mw}
S.~Chekanov {\it et al.}  [ZEUS Collaboration],
%``Exclusive electroproduction of J/psi mesons at HERA,''
Nucl.\ Phys.\ B {\bf 695} (2004) 3
[hep-ex/0404008].
%%CITATION = HEP-EX 0404008;%%

%%%%%%%%%%%%%%%%%%%%%%%%%%%%%%%%%%%%%%%%%%%%%%%%%%%%%%%%%%%%%%%%%%
% theory on Diffractive VM Production


%%%%%%%%%%%%%%%%%%%%%%%%%%%%%%%%%%%%%%%%%%%%%%%%%%%%%%%%%%%%%%%%%%
% QCD calculations of Diffractive VM Production

%\cite{Frankfurt:1997fj}
\bibitem{Frankfurt:1997fj}
L.~Frankfurt, W.~Koepf and M.~Strikman,
%``Diffractive heavy quarkonium photo- and electroproduction in QCD,''
Phys.\ Rev.\ D {\bf 57} (1998) 512 
[hep-ph/9702216].
%%CITATION = HEP-PH 9702216;%%

%\cite{McDermott:1999fa}
\bibitem{McDermott:1999fa}
M.~McDermott, L.~Frankfurt, V.~Guzey and M.~Strikman,
%``Unitarity and the QCD-improved dipole picture,''
Eur.\ Phys.\ J.\ C {\bf 16} (2000) 641 
[hep-ph/9912547].
%%CITATION = HEP-PH 9912547;%%

%\cite{Frankfurt:2000ez}
\bibitem{Frankfurt:2000ez}
L.~Frankfurt, M.~McDermott and M.~Strikman,
 %``A fresh look at diffractive J/psi photoproduction at HERA, with  predictions
%for THERA,''
JHEP {\bf 0103} (2001) 045 
[hep-ph/0009086].
%%CITATION = HEP-PH 0009086;%%

%\cite{Ryskin:1995hz}
\bibitem{Ryskin:1995hz}
M.~G.~Ryskin, R.~G.~Roberts, A.~D.~Martin and E.~M.~Levin,
%``Diffractive $J/\psi$ photoproduction as a probe of the gluon density,''
Z.\ Phys.\ C {\bf 76} (1997) 231 
[hep-ph/9511228].
%%CITATION = HEP-PH 9511228;%%

%\cite{Martin:1997wy}
\bibitem{Martin:1997wy}
A.~D.~Martin and M.~G.~Ryskin,
 %``The effect of off-diagonal parton distributions in diffractive vector  meson
%electroproduction,''
Phys.\ Rev.\ D {\bf 57} (1998) 6692 
[hep-ph/9711371].
%%CITATION = HEP-PH 9711371;%%

%\cite{Martin:1999wb}
\bibitem{Martin:1999wb}
A.~D.~Martin, M.~G.~Ryskin and T.~Teubner,
%``Q**2 dependence of diffractive vector meson electroproduction,''
Phys.\ Rev.\ D {\bf 62} (2000) 014022 
[hep-ph/9912551].
%%CITATION = HEP-PH 9912551;%%

%\cite{Ivanov:2004vd}
\bibitem{Ivanov:2004vd}
D.~Y.~Ivanov, A.~Sch\"afer, L.~Szymanowski and G.~Krasnikov,
%``Exclusive photoproduction of a heavy vector meson in QCD,''
Eur.\ Phys.\ J.\ C {\bf 34} (2004) 297 
[hep-ph/0401131].
%%CITATION = HEP-PH 0401131;%%


%%%%%%%%%%%%%%%%%%%%%%%%%%%%%%%%%%%%%%%%%%%%%%%%%%%%%%%%%%%%%%%%%%
% Regge picture of Diffractive VM Production

%\cite{Regge:mz}
\bibitem{Regge:mz}
T.~Regge,
%``Introduction To Complex Orbital Momenta,''
Nuovo Cim.\  {\bf 14} (1959) 951.  
%%CITATION = NUCIA,14,951;%%

%\cite{Regge:1960zc}
\bibitem{Regge:1960zc}
T.~Regge,
%``Bound States, Shadow States And Mandelstam Representation,''
Nuovo Cim.\  {\bf 18} (1960) 947.
%%CITATION = NUCIA,18,947;%%

%\cite{Sakurai:1969ss}
\bibitem{Sakurai:1969ss}
J.~J.~Sakurai,
 %``Vector Meson Dominance And High-Energy Electron Proton Inelastic
%Scattering,''
Phys.\ Rev.\ Lett.\  {\bf 22} (1969) 981.
%%CITATION = PRLTA,22,981;%%

%\cite{Donnachie:1992ny}
\bibitem{Donnachie:1992ny}
A.~Donnachie and P.~V.~Landshoff,
%``Total cross-sections,''
Phys.\ Lett.\ B {\bf 296} (1992) 227 
[hep-ph/9209205].
%%CITATION = HEP-PH 9209205;%%

%\cite{Donnachie:1998gm}
\bibitem{Donnachie:1998gm}
A.~Donnachie and P.~V.~Landshoff,
%``Small x: Two pomerons!,''
Phys.\ Lett.\ B {\bf 437} (1998) 408 
[hep-ph/9806344].
%%CITATION = HEP-PH 9806344;%%

\bibitem{FKL:1976}
E.~A.~Kuraev, L.~N.~Lipatov and V.~S.~Fadin,
%``Multi - Reggeon Processes In The Yang-Mills Theory,''
Sov.\ Phys.\ JETP {\bf 44} (1976) 443
[Zh.\ Eksp.\ Teor.\ Fiz.\  {\bf 71} (1976) 840].
%%CITATION = SPHJA,44,443;%%

%\cite{Kuraev:fs}
\bibitem{Kuraev:fs}
E.~A.~Kuraev, L.~N.~Lipatov and V.~S.~Fadin,
%``The Pomeranchuk Singularity In Nonabelian Gauge Theories,''
Sov.\ Phys.\ JETP {\bf 45} (1977) 199
[Zh.\ Eksp.\ Teor.\ Fiz.\  {\bf 72} (1977) 377].
%%CITATION = SPHJA,45,199;%%

\bibitem{BL:1978}
I.~I.~Balitsky and L.~N.~Lipatov,
%``The Pomeranchuk Singularity In Quantum Chromodynamics,''
Sov.\ J.\ Nucl.\ Phys.\  {\bf 28} (1978) 822
[Yad.\ Fiz.\  {\bf 28} (1978) 1597].
%%CITATION = SJNCA,28,822;%%

%\cite{Mueller:1994jq}
\bibitem{Mueller:1994jq}
A.~H.~Mueller and B.~Patel,
 %``Single and double BFKL pomeron exchange and a dipole picture of high-energy
%hard processes,''
Nucl.\ Phys.\ B {\bf 425} (1994) 471 
[hep-ph/9403256].
%%CITATION = HEP-PH 9403256;%%

%\cite{Nikolaev:1993th}
\bibitem{Nikolaev:1993th}
N.~N.~Nikolaev and B.~G.~Zakharov,
 %``The Triple pomeron regime and the structure function of the pomeron in the
%diffractive deep inelastic scattering at very small x,''
Z.\ Phys.\ C {\bf 64} (1994) 631 
[hep-ph/9306230].
%%CITATION = HEP-PH 9306230;%%

\bibitem{Gribov:ri}
V.~N.~Gribov and L.~N.~Lipatov,
%``Deep Inelastic E P Scattering In Perturbation Theory,''
Yad.\ Fiz.\  {\bf 15} (1972) 781 and 1218
[Sov.\ J.\ Nucl.\ Phys.\  {\bf 15} (1972) 438 and 675].
%%CITATION = YAFIA,15,781;%%

%%% 4.4. Prospects for the upgraded HERA collider

\bibitem{Cacciari:1996dy}
M.~Cacciari and M.~Kr\"amer, Talk given at the Workshop on Future
Physics at HERA, Hamburg, Germany, 30-31 May 1996
%``Prospects for quarkonium physics at HERA,''
[hep-ph/9609500].
%%CITATION = HEP-PH 9609500;%%

\bibitem{Fleming:1998md}
S.~Fleming and T.~Mehen,
%``Photoproduction of h/c,''
Phys.\ Rev.\ D {\bf 58} (1998) 037503
[hep-ph/9801328].
%%CITATION = HEP-PH 9801328;%%
 
\bibitem{Hao:1999kq}
L.~Hao, F.~Yuan and K.~Chao,
%``Photoproduction of eta/c in {NRQCD},''
Phys.\ Rev.\ Lett.\  {\bf 83} (1999) 4490
[hep-ph/9902338].
%%CITATION = HEP-PH 9902338;%%
 
\bibitem{Hao:2000ci}
L.~Hao, F.~Yuan and K.~Chao,
%``Inelastic electroproduction of eta/c at e p colliders,''
Phys.\ Rev.\ D {\bf 62} (2000) 074023
[hep-ph/0004203].
%%CITATION = HEP-PH 0004203;%%
 
\bibitem{Kim:1993at}
C.~S.~Kim and E.~Reya,
%``Associate J / psi + gamma production as a clean probe of heavy quark production models,''
Phys.\ Lett.\ B {\bf 300} (1993) 298.
%%CITATION = PHLTA,B300,298;%%
 
\bibitem{Mehen:1997vx}
T.~Mehen,
%``Testing quarkonium production with photoproduced J/psi + gamma,''
Phys.\ Rev.\ D {\bf 55} (1997) 4338
[hep-ph/9611321].
%%CITATION = HEP-PH 9611321;%%

\bibitem{Cacciari:1997zu}
M.~Cacciari, M.~Greco and M.~Kr\"amer,
%``Associated J/psi + gamma photoproduction as a probe of the  colour-octet mechanism,''
Phys.\ Rev.\ D {\bf 55} (1997) 7126
[hep-ph/9611324].
%%CITATION = HEP-PH 9611324;%%

%%% Section 5 Quarkonium Production at LEP

%%% 5.1 J/psi production

%\cite{ALEPH:1997zj}
\bibitem{ALEPH:1997zj}
  [ALEPH Collaboration],
%``Study of prompt J/psi production in hadronic Z decays,''
CERN-OPEN-99-343
{\it Prepared for International Europhysics Conference 
on High-Energy Physics (HEP 97), Jerusalem, Israel, 19-26 Aug 1997}.

%\cite{Abreu:1994rk}
\bibitem{Abreu:1994rk}
P.~Abreu {\it et al.}  [DELPHI Collaboration],
%``J / psi production in the hadronic decays of the Z,''
Phys.\ Lett.\ B {\bf 341} (1994) 109.
%%CITATION = PHLTA,B341,109;%%

%\cite{Wadhwa:1998mt}
\bibitem{Wadhwa:1998mt}
M.~Wadhwa  [L3 Collaboration],
%``Charmonium and bottomonium production in the L3 detector,''
Nucl.\ Phys.\ Proc.\ Suppl.\  {\bf 64} (1998) 441.
%%CITATION = NUPHZ,64,441;%%

%\cite{Alexander:1996jp}
\bibitem{Alexander:1996jp}
G.~Alexander {\it et al.}  [OPAL Collaboration],
%``Prompt J/psi production in hadronic Z0 decays,''
Phys.\ Lett.\ B {\bf 384} (1996) 343.
%%CITATION = PHLTA,B384,343;%%

%\cite{Boyd:1998km}
\bibitem{Boyd:1998km}
C.~G.~Boyd, A.~K.~Leibovich, and I.~Z.~Rothstein,
%``J/psi production at LEP: Revisited and resummed,''
Phys.\ Rev.\ D {\bf 59} (1999) 054016
[hep-ph/9810364].
%%CITATION = HEP-PH 9810364;%%

%\cite{Todorova-Nova:2001pt}
\bibitem{Todorova-Nova:2001pt}
S.~Todorova-Nova,
%``(Some of) recent gamma gamma measurements from LEP,''
hep-ph/0112050.
%%CITATION = HEP-PH 0112050;%%

%\cite{Abdallah:2003du}
\bibitem{Abdallah:2003du}
J.~Abdallah {\it et al.}  [DELPHI Collaboration],
%``Study of inclusive J/psi production in two-photon collisions at LEP II with
%the DELPHI detector,''
Phys.\ Lett.\ B {\bf 565} (2003) 76
[hep-ex/0307049].
%%CITATION = HEP-EX 0307049;%%

%\cite{Ma:1997bi}
\bibitem{Ma:1997bi}
J.~P.~Ma, B.~H.~J.~McKellar and C.~B.~Paranavitane,
%``J/psi production at photon photon colliders as a probe of the color  octet
%mechanism,''
Phys.\ Rev.\ D {\bf 57} (1998) 606
[hep-ph/9707480].
%%CITATION = HEP-PH 9707480;%%

%\cite{Japaridze:1998ss}
\bibitem{Japaridze:1998ss}
G.~Japaridze and A.~Tkabladze,
%``Color octet contribution to J/psi production at a photon linear  collider,''
Phys.\ Lett.\ B {\bf 433} (1998) 139
[hep-ph/9803447].
%%CITATION = HEP-PH 9803447;%%

%\cite{Godbole:2001pj}
\bibitem{Godbole:2001pj}
R.~M.~Godbole, D.~Indumathi and M.~Kr\"amer,
%``J/psi production through resolved photon processes at e+ e- colliders,''
Phys.\ Rev.\ D {\bf 65} (2002) 074003
[hep-ph/0101333].
%%CITATION = HEP-PH 0101333;%%

%\cite{Klasen:2001mi}
\bibitem{Klasen:2001mi}
M.~Klasen, B.~A.~Kniehl, L.~Mihaila and M.~Steinhauser,
%``J/psi plus dijet associated production in two-photon collisions,''
Nucl.\ Phys.\ B {\bf 609} (2001) 518
[hep-ph/0104044].
%%CITATION = HEP-PH 0104044;%%

%\cite{Klasen:2001cu}
\bibitem{Klasen:2001cu}
M.~Klasen, B.~A.~Kniehl, L.~N.~Mihaila, and M.~Steinhauser,
%``Evidence for colour-octet mechanism from CERN LEP2 gamma gamma $\to$  
%J/psi + X data,''
Phys.\ Rev.\ Lett.\  {\bf 89} (2002) 032001
[hep-ph/0112259].
%%CITATION = HEP-PH 0112259;%%

%\cite{Klasen:2004tz}
\bibitem{Klasen:2004tz}
M.~Klasen, B.~A.~Kniehl, L.~N.~Mihaila and M.~Steinhauser,
%``J/psi plus jet associated production in two-photon collisions at
%next-to-leading order,''
[hep-ph/0407014].
%%CITATION = HEP-PH 0407014;%%

%%% 5.2 Upsilon production

%\cite{Alexander:1995vh}
\bibitem{Alexander:1995vh}
G.~Alexander {\it et al.}  [OPAL Collaboration],
%``Observation of Upsilon production in hadronic Z0 decays,''
Phys.\ Lett.\ B {\bf 370} (1996) 185.
%%CITATION = PHLTA,B370,185;%%

%\cite{Cho:1995vv}
\bibitem{Cho:1995vv}
P.~L.~Cho,
%``Prompt Upsilon and Psi Production at LEP,''
Phys.\ Lett.\ B {\bf 368} (1996) 171
[hep-ph/9509355].
%%CITATION = HEP-PH 9509355;%%

\bibitem{Keung:1980ev}
W.~Y.~Keung,
%``Off Resonance Production Of Heavy Vector Quarkonium States In E+ E- 
%Annihilation,''
Phys.\ Rev.\ D {\bf 23} (1981) 2072.
%%CITATION = PHRVA,D23,2072;%%

\bibitem{Kuhn:1981jy}
J.~H.~K\"uhn and H.~Schneider,
%``Testing QCD Through Inclusive J / Psi Production In E+ E- 
%Annihilations,''
Z.\ Phys.\ C {\bf 11} (1981) 263.
%%CITATION = ZEPYA,C11,263;%%

\bibitem{Abraham:1989ri}
K.~J.~Abraham,
%``Bottomonium Production At Lep,''
Z.\ Phys.\ C {\bf 44} (1989) 467.
%%CITATION = ZEPYA,C44,467;%%

\bibitem{Hagiwara:1991mt}
K.~Hagiwara, A.~D.~Martin and W.~J.~Stirling,
%``J / psi production from gluon jets at LEP,''
Phys.\ Lett.\ B {\bf 267} (1991) 527
[Erratum-ibid.\ B {\bf 316} (1993) 631].

%\cite{Braaten:1993mp}
\bibitem{Braaten:1993mp}
E.~Braaten, K.~m.~Cheung, and T.~C.~Yuan,
%``Z0 decay into charmonium via charm quark fragmentation,''
Phys.\ Rev.\ D {\bf 48} (1993) 4230
[hep-ph/9302307].
%%CITATION = HEP-PH 9302307;%%

%%%%%%%%%%%%%%%%%%%%%%%%%%%%%%%%%%%%%%%%%%%%%%%%%%%%%%%%%%%%%%%%%%%%%%%%%%%

\bibitem{Abe:2001za}
K.~Abe {\it et al.}  [BELLE Collaboration],
%``Production of prompt charmonia in e+ e- annihilation at s**(1/2) =  
%10.6-GeV,''
Phys.\ Rev.\ Lett.\  {\bf 88} (2002) 052001
[hep-ex/0110012].
%%CITATION = HEP-EX 0110012;%%

\bibitem{Aubert:2001pd}
B.~Aubert {\it et al.}  [BABAR Collaboration],
%``Measurement of J/psi production in continuum e+ e- annihilations near 
% s**(1/2) = 10.6-GeV,''
Phys.\ Rev.\ Lett.\  {\bf 87} (2001) 162002
[hep-ex/0106044].
%%CITATION = HEP-EX 0106044;%%

\bibitem{Cho:1996cg}
P.~L.~Cho and A.~K.~Leibovich,
%``Color-singlet psi(Q) production at e+ e- colliders,''
Phys.\ Rev.\ D {\bf 54} (1996) 6690
[hep-ph/9606229].
%%CITATION = HEP-PH 9606229;%%

\bibitem{Yuan:1996ep}
F.~Yuan, C.~F.~Qiao, and K.~T.~Chao,
%``Prompt J/psi production at e+ e- colliders,''
Phys.\ Rev.\ D {\bf 56} (1997) 321
[hep-ph/9703438].
%%CITATION = HEP-PH 9703438;%%

\bibitem{Yuan:1997sn}
F.~Yuan, C.~F.~Qiao, and K.~T.~Chao,
%``Determination of color-octet matrix elements from e+ e- process at 
%low  energies,''
Phys.\ Rev.\ D {\bf 56} (1997) 1663
[hep-ph/9701361].
%%CITATION = HEP-PH 9701361;%%

\bibitem{Schuler:1998az}
G.~A.~Schuler,
%``Testing factorization of charmonium production,''
Eur.\ Phys.\ J.\ C {\bf 8} (1999) 273
[hep-ph/9804349].
%%CITATION = HEP-PH 9804349;%%

\bibitem{Braaten:1995ez}
E.~Braaten and Y.~Q.~Chen,
%``Signature for color octet production of J / Psi in e+ e- 
%annihilation,''
Phys.\ Rev.\ Lett.\  {\bf 76} (1996) 730
[hep-ph/9508373].
%%CITATION = HEP-PH 9508373;%%

\bibitem{belle-eps2003}
K.~Abe {\it et al.}  [Belle Collaboration], BELLE-CONF-0331, contributed
paper, International Europhysics Conference on High Energy Physics
(EPS 2003), Aachen, Germany, 2003.

%\cite{Baek:1996kq}
\bibitem{Baek:1996kq}
S.~Baek, P.~Ko, J.~Lee, and H.~S.~Song,
%``Color-octet heavy quarkonium productions in Z0 decays at LEP,''
Phys.\ Lett.\ B {\bf 389} (1996) 609 
[hep-ph/9607236].
%%CITATION = HEP-PH 9607236;%%

\bibitem{Abe:2002rb}
K.~Abe {\it et al.}  [Belle Collaboration],
%``Observation of double c anti-c production in e+ e- annihilation at  s**(1/2)
%approx. 10.6-GeV,''
Phys.\ Rev.\ Lett.\  {\bf 89} (2002) 142001
[hep-ex/0205104].
%%CITATION = HEP-EX 0205104;%%

\bibitem{Liu:2003jj}
K.~Y.~Liu, Z.~G.~He and K.~T.~Chao,
%``Inclusive charmonium production via double c anti-c in e+ e-
%annihilation,''
Phys.\ Rev.\ D {\bf 69} (2004) 094027.
%%CITATION = HEP-PH 0301218;%%

\bibitem{Abe:2004ww}
K.~Abe {\it et al.}  [Belle Collaboration],
%``Study of double charmonium production in e+ e- annihilation at s**(1/2)
%approx. 10.6-GeV,''
hep-ex/0407009.
%%CITATION = HEP-EX 0407009;%%

%\cite{Braaten:2002fi}
\bibitem{Braaten:2002fi}
E.~Braaten and J.~Lee,
%``Exclusive double-charmonium production in e+ e- annihilation,''
Phys.\ Rev.\ D {\bf 67} (2003) 054007 
[hep-ph/0211085].
%%CITATION = HEP-PH 0211085;%%

%\cite{Liu:2002wq}
\bibitem{Liu:2002wq}
K.~Y.~Liu, Z.~G.~He, and K.~T.~Chao,
%``Problems of double charm production in e+ e- annihilation at s**(1/2)
%= 10.6-GeV. ((V)),''
Phys.\ Lett.\ B {\bf 557} (2003) 45 
[hep-ph/0211181].
%%CITATION = HEP-PH 0211181;%%

\bibitem{brodsky-ji-lee}
S.~J.~Brodsky, C.-R.~Ji, and J.~Lee, private communication.

%\cite{Bodwin:2002fk}
\bibitem{Bodwin:2002fk}
G.~T.~Bodwin, J.~Lee, and E.~Braaten,
%``e+ e- annihilation into J/psi + J/psi,''
Phys.\ Rev.\ Lett.\  {\bf 90} (2003) 162001 
[hep-ph/0212181].
%%CITATION = HEP-PH 0212181;%%

%\cite{Bodwin:2002kk}
\bibitem{Bodwin:2002kk}
G.~T.~Bodwin, J.~Lee, and E.~Braaten,
%``Exclusive double-charmonium production from e+ e- annihilation into
%two virtual photons,''
Phys.\ Rev.\ D {\bf 67} (2003) 054023 
[hep-ph/0212352].
%%CITATION = HEP-PH 0212352;%%

%\cite{Luchinsky:2003yh}
\bibitem{Luchinsky:2003yh}
A.~V.~Luchinsky,
%``On double 1-- charmonium production through two-photon e+ e-
%annihilation at s**(1/2) = 10.6-GeV,''
hep-ph/0301190.
%%CITATION = HEP-PH 0301190;%%

%\cite{Abe:2003ja}
\bibitem{Abe:2003ja}
K.~Abe {\it et al.}  [Belle Collaboration],
%``Comment on 'e+ e- annhilation into J/psi + J/psi',''
hep-ex/0306015.
%%CITATION = HEP-EX 0306015;%%

%\cite{Ma:2004qf}
\bibitem{Ma:2004qf}
J.~P.~Ma and Z.~G.~Si,
%``Predictions for e+ e- $\to$ J/psi eta/c with light-cone
% wave-functions,''
hep-ph/0405111.
%%CITATION = HEP-PH 0405111;%%

%% B Decay references

%\cite{Balest:1994jf}
\bibitem{Balest:1994jf}
R.~Balest {\it et al.}  [CLEO Collaboration],
%``Inclusive decays of B mesons to charmonium,''
Phys.\ Rev.\ D {\bf 52} (1995) 2661.
%%CITATION = PHRVA,D52,2661;%%

%\cite{Chen:2000ri}
\bibitem{Chen:2000ri}
S.~Chen {\it et al.}  [CLEO Collaboration],
%``Study of chi/c1 and chi/c2 meson production in B meson decays,''
Phys.\ Rev.\ D {\bf 63} (2001) 031102
[hep-ex/0009044].
%%CITATION = HEP-EX 0009044;%%

%\cite{Buskulic:1992wp}
\bibitem{Buskulic:1992wp}
D.~Buskulic {\it et al.}  [ALEPH Collaboration],
%``Measurements of mean lifetime and branching fractions of b hadrons decaying
%to J / psi,''
Phys.\ Lett.\ B {\bf 295} (1992) 396.
%%CITATION = PHLTA,B295,396;%%

%\cite{Adriani:1993ta}
\bibitem{Adriani:1993ta}
O.~Adriani {\it et al.}  [L3 Collaboration],
%``Chi(c) production in hadronic Z decays,''
Phys.\ Lett.\ B {\bf 317} (1993) 467.
%%CITATION = PHLTA,B317,467;%%

%\cite{Ko:1995iv}
\bibitem{Ko:1995iv}
P.~Ko, J.~Lee and H.~S.~Song,
%``Inclusive $S-$wave charmonium productions in $B$ decays,''
Phys.\ Rev.\ D {\bf 53} (1996) 1409 
[hep-ph/9510202].
%%CITATION = HEP-PH 9510202;%%

%\cite{Ko:1999zx}
\bibitem{Ko:1999zx}
P.~Ko, J.~Lee and H.~S.~Song,
%``Testing J/psi production mechanisms in B $\to$ J/psi + X,''
J.\ Korean Phys.\ Soc.\  {\bf 34} (1999) 301.
%%CITATION = JKPSD,34,301;%%

%\cite{Anderson:2002md}
\bibitem{Anderson:2002md}
S.~Anderson {\it et al.}  [CLEO Collaboration],
%``Measurements of inclusive B $\to$ psi production,''
Phys.\ Rev.\ Lett.\  {\bf 89} (2002) 282001.
%%CITATION = PRLTA,89,282001;%%

%\cite{Beneke:1998ks}
\bibitem{Beneke:1998ks}
M.~Beneke, F.~Maltoni and I.~Z.~Rothstein,
%``{QCD} analysis of inclusive B decay into charmonium,''
Phys.\ Rev.\ D {\bf 59} (1999) 054003
[hep-ph/9808360].
%%CITATION = HEP-PH 9808360;%%

%\cite{Ma:2000bz}
\bibitem{Ma:2000bz}
J.~P.~Ma,
%``Effects of the initial hadron in B $\to$ J/psi + X,''
Phys.\ Lett.\ B {\bf 488} (2000) 55
[hep-ph/0006060].
%%CITATION = HEP-PH 0006060;%%

%\cite{Fleming:1996pt}
\bibitem{Fleming:1996pt}
S.~Fleming, O.~F.~Hernandez, I.~Maksymyk and H.~Nadeau,
%``NRQCD matrix elements in polarization of J/psi produced from b decay,''
Phys.\ Rev.\ D {\bf 55} (1997) 4098 
[hep-ph/9608413].
%%CITATION = HEP-PH 9608413;%%

%\cite{Bodwin:1992qr}
\bibitem{Bodwin:1992qr}
G.~T.~Bodwin, E.~Braaten, T.~C.~Yuan and G.~P.~Lepage,
%``P wave charmonium production in B meson decays,''
Phys.\ Rev.\ D {\bf 46} (1992) 3703 
[hep-ph/9208254].
%%CITATION = HEP-PH 9208254;%%

%\cite{Kuhn:1983ar}
\bibitem{Kuhn:1983ar}
J.~H.~K\"uhn and R.~R\"uckl,
%``Clues On Color Suppression From B $\to$ J / Psi X,''
Phys.\ Lett.\  {\bf 135B} (1984) 477
[Erratum-ibid.\ B {\bf 258} (1991) 499].
%%CITATION = PHLTA,135B,477;%%

%% B_C

%\cite{Chang:bb}
\bibitem{Chang:bb}
C.~H.~Chang and Y.~Q.~Chen,
%``The Production Of B(C) Or Anti-B(C) Meson Associated With Two Heavy Quark
%Jets In Z0 Boson Decay,''
Phys.\ Rev.\ D {\bf 46} (1992) 3845
[Erratum-ibid.\ D {\bf 50} (1994) 6013].
%%CITATION = PHRVA,D46,3845;%%

%\cite{Abreu:1996nz}
\bibitem{Abreu:1996nz}
P.~Abreu {\it et al.}  [DELPHI Collaboration],
%``Search for the B/c meson,''
Phys.\ Lett.\ B {\bf 398} (1997) 207.
%%CITATION = PHLTA,B398,207;%%

%\cite{Barate:1997kk}
\bibitem{Barate:1997kk}
R.~Barate {\it et al.}  [ALEPH Collaboration],
%``Search for the B/c meson in hadronic Z decays,''
Phys.\ Lett.\ B {\bf 402} (1997) 213.
%%CITATION = PHLTA,B402,213;%%

%\cite{Ackerstaff:1998zf}
\bibitem{Ackerstaff:1998zf}
K.~Ackerstaff {\it et al.}  [OPAL Collaboration],
%``Search for the B/c meson in hadronic Z0 decays,''
Phys.\ Lett.\ B {\bf 420} (1998) 157
[hep-ex/9801026].
%%CITATION = HEP-EX 9801026;%%

%\cite{Abe:1998wi}
\bibitem{Abe:1998wi}
F.~Abe {\it et al.}  [CDF Collaboration],
%``Observation of the B/c meson in p anti-p collisions at s**(1/2) =  1.8-TeV,''
Phys.\ Rev.\ Lett.\  {\bf 81} (1998) 2432
[hep-ex/9805034].
%%CITATION = HEP-EX 9805034;%%

%\cite{Abe:1998fb}
\bibitem{Abe:1998fb}
F.~Abe {\it et al.}  [CDF Collaboration],
%``Observation of B/c mesons in p anti-p collisions at s**(1/2) = 1.8-TeV,''
Phys.\ Rev.\ D {\bf 58} (1998) 112004
[hep-ex/9804014].
%%CITATION = HEP-EX 9804014;%%

%\cite{Braaten:1993jn}
\bibitem{Braaten:1993jn}
E.~Braaten, K.~m.~Cheung and T.~C.~Yuan,
%``Perturbative QCD fragmentation functions for B(c) and B(c)* production,''
Phys.\ Rev.\ D {\bf 48} (1993) 5049
[hep-ph/9305206].
%%CITATION = HEP-PH 9305206;%%

%\cite{Chang:1992jb}
\bibitem{Chang:1992jb}
C.~H.~Chang and Y.~Q.~Chen,
%``The hadronic production of the B(c) meson at Tevatron, LHC and SSC,''
Phys.\ Rev.\ D {\bf 48} (1993) 4086.
%%CITATION = PHRVA,D48,4086;%%

%\cite{Chang:1994aw}
\bibitem{Chang:1994aw}
C.~H.~Chang, Y.~Q.~Chen, G.~P.~Han and H.~T.~Jiang,
%``On hadronic production of the B(c) meson,''
Phys.\ Lett.\ B {\bf 364} (1995) 78
[hep-ph/9408242].
%%CITATION = HEP-PH 9408242;%%

%\cite{Chang:1996jt}
\bibitem{Chang:1996jt}
C.~H.~Chang, Y.~Q.~Chen and R.~J.~Oakes,
%``Comparative Study of the Hadronic Production of $B_c$ Mesons,''
Phys.\ Rev.\ D {\bf 54} (1996) 4344
[hep-ph/9602411].
%%CITATION = HEP-PH 9602411;%%

%\cite{Kolodziej:1995nv}
\bibitem{Kolodziej:1995nv}
K.~Kolodziej, A.~Leike and R.~R\"uckl,
%``Production of B(c) mesons in hadronic collisions,''
Phys.\ Lett.\ B {\bf 355} (1995) 337
[hep-ph/9505298].
%%CITATION = HEP-PH 9505298;%%

\bibitem{Lusignoli1}
M.~Lusignoli and M.~Masetti,
Z.\ Phys.\ C {\bf 51} (1991) 546.

\bibitem{Scora}
D.~Scora and N.~Isgur,
Phys.\ Rev.\ D {\bf 52} (1995) 2783.

\bibitem{PDG96}
R.~M.~Barnett {\it et al.}, ``Review of Particle Physics'',
Phys.\ Rev.\ D {\bf 54} (1996) 1.

\bibitem{Lusignoli2}
M.~Lusignoli, M.~Masetti and S.~Petrarca,
Phys.\ Lett.\ B {\bf 266} (1991) 142.

%\cite{Berezhnoy:an}
\bibitem{Berezhnoy:an}
A.~V.~Berezhnoy, V.~V.~Kiselev and A.~K.~Likhoded,
%``Photonic Production Of S- And P Wave B/C States And Doubly Heavy Baryons,''
Z.\ Phys.\ A {\bf 356} (1996) 89.
%%CITATION = ZEPYA,A356,89;%%

%\cite{Baranov:wy}
\bibitem{Baranov:wy}
S.~P.~Baranov,
%``Pair Production Of B(C)* Mesons In P P And Gamma Gamma Collisions,''
Phys.\ Rev.\ D {\bf 55} (1997) 2756.
%%CITATION = PHRVA,D55,2756;%%

%\cite{Berezhnoi:1997fp}
\bibitem{Berezhnoi:1997fp}
A.~V.~Berezhnoi, V.~V.~Kiselev, A.~K.~Likhoded and A.~I.~Onishchenko,
%``B/c meson at LHC,''
Phys.\ Atom.\ Nucl.\  {\bf 60} (1997) 1729
[hep-ph/9703341].
%%CITATION = HEP-PH 9703341;%%

%\cite{Cheung:1999ir}
\bibitem{Cheung:1999ir}
K.~m.~Cheung,
%``B/c meson production at the Tevatron revisited,''
Phys.\ Lett.\ B {\bf 472} (2000) 408
[hep-ph/9908405].
%%CITATION = HEP-PH 9908405;%%

%\cite{Chang:2003cq}
\bibitem{Chang:2003cq}
C.~H.~Chang, C.~Driouichi, P.~Eerola and X.~G.~Wu,
%``BCVEGPY: An event generator for hadronic production of the B/c meson,''
Comput.\ Phys.\ Commun.\  {\bf 159} (2004) 192
[hep-ph/0309120].
%%CITATION = HEP-PH 0309120;%%

%\cite{Mangano:1990by}
\bibitem{Mangano:1990by}
M.~L.~Mangano and S.~J.~Parke,
%``Multiparton Amplitudes In Gauge Theories,''
Phys.\ Rept.\  {\bf 200} (1991) 301,
%%CITATION = PRPLC,200,301;%%
and references therein.

%\cite{Sjostrand:1993yb}
\bibitem{Sjostrand:1993yb}
T.~Sj\"ostrand,
%``High-energy physics event generation with PYTHIA 5.7 and JETSET 7.4,''
Comput.\ Phys.\ Commun.\  {\bf 82} (1994) 74.
%%CITATION = CPHCB,82,74;%%

%\cite{Chang:2003cr}
\bibitem{Chang:2003cr}
C.~H.~Chang and X.~G.~Wu,
%``Uncertainties in estimating hadronic production of the meson B/c and
%comparisons between TEVATRON and LHC,''
hep-ph/0309121.
%%CITATION = HEP-PH 0309121;%%

%\cite{Guberina:1980fn}
\bibitem{Guberina:1980fn}
B.~Guberina, R.~Meckbach, R.~D.~Peccei and R.~R\"uckl,
%``Quarkonium Sum Rules: A Critical Reappraisal,''
Nucl.\ Phys.\ B {\bf 184} (1981) 476.
%%CITATION = NUPHA,B184,476;%%

%\cite{Chen:fq}
\bibitem{Chen:fq}
Y.~Q.~Chen and Y.~P.~Kuang,
%``Improved QCD Motivated Heavy Quark Potentials With Explicit Lambda(Ms)
%Dependence,''
Phys.\ Rev.\ D {\bf 46} (1992) 1165
[Erratum-ibid.\ D {\bf 47} (1993) 350].
%%CITATION = PHRVA,D46,1165;%%

%\cite{Eichten:1994gt}
\bibitem{Eichten:1994gt}
E.~J.~Eichten and C.~Quigg,
%``Mesons with beauty and charm: Spectroscopy,''
Phys.\ Rev.\ D {\bf 49} (1994) 5845
[hep-ph/9402210].
%%CITATION = HEP-PH 9402210;%%

%\cite{Kiselev:1994rc}
\bibitem{Kiselev:1994rc}
V.~V.~Kiselev, A.~K.~Likhoded and A.~V.~Tkabladze,
%``B(c) spectroscopy,''
Phys.\ Rev.\ D {\bf 51} (1995) 3613
[hep-ph/9406339].
%%CITATION = HEP-PH 9406339;%%

%\cite{AbdEl-Hady:1999xh}
\bibitem{AbdEl-Hady:1999xh}
A.~Abd El-Hady, J.~H.~Munoz and J.~P.~Vary,
%``Semileptonic and non-leptonic B/c decays,''
Phys.\ Rev.\ D {\bf 62} (2000) 014019
[hep-ph/9909406].
%%CITATION = HEP-PH 9909406;%%

%\cite{Davies:1996gi}
\bibitem{Davies:1996gi}
C.~T.~H.~Davies, K.~Hornbostel, G.~P.~Lepage, A.~J.~Lidsey, 
        J.~Shigemitsu and J.~H.~Sloan,
%``$B_c$ Spectroscopy from Lattice QCD,''
Phys.\ Lett.\ B {\bf 382} (1996) 131
[hep-lat/9602020].
%%CITATION = HEP-LAT 9602020;%%

\end{thebibliography}

