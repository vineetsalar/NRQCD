\section{Introduction}
\label{intro}

Measurements of heavy quark--antiquark bound states (quarkonia) production processes provide 
an insight into the nature of quantum
chromodynamics (QCD) close to the boundary between the perturbative and non-perturbative regimes. 
More than forty years since the discovery of the $\jpsi$, the investigation of
hidden heavy-flavour production in hadronic collisions 
still presents significant challenges to both theory and experiment.

In high-energy hadronic collisions, charmonium states can be produced either directly by 
short-lived QCD sources (``prompt'' production), or by long-lived sources in the decay chains of beauty hadrons (``non-prompt'' production). 
These can be separated experimentally using the
distance between the proton--proton primary interaction and the decay vertex of the quarkonium state.
While {\em Fixed-Order with Next-to-Leading-Log} (FONLL) calculations \cite{FONLL_2001,Cacciari:2012ny}, made within the framework of perturbative QCD, have
been quite successful in describing non-prompt production of various quarkonium states, a satisfactory understanding of the prompt production mechanisms is still to be achieved.


The $\psiprime$ meson is the only vector charmonium state that 
is produced with no significant  contributions from decays of higher-mass quarkonia, 
referred to as feed-down contributions. This
provides a unique opportunity to study production mechanisms specific to
$J^{PC}=1^{--}$ states~\cite{CDFjpsianomaly1,Abulencia:2007us,Abelev:2014qha,Chatrchyan:2011kc,Chatrchyan:2013cla,Aaij:2012ag,Aad:2014fpa,Aaij:2013jxj,Khachatryan:2015rra,Aaij:2013yaa}.
Measurements of the production of $J^{++}$ states with $J=0, 1, 2$,
\cite{CDFjpsianomaly2,ATLAS:2014ala,Chatrchyan:2012ub,Aaij:2013yaa,Aaij:2013dja,LHCb:2012ac},
strongly coupled to the two-gluon channel, allow similar studies in the $CP$-even sector, complementary to the $CP$-odd vector sector. 
Production of $\jpsi$ mesons
\cite{Aad:2011sp,Abulencia:2007us,Aaij:2011jh,Chatrchyan:2011kc,Abelev:2012gx,Chatrchyan:2013cla,Aaij:2013jxj,Abelev:2014qha,CDFjpsianomaly3,CDFjpsianomaly1,D0jpsi1,D0jpsi2,Khachatryan:2015rra,Aad:2014fpa,CDFjpsianomaly2,LHCb:2012af}
arises from a mixture of different sources, receiving contributions from the production of $1^{--}$ and $J^{++}$ states in comparable amounts.


Early attempts to describe the formation of charmonium
\cite{CSM1,CSM2,CSM3,CSM4,CSM5,CSM7,CEM1,CEM2}
using leading-order perturbative QCD 
gave rise to a variety of models, none of which could explain the
large production cross-sections measured at the Tevatron
\cite{CDFjpsianomaly1,CDFjpsianomaly2,CDFjpsianomaly3,D0jpsi1,D0jpsi2}.
Within the colour-singlet model (CSM) \cite{Lansberg:2008gk}, next-to-next-to-leading-order (NNLO) contributions to the hadronic production of S-wave
quarkonia were calculated without introducing any new phenomenological parameters. However,
technical difficulties have so far made it impossible to perform the full NNLO calculation, or to 
extend those calculations to the P-wave states. 
So it is not entirely surprising that the predictions of the model underestimate the experimental data for inclusive production of
\jpsi\ and $\Upsilon$ states, where the feed-down is significant, but offer a better description  for \psiprime\ production \cite{Aad:2011sp,Aad2012dlq}.

Non-relativistic QCD (NRQCD) calculations that include colour-octet (CO) contributions 
\cite{Bodwin:1994jh} introduce a number of phenomenological parameters --- long-distance matrix elements
(LDMEs) --- which are determined from fits to the experimental data, and can hence
describe the cross-sections and differential spectra satisfactorily 
{\cite{CO_LDME1}}. 
However, the attempts to describe the polarization of S-wave quarkonium states using this approach
have not been so successful~\cite{Gong:2012ug},
prompting a suggestion~\cite{Faccioli:2014cqa} that a more coherent approach is needed for the treatment of polarization
within the QCD-motivated models of quarkonium production. 

Neither the CSM nor the NRQCD model gives a satisfactory explanation for the measurement of prompt
\jpsi\ production in association with the $W$~\cite{Aad:2014rua} and $Z$~\cite{Aad:2014kba} bosons: in both cases,
the measured differential cross-section is larger than theoretical expectations~\cite{Li:2010hc,Lansberg:2013wva,Gong:2012ah, Mao:2011kf}.
It is also important to broaden the scope of comparisons 
between theory and experiment by providing a variety of experimental information
about quarkonium production across a wider kinematic range.
In this context, ATLAS has measured the inclusive differential cross-section of \jpsi\  production, with $2.3$~pb$^{-1}$ of integrated luminosity~\cite{Aad:2011sp}, 
at $\sqrt{s} = 7$~\TeV\ using the data collected in 2010, 
as well as the differential cross-sections of the production of  $\chi_c$ states~(4.5~fb$^{-1}$)~\cite{ATLAS:2014ala}, and of the \psiprime\ in its $J/\psi\pi\pi$ decay mode (2.1 fb$^{-1})$~\cite{Aad:2014fpa}, at $\sqrt{s} = 7$~\TeV\ with data collected in 2011.
The cross-section and polarization measurements 
from CDF~\cite{Abulencia:2007us},
CMS~\cite{Chatrchyan:2013cla,Khachatryan:2010yr,CMS},
LHCb~\cite{Aaij:2012ag,Aaij:2013jxj,Aaij:2013yaa,Aaij:2012asz,Aaij:2013nlm,Aaij:2014qea}  
and ALICE~\cite{Abelev:2014qha,Abelev:2011md,Aamodt:2011gj},
cover a considerable variety of charmonium production characteristics in a wide 
kinematic range (transverse momentum $\pt\leq 100$ \GeV\ and rapidities $|y|<5$), thus providing a wealth of
information for a new generation of theoretical models.

This paper presents a precise measurement of
\jpsi\ and  \psiprime\ production in the dimuon decay mode, both at $\sqrt{s} = 7$~TeV
and at $\sqrt{s} = 8$~TeV.
It is presented as a double-differential measurement in transverse momentum and rapidity of the quarkonium state, 
separated into prompt and non-prompt contributions,
covering a range of transverse momenta $8 < \pt\leq 110$ \GeV\ and rapidities $|y|<2.0$. 
The ratios of $\psiprime$ to $\jpsi$  cross-sections for prompt and non-prompt processes are also reported, as well as the 
non-prompt fractions of $\jpsi$ and $\psiprime$.