\section{Candidate selection}
\label{sec:data}

The analysis is based on data recorded at the LHC in 2011 and 2012 during proton--proton collisions at centre-of-mass energies of $7$~\TeV\ and $8$~\TeV, respectively.
This data sample corresponds to a total integrated luminosity of \lumiA\ and \lumiB\ for  $7$~\TeV\ data and $8$~\TeV\ data, respectively.

Events were selected  using a trigger requiring two oppositely charged muon candidates, each passing the requirement $\pt>4$~\GeV. 
The muons are constrained to originate from a common vertex, which is fitted with the track parameter uncertainties taken into account. 
The fit is required to satisfy $\chi^2 < 20$ for the one degree of freedom.

For $7$~\TeV\ data, the Level-1 trigger required only spatial coincidences in the MS~\cite{ATLAS:2010kba}. For $8$~\TeV\ data, a $4$~\GeV\ muon $\pt$ threshold was also applied at Level-1, which reduced the trigger efficiency for low-$\pt$ muons.

The offline analysis requires events to have at least two muons, identified by the muon spectrometer and with matching tracks reconstructed in the ID~\cite{muons}.
Due to the ID acceptance, muon reconstruction is possible only for $|\eta| <$ 2.5. 
The selected muons are further restricted to $|\eta| <$ 2.3 to ensure high-quality tracking and triggering, and to reduce the contribution from misidentified muons.
For the momenta of interest in this analysis (corresponding to muons with a transverse momentum of at most $O(100)$ \GeV), measurements of the
muons are degraded by multiple scattering within the MS and so only the ID tracking
information is considered. 
To ensure accurate ID measurements, each muon track must fulfil muon reconstruction and selection requirements~\cite{muons}.
The pairs of muon candidates satisfying these quality criteria are required to have opposite charges.

In order to allow an accurate correction for trigger inefficiencies, each reconstructed muon candidate is required to match a trigger-identified muon
candidate within a cone of $\Delta R = \sqrt{(\Delta\eta)^2 + (\Delta\phi)^2}=0.01$.
Dimuon candidates are obtained from muon pairs, constrained to originate from a common vertex using ID track parameters and uncertainties, with a 
requirement of $\chi^2 < 20$ of the vertex fit for the one degree of freedom.
All dimuon candidates with an invariant mass
within $2.6 < m(\mu\mu) < 4.0$ \GeV\ and within the kinematic range $\pt(\mu\mu) > 8$ \GeV, $|y(\mu\mu)| < 2.0$ are retained for the analysis.
If multiple candidates are found in an event (occurring in approximately $10^{-6}$ of selected events), all candidates are retained.
The properties of the dimuon system, such as invariant mass $m(\mu\mu)$, transverse momentum $\pt(\mu\mu)$, and rapidity $|y(\mu\mu)|$ are determined from the result of the vertex fit.


