\documentclass[aps,prc,preprint]{revtex4}
\begin{document}

{\bf Report of Referee A -- CR10456D/Kumar } \\


The present manuscript analyzes charmonium production in hadronic collisions
within the framework of Non-Relativistic QCD (NRQCD). The authors use a
leading-order (LO) NRQCD analysis throughout the paper. I fail to see how such
an analysis brings anything new at this point, since the current state of the
art is to use next-to-leading-order (NLO) full NRQCD analyses. The NLO NRQCD
computation has already been performed by three different groups, and all of
them have extensively analyzed the available data (including LHC data) and
performed several phenomenological studies. In fact, there has been much
discussion lately about finally being able to prove, or disprove, the validity
of the NRQCD factorization formalism. All these discussions consider NLO
analyses and the LHC data. I do not see any reason why a LO analysis, such as
the present one, brings any new significant piece of information. In summary, I
do not think this manuscript contains any piece of new and significant
information that could justify its publication.

Ans: We thanks the referee for his comments. Our response is as follows:
 An LO NRQCD analyis is useful as it is straightforward and once
the parameters are obtained by fitting over large datasets it has excellent 
predictability power for unknown cross sections. They provide a reference
for comparison with NLO calculations which vary since the NLO techniques are
not unique. Example, results LDMEs from Ref. .. and from Ref. .. do noth 
match.
 The shortcomings of NLO analysis .......[28] and [29] of our paper.
 Thw work includes most uptodate datasets for fitting.

\ \\

{\bf Report of Referee B -- CR10456D/Kumar} \\


This manuscript calculated the hadroproduction cross sections of the 
$J/\psi$ ,  $\psi$(2s) and $\chi_c$ mesons at QCD LO. These calculations have been 
performed 20 years ago in e.g. Ref. [1–4] and also in a recent paper [5]. Neither 
the inclusion of the LHC data in the fitting nor the prediction of 
$\sigma(\psi(2s))/(J/\psi)$ is new.  Actually, they have been studied at QCD NLO
level [6, 7]. Nevertheless, I believe that an updated QCD LO study on the charmonium
hadroproduction is necessary for the convenience of some researchers who have not been
equipped with the tools to do QCD NLO computation. But just these materials do not
justify the publication in Physical Review D. I suggest that the authors could publish 
their paper in some lower-level journal. However, if they can make some major 
extensions, I can also recommend publication of their paper in Physical Review D.
 In addition, I provide some comments regarding more specific issues of this paper 
as follows.

Ans: We thank the referee for his thorough and useful comments. We sincerly attempt
to address all his comments.
Add reference 1-4 and see reference 7. 


1. The authors should provide the complete definitions of the variables used in their
manuscript, for example, the definition of $M$, $\sigma$, $p_T$ , $m_T$ , and $m_H$. 
The symbol $\times$ in Eq.(1) is confusing. According to my calculation, it should be 
simply ”times”.

Ans: $M$ is the mass ....   $m_H$ is the mass of the heavy meson. 
$m_T=\sqrt{p_T^2 + M^2}$ ........
 Yes $\times$ should be replaced 
\ \\

2. The statement below Eq.(1), ”... depends ... on the renormalization scale $\mu_R$” 
is wrong. The PDF depends on $\mu_F$ , however, does not depend on $\mu_R$.

Ans: Agreed.

\ \\

3. The authors used $M_L(QQ(n) \rightarrow H)$ to denote the LDMEs. Although optional, 
I believe that using the generally used notations would improve the readability of 
their paper.

Ans: The generally used notations ....

\ \\

4. The relations between the LDMEs for $^3S_1^{[8]}$ and $^3P^{[8]}_J$ to $J/\psi$  
and  $\psi$(2s) in Eq.(10) and Eq.(11) are strange. Actually, there is neither 
theoretical nor phenomenological evidence for these relations. More confusingly, 
even their own results in Eq.(14) and Eq.(15) do not support their relations.

Ans: There is inconsistency in these equations .. check

\ \\

5. Another optional suggestion is that the authors could include the RHIC data, 
which is also suitable for perturbative calculations.

Ans: The RHIC data are very low pT and hence are not included in the fit.
  But we give a comparison of calculations with RHIC data.

\ \\


6. I don’t know why the authors ignored the copius $\chi_c$ data at the Tevatron 
and the LHC, namely Ref. [8–12]. The authors should at least address this, and 
compare there results with Ref. [13]. They should also notice that Ref. [14] only 
measured four points, namly the ratio $\sigma(\chi_c)/(J/\psi)$. The differential 
cross section for $\chi_c$ production was obtained by extrapolation.

Ans: This dataset is used in the present work now.

\ \\


7. The presentation of the LDMEs, $M_L(^1S^{[8]}_0 ...) = M_L(^3P^{[8]}_0 ...)/m^2_{charm}$
in Eq.(14) and Eq.(15) seems strange to me. This equation has no foundation. 
So, I suggest they present their results following the form in Ref. [2] or Ref. [1]. 
Actually, they can use the $\eta_c$ hadroproduction data to fix these LDMEs, as 
Ref. [15] did. It would be interesting to see whether this approach also applies 
at QCD LO.

Ans: Check these Eqs.  Check Ref. 1 and 2.

\ \\

8. The authors should compare their results with Ref. [1, 4, 5]. Once the authors 
can address all the issues raised above and make major extensions, I can recommend 
publication of this paper in Physical Review D.

Ans: We give a table for LDMES with results from LO, Ref[1,4,5] and 
NLO from our paper Ref [28, 29].\\
Our values of LDMEs
1. J/$\psi$ 





The way, we are doing the fitting is not proper. We have to fit the linear combintion of LDMEs.\\







Ref [1] defined a quantity M$_{J/\psi}$ as 
\begin{equation}
M_{J/\psi} = \frac{3P0_8}{m_{c}^2} + \frac{1S0_8}{3} \\ \nonumber
\end{equation}









\ \\


[1] P. L. Cho and A. K. Leibovich, Phys. Rev. D53, 6203 (1996), hep-ph/9511315.

[2] M. Beneke and M. Kramer, 1, Phys. Rev. D55, 5269 (1997), hep-ph/9611218.

[3] A. K. Leibovich, Phys. Rev. D56, 4412 (1997), hep-ph/9610381.

[4] E. Braaten, B. A. Kniehl, and J. Lee, Phys. Rev. D62, 094005 (2000), hep-ph/9911436.

[5] R. Sharma and I. Vitev, Phys. Rev. C87, 044905 (2013), 1203.0329.

[6] H.-S. Shao, H. Han, Y.-Q. Ma, C. Meng, Y.-J. Zhang, and K.-T. Chao, JHEP 05, 103 (2015),
1411.3300.

[7] Z. Sun and H.-F. Zhang (2015), 1505.02675.

[8] A. Abulencia et al. (CDF), Phys. Rev. Lett. 98, 232001 (2007), hep-ex/0703028.

[9] R. Aaij et al. (LHCb), Phys. Lett. B718, 431 (2012), 1204.1462.

[10] R. Aaij et al. (LHCb), JHEP 10, 115 (2013), 1307.4285.

[11] S. Chatrchyan et al. (CMS), Eur. Phys. J. C72, 2251 (2012), 1210.0875.

[12] G. Aad et al. (ATLAS), JHEP 07, 154 (2014), 1404.7035.

[13] H.-F. Zhang, L. Yu, S.-X. Zhang, and L. Jia, Phys. Rev. D93, 054033 (2016), 
 [Addendum: Phys. Rev.D93,no.7,079901(2016)], 1410.4032.

[14] F. Abe et al. (CDF), Phys. Rev. Lett. 79, 578 (1997).

[15] H.-F. Zhang, Z. Sun, W.-L. Sang, and R. Li, Phys. Rev. Lett. 114, 092006 (2015), 1412.0508.

\end{document}
