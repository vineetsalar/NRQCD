\documentclass[aps,prc,preprint,superscriptaddress,showpacs,showkeys,amsmath]{revtex4-1}
\usepackage{graphicx}
\usepackage[usenames,dvipsnames,svgnames,table]{xcolor}
\usepackage{rotating}
\usepackage{graphicx}% Include  files
\usepackage{dcolumn}% Align table columns on decimal point
\usepackage{bm}% bold math
\usepackage{epsfig}
\usepackage{hyperref}
\usepackage{ulem}
\usepackage{appendix}



%\documentclass[aps,prc,preprint]{revtex4}
%\usepackage{graphicx}
%\usepackage[usenames,dvipsnames,svgnames,table]{xcolor}

\begin{document}
%%%%%\newcommand{\1S08And3P08}{$\frac{[^1S_{0}]_{8}}{3}+\frac{[^3P_0]_{8}}{m_{c}^2}$}

{\bf Report of Referee A -- CR10456D/Kumar } \\


The present manuscript analyzes charmonium production in hadronic collisions
within the framework of Non-Relativistic QCD (NRQCD). The authors use a
leading-order (LO) NRQCD analysis throughout the paper. I fail to see how such
an analysis brings anything new at this point, since the current state of the
art is to use next-to-leading-order (NLO) full NRQCD analyses. The NLO NRQCD
computation has already been performed by three different groups, and all of
them have extensively analyzed the available data (including LHC data) and
performed several phenomenological studies. In fact, there has been much
discussion lately about finally being able to prove, or disprove, the validity
of the NRQCD factorization formalism. All these discussions consider NLO
analyses and the LHC data. I do not see any reason why a LO analysis, such as
the present one, brings any new significant piece of information. In summary, I
do not think this manuscript contains any piece of new and significant
information that could justify its publication.\\
{\color{blue}
Ans: We thanks the referee for his comments. Our response is as follows:\\
  A LO NRQCD analyis is useful as it is straightforward and unique and once
the parameters are obtained by fitting over large datasets it has excellent 
predictability power for unknown cross sections.
 The short distance QCD cross-sections calculation techniques at NLO are not unique. 
Moreover the NLO cross sections are not available in public. 
 We compare here the work of two groups ~\cite{Butenschoen:2010rq,Ma:2010jj} who 
calculate the J/$\psi$ cross-section at NLO. Ref.~\cite{Butenschoen:2010rq} fit 
all three color-octet ($[^3S_1]_{8}$, $[^1S_0]_{8}$ and $[^3P_0]_{8}$ ) LDMEs independently 
while Ref.~\cite{Ma:2010jj} define two quantities M$^{J/\psi}_{1,r_{1}}$  and  M$^{J/\psi}_{0,r_{0}}$
as a linear combination of $[^3S_1]_{8}$, $[^1S_0]_{8}$ and $[^3P_0]_{8}$.
\begin{equation}
M^{J/\psi}_{0,r_{0}} = [^1S_0]_{8}  + \frac{r_{0}}{m_{c}^{2}}[^3P_0]_{8}\\ \nonumber
\end{equation}
and
\begin{equation}
M^{J/\psi}_{1,r_{0}} = [^3S_1]_{8}  + \frac{r_{1}}{m_{c}^{2}}[^3P_0]_{8}\\ \nonumber
\end{equation}
It uses values of r$_{0}$=3.9 and r$_{1}$=-0.56. 
\begin{table}[h]
\caption{Comparison of J/$\psi$ LDMEs at NLO}
\begin{tabular}{|l|c|c|c|}
\hline            
                             &$[^3S_1]_{1}$(GeV$^3$)                     &$M^{J/\psi}_{1,r_{1}}$(10$^{-2}$GeV$^3$)    &$M^{J/\psi}_{0,r_{0}}$(10$^{-2}$GeV$^3$)  \\        
\hline
\cite{Ma:2010jj}             &1.16                                     &0.05$\pm$0.02$\pm$0.02              &7.4 $\pm$ 1.9 $\pm$ 0.4 \\
\cite{Butenschoen:2010rq}    &1.32                                     &0.594                               &2.47 \\
\hline
\end{tabular}
\label{table:LDMEJPsiNLO_I}
\end{table}
We compare the values of Ref.~\cite{Butenschoen:2010rq} and Ref.~\cite{Ma:2010jj} in 
the Table~\ref{table:LDMEJPsiNLO_I} which clearly shows that LDMEs from the two works 
are different from each other. 
 An updated QCD LO study on the charmonium hadroproduction is useful as it
provides a reference for comparison with NLO calculations.
 Many NLO analysis do not include the feed down contribution from the higher charmonia 
states. Our work includes most uptodate datasets and include feeddown contributions.
}
 \\

{\bf Report of Referee B -- CR10456D/Kumar} \\

This manuscript calculated the hadroproduction cross sections of the 
$J/\psi$ ,  $\psi$(2s) and $\chi_c$ mesons at QCD LO. These calculations have been 
performed 20 years ago in e.g.~\cite{Cho:1995vh,Beneke:1996yw,Leibovich:1996pa,Braaten:1999qk} 
and also in a recent paper~\cite{Sharma:2012dy}. Neither the inclusion of the LHC data in the 
fitting nor the prediction of $\sigma(\psi(2s))/(J/\psi)$ is new.  Actually, they have 
been studied at QCD NLO level~\cite{Shao:2014yta,Sun:2015pia}. Nevertheless, I believe 
that an updated QCD LO study on the charmonium
hadroproduction is necessary for the convenience of some researchers who have not been
equipped with the tools to do QCD NLO computation. But just these materials do not
justify the publication in Physical Review D. I suggest that the authors could publish 
their paper in some lower-level journal. However, if they can make some major 
extensions, I can also recommend publication of their paper in Physical Review D.
 In addition, I provide some comments regarding more specific issues of this paper 
as follows.

Ans: We thank the referee for his thorough and useful comments. We sincerly attempt
to address all his comments.

%Add reference 1-4 and see reference 7. 

\begin{enumerate}

\item The authors should provide the complete definitions of the variables used in their
manuscript, for example, the definition of $M$, $\sigma$, $p_T$ , $m_T$ , and $m_H$. 
The symbol $\times$ in Eq.(1) is confusing. According to my calculation, it should be 
simply ”times”.\\
{\color{blue}
Ans: We have defined all the variables in the equation (1) and replace the $\times$
symbol. 
%$M$ is the mass ....   $m_H$ is the mass of the heavy meson. 
%$m_T=\sqrt{p_T^2 + M^2}$ ........
%Yes $\times$ should be replaced 
}
\item The statement below Eq.(1), ”... depends ... on the renormalization scale $\mu_R$” 
is wrong. The PDF depends on $\mu_F$ , however, does not depend on $\mu_R$. \\
{\color{blue}
Ans: We agree with the refree and modify the statement accordingly.
}
\item The authors used $M_L(QQ(n) \rightarrow H)$ to denote the LDMEs. Although optional, 
I believe that using the generally used notations would improve the readability of 
their paper. \\
{\color{blue}
Ans: There is no fixed notation for LDMEs. Different authors use different notations according to
their tastes. In this version we choose to keep our notations. 
%We will prefer our notations but if refree feels very strongly about it We will 
%change them. 
}

\item The relations between the LDMEs for $^3S_1^{[8]}$ and $^3P^{[8]}_J$ to $J/\psi$  
and  $\psi$(2s) in Eq.(10) and Eq.(11) are strange. Actually, there is neither 
theoretical nor phenomenological evidence for these relations. More confusingly, 
even their own results in Eq.(14) and Eq.(15) do not support their relations.\\
{\color{blue}
Ans: As per last version, we were using equation 14 and 15 for the fitting purposes. There was
a typo in equation 10 and 11. We have corrected it. However now we are fitting the 
linear combination of the LDMEs according to your suggestion in point no. 7.
}
\item Another optional suggestion is that the authors could include the RHIC data, 
  which is also suitable for perturbative calculations.\\
  {\color{blue}
    Ans: The RHIC data are at very low $p_{T}$ and hence are not included in the fit.
    %But we give a comparison of calculations with RHIC data.
  }


\item I don’t know why the authors ignored the copius $\chi_c$ data at the Tevatron 
and the LHC, namely~\cite{LHCb:2012af,Aaij:2013dja,Chatrchyan:2012ub,ATLAS:2014ala}. The authors 
should at least address this, and compare there results with ~\cite{Jia:2014jfa}. They should also notice that~\cite{Abe:1997yz} 
only measured four points, namly the ratio $\sigma(\chi_c)/(J/\psi)$. The differential 
cross section for $\chi_c$ production was obtained by extrapolation. \\
{\color{blue}
Ans: Now we use the dataset of ~\cite{LHCb:2012af,Aaij:2013dja,Chatrchyan:2012ub,ATLAS:2014ala} for fitting and
comparison tables are given.%
\ref{table:LDMEChic0}.
}

\item The presentation of the LDMEs, $M_L(^1S^{[8]}_0 ...) = M_L(^3P^{[8]}_0 ...)/m^2_{charm}$
in Eq.(14) and Eq.(15) seems strange to me. This equation has no foundation. 
So, I suggest they present their results following the form in Ref.~\cite{Beneke:1996yw} 
or Ref.~\cite{Cho:1995vh}. 
Actually, they can use the $\eta_c$ hadroproduction data to fix these LDMEs, as 
Ref.~\cite{Zhang:2014ybe} did. It would be interesting to see whether this approach also applies 
at QCD LO.\\

{\color{blue}
Ans: We are following the method described in Ref.~\cite{Cho:1995vh,Beneke:1996yw} and
presenting the results accordingly as per your suggestion.
}

\item The authors should compare their results with Ref.~\cite{Cho:1995vh, Braaten:1999qk, Sharma:2012dy}. Once the authors 
can address all the issues raised above and make major extensions, I can recommend 
publication of this paper in Physical Review D.\\


{\color{blue}

Ans: We give comparison tables for LDMEs with our results from LO, Ref.~\cite{Cho:1995vh, Braaten:1999qk, Sharma:2012dy} and 
NLO from Ref.~\cite{Butenschoen:2010rq,Ma:2010jj} in the paper. Here we compare the NLO LDMEs given 
in Ref.~\cite{Butenschoen:2010rq} and Ref.~\cite{Ma:2010jj} with each other. Ref.~\cite{Butenschoen:2010rq} fit all three 
$[^3S_1]_{8}$, $[^1S_0]_{8}$ and $[^3P_0]_{8}$ LDMEs independently while Ref.~\cite{Ma:2010jj} defines two 
quantities M$^{J/\psi}_{1,r_{1}}$  and  M$^{J/\psi}_{0,r_{0}}$ as a linear combination 
of $[^3S_1]_{8}$, $[^1S_0]_{8}$ and $[^3P_0]_{8}$ color octet LDMEs as
  
\begin{equation}
M^{J/\psi}_{0,r_{0}} = [^1S_0]_{8}  + \frac{r_{0}}{m_{c}^{2}}[^3P_0]_{8}\\ \nonumber
\end{equation}
and
\begin{equation}
M^{J/\psi}_{1,r_{0}} = [^3S_1]_{8}  + \frac{r_{1}}{m_{c}^{2}}[^3P_0]_{8}\\ \nonumber
\end{equation}

It uses values of $r_{0}$=3.9 and $r_{1}$=-0.56. We compare the values of Ref~\cite{Butenschoen:2010rq}
and Ref~\cite{Ma:2010jj} here.

}
\begin{table}[h]
\caption{Comparison of J/$\psi$ LDMEs}
\begin{tabular}{|l|c|c|c|}
\hline            
                             &$[^3S_1]_{1}$(GeV$^3$)                     &$M^{J/\psi}_{1,r_{1}}$(10$^{-2}$GeV$^3$)    &$M^{J/\psi}_{0,r_{0}}$(10$^{-2}$GeV$^3$)  \\        
\hline
\cite{Ma:2010jj}             &1.16                                     &0.05$\pm$0.02$\pm$0.02              &7.4 $\pm$ 1.9 $\pm$ 0.4 \\
\cite{Butenschoen:2010rq}    &1.32                                     &0.594                               &2.47 \\
\hline
\end{tabular}
\label{table:LDMEJPsiNLO}
\end{table}



\end{enumerate}


%\\

\noindent
\begin{thebibliography}{100}
\medskip



\bibitem{Cho:1995vh} 
  P.~L.~Cho and A.~K.~Leibovich,
  ``Color octet quarkonia production,''
  Phys.\ Rev.\ D {\bf 53}, 150 (1996),
  [hep-ph/9505329].

\bibitem{Beneke:1996yw} 
  M.~Beneke and M.~Kramer, 1,
  ``Direct $J/\psi$ and $\psi^\prime$ polarization and cross-sections at the Tevatron,''
  Phys.\ Rev.\ D {\bf 55}, 5269 (1997), [hep-ph/9611218].

%[3] A. K. Leibovich, Phys. Rev. D56, 4412 (1997), hep-ph/9610381.
\bibitem{Leibovich:1996pa} 
  A.~K.~Leibovich,
  ``Psi-prime polarization due to color octet quarkonia production,''
  Phys.\ Rev.\ D {\bf 56}, 4412 (1997), [hep-ph/9610381].

%[4] E. Braaten, B. A. Kniehl, and J. Lee, Phys. Rev. D62, 094005 (2000), hep-ph/9911436.
\bibitem{Braaten:1999qk} 
  E.~Braaten, B.~A.~Kniehl and J.~Lee,
  ``Polarization of prompt $J/\psi$ at the Tevatron,''
  Phys.\ Rev.\ D {\bf 62}, 094005 (2000),[hep-ph/9911436].

%[5] R. Sharma and I. Vitev, Phys. Rev. C87, 044905 (2013), 1203.0329.
\bibitem{Sharma:2012dy} 
  R.~Sharma and I.~Vitev,
  ``High transverse momentum quarkonium production and dissociation in heavy ion collisions,''
  Phys.\ Rev.\ C {\bf 87}, no. 4, 044905 (2013)
  [arXiv:1203.0329 [hep-ph]].


%[6] H.-S. Shao, H. Han, Y.-Q. Ma, C. Meng, Y.-J. Zhang, and K.-T. Chao, JHEP 05, 103 (2015),1411.3300.
\bibitem{Shao:2014yta} 
  H.~S.~Shao, H.~Han, Y.~Q.~Ma, C.~Meng, Y.~J.~Zhang and K.~T.~Chao,
  ``Yields and polarizations of prompt $J/\psi$ and $\psi(2S)$ production in hadronic collisions,''
  JHEP {\bf 1505}, 103 (2015)
  doi:10.1007/JHEP05(2015)103
  [arXiv:1411.3300 [hep-ph]].


%[7] Z. Sun and H.-F. Zhang (2015), 1505.02675.
\bibitem{Sun:2015pia} 
  Z.~Sun and H.~F.~Zhang,
  ``Reconciling charmonium production and polarization data within the nonrelativistic QCD framework,'',
  arXiv:1505.02675 [hep-ph].


%[8] A. Abulencia et al. (CDF), Phys. Rev. Lett. 98, 232001 (2007), hep-ex/0703028.
\bibitem{Abulencia:2007bra} 
  A.~Abulencia {\it et al.} [CDF Collaboration],
  ``Measurement of $\sigma_{\chi_{c2}}{\cal B}(\chi_{c2} \to J/\psi \gamma)/\sigma_{\chi_{c1}} {\cal B}(\chi_{c1} \to J/\psi \gamma)$ 
  in $p \bar{p}$ collisions at $\sqrt{s}$ = 1.96-TeV,''
  Phys.\ Rev.\ Lett.\  {\bf 98}, 232001 (2007),[hep-ex/0703028 [HEP-EX]].

%[9] R. Aaij et al. (LHCb), Phys. Lett. B718, 431 (2012), 1204.1462.

\bibitem{LHCb:2012af} 
  R.~Aaij {\it et al.} [LHCb Collaboration],
  ``Measurement of the ratio of prompt $\chi_{c}$ to $J/\psi$ production in $pp$ collisions at $\sqrt{s}=7$ TeV,''
  Phys.\ Lett.\ B {\bf 718}, 431 (2012),[arXiv:1204.1462 [hep-ex]].


 %[10] R. Aaij et al. (LHCb), JHEP 10, 115 (2013), 1307.4285.
\bibitem{Aaij:2013dja} 
  R.~Aaij {\it et al.} [LHCb Collaboration],
  ``Measurement of the relative rate of prompt $\chi_{c0}$, $\chi_{c1}$ and $\chi_{c2}$ production at $\sqrt{s}=7$TeV,''
  JHEP {\bf 1310}, 115 (2013), [arXiv:1307.4285 [hep-ex]].
 
%[11] S. Chatrchyan et al. (CMS), Eur. Phys. J. C72, 2251 (2012), 1210.0875.
\bibitem{Chatrchyan:2012ub} 
  S.~Chatrchyan {\it et al.} [CMS Collaboration],
  ``Measurement of the relative prompt production rate of $\chi_{c2}$ and $\chi_{c1}$ 
  in $pp$ collisions at $\sqrt{s}=7$ TeV,''
  Eur.\ Phys.\ J.\ C {\bf 72}, 2251 (2012),[arXiv:1210.0875 [hep-ex]].



%[12] G. Aad et al. (ATLAS), JHEP 07, 154 (2014), 1404.7035.
\bibitem{ATLAS:2014ala} 
  G.~Aad {\it et al.} [ATLAS Collaboration],
  ``Measurement of $\chi_{c1}$ and $\chi_{c2}$ production with $\sqrt{s}$ = 7 TeV $pp$ collisions at ATLAS,''
  JHEP {\bf 1407}, 154 (2014), [arXiv:1404.7035 [hep-ex]].

%[13] H.-F. Zhang, L. Yu, S.-X. Zhang, and L. Jia, Phys. Rev. D93, 054033 (2016), 
 [Addendum: Phys. Rev.D93,no.7,079901(2016)], 1410.4032.
\bibitem{Jia:2014jfa} 
  H.~F.~Zhang, L.~Yu, S.~X.~Zhang and L.~Jia,
  ``Global analysis of the experimental data on $\chi_c$ meson hadroproduction,''
  Phys.\ Rev.\ D {\bf 93}, no. 5, 054033 (2016) Addendum: [Phys.\ Rev.\ D {\bf 93}, no. 7, 079901 (2016)], 
  [arXiv:1410.4032 [hep-ph]].

%[14] F. Abe et al. (CDF), Phys. Rev. Lett. 79, 578 (1997).
\bibitem{Abe:1997yz} 
  F.~Abe {\it et al.} [CDF Collaboration],
  ``Production of $J/\psi$ mesons from $\chi_c$ meson decays in $p\bar{p}$ collisions at $\sqrt{s} = 1.8$ TeV,''
  Phys.\ Rev.\ Lett.\  {\bf 79}, 578 (1997).


%[15] H.-F. Zhang, Z. Sun, W.-L. Sang, and R. Li, Phys. Rev. Lett. 114, 092006 (2015), 1412.0508.
\bibitem{Zhang:2014ybe} 
  H.~F.~Zhang, Z.~Sun, W.~L.~Sang and R.~Li,
  ``Impact of $\eta_c$ hadroproduction data on charmonium production and polarization within NRQCD framework,''
  Phys.\ Rev.\ Lett.\  {\bf 114}, no. 9, 092006 (2015), [arXiv:1412.0508 [hep-ph]].
  


\bibitem{Butenschoen:2010rq} 
  M.~Butenschoen and B.~A.~Kniehl,
  ``Reconciling $J/\psi$ production at HERA, RHIC, Tevatron, and LHC with NRQCD factorization at next-to-leading order,''
  Phys.\ Rev.\ Lett.\  {\bf 106}, 022003 (2011). 
[arXiv:1009.5662 [hep-ph]].
 

\bibitem{Ma:2010jj} 
  Y.~Q.~Ma, K.~Wang and K.~T.~Chao,
  ``A complete NLO calculation of the $J/\psi$ and $\psi'$ production at hadron colliders,''
  Phys.\ Rev.\ D {\bf 84}, 114001 (2011),
  [arXiv:1012.1030 [hep-ph]].








\end{thebibliography}

\end{document}
